\section{Preliminaries}

$\Int^+$ is the set of positive integers. $\nat$ is the set of natural numbers. $[n]:=\{1, \ldots, n\}$. 

\begin{definition}[Finite-state automata] \label{def:nfa}
	A \emph{(nondeterministic) finite-state automaton}
	(\FA{}) over a finite alphabet $\ialphabet$ is a tuple $\Aut =
	(\ialphabet, \controls, q_0, \finals, \transrel)$ where 
	$\controls$ is a finite set of 
	states, $q_0\in \controls$ is
	the initial state, $\finals\subseteq \controls$ is a set of final states, and 
	$\transrel\subseteq \controls \times 
	\ialphabet \times  \controls$ is the
	transition relation. 
\end{definition}

For an input string $w=a_1 \dots a_n$, a \emph{run} of $\Aut$ on $w$
%(with $a_0 = \EndLeft$ and $a_{n+1} = \EndRight$)
is a sequence of states $q_0, \ldots, q_n$ such that $(q_{j-1}, a_{j}, q_{j}) \in
\transrel$  for every $j \in [n]$.
The run is said to be \defn{accepting} if $q_n \in \finals$.
A string $w$ is \defn{accepted} by $\Aut$ if there is an accepting run of
$\Aut$ on $w$. In particular, the empty string $\varepsilon$ is accepted by $\Aut$ iff $q_0 \in F$. The set of strings accepted by $\Aut$ is denoted by $\Lang(\Aut)$, i.e., the language \defn{recognised} by $\Aut$.
%Since we deal with computational complexity in the sequel, we define
The \defn{size} $|\Aut|$ of $\Aut$ is defined to be $|\controls|$; we will
use this when we discuss computational complexity.

For convenience, for $a \in \Sigma$, we use $\delta^{(a)}$ to denote the  relation $\{(q, q') \mid (q, a, q') \in \delta\}$.

For a finite set $Q$, let $\overline{Q} = \{ (q_1, \ldots, q_n) \mid n \in \nat \wedge \forall i \in [n], q_i \in Q \wedge \forall i,j \in[n], i
  \neq j \rightarrow q_i \neq q_j \}$. Intuitively, $\overline{Q}$ is the set of
sequences of non-repetitive elements from $Q$. Note that the length of each sequence from $\overline{Q}$ is bounded by 
  $| Q |$.

% \bar vs. \overline 

For some sequence $P = (q_0 \ldots q_n) \in \overline{Q}$ and  $q \in Q$, we write $q \in P$ if 
  $ q = q_i$ for some $i \in [n]$. 

\begin{definition}[Prioritized Finite-state automata]
  A \emph{prioritized finite-state automaton} (pNFA) over a finite alphabet $\Sigma$ is a tuple $\pnfa=(Q, \Sigma, \delta, q_0, F)$ where $\delta \in Q
  \times \Sigma \rightarrow \overline{Q}$. The definition of $Q, q_0$ and $F$ is the same as ordinary NFA.
\end{definition}

A run of $\pnfa$ is the sequence $q_0 \sigma_1 q_1 \ldots \sigma_m q_m$, where $q_m \in F$ and for any $i \in [m], q_i \in \delta (q_{i - 1}, \sigma_i)$. For any two runs $p = q_0 \sigma_1 q_1 \ldots \sigma_m q_m$ and $p' =
  q_0 \sigma_1 q_1' \ldots \sigma_m q_m'$ on $w = \sigma_1 \ldots \sigma_m$, we say that $p$ is of a higher priority over
  $p'$ if $p \neq p'$, and for the smallest index $j$ with $q_j \neq q_j'$,
  $\delta (q_{j - 1}, \sigma_j) = \ldots q_j \ldots q_j' \ldots$
  
  The accepting run of $\pnfa$ on $w$ is the one with the highest priority. The language of pNFA $\pnfa$ is the set of
  strings which have an accepting run.
  