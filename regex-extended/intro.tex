
\section{Introduction}



Strings are a fundamental data type in virtually all programming languages.
Their generic nature can, however, lead to many subtle programming
bugs, some with security consequences, e.g., cross-site scripting
(XSS), which is among the OWASP Top 10 Application Security Risks
\cite{owasp17}. 

One effective
automatic testing method for identifying subtle programming errors
%automatically generating test cases
%generation with a good coverage
is based on \emph{symbolic execution}
\cite{king76} and combinations with dynamic analysis
called \emph{dynamic symbolic execution} \cite{jalangi,DART,EXE,CUTE,KLEE}.
See \cite{symbex-survey} for an excellent survey. Unlike purely random testing,
which runs only \emph{concrete} program executions on different
inputs, the techniques of symbolic execution analyse \emph{static} paths
(also called symbolic executions) through the software system under test.
%Since such a path is simply a sequence of assignments and
%conditionals/assertions,
Such a path can be viewed as a constraint $\varphi$ (over
appropriate data domains) and the hope is that a fast
solver is available for checking the satisfiability of $\varphi$ (i.e. to check
the \emph{feasibility} of the static path), which can be used for generating
inputs that lead to certain parts of the program or an erroneous behaviour.
%a undesirable program behaviour.
%or an exploration of a new part of the
%system.


A regular expression (shortened as regex) is a sequence of characters that define a search pattern. Usually such patterns are used by string-searching algorithms for "find" or "find and replace" operations on strings, or for input validation.  


In programming language, stress regex, which admits features such as capturing groups, reference


symbolic execution requires SMT solving 

the main contribution is to introduce  a new transducer model, prioritized streaming string transducer (PSST) with the  motivation 
(1) using the priority to model regular expressions features such as capturing groups and (2) streaming transducers as a generic and expressive model for string manipulation. For instance, it can capture, among others, the reverse operation. 

The new 

contribution: 

\begin{itemize}
	\item extending Streaming String Transducers: \cite{FR17}
	
	\item  universal solution 
	
	\item 
\end{itemize} 