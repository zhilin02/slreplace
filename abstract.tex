\begin{abstract}

The theory of strings with concatenation has been widely argued as the basis of
constraint solving for verifying string-manipulating programs. However, 
this theory is far from adequate for expressing many string constraints that
are needed for analysing string-manipulating programs, which also require
the use of regular constraints (pattern matching against a regular expression), 
and the string-replace function (replacing either the first occurrence or all
occurrences of a string constant/variable possibly satisfying some pattern by 
another string constant/variable), among many others. Both regular constraints
and the string-replace function are crucial for such applications as analysis
of JavaScript (or more generally HTML5 applications) against cross-site
scripting (XSS) vulnerabilities, which motivates us to consider a richer
class of string constraints.
Unfortunately, 
although adding regular constraints to the theory of concatenation preserves 
decidability, it was recently shown that adding even a restricted version
of the string-replace function (where the strings to replace and to be replaced
by are both constant strings) results in an undecidable theory of strings. 
In this paper, we offer a different
viewpoint of string constraints for verifying string-manipulating programs
using the string-replace function and regular constraints as the basis. 

Our goal in this paper is to investigate extensively the decidability and complexity of the satisfiability problem of string constraints with the function $\replaceall$. We show that while it is undecidable in general, the satisfiability problem for the straight-line fragment is in EXPSPACE, by following an automata-theoretical approach.
\end{abstract}
