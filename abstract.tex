\begin{abstract}

The theory of strings with concatenation has been widely argued as the basis of
constraint solving for verifying string-manipulating programs. However, 
this theory is far from adequate for expressing many string constraints that
are needed for analysing string-manipulating programs, which also require
the use of regular constraints (pattern matching against a regular expression), 
and the string-replace function (replacing either the first occurrence or all
occurrences of a ``pattern'' string constant/variable possibly satisfying some 
regular expression by 
a ``replacement'' string constant/variable), among many others. Both regular constraints
and the string-replace function are crucial for such applications as analysis
of JavaScript (or more generally HTML5 applications) against cross-site
scripting (XSS) vulnerabilities, which motivates us to consider a richer
class of string constraints. The importance of the string replace function 
(especially the replace-all facility) is increasingly recognised, which can
be witnessed by the incorporation of the function in input language of
several string solvers. Unfortunately, 
although adding regular constraints to the theory of concatenation preserves 
decidability, it was recently shown that adding even a restricted version
of the string-replace function (where the pattern/replacement strings 
are both constant strings) results in an undecidable theory of strings. 

In this paper, we revisit 
    %offer a different
%viewpoint of string constraints for 
    the theoretical foundation of string constraints for verifying 
    string-manipulating programs and 
    offer a different viewpoint:  
    the string-replace function and regular constraints (i.e. 
    not concatenation) should be the basic operations. 
    In fact, since
the most general version of string-replace function (where a replacement string
variable is) 
is sufficiently powerful to express the concatenation operator, 
solving such constraints is undecidable in general. 
We then impose a straight-line restriction on the formulas (a shape of
formulas typically generated by symbolic execution), and delineate
the decidability boundary of the resulting fragments. Our main result is 
an algorithm for deciding satisfiability for the straight-line fragment
of formulas, wherein each pattern parameter is either a constant or a 
    a ``source'' variable (i.e. not defined in terms of other variables using
    the replace function). On one hand, our decidability result strictly 
    subsumes a recently proposed
%A large 
decidable subclass of the theory of strings with concatenation, replace 
(where both pattern/replacement strings must be constant strings), and 
regular constraints, which could express the program logic of scripts with 
subtle DOM-based XSS vulnerabilities. On the other hand, our decidability
allows a natural usage of the replace function to be modelled (e.g., 
replacing each occurrence of the string \texttt{'username'} by a username
variable).
%particular,
    %by imposing the straight-line
%restriction on the formulas. This class of formulas 
%with equality of variables permitted solving such constraints is undecidable 
%in general. 
    %Concatenation can easily be
    %simulated by the most general version of string 
    %We observe that concatenation can
%be easily simulated by 
%The string-replace function in its most generality.  

    \OMIT{
Our goal in this paper is to investigate extensively the decidability and complexity of the satisfiability problem of string constraints with the function $\replaceall$. We show that while it is undecidable in general, the satisfiability problem for the straight-line fragment is in EXPSPACE, by following an automata-theoretical approach.
}
\end{abstract}
