%!TEX root = main.tex

\section{Symbolic extensions}
\label{sec:symbolic}


In this section, we consider an extension of parametric transducers to infinite alphabets, called \emph{symbolic parametric transducers}, where the input letters are replaced by formulae over an infinite data domain and the output letters are replaced by terms. 
%
Symbolic parametric transducers can be seen as an extension of both parametric transducers introduced in this paper and symbolic transducers introduced in \cite{VHLMB12}.

%Let $\data$ be an infinite data domain.  We define data strings as elements of $\data^\ast$.

\paragraph{Many-sorted first-order logic.}
We assume a signature $\signature=(\sorts, \functions, \predicates)$, where $\sorts$ is a countable set of \emph{sorts}, $\functions$ is a countable set of \emph{function symbols}, and $\predicates$ is a countable set of \emph{predicate symbols}. Each function and predicate symbol has an associated \emph{arity}, which is a tuple of sorts in $\sorts$.  A function symbol with a single sort is called a \emph{constant}. A predicate symbol with a single sort is called a \emph{set}, which intuitively denotes a set of elements of that sort.

An $\signature$-term is built as usual from the function symbols in $\functions$ and variables taken from a set $\mathcal{X}$ that is disjoint from $\sorts$, $\functions$, and $\predicates$. Each variable $x \in \mathcal{X}$ has an associated sort in $\sorts$. In addition, we assume that the variables in $\mathcal{X}$ are linearly ordered $\preceq_{\mathcal{X}}$. When writing $t(\vec{x})$ for a vector of distinct variables $\vec{x}$ such that $\vec{x} = (x_1,\dots, x_n)$ follows the ascending order of the linear order $\preceq_{\mathcal{X}}$, we assume that the variables occurring in the term $t$ are from $\vec{x}$. For a term $t(\vec{x})$ of sort $s$ such that $\vec{x} = (x_1, \dots, x_n)$ and each $x_i$ for $i \in [n]$ is of sort $s_i \in \sorts$, the term $t$ is said to be \emph{of arity} $(s_1 \times \dots \times s_n) \rightarrow s$. In addition, for a vector of terms $(t_1, \dots, t_m)$ such that all the variables of $t_1 ,\dots, t_m$ are from $\vec{x} = (x_1, \dots, x_n)$, if $x_1 \preceq_{\mathcal{X}} x_2  \preceq_{\mathcal{X}} \dots  \preceq_{\mathcal{X}} x_n$, each $x_i$ for $i \in [n]$ is of sort $s_i$, and each $t_j$ for $j \in [m]$ is of sort $s'_j$, then $(t_1,\dots, t_m)$ is said to be a term of arity $(s_1,\dots, s_n) \rightarrow (s'_1,\dots, s'_m)$. For readability, a term of arity $(s_1,\dots, s_n) \rightarrow (s'_1,\dots, s'_m)$ is also called a $(s_1,\dots, s_n) \big/ (s'_1,\dots, s'_m)$-term. We use $(t_1,\dots, t_m)(\vec{x})$ to denote a vector of terms whose variables are from $\vec{x}$.  For convenience, we also write $t(\vec{x})$ as $\lambda \vec{x}.\ t$ and $(t_1,\dots, t_m)(\vec{x})$ as $\lambda \vec{x}.\ (t_1,\dots, t_m)$. 

We assume the standard notions of $\signature$-atoms, $\signature$-literals, and $\signature$-formulae, whose definitions can be found in some textbooks on mathematical logic (see e.g. \cite{Gal85}). The set of free variables of a $\signature$-formula $\psi$ is denoted by $\free(\psi)$. When writing $\psi(\vec{x})$, we assume that the free variables of $\psi$ are from $\vec{x}$. For a formula  $\psi(\vec{x})$ such that $\vec{x} = (x_1, \dots, x_n)$ and each $x_i$ for $i \in [n]$ is of sort $s_i \in \sorts$, the formula $\psi$ is said to be \emph{of arity} $s_1 \times \dots \times s_n$.
A formula $\psi$ that contains exactly one free variable (resp. two, $n \ge 3$ free variables) is called a \emph{unary} (resp. \emph{binary}, $n$-ary) $\signature$-formula. A formula $\psi$ contains no free variables is called a $0$-ary formula, aka a sentence. For $i, j \in \Nat \backslash \{0\}$, a formula $\psi(\vec{x})$ of arity $s^j$ (where $\vec{x}=(x_1, \dots, x_j)$), and an $s^i/s^j$-term $\vec{f}=(f_1,\dots, f_j)$, we use $\psi[\vec{f}/\vec{x}]$ to denote the formula obtained from $\psi$ by simultaneously replacing $x_1$ with $f_1$, $\dots$, and $x_j$ with $f_j$.

An $\signature$-interpretation $I$ maps: (i) each sort $s \in \sorts$  to a set $s^{I}$, (ii) each function symbol $f \in \functions$ of arity $s_1 \times \cdots \times s_n \rightarrow s$ to a total function $f^I: s_1^I \times \cdots \times s^I_n \rightarrow s^I$ if $n>0$, and to an element of $s^I$ if $n = 0$, and (iii) each predicate symbol $p \in \predicates$ of sort $s_1 \times \cdots \times s_n$ to a  subset of $p^I \subseteq s^I_1 \times \cdots s^I_n$.
An $\signature$-assignment $\eta$ maps each variable $x \in \mathcal{X}$ of sort $s \in \sorts$ to an element of $s^I$.
\begin{itemize}
\item For a term $t$, the interpretation of $t$ under $(I, \eta)$ for an $\signature$-interpretation $I$ and $\signature$-assignment $\eta$, denoted by $t^{(I,\eta)}$, can be defined inductively on the syntax of terms.
\item The satisfiability relation between pairs of an $\signature$-interpretation and an $\signature$-assignment, and $\signature$-formulae, written $I \models_{\eta} \psi$,
is defined inductively, as usual.
\end{itemize}
We say that $(I,\eta)$ is a model of $\psi$ if $I \models_{\eta} \psi$. For an $\signature$-sentence $\psi$, we also write $I \models \psi$ if there is an $\signature$-assignment $\eta$ such that $I \models_\eta \psi$.

Let $\signature$ be a signature and $\cI$ be a set of $\signature$-interpretations. Then $\theory(\cI)$, \emph{the $\signature$-theory associated with $\cI$}, is the set of  $\signature$-sentences $\psi$ such that for each $I \in \cI$, $I \models \psi$.


For the definition of symbolic transducers, we introduce the concept of background theories and label theories. Intuitively, background theories are many-sorted Boolean algebra satisfying the additional constraint that the set of formulae is closed under substitutions. Label theories extend background theories further by adding inequalities of terms into the set of formulae.

\begin{definition}[Background theories]
A background theory  $\Upsilon$ is a tuple $(\signature, \| \circ \|, \Psi)$ satisfying the following constraints:
\begin{itemize}
\item $\signature=(\sorts, \functions, \predicates)$ is a signature satisfying that
%\begin{itemize}
%\item
each of $\sorts, \functions, \predicates$ is a recursively enumerable set.
%
%\item for each $\vec{s} = s_1 \times \dots \times s_n$ and $i \in [n]$, there is a function $\pi_{\vec{s}, i}$ of arity $s_1 \times \dots \times s_n \rightarrow s_i$ in $\functions$ such that $\pi_{\vec{s},i}(\vec{t}) = t_i$ for each term $\vec{t} = (t_1,\dots, t_n)$ of sort $s_1 \times \dots \times s_n$.
%\end{itemize}

\item $\|\circ\|$ is an $\signature$-interpretation such that for each $s \in \sorts$, $\| s \|$ is a recursively enumerable set (denoted by $\data_s$).
%
\item $\Psi = \bigcup \limits_{\vec{s} \in \sorts^+} \Psi^{(\vec{s})}$ such that for each $\vec{s} = (s_1, \dots, s_i) \in \sorts^+$, $\Psi^{(\vec{s})}$ is a recursively enumerable set of $\signature$-formulae of arity $s_1 \times \dots \times s_i$ closed under Boolean connectives $\vee, \wedge, \neg$. In addition, $\Psi$ is closed under substitutions, that is, for each $\vec{s}/\vec{s'}$-term $\vec{f}$ and $\psi(\vec{x}) \in \Psi^{(\vec{s'})}$, we have $\psi[\vec{f}/\vec{x}] \in \Psi^{(\vec{s})}$.
%In addition, it is assumed that each formula $\psi \in \Psi$ of arity $s^\ell$ (where $\ell > 0$) has the same set of free variables $\{x_1,\dots, x_\ell\}$.
For each $\psi(\vec{x}) \in \Psi$, we use $\|\psi\|$ to denote the set $\{\eta(\vec{x}) \mid \eta \mbox{ is an } \signature \mbox{-assignment}, \mbox{ and } \| \circ \| \models_\eta \psi(\vec{x})\}$. Elements of $\| \psi\|$ are called the \emph{witnesses} of $\psi$.
\end{itemize}
%For $(s_1,\dots, s_i) \in \sorts^i$ and $(s'_1, \dots, s'_j) \in \sorts^j$, a $(s_1,\dots, s_i)\big/(s'_1,\dots, s'_j)$-term is a term of arity $(s_1,\dots, s_i) \rightarrow (s'_1, \dots, s'_j)$.
\end{definition}
Let $\Upsilon=(\signature, \| \circ \|, \Psi)$ be a background theory and $\psi \in \Psi$.
Then $\psi$ is \emph{satisfiable}, denoted by $\issat(\psi)$, if $\|\psi\| \neq \emptyset$. In addition, $\Upsilon$ is \emph{decidable} iff it is decidable to check $\issat(\psi)$ for  $\psi \in\Psi$.
%The notion $\issat(\psi)$ for $\psi \in \Psi$ and the decidability of $\Upsilon$ can be defined similarly as effective Boolean algebra.

For a background theory $(\signature, \| \circ \|, \Psi)$ with $\signature=(\sorts, \functions, \predicates)$ and a sort $s \in \sorts$, we derive a new sort $s^*$ such that $\|s^* \| = \bigcup \limits_{i \in \Nat}\data_s^i$, where $\data^i_s = \{\varepsilon\}$. An element of $\|s^* \|$ is called a \emph{data string} of sort $s$.

\begin{definition}[Label theories]
A label theory  $\Upsilon$ with the input and output sort $s$ is a tuple $(\signature, \| \circ \|, \Psi, \Psi')$ satisfying the following constraints:
\begin{itemize}
\item $(\signature, \| \circ \|, \Psi)$ is a background theory such that $\signature = (\sorts, \functions, \predicates)$ and $s  \in \sorts$,
%
\item $\Psi' = \bigcup \limits_{i \in \Nat \backslash \{0\}} (\Psi')^{(s^i)}$ such that for each $i \in \Nat \backslash \{0\}$, $ (\Psi')^{(s^i)}$ comprises the formulae of the form $\psi(\vec{x}) \wedge f(\vec{x}) \neq g(\vec{x})$, where $\psi(\vec{x}) \in \Psi^{(s^i)}$ and $f, g $ are $s^i/s$-terms.
\end{itemize}
A label theory is decidable if it is decidable to check $\issat(\psi)$ for $\psi \in \Psi \cup \Psi'$.
\end{definition}

Given a formula $\psi(\vec{x}) \in (\Psi)^{(s^i)}$ and two $s^i /s$-terms $f(\vec{x}), g(\vec{x})$,
$f$ and $g$ are \emph{equivalent up to $\psi$}, denoted by $f \simeq_\psi g$, if
 $\issat(\psi(\vec{x}) \wedge f(\vec{x}) \neq g(\vec{x}))$ does not hold.
Two sequences of $s^i/ s$-terms $\vec{f}=f_1...f_n$ and $\vec{g}=g_1...g_m$ are \emph{equivalent up to $\psi$}, denoted by $\vec{f}\simeq_\psi \vec{g}$,
iff $n=m$ and for every $j \in [n]$, $f_j \simeq_\psi g_j$.

Given a $s^i/ s$-term sequence $\vec{f}=f_1...f_n$ and a sequence of data values $\vec{d} = (d_1, \dots, d_i) \in (\data_{s_{\sf in}})^i$,
let $\|\vec{f}\|(\vec{d})$ denote the sequence $\|f_1\|(\vec{d})...\|f_n\|(\vec{d})$, that is, a data string of sort $s$.

\zhilin{2NST has not been considered before and is new model}
\begin{definition}[Symbolic finite-state transducers]
    A \emph{nondeterministic two-way  symbolic \emph{transducer}} (2NST) is a tuple $\Transducer = (\Upsilon, s, \EndLeft, \EndRight, \controls, q_0, \finals, \transrel)$ where  
\begin{itemize}
\item $\Upsilon=(\signature, \| \circ \|, \Psi, \Psi')$ is a decidable label theory with the input and output sort $s$,
%
\item $\EndLeft$, $\EndRight$, $\controls$, $q_0$, $\finals$ are defined precisely as in 2NFT, 
%
\item $\transrel$ is a finite set of  transitions of one of the following forms,
\begin{itemize}
\item symbolic transitions $(q, \psi, dir, q', f)$ such that $q, q' \in Q$, $dir \in \{\Left, \Stay, \Right\}$, $\psi \in \Psi^{(s)}$ and
$f$ is a $s /s$-term, 
\item non-symbolic transitions $(q, \EndLeft, \Right, q', c)$ and $(q, \EndLeft, \Left, q', c)$, where $c$ is a constant of sort $s$. 
\end{itemize}
\end{itemize}
A 2NST is an NST if for each $(q, \psi, dir, q', f)$, we have $dir \in \{\Stay, \Right\}$.
\end{definition}

\paragraph{Semantics of 2NST.}
A symbolic transition $t=(q_1,\psi, dir, q_2, f) \in \transrel$ in the ST $\Transducer$ can be concretised
into a potentially infinite set of \emph{concrete} transitions $\|t\| \subseteq Q \times \data_{s} \times \{\Left, \Stay, \Right\} \times Q \times \data_{s}$, where $(q_1, d, dir, q_2, d')  \in \|t\|$ iff $d \in \|\psi\|$ and $d' = \|f\|(d)$.
Intuitively, suppose $\Transducer$ is at the state $q_1$ and reading the input data value $d \in \data_{s}$,
if there is a transition $(q_1, \psi, dir, q_2, f )\in \transrel$ such that $d \in \|\psi\|$, then $\Transducer$ can move its reading head according to $dir$, and moves from the state
$q_1$ to the state $q_2$, moreover it produces a data value $d' \in \data_{s}$.

Given a data string $w = d_1 \dots d_n$, a \emph{run} of $\Transducer$ on $w$
is a sequence of tuples $(q_0, i_0, d'_0), \ldots, (q_m, i_m, d'_m) \in \controls \times [n+1] \times \data_s$ 
such that, let $d_0 = \EndLeft$ and $d_{n+1} = \EndRight$, %. The following conditions, then, have to be satisfied:
\begin{itemize}
    \item $i_0 = 0$, and
    \item for every $j \in [m-1]$, $t=(q_j, \psi, dir, q_{j+1}, f) \in \transrel$ for some $\psi \in \Psi^{(s)}$, $dir \in \{\Left, \Stay, \Right\}$, and $s/s$ term $f$, such that $(q_j, d_{i_j}, dir, q_{j+1}, d'_j) \in \|t\|$, and $i_{j+1} = i_j + dir$.
\end{itemize}
The run is said to be \defn{accepting} if $i_m = n+1$ and $q_m \in \finals$. When a run is accepting, $d'_0 \cdots d'_m$ is said to be the \emph{output} of the run.
A data string $w'$ is said to be an output of $\Transducer$ on $w$ if there is an accepting run of
$\Transducer$ on $w$ with output $w'$. We use $\Tran(\Transducer)$ to denote the \emph{transduction} defined by $\Transducer$, that is, the relation comprising the pairs $(w, w')$ such that $w'$ is an output of $\Transducer$ on $w$.

\begin{proposition}
Each 2NST can be turned into an equivalent NST. 
\end{proposition}


\begin{definition}[Symbolic parametric transducers]
A \emph{nondeterministic two-way symbolic parametric transducer} (2NSPT) is a tuple
$\Transducer=(\Upsilon, s,  \EndLeft, \EndRight, X, \controls, q_0, \finals, \transrel)$, where:
\begin{itemize}
\item $\Upsilon=(\signature, \| \circ \|, \Psi, \Psi')$ is a decidable label theory with the input and output sort $s$,
%
\item $\EndLeft$, $\EndRight$, $\controls$, $q_0$, $\finals$ are defined precisely as in 2NST, 
%
\item $X=\{x_1,\cdots, x_m\}$ is a finite set of parameters (variables) of sort $s^*$, 
%
\item $\transrel$ is a finite set of transitions of one of the following forms: 
\begin{itemize}
\item symbolic transitions $(q, \psi, q', f)$ or $(q, \psi, q', x)$ such that $q,q' \in Q$, $\psi \in \Psi^{(s)}$,
$f$ is a $s/s$-term, and $x \in X$,
%
\item non-symbolic transitions $(q, \EndLeft, \Right, q', c)$ and $(q, \EndLeft, \Left, q', c)$, where $c$ is a constant of sort $s$. 
\end{itemize}
\end{itemize}
\end{definition}

Notice that parameters are only allowed in the output track.
Intuitively, each instantiation of the parameters $x_1,\ldots, x_m$ with data strings 
$w_1,\ldots, w_m$ gives rise to a nondeterministic two-way symbolic transducer which outputs
the data string $w_j$, whenever a transition of the form $(p, \psi, dir, q, x_j)$ is
taken by the transducer. This instantiation of the parameters is only done 
\emph{once} before the symbolic parametric transducer is run.

\zhilin{extend the generic decision procedure to 2NSPT.} 
