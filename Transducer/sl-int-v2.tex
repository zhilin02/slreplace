 \documentclass{llncs}

%\documentclass[envcountsame, fleqn]{llncs}
  \usepackage{amsmath, latexsym}
  \usepackage{graphicx}
\usepackage{epic,eepic}

\pagestyle{plain}
\usepackage{listings}
\usepackage{psfrag}
\usepackage{rotating}

\usepackage{url}
\usepackage{amssymb,epsfig,amstext}
\usepackage{txfonts}
\usepackage{algorithmic}
\usepackage{algorithm}
\usepackage{graphicx}

\usepackage{todonotes}
\usepackage{multirow}
\usepackage{float,color}
\usepackage{picinpar,color,xcolor,wrapfig}
\setcounter{secnumdepth}{4}
\sloppy

\pagestyle{plain}

%%%%%%%%%%%%%%%   MACROS  %%%%%%%%%%%%%%%%%%


%!TEX root = main.tex

\newcommand{\set}[1]{\{ #1 \}}
\newcommand{\sequence}[2]{(#1, \ldots, #2)}
\newcommand{\couple}[2]{(#1,#2)}
\newcommand{\pair}[2]{(#1,#2)}
\newcommand{\triple}[3]{(#1,#2,#3)}
\newcommand{\quadruple}[4]{(#1,#2,#3,#4)}
\newcommand{\tuple}[2]{(#1,\ldots,#2)}
\newcommand{\Nat}{\ensuremath{\mathbb{N}}}
\newcommand{\Rat}{\ensuremath{\mathbb{Q}}}
\newcommand{\Rea}{\ensuremath{\mathbb{R}}}
\newcommand{\Int}{\ensuremath{\mathbb{Z}}}
%\newcommand{\true}{\top}
%\newcommand{\false}{\perp}
\newcommand{\bottom}{\perp}
%% \newcommand{\powerset}[1]{{\cal P}(#1)}
\newcommand{\npowerset}[2]{{\cal P}^{#1}(#2)}
\newcommand{\finitepowerset}[1]{{\cal P}_f(#1)}
\newcommand{\level}[2]{L_{#1}(#2)}
\newcommand{\card}[1]{\mbox{card}(#1)}
\newcommand{\range}[1]{\mathtt{ran}(#1)}
\newcommand{\astring}{s}

\newcommand{\Cc}{\mathcal{C}}

\newcommand{\intnum}{\mathbb{Z}}


\newcommand {\notof}{\ensuremath{\neg}}
\newcommand {\myand}{\ensuremath{\wedge}}
\newcommand {\myor}{\ensuremath{\vee}}
\newcommand {\mynext}{\mbox{{\sf X}}}
\newcommand {\until}{\mbox{{\sf U}}}
\newcommand {\sometimes}{\mbox{{\sf F}}}
\newcommand {\previous}{\mynext^{-1}}
\newcommand {\since}{\mbox{{\sf S}}}
\newcommand {\fminusone}{\mbox{{\sf F}}^{-1}}
\newcommand {\everywhere}[1]{\mbox{{\sf Everywhere}}(#1)}



\newcommand{\aatomic}{{\rm A}}
\newcommand{\aset}{X}
\newcommand{\asetbis}{Y}
\newcommand{\asetter}{Z}

\newcommand{\avarprop}{p}
\newcommand{\avarpropbis}{q}
\newcommand{\avarpropter}{r}
\newcommand{\varprop}{{\rm PROP}} % Set of atomic propositions (for a given logic)

% formulae

\newcommand{\aformula}{\astateformula} % a formula
\newcommand{\aformulabis}{\astateformulabis} % another formula (when at least 2 are present)
\newcommand{\aformulater}{\astateformulater} % another formula (when at least 3 are present)
\newcommand{\asetformulae}{X}
\newcommand{\subf}[1]{sub(#1)}

\newcommand{\aautomaton}{{\mathbb A}}
\newcommand{\aautomatonbis}{{\mathbb B}}

%\newcommand {\length}[1] {\ensuremath{|#1|}}



% Equivalences
\newcommand{\egdef}{\stackrel{\mbox{\begin{tiny}def\end{tiny}}}{=}} % =def=
\newcommand{\eqdef}{\stackrel{\mbox{\begin{tiny}def\end{tiny}}}{=}} % =def=
\newcommand{\equivdef}{\stackrel{\mbox{\begin{tiny}def\end{tiny}}}{\equivaut}} % <=def=>
\newcommand{\equivaut}{\;\Leftrightarrow\;}

\newcommand{\ainfword}{\sigma}

\newcommand{\amap}{\mathfrak{f}}
\newcommand{\amapbis}{\mathfrak{g}}

\newcommand{\step}[1]{\xrightarrow{\!\!#1\!\!}}
\newcommand{\backstep}[1]{\xleftarrow{\!\!#1\!\!}}

\newcommand {\aedge}[1] {\ensuremath{\stackrel{#1}{\longrightarrow}}}
\newcommand {\aedgeprime}[1] {\ensuremath{\stackrel{#1}{\longrightarrow'}}}
\newcommand {\afrac}[1] {\ensuremath{\mathit{frac}(#1)}}
\newcommand {\cl}[1] {\ensuremath{\mathit{cl}(#1)}}
\newcommand {\sfc}[1] {\ensuremath{\mathit{sfc}(#1)}}
\newcommand {\dunion} {\ensuremath{\uplus}}
\newcommand {\edge} {\ensuremath{\longrightarrow}}
\newcommand {\emptyword}{\ensuremath{\epsilon}}
\newcommand {\floor}[1] {\ensuremath{\lfloor #1 \rfloor}}
\newcommand {\intersection} {\ensuremath{\cap}}
\newcommand {\union} {\ensuremath{\cup}}
\newcommand {\vals}[2] {\ensuremath{\mathit{val}_{#2}(#1)}}



\newcommand {\pspace} {\textsc{pspace}}
\newcommand {\nlogspace} {\textsc{nlogspace}}
\newcommand {\logspace} {\textsc{logspace}}
\newcommand {\expspace} {\textsc{expspace}}
\newcommand {\nexpspace} {\textsc{nexpspace}}
\newcommand {\exptime} {\textsc{exptime}}
\newcommand {\np} {\textsc{np}}
\newcommand {\threeexptime} {\textsc{3exptime}}
\newcommand {\polytime} {\textsc{p}}
\newcommand{\twoexpspace}{\textsc{2expspace}}
\newcommand{\threeexpspace}{\textsc{3expspace}}
\newcommand {\nexptime} {\textsc{nexptime}}

\newcommand {\nonelementary} {\textsc{non-elementary}}

\newcommand {\elementary} {\textsc{elementary}}


\newcommand{\aalphabet}{\Sigma}     % an alphabet, A is already used for atoms
\newcommand{\aword}{\mathfrak{u}}
\newcommand{\awordbis}{\mathfrak{v}}



\newcommand{\aassertion}{P}
\newcommand{\aassertionbis}{Q}
\newcommand{\aexpression}{e}
\newcommand{\aexpressionbis}{f}
\newcommand{\avariable}{\mathtt{x}}
\newcommand{\uniquevar}{\mathtt{u}}
\newcommand{\uniquevarbis}{\mathtt{v}}
\newcommand{\avariablebis}{\mathtt{y}}
\newcommand{\avariableter}{\mathtt{z}}
\newcommand{\nullconstant}{\mathtt{null}}
\newcommand{\nilvalue}{nil}
\newcommand{\emptyconstant}{\mathtt{emp}}
\newcommand{\infheap}{\mathtt{inf}}
\newcommand{\saturated}{\mathtt{Saturated}}

\newcommand{\astateformula}{\phi}
\newcommand{\astateformulabis}{\psi}
\newcommand{\astateformulater}{\varphi}
%%
\newcommand{\separate}{\ast}
\newcommand{\sep}{\separate}
\newcommand{\size}{\mathtt{size}}
\newcommand{\sizeeq}[1]{\mathtt{size} \ = \ #1}
\newcommand{\alloc}[1]{\mathtt{alloc}(#1)}
\newcommand{\allocb}[2]{\mathtt{alloc}^{-1}[#2](#1)}
\newcommand{\isol}[1]{\mathtt{isoloc}(#1)}
\newcommand{\icell}{\mathtt{isocell}}
\newcommand{\malloc}{\mathtt{malloc}}
\newcommand{\cons}{\mathtt{cons}}
\newcommand{\new}{\mathtt{new}}
\newcommand{\free}[1]{\mathtt{free} \ #1}
\newcommand{\maxform}[1]{\mathtt{maxForms}(#1)}
\newcommand{\locations}[1]{\mathtt{loc}(#1)}
\newcommand{\values}{\mathtt{Val}}
\newcommand{\aheap}{\mathfrak{h}}
\newcommand{\avaluation}{\mathfrak{V}}
\newcommand{\heaps}{\mathcal{H}}
\newcommand{\astore}{\mathfrak{s}}
\newcommand{\stores}{\mathcal{S}}
\newcommand{\amodel}{\mathfrak{M}}
\newcommand{\alabel}{\ell}

\newcommand{\aprogram}{\mathtt{PROG}}
\newcommand{\programs}{\mathtt{P}}
\newcommand{\ctprograms}{\programs^{ct}}
\newcommand{\aninstruction}{\mathtt{instr}}
\newcommand{\ainstruction}{\mathtt{instr}}
\newcommand{\instructions}{\mathtt{I}}
\newcommand{\aguard}{\ensuremath{g}}
\newcommand{\guards}{\ensuremath{G}}
\newcommand{\domain}[1]{\mathtt{dom}(#1)}
\newcommand{\memory}{\stores\times\heaps}
\newcommand{\skipinstruction}{\mathtt{skip}}

\newcommand{\execution}{\mathtt{comp}}
\newcommand{\aux}{\mathtt{embd}}
\newcommand{\runof}{run}
\newcommand{\anexecution}{e}


\newcommand{\aletter}{\ensuremath{a}}
\newcommand{\aletterbis}{\ensuremath{b}}
\newcommand{\alocation}{\mathfrak{l}}

\newcommand{\pointsl}[1]{\stackrel{#1}{\hookrightarrow}}
\newcommand{\ppointsl}[1]{\stackrel{#1}{\mapsto}}
\newcommand{\ourhook}[1]{\stackrel{#1}{\hookrightarrow}}
\newcommand{\ltrue}{{\sf true}}
\newcommand{\lfalse}{{\sf false}}


\newcommand{\variables}{\mathtt{FVAR}}
\newcommand{\pvariables}{\mathtt{PVAR}}
\newcommand{\secvariables}{\mathtt{SVAR}}
\newcommand{\logique}[1]{\mathtt{FO}(#1)}



\newcommand{\atranslation}{\mathfrak{t}}
\newcommand{\nbpred}[1]{\widetilde{\sharp #1}}
\newcommand{\nbpredstar}[1]{\widetilde{\sharp #1}^{\star}}
\newcommand{\isolated}{\mathtt{isol}}
\newcommand{\stdmarks}{\mathtt{envir}}
\newcommand{\relation}[1]{\mathtt{relation}_{#1}}
\newcommand{\freevar}{\mathtt{FV}}
\newcommand{\notonmark}{\mathtt{notonenv}}
\newcommand{\InVal}[1]{\mathtt{InVal}\!\left(#1\right)}
\newcommand{\NotOnEnv}[1]{\mathtt{NotOnEnv}\!\left(#1\right)}
\newcommand{\PartOfVal}[1]{\mathtt{PartOfVal}\!\left(#1\right)}
%\newcommand{\nbpreds}[3]{\sharp #1 \geq #2}
\newcommand{\defstyle}[1]{{\emph{#1}}}

\newcommand{\cut}[1]{}
\newcommand{\interval}[2]{[#1,#2]}
\newcommand{\buniquevar}{\overline{\uniquevar}}
\newcommand{\bbuniquevar}{\overline{\overline{\uniquevar}}}
\newcommand{\magicwand}{\mathop{\mbox{$\mbox{$-~$}\!\!\!\!\ast$}}}
\newcommand{\wand}{\magicwand}
\newcommand{\septraction}{\stackrel{\hsize0pt \vbox to0pt{\vss\hbox to0pt{\hss\raisebox{-6pt}{\footnotesize$\lnot$}\hss}\vss}}{\magicwand}}
%% \newcommand{\reach}{\mathtt{reach}}
\mathchardef\mhyphen="2D % hyphen while in math mode

\newcommand{\adataword}{\mathfrak{dw}}
\newcommand{\adatum}{\mathfrak{d}}

\newcommand{\collectionknives}{\mathtt{ks}}
\newcommand{\collectionknivesfork}[1]{\mathtt{ksfs}_{=#1}}
\newcommand{\collectionknivesforks}{\mathtt{ksfs}}
\newcommand{\collectionkniveslargeforks}{\mathtt{kslfs}}


\newcommand{\acounter}{\mathtt{C}}

\newcommand{\fotwo}[3]{{\mbox{FO2}_{#1,#2}(#3)}}
\newcommand{\mtrans}[1]{t\!\left(#1\right)^{\Box}}
\newcommand{\mbtrans}[2]{\mtrans{#2}_{#1}}


\newcommand{\alogic}{\mathfrak{L}}


\newcommand{\semantics}[1]{\ensuremath{[ #1 ]}}


\newcommand{\adomino}{\mathfrak{d}}
\newcommand{\atile}{\mathfrak{d}}
\newcommand{\atiling}{\mathfrak{t}}

\newcommand{\hori}{\mathtt{h}}
\newcommand{\verti}{\mathtt{v}}
\newcommand{\domi}{\mathtt{d}}

\newcommand{\cpyrel}{\mathfrak{cp}}

\newcommand{\cntcmp}{\mathfrak{C}}

\newcommand{\heapdag}{\mathfrak{G}}

\newcommand{\onmainpath}{\mathtt{mp}}

\newcommand{\tree}{\mathtt{tree}}

%\newcommand{\tile}{\mathtt{tile}}

\newcommand{\type}{\mathtt{type}}

\newcommand{\ptype}{\mathtt{ptype}}

\newcommand{\exttype}{\mathtt{exttype}}

\newcommand{\anctypes}{\mathtt{AncTypes}}

\newcommand{\destypes}{\mathtt{DesTypes}}

\newcommand{\inctypes}{\mathtt{IncTypes}}

\newcommand{\treeic}{\mathtt{treeIC}}

\newcommand{\trs}{\mathfrak{trs}}


\newcommand{\nin}{\not \in}
\newcommand{\cupplus}{\uplus}
\newcommand{\aunarypred}{\mathtt{P}}


\newcommand{\hide}[1]{}

\newcommand{\eval}[2]{\llbracket#1\rrbracket_{#2}}
\newcommand\cur{\mathsf{cur}}
\newcommand\dom{\mathsf{dom}}
\newcommand\rng{\mathsf{rng}}

\newcommand\dd{\mathbb{D}}
\newcommand\nat{\mathbb{N}}


\newcommand\cA{\mathcal{A}}
\newcommand\cB{\mathcal{B}}
\newcommand\cC{\mathcal{C}}
\newcommand\cE{\mathcal{E}}
\newcommand\cG{\mathcal{G}}
\newcommand\Ll{\mathcal{L}}
\newcommand\cM{\mathscr{M}}
\newcommand\cP{\mathcal{P}}
\newcommand\cR{\mathcal{R}}
\newcommand\cS{\mathcal{S}}
\newcommand\cT{\mathcal{T}}

\newcommand\vard{\mathfrak{d}}

\newcommand\replace{\mathsf{replace}}
\newcommand\replaceall{\mathsf{replaceAll}}
\newcommand\sreplaceall{\mathsf{sreplaceAll}}
\newcommand\reverse{\mathsf{reverse}}
\newcommand\indexof{\mathsf{indexOf}}
\newcommand\length{\mathsf{length}}
\newcommand\substring{\mathsf{substring}}
\newcommand\charat{\mathsf{charAt}}
\newcommand\extract{\mathsf{extract}}

\newcommand\revsym{\pi}

\newcommand\strline{\mathsf{SL}}

\newcommand\pstrline{\mathsf{SL_{pure}}}

\newcommand\search{\mathsf{search}}

\newcommand\verify{\mathsf{vfy}}

\newcommand\searchleft{\mathsf{left}}

\newcommand\searchlong{\mathsf{long}}


\newcommand\pref{\mathsf{Pref}}

\newcommand\wprof{\mathsf{WP}}

\newcommand\vars{\mathsf{Vars}}

\newcommand\dep{\mathsf{Dep}}
\newcommand\ptn{\mathsf{Ptn}}

\newcommand\src{\mathsf{src}}
\newcommand\strtorep{\mathsf{strToRep}}

\newcommand\rpleft{\mathsf{l}}
\newcommand\rpright{\mathsf{r}}


\newcommand\srcnd{\mathsf{srcND}}

\newcommand\ctxt{\mathsf{ctxt}}


\newcommand\ctxts{\mathsf{Ctxts}}

\newcommand\sprt{\mathsf{sprt}}

\newcommand\val{\mathsf{val}}

\newcommand\srclen{\mathsf{srcLen}}

\newcommand\rpleftlen{\mathsf{lLen}}


\newcommand\dfs{\mathsf{DFS}}

\newcommand\repr{\mathsf{rep}}

\newcommand\red{\mathsf{red}}

\newcommand\gfun{\mathcal{F}}


\newcommand{\leftmost}{{\sf leftmost}}
\newcommand{\longest}{{\sf longest}}

\newcommand{\arbidx}{{\sf Idx_{arb}}}
\newcommand{\dmdidx}{{\sf Idx_{dmd}}}
\newcommand{\lftlen}{{\sf Len_{lft}}}


\newcommand{\ASSERT}[1]{\textsf{assert}\left(#1\right)}

\newcommand{\concat}{\cdot}

\newcommand{\parabs}{\Theta} % parametr abstraction
\newcommand{\arity}{r}

\newcommand{\FA}{FA}
\newcommand{\FFA}{2FA}
\newcommand{\SFFA}{S2FA}
\newcommand{\FT}{FT}
\newcommand{\FFT}{2FT}
\newcommand{\FunFT}{FFT}
\newcommand{\SFFT}{S2FT}
\newcommand{\PT}{PT}
\newcommand{\PPT}{2PT}
\newcommand{\SPPT}{S2PT}
\newcommand{\RBPPT}{RB2PT}
\newcommand{\RBSPPT}{RBS2PT}
\newcommand{\SA}{SA}
\newcommand{\SSA}{2SA}
\newcommand{\ST}{ST}
\newcommand{\SST}{2ST}
\newcommand{\SPT}{SPT}
\newcommand{\SSPT}{2SPT}
\newcommand{\RBSSPT}{RB2SPT}

\newcommand{\ialphabet}{\Sigma}
\newcommand{\oalphabet}{\Gamma}


\newcommand{\EndLeft}{\ensuremath{\vartriangleright}}
\newcommand{\EndRight}{\ensuremath{\vartriangleleft}}

\newcommand{\Lang}{\mathscr{L}}
\newcommand{\Tran}{\mathscr{T}}

\newcommand{\NFA}{\mathcal{A}}
\newcommand{\NFAB}{\mathcal{B}}
\newcommand{\NFT}{\mathcal{T}}

\newcommand{\CEFA}{\mathcal{A}}

\newcommand{\Transducer}{\ensuremath{T}}
\newcommand{\controls}{\ensuremath{Q}}
\newcommand{\finals}{\ensuremath{F}}
\newcommand{\transrel}{\ensuremath{\delta}}


\newcommand{\Left}{\ensuremath{-1}}
\newcommand{\Right}{\ensuremath{1}}
\newcommand{\Stay}{\ensuremath{0}}


\newcommand{\defn}[1]{\emph{#1}}

\newcommand{\conacc}{\Omega}

\newcommand{\reginvrel}{\textbf{RegInvRel}}

\newcommand{\prerec}{\reginvrel}

\newcommand{\regmondec}{\textbf{RegMonDec}}

\newcommand\rcdim{\mathsf{art}}

\newcommand\rcdep{\mathsf{asgn}}

\newcommand\rcasrt{\mathsf{asrt}}

\newcommand\rcreg{\mathsf{reg}}

\newcommand\rcsreg{\mathsf{sreg}}


\newcommand\tower{\mathrm{Tower}}

\newcommand\rcphi{\mathsf{fnsize}}
\newcommand\rcpsi{\mathsf{fasize}}

%%%%%%%%%%%%
% Auto escape eg

\newcommand\linkvar{\mintinline{html}{$li}}
\newcommand\linktextvar{\mintinline{html}{$txt}}
\newcommand\urlstarttag{\texttt{[URL]}}
\newcommand\urlendtag{\texttt{[LRU]}}
\newcommand\htmlstarttag{\texttt{[HTML]}}
\newcommand\htmlendtag{\texttt{[LMTH]}}

\newcommand{\Pre}{\textsf{Pre}}

\newcommand\bigO{\mathcal{O}}

\newcommand\Aut{\mathcal{A}}

\newcommand{\tup}[1]{\left( #1 \right)}

\newcommand\ap[2]{{#1}\mathord{\brac{#2}}}
%\newcommand\ap[2]{{#1}\mathord{\brac{#2}}}

\newcommand{\opset}{\mathscr{O}}

\newcommand{\strlineall}{$\strline$[\FT, $\replaceall$, $\reverse$]}

\newcommand{\strlineconcat}{$\strline$[$\concat$, $\sreplaceall$, $\reverse$, \FT]}

\newcommand{\strlinefft}{$\strline$[$\concat$, $\replaceall$, $\reverse$, \FunFT]}


%%% Macros for expspace lower bound with replaceall

% General
\newcommand\brac[1]{\left(#1\right)}

% \setcomp{ele}{comp} = { ele | comp }
\newcommand\setcomp[2]{\left\{{#1}\ \middle|\ {#2}\right\}}

% \lang{A} = L(A)
\newcommand\lang[1]{\mathcal{L}\mathord{\brac{#1}}}

%% Optimisations

\newcommand\caleybox[1]{\llparenthesis #1 \rrparenthesis}
\newcommand\internalchar{\flat}

% Tools

\newcommand\stranger{Stranger}

\newcommand\transducerbench{\textsc{Transducer}}
\newcommand\slogbench{\textsc{SLOG+}}
\newcommand\slogbenchr{\textsc{SLOG+(replace)}}
\newcommand\slogbenchra{\textsc{SLOG+(replaceall)}}
\newcommand\kaluzabench{\textsc{Kaluza}}
\newcommand\pyexbench{\textsc{PyEx}}
\newcommand\pyextdbench{\textsc{PyEx-}td}
\newcommand\pyexztbench{\textsc{PyEx-}z3}
\newcommand\pyexzzbench{\textsc{PyEx-}zz}

\newtheorem{fact}{Fact}
\newtheorem{remark}{Remark}


\newcommand\slint{${\rm SL}_{\rm int}$}

\newcommand\cslint{SL$^{\dag}_{int}$}

\newcommand{\regexp} {{\sf Regex}}
%\newcommand{\cgexp} {{\sf RWRE_{reg}}}
\newcommand{\pcre} {{\sf PCRE}}

\newcommand{\lasat}{${\rm SAT}_{\rm CEFA}[{\rm LIA}]$}

\newcommand{\prjnum}{{\rm Prj}_{\rm num}}

\newcommand{\uwp}{{\rm uwp}}

\newcommand\urlxsssanitise{{\sf urlXssSanitise}}

%%%%%%%%%% Start TeXmacs macros
\newcommand{\colons}{\,:\,}
\newcommand{\tmop}[1]{\ensuremath{\operatorname{#1}}}
\newcommand{\tmtextit}[1]{{\itshape{#1}}}
\newcommand{\tmtextbf}[1]{{\bfseries{#1}}}
%\newtheorem{definition}{Definition}
%{\theorembodyfont{\rmfamily}\newtheorem{note}{Note}}
%{\theorembodyfont{\rmfamily}\newtheorem{remark}{Remark}}
%\newtheorem{theorem}{Theorem}
%\newtheorem{note}{Note}
%\newtheorem{remark}{Remark}
\newcommand\NSST{{\sf NSST}}
\newcommand\PSST{{\sf PSST}}
\newcommand\refexp{{\sf REP}}
\newcommand\ssym{{\sf Start}}
\newcommand\esym{{\sf End}}

\newcommand\pnfa{\mathcal{A}}
\newcommand\psst{\mathcal{T}}

\newcommand\pat{\mathsf{pat}}
\newcommand\rep{\mathsf{rep}}

\newcommand\refbefore{\$^{\leftarrow}}
\newcommand\refafter{\$^{\rightarrow}}

\newcommand\nullchar{\mathsf{null}}

\newcommand\idxexp{{\sf idx}}


\newcommand\cvc{CVC4}
\newcommand\zthree{Z3-str3}
\newcommand\trau{Trau}
\newcommand\trauplus{Trau+}
\newcommand\zthreetrau{Z3-Trau}
\newcommand\ostrich{EMU}
\newcommand\expose{ExpoSE}
\newcommand\sloth{Sloth}
\newcommand\slent{Slent}
\newcommand\Tool{\text{OSTRICH+}}
\newcommand{\OMIT}[1]{}

\newcommand{\seq}[1]{\ensuremath{#1}}
\newcommand{\seqq}[1]{\seq{\Gamma\ifx#1\relax\else,#1\fi}}

\newcommand{\infer}[3][]{%
  \AxiomC{$#3$}
  \ifx#1\relax\else\LeftLabel{\textsc{#1}}\fi
  \UnaryInfC{$#2$}
  \DisplayProof
}
\newcommand{\inferC}[3]{%
  \AxiomC{$#3$}
  \RightLabel{$~~#1$}
  \UnaryInfC{$#2$}
  \DisplayProof
}
\newcommand{\inferii}[4][]{%
  \AxiomC{$#3$}
  \AxiomC{$#4$}
  \ifx#1\relax\else\LeftLabel{\textsc{#1}}\fi
  \BinaryInfC{$#2$}
  \DisplayProof
}


%%%%%%%%%%%%%%%%%%%%%%%%%%%%%%%%%%%%%%%%%%%%



\title{Decision procedure for string constraints \\
involving the integer data type}


\author{}

\institute{ }

\begin{document}


\maketitle

\begin{abstract}
In this note, we consider straight-line string constraints involving string and integer data types. 
%We define both the concrete and abstract version of the straight-line fragment. 
We propose semantic conditions and a generic decision procedure for the path feasibility of the symbolic execution of programs satisfying the semantic conditions. 
Furthermore, we show that common string operations,  including concat, replaceall, transducers, reverse, substring, indexof, and length, satisfy the semantic conditions.
%the concrete version of straight-line fragment satisfies the semantic conditions, thus enjoys a decidable satisfiability problem. 
Our approach is based on a variant of cost register automata.
\end{abstract}

%\section{Introduction}
%\label{intro}
%
%\section{Preliminaries}
%\label{prel}
%
%Definition for NFA, NFT.

\section{Preliminaries}

For $n \in \nat$ with $n \ge 1$, we use $[n]$ to denote $\{1, \cdots, n\}$.

A string over $\Sigma$ is a (possibly empty) sequence of elements from $\Sigma$. Let $w=a_1\cdots a_n$ be a string. The reserve of $w$, denoted by $w^{(r)}$, is $a_n \cdots a_1$.

We consider two data types, the string data type and the integer data type. We will use $c, d,\dots$ to denote integer constants, $u, v, \dots$ to denote string constants,  $i, j, \dots$ to denote the  integer variables, and $x, y, \dots$ to denote the string variables.

A finite automaton (FA) $\Aut$ is a tuple $(Q, \Sigma, \delta, I, F)$, where $Q$ is a finite set of states, $\Sigma$ is a finite alphabet, $\delta \subseteq Q \times \Sigma \times Q$ is the transition relation, $I,F \subseteq Q$ are the set of initial and final states respectively. A string $w=\sigma_1 \cdots \sigma_n$ is accepted by $\Aut$ if there is a state sequence $q_0 \cdots q_n$ such that $q_0 \in I$, $q_n \in F$, and $(q_{i-1}, \sigma_i, q_i) \in \delta$ for each $i \in [n]$. In particular, an empty string $\varepsilon$ is accepted by $\Aut$ if $I \cap F \neq \emptyset$. The language defined by $\Aut$, denoted by $\Lang(\Aut)$, is defined as the set of strings accepted by $\Aut$.

A finite transducer (FT) $T$ is a tuple $(Q, \Sigma, \delta, I, F)$, where $Q$ is a finite set of states, $\Sigma$ is a finite alphabet, $\delta$ is the transition relation, which is a finite subset of $Q \times \Sigma \times Q \times \Sigma^*$, $I,F \subseteq Q$ are the set of initial and final states respectively. For readability, we write a transition $(q, \sigma, q', u)$ as $q \xrightarrow{\sigma, u} q'$. A run of $T$ over a string $w=\sigma_1 \cdots \sigma_n$ is a state sequence of transitions $q_0 \xrightarrow{\sigma_1, u_1} q_1 \cdots q_n \xrightarrow{\sigma_n, u_n} q_n$. The run is accepting if $q_0 \in I$ and $q_n \in F$. The string $u_1 \cdots u_n$ is called the output of the run. We use $\cT(T)$ to denote the set of pairs $(w, u)$ such that there is an accepting run of $T$ on $w$, with the output $u$.

A linear arithmetic formula $\phi$ is defined by the rules: $\phi::= t \ o \ t \mid \neg \phi \mid \phi \vee \phi \mid \exists i.\ \phi$, where $o \in \{=, \neq, \le, \ge, <, >\}$ and $t$ is defined by the rules $t::= i \mid c \mid ct \mid t + t $.  
For a quantifier free linear integer arithmetic formula $\phi$ that contains the free variables $i_1, \cdots, i_k$, we use $\cM(\phi)$ to denote the set of models of $\phi$, namely, $\{(c_1, \cdots ,c_k) \mid \phi[c_1/i_1, \cdots, c_k/i_k] \mbox{ holds}\}$. An existential linear arithmetic formula is a linear arithmetic formula where all the existential quantifiers are under the scope of even number of negation symbols.

\section{The logic \slint}


We consider two types of functions, string functions that return strings and integer functions that return integers. Specifically, we consider 
\begin{itemize}
\item string functions $f(x_1, \vec{i_1}, \cdots, x_k, \vec{i_k})$, where $f$ is of the arity $\Sigma^* \times \intnum^{n_1} \times \cdots \times \Sigma^* \times \intnum^{n_k} \rightarrow 2^{\Sigma^*}$, and
\item  integer functions $g(x_1, \vec{i_1}, \cdots, x_k, \vec{i_k})$, where $g$ is of the arity $\Sigma^* \times \intnum^{n_1} \times \cdots \times \Sigma^* \times \intnum^{n_k} \rightarrow 2^\intnum$.
\end{itemize} 
Note that $f$ and $g$ can be nondeterministic.
%\subsection{The abstract version}

We consider string constraints where the formulae are of the form $S \wedge A$ defined by the following rules,
\[
\begin{array}{l c l}
t  &::=& i \mid c \mid  g(x_1, \vec{t_1}, \cdots, x_k, \vec{t_k}) \mid ct \mid t + t,   \\
S &::= &  x:=f(x_1, \vec{t_1}, \cdots, x_k, \vec{t_k}) \mid S;S, \\
A_r & ::= & x \in \Aut  \mid A_r \wedge A_r, \\
A_i & ::= & t\ o\ t \mid A_i \wedge A_i \mid A_i \vee A_i,\\
A & ::= &   A_r \wedge A_i, 
\end{array}
\]
where $f$ is a string function and $g$ is an integer function, $\vec{t_j} = t_{j,1}, \cdots, t_{j, n_j}$ for each $j \in [k]$, $\Aut$ is a finite-state automaton, and $o \in \{=, \neq, \ge, \le, >, <\}$.

The logic {\slint} is defined as straight-line fragment of the aforementioned string constraints, specifically, {\slint} is defined as the collection of the formulae $S \wedge A$ satisfying that {\bf $S$ is in single static assignment (SSA) form}.  Note that in {\slint}, the straight-line restriction is applied only on $S$, which contains only the assignments to string variables (but not integer variables). No restrictions are put on the integer constraints in $A_i$.
%Intuitively speaking, the integer constraints in $S \wedge A$ are split into the integer assignments in $S$ where the right-hand side is of the form $g(x_1, \vec{i_1}, \cdots, x_k, \vec{i_k})$ and the constraints $t\ o\ t$ in $A$ where the integer functions $g$ do not occur.
%\begin{itemize}
%\item $S$ is in single static assignment (SSA) form,
%\item all the assignments $i: = t$ in $S$ satisfy that either $t$ is of the form $g(x_1, \vec{i_1}, \cdots, x_k, \vec{i_k})$, or $t$ contains no occurrences of the functions of the form $g(x_1, \vec{i_1}, \cdots, x_k, \vec{i_k})$, namely, $t$ is an integer term built from integer variables and constants ,
%
%\item $A$ contains no occurrences of the functions $g(x_1, \vec{i_1}, \cdots, x_k, \vec{i_k})$,
%
%\item all the string variables in $A$ also occur in $S$.
%\end{itemize} 

%%%%%%%%%%%%%%%%%%%%%%%%%%%%%%%%%%%%%%%%%%%%%%%%%%
%%%%%%%%%%%%%%%%%%%%%%%%%%%%%%%%%%%%%%%%%%%%%%%%%%
\hide{
\subsection{The concrete version}

SL$_{int}$ comprises all the formulae $S \wedge A$ defined by the following rules,
\[
\begin{array}{l c l}
t  &::=& i \mid c \mid \length(x) \mid \indexof_u(x, i) \mid t + t,   \\
S &::= & i:= t \mid x := \substring(y, i, j)  \mid x:= y \concat z \mid x:= \replaceall_{e,u}(y) \mid\\
&&  x:=\reverse(y) \mid x:=T(y) \mid S;S, \\
A & ::= &   x \in \Aut \mid t\ o\ t \mid A \wedge A,
\end{array}
\]
where $c$ is an integer constant, $e$ is a regular expression,  $u$ is a string constant, $T$ is an NFT, $\Aut$ is an NFA, and $o \in \{=, \neq, \le, \ge, <, >\}$.
Note that $\replaceall_{e,u}$ is the replaceAll function where $e$ and $u$ are the pattern and the replacement parameters.

For technical convenience, we assume that 
\begin{itemize}
\item $S$ is in single static assignment (SSA) form,
%
\item all the assignments $i: = t$ in $S$ satisfy that either $t = \length(x)$, $t=\indexof_u(x, j)$, or $t$ contains no occurrences of these two functions,
%
\item $A$ contains no occurrences of the functions $\length$ or $\indexof$,
%
\item all the variables in $A$ also occur in $S$.
\end{itemize} 
}
%%%%%%%%%%%%%%%%%%%%%%%%%%%%%%%%%%%%%%%%%%%%%%%%%%
%%%%%%%%%%%%%%%%%%%%%%%%%%%%%%%%%%%%%%%%%%%%%%%%%%

\begin{example}
The formula $x:= y \concat z \wedge y := \substring(y', \indexof(x, c), j)  \wedge y' \in (ab)^* \wedge z \in a^* c b^* \wedge   j = 2 \indexof(x, c)$ belongs to \slint.
\end{example}

In the next section, we specify the semantic conditions for {\slint} in order to achieve decision procedures. For this purpose, we need the concepts of cost-enriched regular languages and recognisable relations. 

\section{The semantic conditions}

\subsection{Cost-enriched regular languages and recognisable relations}

A cost-enriched string is $(w, (n_1, \cdots, n_k))$ with $w$  a string and $n_i \in \intnum$ for all $i \in [k]$. 
A cost-enriched language $L$ is a subset of $\Sigma^* \times \intnum^k$ for some $k$. Note that all the cost-enriched strings in $L$ have the same number of costs, namely $k$.
A cost-enriched relation $\cR$ is a subset of $\Sigma^* \times \intnum^{k_1} \times \cdots \Sigma^* \times \intnum^{k_l}$.

\begin{definition}[Cost-enriched finite automata and regular languages]
A cost-enriched finite automaton (CEFA) $\Aut$ is a tuple $(Q, \Sigma, R, \delta, I, F)$ where $Q, \Sigma, I, F$ are as in FA, $R=(r_1, \cdots, r_k)$ is a vector of (mutually distinct) registers, $\delta$ is the transition relation which is a finite set of tuples $(q, \sigma, q', \eta)$ where $q, q' \in Q$, $\sigma \in \Sigma$, and $\eta: R \rightarrow \intnum$ is the cost-update function. For convenience, we usually write $(q, \sigma, q', \eta) \in \Delta$ as $q \xrightarrow{\sigma, \eta} q'$.
\\
A cost-enriched string $(w, (n_1, \cdots,n_k)) \in \Sigma^* \times \intnum^k$ with $w=\sigma_1 \cdots \sigma_m$ is accepted by $\Aut$ if there is a sequence of transitions $q_0 \xrightarrow{\sigma_1, \eta_1} q_1 \cdots q_{m-1} \xrightarrow{\sigma_m, \eta_m} q_m$ such that $n_i = \eta_1(r_i) + \cdots + \eta_m(r_i)$ for each $i \in [k]$. The set of cost-enriched strings accepted by $\Aut$ is denoted as $\Lang(\Aut)$. A cost-enriched language $L \subseteq \Sigma^* \times \intnum^k$ is called a cost-enriched regular language (CERL) if there is a CEFA $\Aut$ such that $L = \Lang(\Aut)$.
\end{definition}
CEFA can be seen as a variant of CRA in \cite{RLJ+13}, by adding the nondeterminism and discarding the partial final cost function. 

For a CEFA $\Aut$, we use $R(\Aut)$ to denote the vector of registers occurring in $\Aut$. Moreover, for a CEFA $\Aut$ and a vector of integer variables $\vec{i}$ such that $R(\Aut) \cap \vec{i} = \emptyset$, we use $\Aut[\vec{i}/R(\Aut)]$ to denote the CEFA obtained from $\Aut$ by replacing the registers in $R(\Aut)$ with those from $\vec{i}$. 

%Let $\Aut = (Q, \Sigma, R, \delta, I, F)$ be a CEFA and $I',F' \in Q$. Then we use $\Aut_{I', F'}$ to denote the CEFA $(Q, \Sigma, R, \delta, I', F')$.

Given two CEFAs $\Aut_1 = ( Q_1, \Sigma, R_1, \delta_1, I_1, F_1)$ and $\Aut_2 = (Q_2, \Sigma, \delta_2, R_2, I_2, F_2)$ with $R_1 \cap R_2 = \emptyset$, we define the product of $\Aut_1$ and $\Aut_2$, denoted by $\Aut_1 \times \Aut_2$, as $(Q_1 \times Q_2, \Sigma, R_1 \cup R_2, \delta, I_1 \times I_2, F_1 \times F_2)$ such that $\delta$ comprises the tuples $((q_1, q_2), \sigma, (q'_1, q'_2), \eta)$ satisfying that $(q_1, \sigma, q'_1, \eta_1) \in \delta_1$, $(q_2, \sigma, q'_2, \eta_2) \in \delta_2$, and $\eta = \eta_1\cup \eta_2$ for some $\eta_1, \eta_2$.

%Note that according to the definition, over each word $w$, $\Aut^{(r)}(w) = \Aut(w)$. 

\begin{definition}[LA-SAT w.r.t. CEFA]\label{def-la-sat-cefa}
Let $\Aut_1=(Q_1, \Sigma, R_1, \delta_1, I_1, F_1)$, $\cdots$, $\Aut_k=(Q_k, \Sigma, R_k, \delta_k, I_k, F_k)$ be CEFAs
%  with $R_i = (r_{i,1}, \cdots, r_{i, l_i})$ for each $i \in [k]$, 
  and $\phi$ be a quantifier-free linear arithmetic formula whose free variables are from  $R_1 \cup \cdots \cup R_k \cup X$ for some $X$ such that $X \cap (R_1 \cup \cdots \cup R_k) = \emptyset$. Then $\phi$ is said to be satisfiable w.r.t. $\Aut_1, \cdots, \Aut_k$ if  there are words $w_1, \cdots, w_k$ and an assignment function $\eta: R_1 \cup \cdots R_k \cup X \rightarrow \intnum$  %integers $\vec{c_1} \in \intnum^{|R_1|}, \cdots, \vec{c_k} \in \intnum^{|R_k|}$, $\vec{d} \in \intnum^{|X|}$
 such that  $(w_1, \eta(R_1)) \in \Lang(\Aut_1)$, $ \cdots$, $(w_k, \eta(R_k)) \in \Lang(\Aut_k)$, and $\phi[\eta(R_1)/R_1, \cdots,\eta(R_k)/R_k, \eta(X)/X]$ holds.
\end{definition}
Note that in Definition~\ref{def-la-sat-cefa}, it may happen that $R_i \cap R_j \neq \emptyset$ for some $i, j \in [k]$ with $i \neq j$.

\begin{theorem}\label{thm-incra-la-sat}
The LA-SAT w.r.t. CEFA problem is decidable.
\end{theorem}

For the proof of Theorem~\ref{thm-incra-la-sat}, we state and prove the following lemma. 

\begin{lemma}\label{lem-incra-la}
Let $\Aut=(Q, \Sigma, R, \delta, I, F)$ be a CEFA with $R= (r_1, \cdots,  r_m)$. Then there is an existential linear arithmetic formula $\varphi_\Aut(r_1, \cdots, r_m)$ such that $\cM(\varphi_\Aut)= \{(c_1, \cdots, c_m) \mid \mbox{there exists } w \mbox{ such that } (w, c_1, \cdots, c_m) \in \Lang(\Aut)\}$.
\end{lemma}

\begin{proof}
Let $\delta = \{\tau_1, \cdots, \tau_l\}$ such that $\tau_j = (p_j, \sigma_j, p'_j, \eta_j)$ and $\eta_j(r_i) =  c_{i,j}$ for each $j \in [l]$ and $i \in [m]$.
According the results on FA, we know that for each pair of states $(q, q') \in I \times F$,  an existential linear arithmetic formula $\varphi_{q,q'}(j_1, \cdots, j_l)$ can be computed in linear time such that $\cM(\varphi_{q,q'})$ is the set of Parikh images of the transition sequences of $\Aut$ starting from $q$ and ending at $q'$. 

Then 
\[\varphi_\Aut(r_1, \cdots, r_m) ::= \bigvee \limits_{(q,q') \in I \times F} \exists j_1 \cdots \exists j_l.\ \left(\varphi_{q,q'}(j_1, \cdots, j_l) \wedge \bigwedge \limits_{i \in [m]} r_i = \sum \limits_{j \in [l]} c_{i,j} j_j \right).\]
\qed
\end{proof}

\begin{proof}[Theorem~\ref{thm-incra-la-sat}]
Let $\Aut_1=(Q_1, \Sigma, R_1, \delta_1, I_1, F_1)$, $\cdots$, $\Aut_k=(Q_k, \Sigma, R_k, \delta_k, I_k, F_k)$ be CEFAs and $\phi$ be a quantifier-free linear arithmetic formula whose free variables are from  $R_1 \cup \cdots \cup R_k \cup X$ for some $X$ such that $X \cap (R_1 \cdots R_k) = \emptyset$.
Suppose that for each $i \in [k]$, $R_i = (r_{i, 1}, \cdots, r_{i, l_i})$. Then the satisfiability of $\phi$ w.r.t. $\Aut_1,\cdots, \Aut_k$ can be reduced to the satisfiability problem of the  existential linear arithmetic formula
$
\phi \wedge \bigwedge \limits_{i \in [k]} \varphi_{\Aut_i}(r_{i,1}, \cdots, r_{i, l_i}).
$
\qed
\end{proof}

\begin{definition}[Cost-enriched recognisable relations]
A cost-enriched relation $\cR \subseteq \Sigma^* \times \intnum^{k_1} \times \cdots  \times \Sigma^* \times \intnum^{k_l}$ is a cost-enriched recognisable relation (CERR)  if it is a finite union of products of cost-enriched regular languages, namely, 
\[\cR = \bigcup \limits_{i=1}^n L_{i,1 } \times \cdots \times L_{i, l},\]
where $L_{i,1} \subseteq \Sigma^* \times \intnum^{k_1}, \cdots, L_{i, l} \subseteq \Sigma^* \times \intnum^{k_l}$ are CERL. 
A CEFA representation of $\cR$ is a collection of CERA tuples $(\Aut_{i,1}, \cdots, \Aut_{i,l})_{i \in [n]}$ such that $\Lang(\Aut_{i,j}) = L_{i,j}$ for each $i \in [n]$ and $j \in [l]$.
\end{definition}



\subsection{The two semantic conditions}
%The first semantic condition we put is as follows: The relation defined by the integer functions $g(x_1, \vec{i_1}, \cdots, x_k, \vec{i_k})$ is 

For specifying the semantic conditions, we introduce two additional concepts. 

\begin{definition}[CERR linear integer functions]
An integer function $g: \Sigma^* \times \intnum^{k_1} \times \Sigma^* \times \intnum^{k_l} \rightarrow 2^\intnum$ is a CERR linear integer function if there is a pair $(\cR, t)$ such that $\cR \subseteq \Sigma^* \times \intnum^{k_1+1} \times \Sigma^* \times \intnum^{k_l+1}$ is a CERR and $t$ a linear integer term over $r^{(1)}, \cdots, r^{(l)}$ such that for all $\vec{c_1} \in \intnum^{k_1}, \cdots, \vec{c_l} \in \intnum^{k_l}$, and $d_1 \in \intnum, \cdots, d_l \in \intnum$, it holds that $(w_1, (\vec{c_1}, d_1), \cdots, w_l, (\vec{c_l}, d_l)) \in \cR$ iff $t[d_1/r^{(1)}, \cdots, d_l/r^{(l)}] \in g(w_1, \vec{c_1}, \cdots, w_l, \vec{c_l})$.  For a CERR linear integer function $g$ witnessed by the pair $(\cR, t)$, a CEFA representation of $g$ is a tuple $((\Aut_{i,1}, \cdots, \Aut_{i, l})_{i \in [n]}, t)$, where $(\Aut_{i,1}, \cdots, \Aut_{i, l})_{i \in [n]}$ is a CEFA representation of $\cR$.

\end{definition}

\begin{example}
The string functions $\length$ and $\indexof_u$ are CERR linear integer functions, whose CEFA representations can be found in Section~\ref{sec-cslint}.
\end{example}

\begin{definition}[Cost enriched pre-image of CERL]
Suppose that $f: \Sigma^* \times \intnum^{k_1} \times \cdots \times \Sigma^* \times \intnum^{k_l} \rightarrow 2^{\Sigma^*}$ is a string function, $L \subseteq \Sigma^* \times \intnum^n$ is a CERL, and $L = \Lang(\Aut)$ for some CEFA $\Aut=(Q, R, \delta, I, F)$ where $R= (r_1, \cdots, r_n)$. Then the $R$-cost enriched pre-image of $L$ under $f$ is a pair $(\cR, \vec{t})$ such that 
\begin{itemize}
\item $\cR \subseteq \Sigma^* \times \intnum^{k_1 + n} \times \cdots \times \Sigma^* \times \intnum^{k_l + n}$,
\item $\vec{t} = (t_1, \cdots ,t_n)$ is a vector of linear integer terms where for each $i \in [n]$, $t_i$ is a term over $\vec{r_i} = (r^{(1)}_i, \cdots, r^{(l)}_i)$,
\item and 

$L = \{(f(w_1, \vec{c_1}, \cdots, w_l, \vec{c_l}), t_1[d_{1,1}/r^{(1)}_1, \cdots, d_{l, 1}/r^{(l)}_1], \cdots, t_n[d_{1,n}/r^{(1)}_n, \cdots, d_{l, n}/r^{(l)}_n]) \mid (w_1, (\vec{c_1}, \vec{d_1}), \cdots, w_l, (\vec{c_l}, \vec{d_l})) \in \cR\}$ 
%
(where $\vec{d_1}=(d_{1,1}, \cdots, d_{1,n})$, $\cdots$, and $\vec{d_l}=(d_{l,1},\cdots, d_{l,n})$).
\end{itemize}
The $R$-cost enriched pre-image of $L$ under $f$, say $(\cR, \vec{t})$, is said to be CERR-definable if $\cR$ is a CERR. If the $R$-cost enriched pre-image of $L$ under $f$, say $(\cR, \vec{t})$, is CERR-definable,  then its CEFA representation is a tuple $((\Aut_{i,1}, \cdots, \Aut_{i, l})_{i \in [m]}, \vec{t})$, where $(\Aut_{i,1}, \cdots, \Aut_{i, l})_{i \in [m]}$ is a CEFA representation of $\cR$. 
\end{definition}


\begin{example}
Let $\Aut = (Q, R, \delta, I, F)$. The $R$-cost enrichment of the pre-image of $\Lang(\Aut)$ under $\substring$ is CERR-definable and its CEFA representation can be found in Section~\ref{sec-cslint}.
\end{example}

Now we are ready to state the two semantic conditions.
\begin{description}
\item [The 1st semantic condition.] Each integer function $g$ is a CERR linear integer function, moreover, a CEFA representation of $g$ can be effectively computed from $g$.
%
\item [The 2nd semantic condition.] Each string function $f$ satisfies that for each CERL $L$, the cost enriched pre-image of $L$ under $f$, say $(\cR,\vec{t})$, satisfies that $\cR$ is a CERR, moreover, a CEFA representation can be effectively computed from $f$ and $L$.
\end{description}

%Let $X$ be a set of registers and $G_{inc}(X)$ denote the set of terms defined by the rules $t::=c \mid x \mid t+c$, where $c$ is an integer constant.  We interpret terms from $G_{inc}(X)$ over the set of integers.

%Let $\Aut=(\Sigma, Q, I, F, R, \delta)$ be an INCRA with $R=r_1\cdots r_m$. Over an input word $w=\sigma_1 \cdots \sigma_n \in \Sigma^+$, a run of $\Aut$ on $w$ is a sequence $q_0 \xrightarrow{\sigma_1, \eta_1} q_1 \cdots q_{n-1} \xrightarrow{\sigma_n, \eta_n} q_n$ such that $q_0 \in I$ and $(q_{i-1}, \sigma_i, \eta_i, q_i) \in \delta$ for each $i \in [n]$. A run is accepting if $q_n \in F$. The output of an accepting run of $\Aut$ on $w$ is a tuple $(i_1,\cdots, i_m)$, where $i_j = \eta_n(r_j) (\cdots \eta_1(r_j)(0)\cdots)$ for each $j \in [m]$. Note that the initial value of each register $r_j$ is zero. We define $\Aut(w)$ as the set of outputs of the accepting runs of $\Aut$ on $w$ (possibly it is an empty set). Note the in general, an output of an INCRA is a tuple, instead of a single integer. Moreover, we also use $\Lang(\Aut)$ to denote $\{w \in \Sigma^* \mid \Aut(w) \neq \emptyset\}$ and $\cR(w) = \{(w, \vec{n}) \mid \vec{n} \in \Aut(w)\}$.

%Note that a nondeterministic finite-state automaton can be seen as an INCRA $\Aut=(\Sigma, Q, I, F, R, \delta)$ where $R$ is empty.

\subsection{A string logic satisfying the semantic conditions}\label{sec-cslint}
The string logic {\cslint} defined by the following rules satisfies the two semantic conditions,
\[
\begin{array}{l c l}
t  &::=& i \mid c \mid \length(x) \mid \indexof_u(x, i) \mid  ct  \mid t + t,   \\
S &::= &  x:= y \concat z \mid x:= \replaceall_{e,u}(y) \mid   x:=\reverse(y) \mid x:=T(y) \mid \\
& & x := \substring(y, t_1, t_2)  \mid S;S, \\
A_r & ::= & x \in \Aut \mid A_r \wedge A_r,\\
A_i & ::= & t\ o\ t \mid A_i \wedge A_i  \mid A_i \vee A_i,\\
A & ::= &   A_r \wedge A_i,
\end{array}
\]
where  $u \in \Sigma^+$, $e$ is a regular expression, $T$ is a finite-state transducer, and $o \in \{=, \neq, \ge, \le, >, <\}$.

In the following, we show that the integer and string operations in {\cslint} satisfy the semantic conditions.

Let $\Aut=(Q, \Sigma, R, \delta, I, F)$ be a CEFA with $R= (r_1, \cdots, r_m)$. 

\smallskip
\noindent \emph{Concatenation $x_1 \concat x_2$}.

\smallskip

Then $((\Aut_{I, q}, \Aut_{q, F})_{q \in Q}, \vec{t})$ is a CEFA representation of the $R$-cost enriched pre-image of $\Lang(\Aut)$ under $\concat$, where $\Aut_{I, q}=(Q, \Sigma, R^{(1)}, \delta^{(1)}, I, \{q\})$ and  $\Aut_{q, F}=(Q, \Sigma, R^{(2)}, \delta^{(2)}, \{q\}, F)$ such that 
\begin{itemize}
\item $R^{(1)} = (r^{(1)}_1, \cdots, r^{(1)}_m)$, $R^{(2)} = (r^{(2)}_1, \cdots, r^{(2)}_m)$, 
\item $\delta^{(1)}$ comprises the tuples $(q, \sigma, q', \eta')$ satisfying that $(q, \sigma, q', \eta) \in \delta$ and for each $j \in [m]$, $\eta'(r^{(1)}_j)=\eta(r_j)$,  similarly for $\delta^{(2)}$,
\end{itemize}
and $\vec{t} = (r^{(1)}_1 + r^{(2)}_1, \cdots, r^{(1)}_m + r^{(2)}_m)$.


\smallskip 

\noindent \emph{Reverse $\reverse(x_1)$}. 

$(\Aut^{(r)}, (r^{(1)}_1, \cdots, r^{(1)}_m))$ is the CEFA representation of the $R$-cost enriched pre-image of $\Lang(\Aut)$ under $\reverse$, where $\Aut^{(r)}$ is $(Q, \Sigma, R, \delta', F, I)$ such that $\delta'$ comprises the set of tuples $(q', \sigma, q, \eta)$ with  $(q, \sigma, q', \eta) \in \delta$. Note that $\Lang(\Aut^{(r)}) = \{(w^{(r)}, \vec{c}) \mid (w, \vec{c}) \in \Lang(\Aut)\}$.


\smallskip

\noindent \emph{Substring $\substring(x_1, i, j)$}.

Intuitively, $\substring(x_1, i, j)$ returns the substring of $x_1$ starting from the position $i$ and ending at the position $j$ (assuming that $i  < j$), with the letter at the position $j$ excluded.

$(\cB, (r^{(1)}_1, \cdots, r^{(1)}_m))$ is the CEFA representation of the $R$-cost enriched pre-image of $\Lang(\Aut)$ under $\substring$, where $\cB = (Q \times \{p_0, p_1, p_2\}, \Sigma, R', \delta', I \times \{p_0\}, F \times \{p_2\})$ such that $R' = (i, j, r^{(1)}_1,\cdots, r^{(1)}_m)$ and $\delta'$ comprises 
\begin{itemize}
\item the tuples $((q, p_0), \sigma, (q, p_0), \eta')$ such that $q \in I$ and $\eta' = \eta_0 \cup \{i \rightarrow 1, j \rightarrow 1\}$, where $\eta_0(r^{(1)}_j)=0$ for each $j \in [m]$,
%
\item the tuples $((q, p_0), \sigma, (q', p_1), \eta')$ such that $(q, \sigma, q', \eta) \in \delta$ and $\eta' = \eta^{(1)} \cup \{i \rightarrow 1, j  \rightarrow 1\}$, where $\eta^{(1)}(r^{(1)}_j)=\eta(r_j)$ for each $j \in [m]$,
%
\item the tuples $((q, p_1), \sigma, (q', p_1), \eta')$ such that $(q, \sigma, q', \eta) \in \delta$ and $\eta' = \eta^{(1)} \cup \{i \rightarrow 0, j  \rightarrow 1\}$, where $\eta^{(1)}(r^{(1)}_j)=\eta(r_j)$ for each $j \in [m]$,
%
\item the tuples $((q, p_1), \sigma, (q, p_2), \eta')$ such that $q \in F$ and $\eta' = \eta_0 \cup \{i \rightarrow 0, j  \rightarrow 1\}$, where $\eta_0(r^{(1)}_j)=0$ for each $j \in [m]$,
%
\item the tuples $((q, p_2), \sigma, (q, p_2), \eta')$ such that $q \in F$ and $\eta' = \eta_0 \cup \{i \rightarrow 0, j  \rightarrow 0\}$, where $\eta_0(r^{(1)}_j)=0$ for each $j \in [m]$.
%
\end{itemize}
%

\smallskip
\noindent \emph{FT $T(x_1)$}.

\smallskip

Let $T= (Q', \Sigma, \delta', I', F')$. Then $(\cB, (r^{(1)}, \cdots, r^{(1)}_m))$ is the CEFA representation of the $R$-cost enriched pre-image of $\Lang(\Aut)$ under $T$, where $ \cB= (Q \times Q', \Sigma, R^{(1)}, \delta'', I \times I', F \times F')$ such that $R^{(1)}  = (r^{(1)}, \cdots, r^{(1)}_m)$, $\delta''$ comprises the tuples $((q_1, q'_1), \sigma, (q_2, q'_2), \eta')$ satisfying that $(q'_1, \sigma, q'_2, u) \in \delta'$ with $u = \sigma_1 \cdots \sigma_i$, and in $\Aut$, we have $p_1 \xrightarrow{\sigma_1, \eta_1} p_2 \cdots \xrightarrow{\sigma_i, \eta_i} p_{i+1}$ with $p_1 = q_1$ and $p_{i+1}= q_2$, and for each $j \in [m]$,  $\eta'(r^{(1)}_j) = \eta_1(r_j) + \cdots + \eta_i(r_j)$.
%

\smallskip 

\noindent \emph{ReplaceAll $\replaceall_{e,u}(x)$}.

\smallskip

Intuitively, $\replaceall_{e,u}(x)$ is the string obtained by replacing every occurrence of $e$ in $x$ with the constant string $u$.

From the results in \cite{CCH+18}, we know that  a FT $T_{e,u}$ can be constructed to simulate $\replaceall_{e,u}$. 
Therefore, a CEFA representation of the $R$-cost enriched pre-image of $\Lang(\Aut)$ under $T$ can be constructed as in the FT case.
% 

\smallskip 

\noindent \emph{Length $\length(x_1)$}.

\smallskip

$(\cB, r^{(1)})$ is a CEFA representation of $\length$, where $\cB = (Q', \Sigma, R^{(1)}, \delta', I', F')$ such that $Q' = \{q'_0\}$, $I'=F'=\{q'_0\}$, $R^{(1)} = (r^{(1)})$, $\delta' = \{(q'_0, \sigma, q'_0, \eta) \mid \sigma \in \Sigma, \eta(r^{(1)}) = 1\}$.

\smallskip 

\noindent \emph{IndexOf $\indexof_u(x_1, i)$}.

\smallskip

Suppose $u = \sigma_1 \cdots \sigma_j$. We use the concept of window profiles of positions w.r.t. $u$, which are elements of $\{\bot, \top\}^{j-1}$, to recognise the first occurrence of $u$ after the position $i$. 

For $\pi \in \{\bot, \top\}^{j-1}$ and $\sigma' \in \Sigma$, $upt(\vec{\pi}, \sigma')$ is the updated window profile after reading the letter $\sigma'$, specifically, $upt(\vec{\pi}, \sigma') = \vec{\pi'}$ such that  
\begin{itemize}
\item $\pi'_1 = \top$ iff $\sigma' = \sigma_1$, 
%
\item for each $j' \in [j-2]$, $\pi'_{j'+1} = \top$ iff $\pi_{j'} = \top$ and $\sigma' = \sigma_{j'+1}$. 
\end{itemize}
The set of window profiles of $u$, denoted by $WP_u$, is computed by setting $WP_0 := \{\bot^{j-1}\}$ and iterating the following procedure, until $WP_i = WP_{i+1}$:
\[WP_{i+1}:=WP_i \cup \{upt(\vec{\pi}, \sigma') \mid \vec{\pi} \in WP_i, \sigma' \in \Sigma\}.\] 
From the results in \cite{CCH+18}, we know that $|WP_u| \le |u|$. Therefore, the aforementioned iteration terminates in at most $|u|$ steps.


Then $(\cB, r^{(1)})$ is a CEFA representation of $\indexof_u$, where 
$\cB= (Q', \Sigma, R', \delta', I', F')$ such that  $Q' = \{q'_0, q'_1\} \cup WP_u \cup WP_u \times [i]$, $R'=(i, r^{(1)})$, $I'=\{q'_0\}$, $F'=\{q'_1\}$, and $\delta'$ comprises 
\begin{itemize}
\item the tuples $(q'_0, \sigma, q'_0, \eta)$ such that $\sigma \in \Sigma$, $\eta(i)=1$, and $\eta(r^{(1)}) = 1$,
%
\item the tuples $(q'_0, \sigma, \vec{\pi}, \eta)$ such that $\sigma \in \Sigma$, $\vec{\pi} = \theta \bot^{j-2}$ where $\theta  = \top$ iff $\sigma = \sigma_1$, $\eta(i) = 1$, and $\eta(r^{(1)})= 1$,
% 
\item the tuples  $(\vec{\pi}, \sigma, upt(\vec{\pi}, \sigma), \eta)$ such that $\vec{\pi} \in WP_u$, $\sigma \in \Sigma$, $\pi_{j-1} = \bot$ or $\sigma \neq \sigma_{j}$, $\eta(i) = 0$, and $\eta(r^{(1)})= 1$,
%
\item the tuples $(\vec{\pi}, \sigma, (upt(\vec{\pi}, \sigma), 1), \eta)$ such that $\vec{\pi} \in WP_u$, $\sigma = \sigma_1$, $\pi_{j-1} = \bot$ or $\sigma \neq \sigma_{j}$, $\eta(i) = 0$, and $\eta(r^{(1)})= 1$,
%
\item the tuples $((\vec{\pi}, j'),  \sigma, (upt(\vec{\pi}, \sigma), j'+1), \eta)$ such that $\vec{\pi} \in WP_u$, $j' \in [j-2]$, $\sigma = \sigma_{j'+1}$, $\pi_{j-1} = \bot$ or $\sigma \neq \sigma_{j}$, $\eta(i) = 0$, and $\eta(r^{(1)})= 0$,
%
\item the tuples $((\vec{\pi}, j-1),  \sigma, q'_1, \eta)$ such that $\vec{\pi} \in WP_u$, $\sigma = \sigma_{j}$, $\eta(i) =0$, and $\eta(r^{(1)})= 0$,
%
\item the tuples  $(q'_1, \sigma, q'_1, \eta)$ such that $\sigma \in \Sigma$, $\eta(i) = 0$, and $\eta(r^{(1)})= 0$.
\end{itemize}



 
\section{Decision procedure}

Let $S':=S$ and $A':=A$. Moreover, let $A'':= \ltrue$. Then execute the following procedure to (partially) flatten the integer terms.
\begin{description}
\item[Step 1.] Recursively apply the following transformation until $S' \wedge A'$ contains no more occurrences of integer functions: Select an occurrence of integer functions, say $g(x_1, \vec{t_1}, \cdots, x_k, \vec{t_k})$, such that 
%it is a \emph{proper} subterm of the other integer term and 
{\it none} of $\vec{t_1}, \cdots, \vec{t_k}$ contains occurrences of integer functions, introduce a fresh integer variable $i$, let $S' \wedge A'$ be the formula obtained by replacing $g(x_1, \vec{t_1}, \cdots, x_k, \vec{t_k})$ with $i$, moreover, let $A'':= A'' \wedge i = g(x_1, \vec{t_1}, \cdots, x_k, \vec{t_k})$.
%
\item[Step 2.] It comprises the following two substeps. 
\begin{enumerate}
\item For each occurrence of string functions in $S'$, say $f(x_1, \vec{t_1}, \cdots, x_k, \vec{t_k})$, suppose $\vec{t_j} = (t_{j,1}, \cdots, t_{j, l_j})$ for each $j \in [k]$, introduce fresh integer variables $i_{j, j'}$ for $j \in [k]$ and $j' \in [l_j]$, replace $f(x_1, \vec{t_1}, \cdots, x_k, \vec{t_k})$ with $f(x_1, \vec{i_1}, \cdots, x_k, \vec{i_k})$ in $S'$, where $\vec{i_j} = (i_{j,1}, \cdots, i_{j, l_j})$ for each $j \in [k]$, and let $A':=A' \wedge \bigwedge \limits_{j \in [k], j' \in [l_j]} i_{j, j'} = t_{j, j'}$. 
\item For each occurrence of integer functions in $A''$, say $g(x_1, \vec{t_1}, \cdots, x_k, \vec{t_k})$, suppose $\vec{t_j} = (t_{j,1}, \cdots, t_{j, l_j})$ for each $j \in [k]$, introduce fresh integer variables $i_{j, j'}$ for $j \in [k]$ and $j' \in [l_j]$, replace $g(x_1, \vec{t_1}, \cdots, x_k, \vec{t_k})$ with $g(x_1, \vec{i_1}, \cdots, x_k, \vec{i_k})$ in $A''$, where $\vec{i_j} = (i_{j,1}, \cdots, i_{j, l_j})$ for each $j \in [k]$, and let $A':=A' \wedge \bigwedge \limits_{j \in [k], j' \in [l_j]} i_{j, j'} = t_{j, j'}$. 
\end{enumerate}
%
\item[Step 3.] Let $S:=S'$ and $A:=A'' \wedge A' $.
\end{description}
The aforementioned flattening procedure is a bit technical, for simplicity, we may assume that the integer terms are fully flattened, including the arithmetic operations.

Note that after the aforementioned flattening procedure, the resulting formula $S \wedge A$ satisfies the following property: 
\begin{quote}
The integer terms in all the occurrences of string and integer functions  are integer variables, moreover, each integer variable occurs at most once in these string and integer functions.  \hfill ($*$)
\end{quote}
Therefore, in the sequel, we assume that $S \wedge A$ satisfies the property ($*$).

\begin{theorem}\label{thm-sl-int-dec}
Path feasibility of {\slint} satisfying the semantic conditions is decidable.
\end{theorem}

\begin{proof}
In the following, we extend the generic decision procedure in \cite{CHL+18}, where NFA is replaced by CEFA.

Let $S \wedge A$ be an {\slint} formula (satisfying the property ($*$)).

For each occurrence of $i = g(x_1, \vec{i'_1}, \cdots, x_k, \vec{i'_k})$ in $A$ with $g$ an integer function, apply the following nondeterministic transformation to $A$: 
\begin{quote}
According to the 1st semantic condition, $g$ is a CERR linear integer function and a CEFA representation of $g$, say $((\Aut_{j,1}, \cdots, \Aut_{j, k})_{j \in [m]}, t)$, can be computed effectively from $g$. Consider $((\Aut'_{j,1}, \cdots, \Aut'_{j, k})_{j \in [m]}, t')$, where $\Aut'_{j,1}=\Aut_{j,1}[\vec{i'_1}/R(\Aut_{j,1})]$, $\cdots$, $\Aut'_{j,k}=\Aut_{j,k}[\vec{i'_k}/R(\Aut_{j,k})]$, and $t' = t[i^{(1)}/r^{(1)}, \cdots, i^{(k)}/r^{(k)}]$.
Nondeterministically choose $j \in [m]$, and replace $i = g(x_1, \vec{i'_1}, \cdots, x_k, \vec{i'_k})$ by $x_1 \in \Aut'_{j,1} \wedge \cdots \wedge x_k \in \Aut'_{j,k} \wedge i = t'$ in $A$.
\end{quote}
Note that after this transformation, $S \wedge A$ contains no occurrences of integer functions, moreover, as a result of the property ($*$), for every variable $x$, all the CEFAs to which $x$ belongs satisfy that their sets of registers are  mutually disjoint.

Then repeat the following procedure until $S$ becomes empty.
%
\begin{quote}
Suppose $y := f(x_1, \vec{i_1}, \cdots, x_k, \vec{i_k})$ is the last assignment of $S$. 
\\
Let $\rho := \{\Aut_1, \cdots, \Aut_s\}$ be the set of all CEFAs such that $y \in \Aut_j$ occurs in $A$ for each $j \in [s]$. Construct $\Aut = \Aut_1 \times \cdots \times \Aut_s$ (Recall that the sets of registers of $\Aut_1$, $\cdots$, $\Aut_s$ are mutually disjoint). Let  the vector of registers in $\Aut$ be $R = (r'_1, \cdots, r'_n)$. Then according to the 2nd semantic condition, 
a CEFA representation of the $R$-cost enriched pre-image of $\Lang(\Aut)$ under $f$, say $((\cB_{j, 1}, \cdots, \cB_{j, k})_{j \in [\ell]}, \vec{t})$, can be effectively computed from $\Aut$ and $f$. Consider $((\cB'_{j, 1}, \cdots, \cB'_{j, k})_{j \in [\ell]}, \vec{t'})$, where $\cB'_{j, 1} = \cB_{j, 1}[\vec{i_1}/R(\cB_{j,1}), \vec{(r')^{(1)}}/\vec{r^{(1)}}]$, $\cdots$, $\cB'_{j,k}=\cB_{j,k}[\vec{i_k}/R(\cB_{j,k}), \vec{(r')^{(k)}}/\vec{r^{(k)}}]$ (with $\vec{r^{(1)}}= (r^{1}_1, \cdots, r^{(1)}_n)$, similarly for $\vec{r^{(2)}}$ and so on), and $\vec{t'} = \vec{t}[\vec{r'_1}/\vec{r_1}, \cdots, \vec{r'_n}/\vec{r_n}]$ (with $\vec{r_1} = (r^{(1)}_1, \cdots, r^{(k)}_1)$, similarly for $\vec{r^{(2)}}$ and so on). 
\\
Nondeterministically choose $j \in [\ell]$ and let 
$$A:= A \wedge x_1 \in \cB'_{j, 1} \wedge \cdots \wedge x_k \in \cB'_{j, k}  \wedge \bigwedge \limits_{j' \in [n]} r'_{j'} = t'_{j'}.$$
%
Remove $y := f(x_1, \vec{i_1}, \cdots, x_k, \vec{i_k})$ from $S$.
\end{quote}

We would like to remark that if all the string functions $f$ in $S \wedge A$ are \emph{deterministic}, then the product of CEFAs before the pre-image computation can be avoided and the pre-image can be computed \emph{distributively} for CEFAs in $\rho$.

In the end, we get a formula $S \wedge A$ where $S$ is empty. Suppose $A = A_r \wedge A_i$, where $A_r$ is a conjunction of atomic formulae of the form $x \in \Aut$, and $A_i$ is linear arithmetic formula (containing no integer functions). By computing the product construction of CEFAs, $A_r$ can be rewritten as $x_1 \in \Aut_1 \wedge \cdots \wedge x_n \in \Aut_n$, where $x_1,\cdots, x_n$ are mutually distinct. Therefore, the path feasibility of $S \wedge A$ is exactly the satisfiability of $A_i$ w.r.t. the CEFAs $\Aut_1, \cdots, \Aut_n$. From Theorem~\ref{thm-incra-la-sat}, we conclude that the path feasibility of  {\slint} is decidable.
\qed
\end{proof}

\begin{corollary}
Path feasibility of {\cslint} is decidable.
\end{corollary}


\bibliographystyle{abbrv}
\bibliography{string}

\iffalse
\newpage
\setcounter{page}{1}

\begin{appendix}
%!TEX root = main.tex


\section{Construction of $\CEFA_{\indexof_v}$} \label{appendix:cefa_indexof}

In this section, we show that the function $\indexof_v(\cdot, \cdot)$ can be captured by CEFA. Technically, for any NFA $\NFA$ and constant string $v$, we can construct a CEFA %$\CEFA'$ such that 
accepting $\{(w, (n, \indexof_v(w, n)))\mid w\in \Lang(\NFA) \}$. %$R(\CEFA')=(r_1,r_2)$ and $\Lang(\CEFA')=\{(w, (n, \indexof_v(w, n)))\mid w\in \Lang(\NFA) \}$. 

%The construction is slightly technical and can be found in Appendix, Section~\ref{appendix:cefa_indexof}.
For this purpose, we need a concept of window profiles of  string positions w.r.t. $v$, which are elements of $\{\bot, \top\}^{n-1}$. The window profiles facilitate recognising the first occurrence of $v$ in the input string. 
Intuitively, given a string $u$, the window profile of a position $i$ in $u$ w.r.t. $v$ encodes the matchings of prefixes of $v$ to the suffixes of $u[0,i]$ (see \cite{CCH+18} for the details). For $\pi = \pi_1 \cdots \pi_{n-1} \in \{\bot, \top\}^{n-1}$ and $b \in \Sigma$, we use $\uwp(\vec{\pi}, b)$ to represent the window profile updated from $\pi$ after reading the letter $b$, specifically, $\uwp(\vec{\pi}, b) = \vec{\pi'}$ such that  
\begin{itemize}
\item $\pi'_1 = \top$ iff $b = a_1$, 
%
\item for each $i \in [n-2]$, $\pi'_{i+1} = \top$ iff $\pi_{i} = \top$ and $b = a_{i+1}$. 
\end{itemize}
Let $WP_v$ denote the set of window profiles of string positions w.r.t. $v$. From the result in \cite{CCH+18}, we know that $|WP_v| \le |v|$. 
%Then the set of window profiles of $v$, denoted by $WP_v$, is computed by setting $WP_0 := \{\bot^{n-1}\}$ and iterating the following procedure, until $WP_i = WP_{i+1}$:
%\[WP_{i+1}:=WP_i \cup \{\uwp(\vec{\pi}, b) \mid \vec{\pi} \in WP_i, b \in \Sigma\}.\] 
%Therefore, the aforementioned iteration terminates in at most $|v|$ steps.\\
%
%

Suppose $v = a_1 \cdots a_n$ with $n \ge 2$. 
Then $\indexof_v$ is captured by the CEFA $\CEFA_{\indexof_v}=(Q, \Sigma, R, \delta, I, F)$, such that 
\begin{itemize}
\item $Q = \{q_0, q_1\} \cup WP_v \cup WP_v \times [n]$, 
\item $R=(r_1, r_2)$ (where $r_1,r_2$ represent the input and output positions of $\indexof_v$ respectively), 
\item $I=\{q_0\}$, 
\item $F=\{q_1\}$, and 
\item $\delta$ comprises 
\begin{itemize}
\item the tuples $(q_0, a, q_0, \eta)$ such that $a \in \Sigma$, $\eta(r_1)=1$, and $\eta(r_2) = 1$,
%
\item the tuples $(q_0, a, \vec{\pi}, \eta)$ such that $a \in \Sigma$, $\vec{\pi} = \theta \bot^{n-2}$ where $\theta  = \top$ iff $a = a_1$, $\eta(r_1) = 0$, and $\eta(r_2)= 0$ (recall that the first position of a string is $0$),
% 
\item the tuples  $(\vec{\pi}, a, \uwp(\vec{\pi}, a), \eta)$ such that $\vec{\pi} \in WP_u$, $a \in \Sigma$, $\pi_{n-1} = \bot$ or $a \neq a_{n}$, $\eta(r_1) = 0$, and $\eta(r_2)= 1$,
%
\item the tuples $(\vec{\pi}, a, (\uwp(\vec{\pi}, a), 1), \eta)$ such that $\vec{\pi} \in WP_u$, $a = a_1$, $\pi_{n-1} = \bot$ or $a \neq a_{n}$, $\eta(r_1) = 0$, and $\eta(r_2)= 1$,
%
\item the tuples $((\vec{\pi}, i),  a, (\uwp(\vec{\pi}, a), i+1), \eta)$ such that $\vec{\pi} \in WP_u$, $i \in [n-2]$, $a = a_{i+1}$, $\pi_{n-1} = \bot$ or $a \neq a_{n}$, $\eta(r_1) = 0$, and $\eta(r_2)= 0$,
%
\item the tuples $((\vec{\pi}, n-1),  a, q_1, \eta)$ such that $\vec{\pi} \in WP_u$, $a = a_{n}$, $\eta(r_1) =0$, and $\eta(r_2)= 0$,
%
\item the tuples  $(q_1, a, q_1, \eta)$ such that $a \in \Sigma$, $\eta(r_1) = 0$, and $\eta(r_2)= 0$.
\end{itemize}
\end{itemize}

%%%%%%%%%%%%%%%%%%%%%%%%%%%%%%%%%%%%%%%%%%%%%%%%%%%%%%%%%%%%%%%%%%%%%%%%%%%%%%


\section{Proof of Proposition~\ref{prop:pre-image}}\label{app:pre-image}


\noindent {\bf Proposition~\ref{prop:pre-image}}.
\emph{Let $L$ be a CERL defined by a CEFA $\CEFA = (Q, \Sigma, R, \delta, I, F)$. Then for each string function $f$ ranging over $\concat$, $\replaceall_{e,u}$, $\reverse$, FFTs $\NFT$, and $\substring$, $f^{-1}_R(L)$ is CERR-definable. In addition,
\begin{itemize}
\item a CEFA representation of $\concat^{-1}_R(L)$ can be computed in time $\bigO(|\CEFA|^2)$, 
%
\item a CEFA representation of $\reverse^{-1}_R(L)$ (resp. $\substring^{-1}_R(L)$) can be computed in time $\bigO(|\CEFA|)$,
%
\item a CEFA representation of  $(\Tran(\NFT))^{-1}_R(L)$ can be computed in time polynomial in $|\CEFA|$ and exponential in $|\NFT|$,
%
\item a CEFA representation of  $(\replaceall_{e,u})^{-1}_R(L)$ can be computed in time polynomial in $|\CEFA|$ and exponential in $|e|$ and $|u|$.
\end{itemize}
}

\begin{proof}
	Let $\CEFA=(Q, \Sigma, R, \delta, I, F)$ be a CEFA with $R= (r_1, \cdots, r_k)$. We show how to construct a CEFA representation of $f^{-1}_R(L)$ for each function $f$ in {\slint}.
	
	%%%%%%%%%%%%%%%%%%%%%%%%%%%%%%%%%%%%%%%%%%%%%%%%%%%%%%%%%%%%%%%%%%%%%%%%%%%%%%
	\paragraph*{$\concat^{-1}_R(L)$.}
	%
	A CEFA representation of $\concat^{-1}_R(L)$ is given by $((\CEFA_{I, q}, \NFA_{q, F})_{q \in Q}, \vec{t})$, where 
	\begin{itemize}
		\item $\CEFA_{I, q}=(Q, \Sigma, R^{(1)}, \delta^{(1)}, I, \{q\})$ and  $\CEFA_{q, F}=(Q, \Sigma, R^{(2)}, \delta^{(2)}, \{q\}, F)$ such that 
		\begin{itemize}
			\item $R^{(1)} = (r^{(1)}_1, \cdots, r^{(1)}_k)$, $R^{(2)} = (r^{(2)}_1, \cdots, r^{(2)}_k)$, 
			\item $\delta^{(1)}$ comprises the tuples $(q, a, q', \eta')$ satisfying that there exists $\eta$ such that $(q, a, q', \eta) \in \delta$ and for each $j \in [k]$, and $\eta'(r^{(1)}_j)=\eta(r_j)$,  similarly for $\delta^{(2)}$,
		\end{itemize}
		\item and $\vec{t} = (r^{(1)}_1 + r^{(2)}_1, \cdots, r^{(1)}_k + r^{(2)}_k)$.
	\end{itemize}
	Note that the size of $((\CEFA_{I, q}, \NFA_{q, F})_{q \in Q}, \vec{t})$ is $\bigO(|\CEFA|^2)$.
	%
	%%%%%%%%%%%%%%%%%%%%%%%%%%%%%%%%%%%%%%%%%%%%%%%%%%%%%%%%%%%%%%%%%%%%%%%%%%%%%%
	%
	\paragraph*{$\reverse^{-1}_R(L)$.} 
	%
	A CEFA representation of $\reverse^{-1}_R(L)$ is given by $(\CEFA^{(r)}, \vec{t})$, where 
	\begin{itemize}
		\item $\CEFA^{(r)}=(Q, \Sigma, R^{(1)}, \delta', F, I)$ such that 
		\begin{itemize}
			\item $R^{(1)}=(r^{(1)}_1,\cdots,r^{(1)}_k)$, and 
			\item $\delta'$ comprises the tuples $(q', a, q, \eta')$ satisfying that there exists $\eta$ such that $(q, a, q', \eta) \in \delta$, and $\eta'(r^{(1)}_i) = \eta(r_i)$ for each $i \in [k]$,
		\end{itemize}
		%
		\item and $\vec{t}=(r^{(1)}_1, \cdots, r^{(1)}_k)$. 
	\end{itemize}
	Note that $\Lang(\CEFA^{(r)}) = \{(w^{(r)}, \vec{n}) \mid (w, \vec{n}) \in \Lang(\CEFA)\}$, and the size of $(\CEFA^{(r)}, \vec{t})$ is $\bigO(|\CEFA|)$.
	
	%%%%%%%%%%%%%%%%%%%%%%%%%%%%%%%%%%%%%%%%%%%%%%%%%%%%%%%%%%%%%%%%%%%%%%%%%%%%%%
	\paragraph*{$\substring^{-1}_R(L)$.}
	A CEFA representation of $\substring^{-1}_R(L)$ is given by $(\cB, \vec{t})$, where 
	\begin{itemize}
		\item $\cB = (Q', \Sigma, R', \delta', I', F')$ such that 
		\begin{itemize}
			\item $Q' = Q \times \{p_0, p_1, p_2\}$, (intuitively, $p_0$, $p_1$, and $p_2$ denote that the current position is before the starting position, between the starting position and ending position, and after the ending position respectively)
			%
			\item $R' = \left(r'_{1,1}, r'_{1,2}, r^{(1)}_1,\cdots, r^{(1)}_k \right)$, (intuitively, $r'_{1,1}$ denotes the starting position, and $r'_{1,2}$ denotes the length of the substring)
			%
			\item $I'=I \times \{p_0\}$, $F'=F' \times \{p_2\} \cup (I \cap F) \times \{p_0\}$,
			%
			\item and $\delta'$ comprises 
			\begin{itemize}
				\item the tuples $((q, p_0), a, (q, p_0), \eta')$ such that $q \in I$, $a \in \Sigma$, and $\eta'$ satisfies that $\eta'(r'_{1,1})= 1$, and $\eta'(r'_{1,2}) = 0$, and $\eta'(r^{(1)}_j)=0$ for each $j \in [k]$, 
				%
				\item the tuples $((q, p_0), a, (q', p_1), \eta')$ such that $q \in I$ and there exists $\eta$ satisfying that $(q, a, q', \eta) \in \delta$, moreover, $\eta'(r'_{1,1})=0$ (recall that the positions of strings start at $0$), $\eta'(r'_{1,2}) = 1$, and $\eta'(r^{(1)}_j)=\eta(r_j)$ for each $j \in [k]$,
				%
				\item the tuples $((q, p_0), a, (q', p_2), \eta')$ such that $q \in I$ and there exists $\eta$ satisfying that $(q, a, q', \eta) \in \delta$, moreover, $q' \in F$, and $\eta'(r'_{1,1})=0$ (recall that the positions of strings start at $0$), $\eta'(r'_{1,2}) = 1$, and $\eta'(r^{(1)}_j)=\eta(r_j)$ for each $j \in [k]$,  
				%
				\item the tuples $((q, p_1), a, (q', p_1), \eta')$ such that there exists $\eta$ satisfying that $(q, a, q', \eta) \in \delta$, $\eta'(r'_{1,1}) = 0$, and $\eta'(r'_{1,2}) = 1$, and $\eta'(r^{(1)}_j)=\eta(r_j)$ for each $j \in [k]$,
				%
				\item the tuples $((q, p_1), a, (q', p_2), \eta')$ such that $q' \in F$, and there exists $\eta$ satisfying that $(q, a, q', \eta) \in \delta$, moreover, $\eta'(r'_{1,1}) = 0$, $\eta'(r'_{1,2}) = 1$, and $\eta'(r^{(1)}_j)=\eta(r_j)$ for each $j \in [k]$,
				%
				%\item the tuples $((q, p_1), a, (q, p_2), \eta')$ such that $q \in F$ and $\eta'(r^{(1)}_j)=0$ for each $j \in [k]$, $\eta'(r'_1) = 0$, and $\eta'(r'_2) = 1$,
				%
				\item the tuples $((q, p_2), a, (q, p_2), \eta')$ such that $q \in F$, $\eta'(r'_{1,1}) = 0$, and $\eta'(r'_{1,2}) = 0$, and $\eta'(r^{(1)}_j)=0$ for each $j \in [k]$,
				%
			\end{itemize}
		\end{itemize}
		\item $\vec{t}=(r^{(1)}_1, \cdots, r^{(1)}_k)$.
	\end{itemize}
	Note that the size of $(\cB, \vec{t})$ is $\bigO(|\CEFA|)$.
\zhilin{substring cleaned, the other operations to be cleaned}	
	%%%%%%%%%%%%%%%%%%%%%%%%%%%%%%%%%%%%%%%%%%%%%%%%%%%%%%%%%%%%%%%%%%%%%%%%%%%%%%
	%
	%
	\paragraph*{$(\Tran(\NFT))^{-1}_R(L)$.}
	%
	Suppose $\NFT = (Q', \Sigma, \delta', I', F')$. Then a CEFA representation of $(\Tran(\NFT))^{-1}_R(L)$ is given by 
	$(\cB, \vec{t})$, where 
	\begin{itemize}
		\item $\cB$ simulates the run of $\NFT$ on the input string, meanwhile, it simulates the run of $\CEFA$ on the output string of $\NFT$, formally, $\cB= (Q' \times Q, \Sigma, R^{(1)}, \delta'', I' \times I, F' \times F)$ such that 
		\begin{itemize}
			\item $R^{(1)}  = (r^{(1)}_1, \cdots, r^{(1)}_k)$, and
			\item $\delta''$ comprises the tuples $((q'_1, q_1), a, (q'_2, q_2), \eta')$ satisfying one of the following conditions,
			\begin{itemize}
				\item there exist $u = a_1 \cdots a_n \in \Sigma^+$ and a transition sequence $p_0 \xrightarrow[\delta]{a_1, \eta_1} p_2 \cdots p_{n-1} \xrightarrow[\delta]{a_n, \eta_n} p_{n}$ in $\CEFA$ such that $(q'_1, a, q'_2, u) \in \delta'$, $p_0 = q_1$, $p_{n}= q_2$, and for each $j \in [k]$,  $\eta'(r^{(1)}_j) = \eta_1(r_j) + \cdots + \eta_n(r_j)$,
				%
				\item $(q'_1, a, q'_2, \varepsilon) \in \delta'$, $q_1 = q_2$, and $\eta'(r^{(1)}_j) =0$ for each $j \in [k]$,
			\end{itemize}
		\end{itemize}
		%
		\item $\vec{t}=(r^{(1)}_1, \cdots, r^{(1)}_k)$.
	\end{itemize}
	Note that the number of transitions of $\cB$ can be exponential in the worst case, since it summarises the updates of cost registers of $\CEFA$ on the output strings of the transitions of $\NFT$. More precisely,  let
	\begin{itemize}
		\item $\ell$ be the maximum length of the output strings of transitions of $\NFT$, 
		\item $N$ be the maximum number of transitions between a given pair of states of $\CEFA$, and
		\item  $C$ be the maximum absolute value of the integer constants occurring in $\CEFA$,
	\end{itemize}
	then $|\delta''|$, the cardinality of $\delta''$, is bounded by $|\delta'| \times |Q|^2 \times N^\ell $, and the integer constants occurring in each transition of $\delta''$ are bounded by $\ell C$. Therefore, 
	the size of $(\cB, \vec{t})$ is 
	\[
	\bigO(|\delta'| \times |Q|^2 \times N^\ell \times k \log_2 (\ell C)).
	\] 
	Since $|\delta'|, \ell \le |\NFT|$, $|Q|, N, k \le |\CEFA|$, and $C \le 2^{|\CEFA|}$, we deduce that the size of $(\cB, \vec{t})$ is 
	$
	\bigO( |\NFT| \times  |\CEFA|^2 \times |\CEFA|^{|\NFT|} \times |\CEFA|^2 \log_2(|\NFT|))= |\CEFA|^{\bigO(|\NFT|)} |\NFT| \log_2(|\NFT|).$
	%
	
	%%%%%%%%%%%%%%%%%%%%%%%%%%%%%%%%%%%%%%%%%%%%%%%%%%%%%%%%%%%%%%%%%%%%%%%%%%%%%%
	\paragraph*{$(\replaceall_{e,u})^{-1}_R(L)$.}
	%
	From the result in \cite{CCH+18}, we know that  a NFT $\NFT_{e,u}=(Q', \Sigma, \delta', I', F')$ can be constructed to capture $\replaceall_{e,u}$.  Moreover, 
	\begin{itemize}
		\item $|Q'|$, as well as $|\delta'|$, is $2^{\bigO(|e|)}$,
		\item $\ell$, the maximum length of the output strings of transitions of $\NFT_{e,u}$, is $|u|$.
	\end{itemize}
	Then a CEFA representation of $(\replaceall_{e,u})^{-1}_R(L)$ can be constructed as that of $(\Tran(\NFT_{e,u}))^{-1}_R(L)$.
	Let $N$ denote the maximum number of transitions between a given pair of states of $\CEFA$, and $C$ be the maximum absolute value of the integer constants occurring in $\CEFA$, which is bounded by $2^{|\CEFA|}$. Then the CEFA representation of $(\replaceall_{e,u})^{-1}_R(L)$ is of size 
	\[
	\bigO(|\delta'| \times |Q|^2 \times N^\ell \times k \log_2 (\ell C)) = 2^{\bigO(|e|)} |\CEFA|^2 |\CEFA|^{|u|} |\CEFA|^2 \log_2 |u|=2^{\bigO(|e|)} |\CEFA|^{\bigO(|u|)}.
	\]
	%
	according to the aforementioned discussion for NFTs.
	% 
	%
\end{proof}

%%%%%%%%%%%%%%%%%%%%%%%%%%%%%%%%%%%%%%%%%%%%%%%%%%%%%%%%%%%%%%%%%%%%%%%%%%%%%%%%%%%%%%%%%%%%%%%%%%%%%%%
\section{Proof of Proposition~\ref{prop:la-sat-cefa-inter}}\label{app:sat-cefa}

\noindent{\bf Proposition~\ref{prop:la-sat-cefa-inter}}.
\emph{The {\lasat} problem is $\pspace$-complete.}

\begin{proof}
	The lower bound follows from the {\pspace}-hardness of the intersection problem of NFAs. 
	
	For the upper bound, let $\{ \CEFA_i^{j} \}_{i\in I,j\in J_i}$ be a family of CEFAs  each of which carries a vector of registers $R_i^j$ and  $\phi$ be a quantifier-free LIA formula such that  $ R_i^{j} $ are pairwise disjoint and the variables of $\phi$ are from $R'=\bigcup_{i,j} R_i^j$. 
	%Deciding whether  there are an assignment function $\theta: R' \rightarrow \Int$ and strings $(w_i)_{i \in I}$ such that  $\phi[\theta(R' )/R']$ holds and $(w_i, \theta(R_i^j)) \in \Lang(\CEFA_{i}^j)$ for every $i \in I$ and $j \in J_i$ is $\pspace$-complete. 
	
	At first, we observe that we can focus on \emph{monotonic CEFAs} where the cost registers are monotone in the sense that their values are non-decreasing, in other words, they can only be updated with natural number constants. This observation is justified by the following reduction.
	
	For each register $r \in R^i_j$, introduce two registers $r^+, r^-$. Let $(R^i_j)^{\pm}$ denote the vector of registers by replacing each $r \in R^i_j$ with $(r^+, r^-)$. Intuitively,  for each $r \in R^i_j$, the updates of $r$ in $\CEFA_i^{j} $ are split into the non-negative ones and negative ones, with the former stored in $r^+$ and the latter in $r^-$. Suppose $(R')^{\pm} = \bigcup_{i,j} (R_i^j)^{\pm}$. Then we construct monotonic CEFAs $(\cB_i^{j})_{i \in I, j \in J_i}$ and an LIA formula $\phi^\pm$ such that
	\begin{quote}
		there are an assignment function $\theta: R' \rightarrow \Int$ and strings $(w_i)_{i \in I}$ such that  $\phi[\theta(R' )/R']$ holds and $(w_i, \theta(R_i^j)) \in \Lang(\CEFA_{i}^j)$ for every $i \in I$ and $j \in J_i$ 
		\begin{center} if and only if \end{center}
		there are an assignment function $\theta^\pm: (R')^\pm \rightarrow \Nat$ and strings $(w_i)_{i \in I}$ such that  $\phi^\pm[\theta^\pm((R')^\pm)/(R')^\pm]$ holds and $(w_i, \theta^\pm((R_i^j)^\pm)) \in \Lang(\cB_{i}^j)$ for every $i \in I$ and $j \in J_i$.
	\end{quote}
	For $i \in I$ and $j \in J_i$, the CEFA $\cB_{i}^j$ is obtained from $\CEFA_{i}^j$ by replacing each transition $(q, a, q', \eta)$ in $\CEFA_i^j$ by the transition $(q, a, q', \eta')$ such that for each $r \in R_j^j$, 
	\[
	\eta'(r^+) = \left\{ \begin{array}{l  l}\eta(r), & \mbox{ if } \eta(r) \ge 0 \\ 0 & \mbox{ otherwise} \end{array}\right.,  \eta'(r^-) = \left\{ \begin{array}{l  l} 0, & \mbox{ if } \eta(r) \ge 0 \\ -\eta(r) & \mbox{ otherwise} \end{array}\right..
	\]
	In addition, $\phi^\pm$ is obtained from $\phi$ by replacing each $r \in R'$ with $r^+-r^-$.
	
	\smallskip
	
	It remains to prove the {\lasat} problem for monotonic CEFAs is in {\pspace}, namely,
	\begin{quote}
		given a family of monotonic CEFAs $\{ \CEFA_i^{j} \}_{i\in I,j\in J_i}$ each of which carries a vector of registers $R_i^j$ and a quantifier-free LIA formula $\phi$ such that  $ R_i^{j} $ are pairwise disjoint,  and the variables of $\phi$ are from $R'=\bigcup_{i,j} R_i^j$, deciding whether  there are an assignment function $\theta: R' \rightarrow \Nat$ and strings $(w_i)_{i \in I}$ such that  $\phi[\theta(R' )/R']$ holds and $(w_i, \theta(R_i^j)) \in \Lang(\CEFA_{i}^j)$ for every $i \in I$ and $j \in J_i$ is in {\pspace}.
	\end{quote}
	
	
	We use Proposition 16 in \cite{LB16} to show the result. Proposition 16 in \cite{LB16} is about monotonic counter machines, which can be seen as monotonic CEFAs where each transition contain no alphabet symbol, moreover, $\eta(r) \in \{0,1\}$ for the update function $\eta$ therein.
	
	For each $i\in I$ and $j\in J_i$, let $(\CEFA')_i^j$ be the monotonic counter machine obtained from $\CEFA_i^{j}$ by the following two-step procedure:
	\begin{enumerate}
		\item {[Remove the alphabet symbols]}: Remove alphabet symbols $a$ in each transition $(q, a, q', \eta)$ of $\CEFA_i^{j}$.
		%
		\item {[From binary encoding to unary encoding]}: Replace each transition $(q, q', \eta)$ such that $\ell = \max_{r \in R_i^j} \eta(r) > 1$ with a sequence of transitions $(q, p_1,\eta'_1), \cdots, (p_{\ell-1}, p_\ell, \eta'_\ell)$, where $p_1, \cdots,p_r$ are the freshly introduced states, moreover, $\eta'_j(r) = 1$ if $\eta(r) \ge j$, and $\eta'_j(r)=0$ otherwise. 
	\end{enumerate}
	
	According to Proposition 16 in \cite{LB16}, we have the following, 
	\begin{quote}
		Given a family of monotonic counter machines $\{ \cC_i \}_{i\in I}$ each of which carries a vector of counters $R_i$ and a quantifier-free LIA formula $\phi$ such that $ R_i$ are pairwise disjoint,  and the variables of $\phi$ are from $R'=\bigcup_{i} R_i$. If there are an assignment function $\theta: R' \rightarrow \Nat$ and strings $(w_i)_{i \in I}$ such that  $\phi[\theta(R' )/R']$ holds and $(w_i, \theta(R_i)) \in \Lang(\cC_{i})$ for every $i \in I$, then there are desired $\theta$ and $(w_i)_{i \in I}$ such that for each $i \in I$ and $r \in R_i$, $\theta(r)$ is at most polynomial in the number of states in $\cC_i $, exponential in $|R_i|$, and exponential in $|\phi|$.
	\end{quote}
	For each $i \in I$, let $\cC_i$ be the product of monotonic counter machines $(\CEFA')_i^j$ for $j \in J_i$. 
	From the fact that the number of states of $(\CEFA')_i^j$ is at most the product of the number of transitions of $\CEFA_i^j$ and $B_{\CEFA_i^j}$ (where $B_{\CEFA_i^j}$ denotes the maximum natural number constants $\eta(r)$ in $\CEFA_i^j$), we deduce the following,
	\begin{quote}
		if there are an assignment function $\theta: R' \rightarrow \Nat$ and strings $(w_i)_{i \in I}$ such that  $\phi[\theta(R' )/R']$ holds and $(w_i, \theta(R_i^j)) \in \Lang(\CEFA_i^j)$ for every $i \in I$ and $j \in J_i$, then there are desired $\theta$ and $(w_i)_{i \in I}$ such that for each $r \in \bigcup_{j \in J_i} R^j_i$, $\theta(r)$ is at most polynomial in the product of the number of transitions in $\CEFA_i^j$ and $B_{\CEFA_i^j}$ for $j \in J_i$, exponential in $\left|\bigcup_{j \in J_i} R^j_i \right|$, and exponential in $|\phi|$.
	\end{quote}
	
	Therefore, one can nondeterministically guess the strings $(w_i)_{i \in I}$, and for each $i \in I$ and $j \in J_i$, simulate the runs of CEFAs $\CEFA_i^j$ on $w_i$, in polynomial space, since  the values of all the registers can be assumed to be at most exponential, thus their binary encodings can be stored in polynomial space. From Savitch's theorem \cite{complexity-book}, we conclude that the {\lasat} problem for monotonic CEFAs is in {\pspace}.
\end{proof}

%%%%%%%%%%%%%%%%%%%%%%%%%%%%%%%%%%%%%%%%%%%%%%%%

\section{Example of {\tt urlXssSanitise(url)} for the decision procedure} \label{app:urlexample}

%\begin{example}
	Consider the program $S$ associated with {\tt urlXssSanitise(url)} in Section~\ref{sec:intro}. % Example~\ref{exmp:running}.
	%\[
	%\begin{array}{l}
	%y := \replaceall_{\Sigma \setminus ), \varepsilon}(x); z:= \replaceall_{\Sigma \setminus (, \varepsilon}(x);\\
	%\ASSERT{\length(y) = \length(z)}; \ASSERT{\indexof_{(}(x,0) < \indexof_{)}(x,0)}.
	%\end{array}
	%\] 
	We show how to decide its path feasibility. % of $S$. 
	\begin{description}
		\item[Step I.]   Vacuous since $S$ contains only atomic assertions already. %neither disjunction nor conjunction.
		%
		\item[Step II.] Nondeterministically choose to replace $\indexof_\#(\mathtt{url1}, 0)$ with $-1$ and add $\ASSERT{\mathtt{url1} \in \NFA_{\overline{\Sigma^*\#\Sigma^*}}}$ to $S$.  
		%
		\item[Step III.] Replace $\length(\mathtt{url1})$ with $i'_1$ and add $\ASSERT{\mathtt{url1} \in \CEFA_{\rm len}[i'_1/r_1]}$ to $S$, moreover, replace $\indexof_?(\mathtt{url1}, 0)$ with $i'_3$ and add $\ASSERT{0= i'_2}; \ASSERT{\mathtt{url1} \in \CEFA_{\indexof}[i'_2/r_1, i'_3/r_2]}$ to $S$, where $i'_1, i'_2, i'_3$ are fresh integer variables. Then we get the following program (still denoted by $S$), 
		\[ 
		\begin{array}{l}
		\ASSERT{\mathtt{prothostpath} \in \NFA_\varepsilon}; \ASSERT{\mathtt{querfrag} \in \NFA_\varepsilon}; \mathtt{url1} := \NFT_{\rm trim}(\mathtt{url}); \\
		\ASSERT{\mathtt{qmarkpos} = i'_3}; \ASSERT{\mathtt{sharppos} =-1 }; \ASSERT{\mathtt{qmarkpos} \ge 0}; \\ 
		\mathtt{prothostpath1} := \substring(\mathtt{url1}, 0, \mathtt{qmarkpos});\\
		\mathtt{querfrag1} := \substring(\mathtt{url1, qmarkpos}, i'_1 - \mathtt{qmarkpos});\\
		\mathtt{querfrag2} := \replaceall_{\mathtt{script},\ \varepsilon}(\mathtt{querfrag1});\\
		\mathtt{url2} := \mathtt{prothostpath1} \concat \mathtt{querfrag2}; \ASSERT{\mathtt{querfrag2} \in  \NFA_{\Sigma^*\mathtt{script}\Sigma^*}};  \\
		\ASSERT{\mathtt{url1} \in  \NFA_{\overline{\Sigma^*\#\Sigma^*}}}; \ASSERT{\mathtt{url1} \in \CEFA_{\rm len}[i'_1/r_1]}; \\
		\ASSERT{0= i'_2}; \ASSERT{\mathtt{url1} \in \CEFA_{\indexof}[i'_2/r_1, i'_3/r_2]}.
		\end{array}
		\]
		%
		\item[Step IV.] Since there is no CEFA constraints for $\mathtt{url2}$, removing the last assignment statement of $S$, i.e. $\mathtt{url2} := \mathtt{prothostpath1} \concat \mathtt{querfrag2}$, is easy and in this case we add no statements to $S$. After this, $\mathtt{querfrag2} := \replaceall_{\mathtt{script},\ \varepsilon}(\mathtt{querfrag1})$ becomes the last assignment statement. Suppose $\NFA'=(Q', \Sigma, \delta', I', F')$ is an NFA representing $(\replaceall_{\mathtt{script},\ \varepsilon})^{-1}_\emptyset(\Lang(\NFA_{\Sigma^*\mathtt{script}\Sigma^*}))$, namely, the pre-image of $\Lang(\NFA_{\Sigma^*\mathtt{script}\Sigma^*})$ under $\replaceall_{\mathtt{script},\ \varepsilon}$. Then we remove this assignment statement and add $\ASSERT{\mathtt{querfrag1 \in \NFA'}}$, resulting into the following program
		\[ 
		\begin{array}{l}
		\ASSERT{\mathtt{prothostpath} \in \NFA_\varepsilon}; \ASSERT{\mathtt{querfrag} \in \NFA_\varepsilon}; \mathtt{url1} := \NFT_{\rm trim}(\mathtt{url}); \\
		\ASSERT{\mathtt{qmarkpos} = i'_3}; \ASSERT{\mathtt{sharppos} =-1 }; \ASSERT{\mathtt{qmarkpos} \ge 0}; \\ 
		\mathtt{prothostpath1} := \substring(\mathtt{url1}, 0, \mathtt{qmarkpos});\\
		\mathtt{querfrag1} := \substring(\mathtt{url1, qmarkpos}, i'_1 - \mathtt{qmarkpos});\\
		%    \mathtt{querfrag2} := \replaceall_{\mathtt{script},\ \varepsilon}(\mathtt{querfrag1});\\
		%    \mathtt{url2} := \mathtt{prothostpath1} \concat \mathtt{querfrag2}; 
		\ASSERT{\mathtt{querfrag2} \in  \NFA_{\Sigma^*\mathtt{script}\Sigma^*}};  
		\ASSERT{\mathtt{url1} \in  \NFA_{\overline{\Sigma^*\#\Sigma^*}}}; \\
		\ASSERT{\mathtt{url1} \in \CEFA_{\rm len}[i'_1/r_1]};  \ASSERT{0= i'_2}; \\
		\ASSERT{\mathtt{url1} \in \CEFA_{\indexof}[i'_2/r_1, i'_3/r_2]};  \ASSERT{\mathtt{querfrag1} \in \NFA'}.
		\end{array}
		\]
		
		From Example~\ref{exm:pre-image}, we know that $\substring^{-1}_\emptyset(\Lang(\NFA'))$ can be represented by some CEFA $\cB=(Q'', R'', \delta'', I'', F'')$ with $Q''= Q' \times \{p_0,p_1,p_2\}$ and $R''=(r'_{1,1}, r'_{1,2})$ (where $r'_{1,1}$ and $r'_{1,2}$ are fresh integer variables). Then we remove $\mathtt{querfrag1} := \substring(\mathtt{url1, qmarkpos}, i'_1 - \mathtt{qmarkpos})$, add $\ASSERT{\mathtt{url1} \in \cB};\ASSERT{\mathtt{r'_{1,1}= qmarkpos}}; \ASSERT{r'_{1,2}=i'_1 - \mathtt{qmarkpos}}$, and get the following program
		\[ 
		\begin{array}{l}
		\ASSERT{\mathtt{prothostpath} \in \NFA_\varepsilon}; \ASSERT{\mathtt{querfrag} \in \NFA_\varepsilon}; \mathtt{url1} := \NFT_{\rm trim}(\mathtt{url}); \\
		\ASSERT{\mathtt{qmarkpos} = i'_3}; \ASSERT{\mathtt{sharppos} =-1 }; \ASSERT{\mathtt{qmarkpos} \ge 0}; \\ 
		\mathtt{prothostpath1} := \substring(\mathtt{url1}, 0, \mathtt{qmarkpos});\\
		%   \mathtt{querfrag1} := \substring(\mathtt{url1, qmarkpos}, i'_1 - \mathtt{qmarkpos});\\
		%    \mathtt{querfrag2} := \replaceall_{\mathtt{script},\ \varepsilon}(\mathtt{querfrag1});\\
		%    \mathtt{url2} := \mathtt{prothostpath1} \concat \mathtt{querfrag2}; 
		\ASSERT{\mathtt{querfrag2} \in  \NFA_{\Sigma^*\mathtt{script}\Sigma^*}};  
		\ASSERT{\mathtt{url1} \in  \NFA_{\overline{\Sigma^*\#\Sigma^*}}}; \\
		\ASSERT{\mathtt{url1} \in \CEFA_{\rm len}[i'_1/r_1]};  \ASSERT{0= i'_2}; \\
		\ASSERT{\mathtt{url1} \in \CEFA_{\indexof}[i'_2/r_1, i'_3/r_2]};  \ASSERT{\mathtt{querfrag1} \in \NFA'};\\
		\ASSERT{\mathtt{url1} \in \cB};\ASSERT{\mathtt{r'_{1,1} = qmarkpos} }; \ASSERT{r'_{1,2}=i'_1 - \mathtt{qmarkpos}}.
		\end{array}
		\]
		We can continue the process until the problem contains no assignment statement.
		%
		\item[Step V.]  Straightforward by utilising Proposition~\ref{prop:la-sat-cefa-inter}. 
	\end{description}
%\end{example}








%%%%%%%%%%%%%%%%%%%%%%%%%%%%%%
\section{Algorithms for case splits in the semantics of $\indexof_v$ and $\substring$}\label{app:case-split-semantics}

\begin{algorithm}[htbp]
  \small
  \KwIn{$active$: set of CEFA constraints,  $arith$: arithmetic constraint,
    $\mathit{funApps}$: acyclic set of assignment statements, and $(\mathcal{I}_l)_{l \in [5]}$: subsets of $\indexof_v(x,i)$ string terms}
  \KwResult{$(active, arith, \mathit{funApps})$\newline}
  
\For{each $\indexof_v(x, i) \in \mathcal{I}_1$}
		{
			$arith \leftarrow arith[\indexof_v(x, 0)/\indexof_v(x,i)] \wedge i < 0$\;
		}
		\For{each $\indexof_v(x, i) \in \mathcal{I}_2$}
		{
			$active \leftarrow active \cup \{x \in \NFA_{\overline{\Sigma^* v \Sigma^*}}\}$\;
			$arith \leftarrow arith[-1/\indexof_v(x,i)] \wedge i < 0$\;
		}
		\For{each $\indexof_v(x, i) \in \mathcal{I}_3$}
		{
			$arith \leftarrow arith[-1/\indexof_v(x,i)] \wedge i \ge \length(x)$\;
		}
		\For{each $\indexof_v(x, i) \in \mathcal{I}_4$}
		{
			$arith \leftarrow arith[-1/\indexof_v(x,i)] \wedge i \ge 0 \wedge i < \length(x)$\;
		}
		\For{each $\indexof_v(x, i) \in \mathcal{I}_5$}
		{
			choose fresh variables $y$ and $j$\;
			$active \leftarrow active \cup \{y \in \NFA_{\overline{\Sigma^* v \Sigma^*}}\}$\;
			$arith \leftarrow arith[-1/\indexof_v(x,i)] \wedge i \ge 0 \wedge i < \length(x) \wedge j = \length(x)-i$\;
			 $\mathit{funApps} \leftarrow \mathit{funApps} \cup \{y:=\substring(x, i, j)\}$\;
		}		
\caption{$\mathit{indexofCaseSplit}$ for case splits in the semantics of $\indexof_v$}\label{alg:indexof}
\end{algorithm}

\begin{algorithm}[htbp]
  \small
  \KwIn{$active$: set of CEFA constraints,  $arith$: arithmetic constraint,
    $\mathit{funApps}$: acyclic set of assignment statements, and $(\mathcal{I}_l)_{l \in [5]}$: subsets of $\indexof_v(x,i)$ string terms}
  \KwResult{$(active, arith, \mathit{funApps})$\newline}

  		\For{each $y:=\substring(x, i, j) \in \mathcal{J}_1$}
		{
			 $arith \leftarrow arith \wedge i \ge 0 \wedge i+j \le \length(x)$;
		}
		\For{each $y:=\substring(x, i, j) \in \mathcal{J}_2$}
		{
			 choose a fresh integer variable $i'$\;
			 $arith \leftarrow arith \wedge i \ge 0 \wedge i \le \length(x) \wedge i+j > \length(x) \wedge i' = \length(x)-i$\;
			 $\mathit{funApps} \leftarrow \mathit{funApps}[y:=\substring(x, i, i')/y:=\substring(x, i, j)]$\;
		}
		\For{each $y:=\substring(x, i, j) \in \mathcal{J}_3$}
		{
			 $arith \leftarrow arith \wedge i < 0$\;
			 $active \leftarrow active \cup \{y \in \NFA_\varepsilon\}$\;
			 $\mathit{funApps} \leftarrow \mathit{funApps} \setminus \{y:=\substring(x, i, j)\}$\;		 
		}
\caption{$\mathit{substringCaseSplit}$  for case splits in the semantics of $\substring$}\label{alg:substring}
\end{algorithm}





%%%%%% The proof of Theorem~\ref{thm-la-sat-cefa} is removed%%%%%%%%%%
%%%%%% The proof of Theorem~\ref{thm-la-sat-cefa} is removed%%%%%%%%%%
%%%%%% The proof of Theorem~\ref{thm-la-sat-cefa} is removed%%%%%%%%%%
%%%%%% The proof of Theorem~\ref{thm-la-sat-cefa} is removed%%%%%%%%%%
\hide{
\section{Proof of Theorem~\ref{thm-la-sat-cefa}} \label{appendix:thm-la-sat-cefa}

For a $k$-cost-enriched language $L$, we define 
\[
\prjnum(L) = \left\{(n_1, \cdots, n_k) \in \Int^k \mid \mbox{ there exist } w \in \Sigma^*.\ (w,(n_1,\cdots,n_k)) \in L \right\}.
\]

\begin{lemma}\label{lem-cefa-la}
	Let $\CEFA=(Q, \Sigma, R, \delta, I, F)$ be a CEFA with $R= (r_1, \cdots,  r_k)$. Then an existential LIA formula $\phi_\CEFA(r_1, \cdots, r_k)$ such that $\cM(\phi_\CEFA)= \prjnum(\Lang(\CEFA))$ can be computed in linear time from $\CEFA$.
\end{lemma}

\begin{proof}
	Suppose $\delta = \{\tau_1, \cdots, \tau_l\}$ such that $\tau_j = (q_j, a_j, q'_j, \eta_j)$ and $\eta_j(r_i) =  c_{j,i}$ for every $j \in [l]$ and $i \in [k]$.
	
	From the results on NFAs (Theorem~1 in \cite{SSMH04}), we know that for each pair of states $(q, q') \in I \times F$,  an existential LIA formula $\phi_{q,q'}(m_1, \cdots, m_l)$ can be computed in linear time such that $\cM(\phi_{q,q'})$ is the set of Parikh images of the runs of $\NFA$ starting from $q$ and ending at $q'$, where the variables $m_1, \cdots, m_l$ represent the numbers of occurrences of $\tau_1,\cdots, \tau_l$ respectively in the run. 
	
	Then the desired existential LIA formula $\phi_\NFA$ is constructed as follows,
	\[\phi_\NFA(r_1, \cdots, r_k) ::= \bigvee \limits_{(q,q') \in I \times F} \exists m_1 \cdots \exists m_l.\ \left(\varphi_{q,q'}(m_1, \cdots, m_l) \wedge \bigwedge \limits_{i \in [k]} r_i = \sum \limits_{j \in [l]} c_{j,i} m_j \right).\]
\end{proof}

We are ready to prove Theorem~\ref{thm-la-sat-cefa}.
\begin{proof}[Proof of Theorem~\ref{thm-la-sat-cefa}]
	The NP lower bound follows from the fact that the satisfiability problem of existential LIA formulas is NP-complete \cite{BT76,GS78} (see also \cite{Haase18}).
	
	For the upper bound, suppose that $\phi$ is a quantifier-free LIA formula and $\CEFA_1,\cdots,\CEFA_m$ are CEFAs such that 
	\begin{itemize}
		\item	$\CEFA_i=(Q_i, \Sigma, R_i, \delta_i, I_i, F_i)$  with $R_i = (r_{i, 1}, \cdots, r_{i, k_i})$, for every $i\in [m]$,
		\item $R_i \cap R_j = \emptyset$ for every $1 \le i < j \le m$, and
		\item the free variables of $\phi$ are from $\bigcup_{i\in [m]} R_i$.
	\end{itemize}
	From Lemma~\ref{lem-cefa-la}, for every $i \in [m]$, an existential LIA formula $\phi_{\CEFA_i}(r_{i,1}, \cdots, r_{i, k_i})$ such that $\cM(\phi_{\CEFA_i})= \prjnum(\Lang(\CEFA_i))$ can be computed in linear time from $\CEFA_i$.
	
	Then the satisfiability of $\phi$ w.r.t. $\CEFA_1,\cdots, \CEFA_m$ is reduced to the satisfiability problem of the  following existential LIA formula
	\[
	\phi' \equiv \phi \wedge \bigwedge \limits_{i \in [m]} \phi_{\CEFA_i}(r_{i,1}, \cdots, r_{i, k_i}).
	\]
	Since the size of $\phi'$ is linear in the size of $\phi$ and those of $\CEFA_1,\cdots,\CEFA_m$, and the satisfiability problem of existential LIA formulas is NP-complete, we conclude that the satisfiability of $\phi$ w.r.t.  $\CEFA_1,\cdots,\CEFA_m$ can be decided in nondeterministic polynomial time.
	
	It remains to prove the correctness of the reduction, namely, $\phi$ is satisfiable w.r.t. $\CEFA_1,\cdots, \CEFA_m$ iff $\phi'$ is satisfiable.
	
	\smallskip
	
	\noindent \emph{``Only if'' direction}. Suppose $\phi$ is satisfiable w.r.t. $\CEFA_1,\cdots, \CEFA_m$. Then there are an assignment function $\theta: \bigcup \limits_{i \in [m]} R_i \rightarrow \Int$ and strings $w_1, \cdots, w_m$  
	such that  $\phi[(\theta(R_i)/R_i)_{i \in [m]}]$ is evaluated to $true$ and $(w_i, \theta(R_i)) \in \Lang(\NFA_i)$ for every $i \in [m]$. For every $i \in [m]$, from $\cM(\phi_{\CEFA_i})=\prjnum(\Lang(\CEFA_i))$, we know that $\theta(R_i)$ satisfies $\phi_{\CEFA_i}$, namely, $\phi_{\CEFA_i}[\theta(R_i)/R_i]$ is evaluated to $true$. Therefore, the assignment $\theta$ makes $\phi'$ satisfied.
	
	\smallskip 
	
	\noindent \emph{``If'' direction}. Suppose $\phi'$ is satisfiable. Then there is an assignment $\theta: \bigcup \limits_{i \in [m]} R_i \rightarrow \Int$ such that $\phi[(\theta(R_i)/R_i)_{i \in [m]}]$, $\phi_{\CEFA_1}[\theta(R_1)/R_1]$, $\cdots$, and $\phi_{\CEFA_m}[\theta(R_m)/R_m]$ are all evaluated to $true$. For every $i \in [m]$, from $\cM(\phi_{\CEFA_i})=\prjnum(\Lang(\CEFA_i))$,  we know that there is a string $w_i$ such that $(w_i, \theta(R_i)) \in \Lang(\CEFA_i)$. From Definition~\ref{def-la-sat-cefa}, we conclude that $\phi$ is satisfiable w.r.t. $\CEFA_1,\cdots, \CEFA_m$.
\end{proof}
}
%%%%%% The proof of Theorem~\ref{thm-la-sat-cefa} is removed%%%%%%%%%%
%%%%%% The proof of Theorem~\ref{thm-la-sat-cefa} is removed%%%%%%%%%%


\end{appendix}

\fi

\end{document}
