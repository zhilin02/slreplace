\documentclass{llncs}


%% Some recommended packages.
\usepackage{booktabs}   %% For formal tables:
                        %% http://ctan.org/pkg/booktabs
%\usepackage{subcaption} %% For complex figures with subfigures/subcaptions
                        %% http://ctan.org/pkg/subcaption
\usepackage{latexsym}
\usepackage{setspace}
\usepackage{cancel}
\usepackage{listings}
\usepackage{graphicx}
\usepackage{appendix}
\usepackage{amssymb}
\usepackage{stmaryrd}
\usepackage{amsmath}
\usepackage{leftidx}
\usepackage{mathtools}
\usepackage{paralist}
\usepackage{color}
\usepackage{mathrsfs}
\usepackage{tikz}
%\usepackage[draft]{minted}
\usetikzlibrary{shapes}
\usepackage[linesnumbered,ruled]{algorithm2e}

%==========================================================
%!TEX root = main.tex

\newcommand{\brac}[1]{\left( #1 \right)}
\newcommand{\tup}[1]{\left( #1 \right)}
\newcommand{\set}[1]{\{ #1 \}}
\newcommand{\sequence}[2]{(#1, \ldots, #2)}
\newcommand{\couple}[2]{(#1,#2)}
\newcommand{\pair}[2]{(#1,#2)}
\newcommand{\triple}[3]{(#1,#2,#3)}
\newcommand{\quadruple}[4]{(#1,#2,#3,#4)}
\newcommand{\tuple}[2]{(#1,\ldots,#2)}
\newcommand{\Nat}{\ensuremath{\mathbb{N}}}
\newcommand{\Rat}{\ensuremath{\mathbb{Q}}}
\newcommand{\Rea}{\ensuremath{\mathbb{R}}}
\newcommand{\Zed}{\ensuremath{\mathbb{Z}}}
%\newcommand{\true}{\top}
%\newcommand{\false}{\perp}
\newcommand{\bottom}{\perp}
%% \newcommand{\powerset}[1]{{\cal P}(#1)}
\newcommand{\npowerset}[2]{{\cal P}^{#1}(#2)}
\newcommand{\finitepowerset}[1]{{\cal P}_f(#1)}
\newcommand{\level}[2]{L_{#1}(#2)}
\newcommand{\card}[1]{\mbox{card}(#1)}
\newcommand{\range}[1]{\mathtt{ran}(#1)}
\newcommand{\astring}{s}

\newcommand{\Cc}{\mathcal{C}}


\newcommand {\notof}{\ensuremath{\neg}}
\newcommand {\myand}{\ensuremath{\wedge}}
\newcommand {\myor}{\ensuremath{\vee}}
\newcommand {\mynext}{\mbox{{\sf X}}}
\newcommand {\until}{\mbox{{\sf U}}}
\newcommand {\sometimes}{\mbox{{\sf F}}}
\newcommand {\previous}{\mynext^{-1}}
\newcommand {\since}{\mbox{{\sf S}}}
\newcommand {\fminusone}{\mbox{{\sf F}}^{-1}}
\newcommand {\everywhere}[1]{\mbox{{\sf Everywhere}}(#1)}



\newcommand{\aatomic}{{\rm A}}
\newcommand{\aset}{X}
\newcommand{\asetbis}{Y}
\newcommand{\asetter}{Z}

\newcommand{\avarprop}{p}
\newcommand{\avarpropbis}{q}
\newcommand{\avarpropter}{r}
\newcommand{\varprop}{{\rm PROP}} % Set of atomic propositions (for a given logic)

% formulae

\newcommand{\aformula}{\astateformula} % a formula
\newcommand{\aformulabis}{\astateformulabis} % another formula (when at least 2 are present)
\newcommand{\aformulater}{\astateformulater} % another formula (when at least 3 are present)
\newcommand{\asetformulae}{X}
\newcommand{\subf}[1]{sub(#1)}

\newcommand{\aautomaton}{{\mathbb A}}
\newcommand{\aautomatonbis}{{\mathbb B}}

\newcommand {\length}[1] {\ensuremath{|#1|}}



% Equivalences
\newcommand{\egdef}{\stackrel{\mbox{\begin{tiny}def\end{tiny}}}{=}} % =def=
\newcommand{\eqdef}{\stackrel{\mbox{\begin{tiny}def\end{tiny}}}{=}} % =def=
\newcommand{\equivdef}{\stackrel{\mbox{\begin{tiny}def\end{tiny}}}{\equivaut}} % <=def=>
\newcommand{\equivaut}{\;\Leftrightarrow\;}

\newcommand{\ainfword}{\sigma}

\newcommand{\amap}{\mathfrak{f}}
\newcommand{\amapbis}{\mathfrak{g}}

\newcommand{\step}[1]{\xrightarrow{\!\!#1\!\!}}
\newcommand{\backstep}[1]{\xleftarrow{\!\!#1\!\!}}

\newcommand {\aedge}[1] {\ensuremath{\stackrel{#1}{\longrightarrow}}}
\newcommand {\aedgeprime}[1] {\ensuremath{\stackrel{#1}{\longrightarrow'}}}
\newcommand {\afrac}[1] {\ensuremath{\mathit{frac}(#1)}}
\newcommand {\cl}[1] {\ensuremath{\mathit{cl}(#1)}}
\newcommand {\sfc}[1] {\ensuremath{\mathit{sfc}(#1)}}
\newcommand {\dunion} {\ensuremath{\uplus}}
\newcommand {\edge} {\ensuremath{\longrightarrow}}
\newcommand {\emptyword}{\ensuremath{\epsilon}}
\newcommand {\floor}[1] {\ensuremath{\lfloor #1 \rfloor}}
\newcommand {\intersection} {\ensuremath{\cap}}
\newcommand {\union} {\ensuremath{\cup}}
\newcommand {\vals}[2] {\ensuremath{\mathit{val}_{#2}(#1)}}



\newcommand {\pspace} {\textsc{pspace}}
\newcommand {\nlogspace} {\textsc{nlogspace}}
\newcommand {\logspace} {\textsc{logspace}}
\newcommand {\expspace} {\textsc{expspace}}
\newcommand {\np} {\textsc{np}}
\newcommand {\threeexptime} {\textsc{3exptime}}
\newcommand {\polytime} {\textsc{p}}
\newcommand{\twoexpspace}{\textsc{2expspace}}
\newcommand{\threeexpspace}{\textsc{3expspace}}
\newcommand {\nexptime} {\textsc{nexptime}}



\newcommand{\aalphabet}{\Sigma}     % an alphabet, A is already used for atoms
\newcommand{\aword}{\mathfrak{u}}
\newcommand{\awordbis}{\mathfrak{v}}



\newcommand{\aassertion}{P}
\newcommand{\aassertionbis}{Q}
\newcommand{\aexpression}{e}
\newcommand{\aexpressionbis}{f}
\newcommand{\avariable}{\mathtt{x}}
\newcommand{\uniquevar}{\mathtt{u}}
\newcommand{\uniquevarbis}{\mathtt{v}}
\newcommand{\avariablebis}{\mathtt{y}}
\newcommand{\avariableter}{\mathtt{z}}
\newcommand{\nullconstant}{\mathtt{null}}
\newcommand{\nilvalue}{nil}
\newcommand{\emptyconstant}{\mathtt{emp}}
\newcommand{\infheap}{\mathtt{inf}}
\newcommand{\saturated}{\mathtt{Saturated}}

\newcommand{\astateformula}{\phi}
\newcommand{\astateformulabis}{\psi}
\newcommand{\astateformulater}{\varphi}
%%
\newcommand{\separate}{\ast}
\newcommand{\sep}{\separate}
\newcommand{\size}{\mathtt{size}}
\newcommand{\sizeeq}[1]{\mathtt{size} \ = \ #1}
\newcommand{\alloc}[1]{\mathtt{alloc}(#1)}
\newcommand{\allocb}[2]{\mathtt{alloc}^{-1}[#2](#1)}
\newcommand{\isol}[1]{\mathtt{isoloc}(#1)}
\newcommand{\icell}{\mathtt{isocell}}
\newcommand{\malloc}{\mathtt{malloc}}
\newcommand{\cons}{\mathtt{cons}}
\newcommand{\new}{\mathtt{new}}
\newcommand{\free}[1]{\mathtt{free} #1}
\newcommand{\maxform}[1]{\mathtt{maxForms}(#1)}
\newcommand{\locations}[1]{\mathtt{loc}(#1)}
\newcommand{\values}{\mathtt{Val}}
\newcommand{\aheap}{\mathfrak{h}}
\newcommand{\avaluation}{\mathfrak{V}}
\newcommand{\heaps}{\mathcal{H}}
\newcommand{\astore}{\mathfrak{s}}
\newcommand{\stores}{\mathcal{S}}
\newcommand{\amodel}{\mathfrak{M}}
\newcommand{\alabel}{\ell}

\newcommand{\aprogram}{\mathtt{PROG}}
\newcommand{\programs}{\mathtt{P}}
\newcommand{\ctprograms}{\programs^{ct}}
\newcommand{\aninstruction}{\mathtt{instr}}
\newcommand{\ainstruction}{\mathtt{instr}}
\newcommand{\instructions}{\mathtt{I}}
\newcommand{\aguard}{\ensuremath{g}}
\newcommand{\guards}{\ensuremath{G}}
\newcommand{\domain}[1]{\mathtt{dom}(#1)}
\newcommand{\memory}{\stores\times\heaps}
\newcommand{\skipinstruction}{\mathtt{skip}}

\newcommand{\execution}{\mathtt{comp}}
\newcommand{\aux}{\mathtt{embd}}
\newcommand{\runof}{run}
\newcommand{\anexecution}{e}


\newcommand{\aletter}{\ensuremath{a}}
\newcommand{\aletterbis}{\ensuremath{b}}
\newcommand{\alocation}{\mathfrak{l}}

\newcommand{\pointsl}[1]{\stackrel{#1}{\hookrightarrow}}
\newcommand{\ppointsl}[1]{\stackrel{#1}{\mapsto}}
\newcommand{\ourhook}[1]{\stackrel{#1}{\hookrightarrow}}
\newcommand{\ltrue}{{\sf true}}
\newcommand{\lfalse}{{\sf false}}


\newcommand{\variables}{\mathtt{FVAR}}
\newcommand{\pvariables}{\mathtt{PVAR}}
\newcommand{\secvariables}{\mathtt{SVAR}}
\newcommand{\logique}[1]{\mathtt{FO}(#1)}



\newcommand{\atranslation}{\mathfrak{t}}
\newcommand{\nbpred}[1]{\widetilde{\sharp #1}}
\newcommand{\nbpredstar}[1]{\widetilde{\sharp #1}^{\star}}
\newcommand{\isolated}{\mathtt{isol}}
\newcommand{\stdmarks}{\mathtt{envir}}
\newcommand{\relation}[1]{\mathtt{relation}_{#1}}
\newcommand{\freevar}{\mathtt{FV}}
\newcommand{\notonmark}{\mathtt{notonenv}}
\newcommand{\InVal}[1]{\mathtt{InVal}\!\left(#1\right)}
\newcommand{\NotOnEnv}[1]{\mathtt{NotOnEnv}\!\left(#1\right)}
\newcommand{\PartOfVal}[1]{\mathtt{PartOfVal}\!\left(#1\right)}
%\newcommand{\nbpreds}[3]{\sharp #1 \geq #2}
\newcommand{\defstyle}[1]{{\emph{#1}}}

\newcommand{\cut}[1]{}
\newcommand{\interval}[2]{[#1,#2]}
\newcommand{\buniquevar}{\overline{\uniquevar}}
\newcommand{\bbuniquevar}{\overline{\overline{\uniquevar}}}
\newcommand{\magicwand}{\mathop{\mbox{$\mbox{$-~$}\!\!\!\!\ast$}}}
\newcommand{\wand}{\magicwand}
\newcommand{\septraction}{\stackrel{\hsize0pt \vbox to0pt{\vss\hbox to0pt{\hss\raisebox{-6pt}{\footnotesize$\lnot$}\hss}\vss}}{\magicwand}}
%% \newcommand{\reach}{\mathtt{reach}}
\mathchardef\mhyphen="2D % hyphen while in math mode

\newcommand{\adataword}{\mathfrak{dw}}
\newcommand{\adatum}{\mathfrak{d}}

\newcommand{\collectionknives}{\mathtt{ks}}
\newcommand{\collectionknivesfork}[1]{\mathtt{ksfs}_{=#1}}
\newcommand{\collectionknivesforks}{\mathtt{ksfs}}
\newcommand{\collectionkniveslargeforks}{\mathtt{kslfs}}


\newcommand{\acounter}{\mathtt{C}}

\newcommand{\fotwo}[3]{{\mbox{FO2}_{#1,#2}(#3)}}
\newcommand{\mtrans}[1]{t\!\left(#1\right)^{\Box}}
\newcommand{\mbtrans}[2]{\mtrans{#2}_{#1}}


\newcommand{\alogic}{\mathfrak{L}}


\newcommand{\semantics}[1]{\ensuremath{[ #1 ]}}


\newcommand{\adomino}{\mathfrak{d}}
\newcommand{\atile}{\mathfrak{d}}
\newcommand{\atiling}{\mathfrak{t}}

\newcommand{\hori}{\mathtt{h}}
\newcommand{\verti}{\mathtt{v}}
\newcommand{\domi}{\mathtt{d}}

\newcommand{\cpyrel}{\mathfrak{cp}}

\newcommand{\cntcmp}{\mathfrak{C}}

\newcommand{\heapdag}{\mathfrak{G}}

\newcommand{\onmainpath}{\mathtt{mp}}

\newcommand{\tree}{\mathtt{tree}}

%\newcommand{\tile}{\mathtt{tile}}

\newcommand{\type}{\mathtt{type}}

\newcommand{\ptype}{\mathtt{ptype}}

\newcommand{\exttype}{\mathtt{exttype}}

\newcommand{\anctypes}{\mathtt{AncTypes}}

\newcommand{\destypes}{\mathtt{DesTypes}}

\newcommand{\inctypes}{\mathtt{IncTypes}}

\newcommand{\treeic}{\mathtt{treeIC}}

\newcommand{\trs}{\mathfrak{trs}}


\newcommand{\nin}{\not \in}
\newcommand{\cupplus}{\uplus}
\newcommand{\aunarypred}{\mathtt{P}}


\newcommand{\hide}[1]{}

\newcommand{\eval}[2]{\llbracket#1\rrbracket_{#2}}
\newcommand\cur{\mathsf{cur}}
\newcommand\dom{\mathsf{dom}}
\newcommand\rng{\mathsf{rng}}

\newcommand\dd{\mathbb{D}}
\newcommand\nat{\mathbb{N}}


\newcommand\cA{\mathcal{A}}
\newcommand\cB{\mathcal{B}}
\newcommand\cC{\mathcal{C}}
\newcommand\cE{\mathcal{E}}
\newcommand\cG{\mathcal{G}}
\newcommand\cI{\mathcal{I}}
\newcommand\Ll{\mathcal{L}}
\newcommand\cM{\mathcal{M}}
\newcommand\cP{\mathcal{P}}
\newcommand\cR{\mathcal{R}}
\newcommand\cS{\mathcal{S}}
\newcommand\cT{\mathcal{T}}


\newcommand\vard{\mathfrak{d}}

\newcommand\replaceall{\mathsf{replaceAll}}
\newcommand\indexof{\mathsf{IndexOf}}



\newcommand\strline{\mathsf{SL}}

\newcommand\pstrline{\mathsf{SL_{pure}}}

\newcommand\search{\mathsf{search}}

\newcommand\verify{\mathsf{verify}}

\newcommand\searchleft{\mathsf{searchLeft}}

\newcommand\searchlong{\mathsf{searchLong}}


\newcommand\pref{\mathsf{Pref}}

\newcommand\wprof{\mathsf{WP}}

\newcommand\vars{\mathsf{Vars}}

\newcommand\dep{\mathsf{Dep}}
\newcommand\ptn{\mathsf{Ptn}}

\newcommand\src{\mathsf{src}}
\newcommand\strtorep{\mathsf{strToRep}}

\newcommand\rpleft{\mathsf{l}}
\newcommand\rpright{\mathsf{r}}


\newcommand\srcnd{\mathsf{srcND}}

\newcommand\ctxt{\mathsf{ctxt}}


\newcommand\ctxts{\mathsf{Ctxts}}

\newcommand\sprt{\mathsf{sprt}}

\newcommand\val{\mathsf{val}}

\newcommand\srclen{\mathsf{srcLen}}

\newcommand\rpleftlen{\mathsf{lLen}}


\newcommand\dfs{\mathsf{DFS}}

\newcommand\repr{\mathsf{rep}}

\newcommand\red{\mathsf{red}}

\newcommand\gfun{\mathcal{F}}


\newcommand{\leftmost}{{\sf leftmost}}
\newcommand{\longest}{{\sf longest}}

\newcommand{\arbidx}{{\sf Idx_{arb}}}
\newcommand{\dmdidx}{{\sf Idx_{dmd}}}
\newcommand{\lftlen}{{\sf Len_{lft}}}

\newcommand\rcdim{\mathsf{dim}}

\newcommand\rcdep{\mathsf{dep}}

\newcommand\tower{\mathrm{Tower}}


%\newtheorem{remark}[theorem]{Remark}

%%%%%%%%%%%%%%%%%%%%%%%%%%%%%%%%%%%%%%%
% Macros for two-way lower bound proof.

    \usepackage{unicode-math}

    \newcommand\ap[2]{{#1}\mathord{\brac{#2}}}

    % Tiling

    \newcommand\tiles{\Theta}
    \newcommand\hrel{H}
    \newcommand\vrel{V}
    \newcommand\tile{t}
    \newcommand\inittile{t_I}
    \newcommand\fintile{t_F}
    \newcommand\tileheight{h}

    % Large numbers

    \newcommand\expheight{n}
    \newcommand\linlen{m}
    \newcommand\tilesnum[1]{\Theta_{#1}}
    \newcommand\hrelnum[1]{H_{#1}}
    \newcommand\vrelnum[1]{V_{#1}}
    \newcommand\inittilenum[1]{\inittile^{#1}}
    \newcommand\fintilenum[1]{\fintile^{#1}}

    % \goodnums{n}{x} means x is good seqs of level-n nums
    \newcommand\goodnums[2]{\ap{\varphi_{#1}}{#2}}

    % \tenc{n}{val} = tile encoding of val in level-n
    \newcommand\tenc[2]{[#2]_{#1}}
    % \exptower{n}{m}  = tower of height n, to the m.
    \newcommand\nexp[2]{2 \uparrow_{#1} \brac{#2}}
    % \nmax{n} -- max number encodable at level n
    \newcommand\nmax[1]{\text{MAX}_{#1}}

    % nested alphabets, argument is level of nesting
    \newcommand\bone[1]{1_{#1}}
    \newcommand\bzero[1]{0_{#1}}
    \newcommand\nestednum[1]{c^{#1}}
    \newcommand\nestedalphabet[1]{\Sigma_{#1}}

    \newcommand\numeq{\circledequal}
    \newcommand\numplus{\oplus}
    \newcommand\numsep{\#}

    \newcommand\probfmla{\varphi}
    \newcommand\tilerow{r}

\newcommand{\ASSERT}[1]{\textbf{assert}(#1)}

\newcommand{\straightline}{\textsf{SL}}
\newcommand{\straightlinesym}{\textsf{SLS}}

\newcommand{\Pre}{\textsf{Pre}}

\newcommand{\twpt}{\textsf{2PT}}
\newcommand{\owpt}{\textsf{PT}}
\newcommand{\rbtwpt}{\textsf{RB2PT}}
\newcommand{\twspt}{\textsf{2SPT}}
\newcommand{\owspt}{\textsf{SPT}}

\newcommand{\transet}{\mathscr{T}}

\newcommand\theory{{\sf Th}}

\newcommand{\signature}{\mathcal{S}}

\newcommand{\sorts}{{\mathfrak{S}}}

\newcommand{\functions}{{\mathcal{F}}}

\newcommand{\predicates}{{\mathcal{P}}}

\newcommand\data{\mathbb{D}}

\newcommand{\interpretation}{\mathcal{I}}

\newcommand{\structure}{\mathcal{D}}

\newcommand{\issat}{\mathsf{isSat}}

\newcommand{\OMIT}[1]{}

\newcommand{\Left}{\ensuremath{\leftarrow}}
\newcommand{\Right}{\ensuremath{\rightarrow}}
\newcommand{\Stay}{\ensuremath{\text{\scshape S}}}

\newcommand{\Aut}{\ensuremath{\mathcal{A}}}
\newcommand{\AutB}{\ensuremath{\mathcal{B}}}
\newcommand{\Transducer}{\ensuremath{T}}
\newcommand{\controls}{\ensuremath{Q}}
\newcommand{\finals}{\ensuremath{F}}
\newcommand{\transrel}{\ensuremath{\delta}}

\newcommand{\Lang}{\mathcal{L}}
\newcommand{\ialphabet}{\Sigma}

\newcommand{\EndLeft}{\ensuremath{\triangleright}}
\newcommand{\EndRight}{\ensuremath{\triangleleft}}

%==========================================================

%\newcommand\shortlong[2]{#2}
\newcommand\shortlong[2]{#1}

\newif\ifdraft\drafttrue
%\newif\ifdraft\draftfalse
\ifdraft
\newcommand{\anthony}[1]{\color{red} {YA: #1 :AY} \color{black}}
\newcommand{\zhilin}[1]{\color{brown} {ZL: #1 :LZ} \color{black}}
\newcommand{\tl}[1]{\color{blue} {TL: #1 :LT} \color{black}}
\newcommand{\mat}[1]{\color{cyan} {MH: #1 :HM} \color{black}}
\else
\newcommand{\anthony}[1]{}
\newcommand{\zhilin}[1]{}
\newcommand{\tl}[1]{}
\newcommand{\mat}[1]{}
\fi

\newcommand{\concat} {\circ}
\newcommand{\replace} {{\sf replace}}
\newcommand{\str} {{\sf Str}}
\newcommand{\intnum} {{\sf Int}}
\newcommand{\regexp} {{\sf RegExp}}
\newcommand{\strarr} {{\sf StringArray}}
\newcommand{\dtypes} {{\sf DataTypes}}
\newcommand{\anarr} {{\mathbb{A}}}

%============================================================


\begin{document}

\title{Parameterised Transducers}
\subtitle{(An Expressive Framework for Symbolic 
Execution Analysis of Programs with Strings)}

\author{}
\institute{}

\maketitle

\zhilin{Parametric versus parameterised ?}

%!TEX root = main.tex

\begin{abstract}
%% some background on regular expressions
Regular expressions (RE) are a classical concept in formal language theory.
%, which are expressions built from characters by the operators of concatenation, union, and Kleene star. 
Real-world regular expressions (RWRE) in programming languages differ from REs 
in the non-standard semantics of operators (e.g. non-commutative union and 
greedy/lazy Kleene star), as well as  the additional features like capturing
groups and backreferences. 
%% symbolic execution requires faithful encoding of regex semantics
%String constraint solvers are one of the cornerstones of the analysis and verification of string manipulating programs. Faithful encoding of the semantics of real-world regular expressions in string constraint solvers facilitates more precise program analysis and verification.
%% state of the art string constraint solvers
%The semantics of real-word regular expressions are tricky and vary in different programming languages. It is challenging for string constraint solvers to support real-world regular expressions.
While REs are supported by every state-of-the-art string constraint solver, 
RWREs are thus far unsupported. Recent works have suggested that 
the mismatch between REs and RWREs
 %in string constraint 
%solvers and RWREs in programming languages 
%hinders the precision and efficiency 
makes it difficult for a symbolic execution engine to deal with RWREs,
which are hitherto approximated by a CEGAR-based approach,\philipp{this should be rephrased, it sounds as if CEGAR is the standard approach to deal with RWREs. It is not, there is only one symex tool doing that.} which performs
many satisfiability checks of string constraints with only REs.
%; hitherto,
%RWREs are either approximated away
%
%by resorting to approximate 
%encoding of RWREs and counter-example guided abstraction refinements (CEGAR).
%
%The semantics of real-world regular expressions are better to be encoded as faithfully as possible. For instance, in dynamic symbolic execution of string manipulating programs, in order to generate the inputs for execution paths and improve coverage,  the semantics of real-world regular expressions are better to be encoded as faithfully as possible in string constraint solvers. 
%% contribution
In this paper, we propose an approach of natively supporting RWREs in string
constraint solving. 
The key idea of our approach is to introduce a new automata model, called \emph{prioritized streaming string transducers} (PSST), 
 to model the string functions involving RWREs.
PSSTs combine \emph{priorities,} which have previously been introduced
in prioritized finite-state automata to capture greedy/non-greediness,
with \emph{string variables} as in streaming string transducers to
model capture groups.
%
%non-standard semantics of regular expression operators can be modeled by priorities and new features of capturing groups and back references can be modeled by string variables. 
%the non-standard semantics of regular expression operators can be modeled by priorities and new features of capturing groups and back references can be modeled by string variables. 
Based on PSSTs, we design a decision procedure for string constraints with 
RWREs and provide its implementation.
%% implementation and experiments
%We implement the decision procedure 
%and do extensive experiments to evaluate its performance. 
We evaluate its performance on over 160,000 string constraints generated 
from RWREs in open-source programs.
Our approach dramatically improves the CEGAR-based approach for RWREs in both 
precision and efficiency. 
%To the best of our knowledge, this work represents the first string constraint solver supporting RWREs.
\end{abstract}


%!TEX root = main.tex

string manipulating programs

symbolic execution

string operations with integer data type

We use the following running example to illustrate the decision procedure in this paper.
%\begin{example}
{\small
\begin{minted}[linenos]{javascript}
function urlSimpleParse(url)
{
  var protocol='', host='';
  url = url.trim();
  var colonpos = url.indexof(':');
  if (colonpos >= 0) 
  {
    protocol = url.substr(0, colonpos).toLowerCase();
    if(/^http$|^https$/.test(protocol))
    {
      url = url.substr(colonpos+3);
      var slashpos = url.indexof('/');
      if (slashpos >= 0)  host = url.substr(0, slashpos); 
    }
    else protocol = '';
    return protocol, host; 
  }
}
\end{minted}
}
%\end{example}

% \in [\backslash w | \backslash x2E]^*$
We expect that host contains only the alphanumeric symbols as well as the dot symbol, but actually this is not the case. This question can be reduced to solving the path feasibility problem of the following program of the SSA (single static assignment) form.

{\small
\begin{minted}{javascript}
  protocol = ''; host = ''; url1 = url.trim(); 
  colonpos = url1.indexof(':'); assert(colonpos >= 0); 
  protocol1 = url1.substr(0, colonpos); 
  protocol2 = protocol1.toLowerCase();
  assert(/^http$|^https$/.test(protocol2));
  url2 = url1.substr(colonpos+3);
  slashpos = url2.indexof('/'); assert(slashpos >= 0);
  host1 = url2.substr(0, slashpos); assert(!/[\w|\x2E]*/.test(host1))
\end{minted}
}

state-of-the-art: heuristics

The contribution of this paper: decision procedure for string constraints involving integer data type

automata-theoretic, cost-enriched regular languages and recognisable relations, backward computation

implementation OSTRICH+, experimental results promising

first decision procedure for such an expressive class of string constraints involving so many different operations, natural extension of the decision procedure of OSTRICH, efficient implementation, extensive experiments, 


related work

SLENT: \cite{WC+18}

CVC4: \cite{cvc4}

TRAU, Z3-TRAU, TRAU+: \cite{Abdulla17,AbdullaA+19}

Z3-STR: \cite{Z3-str}

OSTRICH: \cite{CHL+19}

%!TEX root = main.tex

\section{Preliminaries}
\label{sec:prelim}

\paragraph{General Notations.}
Let $\mathbb{Z}$ and $\Nat$ denote the set of integers and natural numbers
respectively. For $i \le j \in \Nat$, let $[i, j]$ denote $\{i,i+1,\ldots,j\}$ and
$[i] = [0, i]$. \zhilin{The only place I notice that $[i] = [0,i]$ is used is the definition of semantics of \FFA{}(T). In most places, $[i]$ is used for $\{1,\ldots,i\}$. I would suggest to define $[i]=\{1,\ldots,i\}$, and if $[0,i]$ should be used, then just use $[0,i]$ there. I will assume this notation through the paper.} For a vector
$\vec{x}=(x_1,\cdots, x_n)$, let $|\vec{x}|$ denote the length of $\vec{x}$
(i.e., $n$) and  $\vec{x}[i]$ denote $x_i$ for each $i \in [n]$. For a set
$S$, we use $S^*$ (resp.~$S^+$) to denote the set of all finite (resp.~finite
and nonempty) sequences over $S$. We use $\epsilon$ for the empty sequence.

%=======================================================================
\OMIT{
\paragraph{Graph-Theoretical Notation.} \tl{not sure whether it is really needed; will see}
A DAG (\emph{directed acyclic graph}) $G$ is a finite directed graph $(V, E)$ with
no directed cycles, where $V$ (resp.~$E \subseteq V \times V$) is a set of vertices (resp.~edges).
%. That is, each DAG consists of finitely many vertices and edges, with each edge directed from one vertex to another, such that there is no way to start at any vertex $\mathit{v}$ and follow a consistently-directed sequence of edges that eventually loops back to $\mathit{v}$ again.
Equivalently, a DAG is a directed graph that has a topological ordering, which
is a sequence of the vertices such that every edge is directed from an earlier
vertex to a later vertex in the sequence. An edge $(\mathit{v},\mathit{v'})$ in
$G$ is called an \emph{incoming} edge of $\mathit{v'}$ and an \emph{outgoing}
edge of $\mathit{v}$. If $(\mathit{v},\mathit{v'}) \in E$, then $\mathit{v'}$ is
called a \emph{successor} of $\mathit{v}$ and $\mathit{v}$ is called a
\emph{predecessor} of $\mathit{v'}$. A \emph{path} $\pi$ in $G$ is a sequence
$\mathit{v}_0 \mathit{e}_1 \mathit{v}_1 \cdots \mathit{v}_{n-1} \mathit{e}_n
\mathit{v}_n$ such that for each $i \in [n]$, we have $\mathit{e}_i =
(\mathit{v}_{i-1},\mathit{v}_i) \in E$. The \emph{length} of the path $\pi$
%$\mathit{v}_0 e_1 \mathit{v}_1 \cdots \mathit{v}_{n-1} e_n \mathit{v}_n$ in $G$
is the number $n$ of edges in $\pi$. If there is a path from
$\mathit{v}$ to $\mathit{v'}$ (resp. from $\mathit{v'}$ to $\mathit{v}$) in $G$,
then $\mathit{v'}$ is said to be \emph{reachable} (resp. \emph{co-reachable})
from $\mathit{v}$ in $G$. If $\mathit{v}$ is reachable from $\mathit{v'}$ in
$G$, then $\mathit{v'}$ is also called an \emph{ancestor} of $\mathit{v}$ in
$G$. In addition, an edge $(\mathit{v'},\mathit{v''})$ is said to be reachable
(resp. co-reachable) from $\mathit{v}$ if $\mathit{v'}$ is reachable from $\mathit{v}$ (resp. $\mathit{v''}$ is co-reachable from $\mathit{v}$). The \emph{in-degree} (resp. \emph{out-degree}) of a vertex $\mathit{v}$ is the number of incoming (resp. outgoing) edges of $\mathit{v}$.
%A vertex $\mathit{v}$ in $G$ is said to be a \emph{join} vertex if the in-degree of $\mathit{v}$ is at least two.
%A DAG $G$ is called an \emph{arborescence} if there is a vertex $v_0$ such that all the vertices are reachable from $v_0$ in $G$, in addition, there are no join vertices in $G$.
A \emph{subgraph} $G'$ of $G=(V,E)$ is a directed graph $(V', E')$ with
$V' \subseteq V$ and $E' \subseteq E$. Let $G'$ be a subgraph of $G$. Then $G \setminus G'$ is the graph obtained from $G$ by removing all the edges in $G'$.
}
%=============================================================================

\paragraph{Automata.} We review some necessary background from automata theory;
for more, see \cite{Kozen-automata,HU79}. Let $\ialphabet$ be a finite set (called
\defn{alphabet}). A sequence over $\ialphabet$ is called a \defn{string}. A \defn{language} is set of strings over $\ialphabet$.
We will review regular languages and necessary models of finite-state automata below.

\begin{definition}[Two-way finite-state automata] \label{def:2nfa}
    A \emph{(nondeterministic) two-way finite-state automaton}
(\FFA{}) over a finite alphabet $\ialphabet$ is a tuple $\Aut =
(\ialphabet, \EndLeft, \EndRight, \controls, q_0, \finals, \transrel)$ where 
    $\controls$ is a finite set of 
    states, $\EndLeft$ (resp.~$\EndRight$) a left (resp.~right) input tape end 
    marker, $q_0\in \controls$ is
the initial state, $\finals\subseteq \controls$ is a set of final states, and 
    $\transrel$ is the
transition relation  $\transrel\subseteq \controls \times 
    \overline{\ialphabet}\times \{\Left, \Stay, \Right\}\times \controls$ with
    $\overline{\ialphabet} := \ialphabet \cup \{\EndLeft,\EndRight\}$.
    Here, we assume $\EndLeft, \EndRight \notin \ialphabet$, and that
    there are no transitions that take the head of the tape past the left/right
    end marker (i.e.~$(p,\EndLeft,\Left,q), (p,\EndRight,\Right,q) \notin
    \transrel$ for every $p, q \in \controls$).
%\tl{not sure you want to use $\leftarrow, \rightarrow$?}
%    \anthony{Just defined macros. Please use generic macros in the definitions
%    guys.}

A (nondeterministic one-way) finite-state automaton (\FA{})
is a \FFA{} such that $\transrel \subseteq \controls \times \overline{\ialphabet} \times
    \{\Right,\Stay\} \times \controls $.
\end{definition}
In the sequel, whenever understood we will only tacitly mention $\ialphabet$, 
$\EndLeft$, and $\EndRight$ in $\Aut$. 

%A \emph{nondeterministic finite automaton} (NFA) $\cA$ on $\Sigma$ is a tuple $(Q, \delta, q_0, F)$, where $Q$ is a finite set of \emph{states}, $q_0 \in Q$ is the \emph{initial} state, $F \subseteq Q$ is the set of \emph{final} states, and $\delta \subseteq Q \times \Sigma \times Q$ is the \emph{transition relation}.

The notion of runs of \FFA{} on an input string is exactly the same as that of
Turing machines on a read-only input tape. More precisely, for a string 
$w = a_1 \dots a_n$, a \emph{run} of $\Aut$ on $w$
%(with $a_0 = \EndLeft$ and $a_{n+1} = \EndRight$)
is a sequence of pairs $(q_0,i_0),\ldots, (q_m,i_m) \in \controls \times [0, n+1]$ 
defined as follows. Let $a_0 = \EndLeft$ and $a_{n+1} = \EndRight$. The
following conditions then have to be satisfied:
\begin{itemize}
    \item $i_0 = 0$, and
    \item for every $j \in [m-1]$, we have $(q_j,a_{i_j}, dir, q_{j+1}) \in
        \transrel$ and $i_{j+1} = i_j + dir$ for some $dir \in  \{\Left, \Stay, \Right\}$.
\end{itemize}
The run is said to be \defn{accepting} if $i_m = n+1$ and $q_m \in \finals$.
A string $w$ is \defn{accepted} by $\Aut$ if there is an accepting run of
$\Aut$ on $w$. The set of strings accepted by $\Aut$ is denoted by $\Lang(\Aut)$,
a.k.a., the language \defn{recognised} by $\Aut$.
%Since we deal with computational complexity in the sequel, we define
The \defn{size} $|\Aut|$ of $\Aut$ is defined to be $|\controls|$; this will
be needed when we talk about computational complexity.

For convenience, we will also refer to an \FA{} without initial and final states, that is, a pair $(Q, \delta)$, as a \emph{transition graph}.

\OMIT{
state sequence $q_0 \dots q_n$ such that for each $i \in [n]$, $(q_{i-1}, a_i, q_i) \in \delta$. A run $q_0 \dots q_n$ is \emph{accepting} if $q_n \in F$. A string $w$ is \emph{accepted} by $\cA$ if there is an accepting run of $\cA$ on $w$. We use $\Ll(\cA)$ to denote the language defined by $\cA$, that is, the set of strings accepted by $\cA$. We will use $\cA, \cB, \cdots$ to denote NFAs.
%An NFA $\cA$ is \emph{deterministic} if for each $(q, \sigma) \in Q \times \Sigma$, there is at most one $q' \in Q$ such that $(q, a, q') \in \delta$. An NFA $\cA$ is \emph{complete} if for each $(q, \sigma) \in Q \times \Sigma$, there is at least one $q' \in Q$ such that $(q, a, q') \in \delta$. We assume that all NFA considered in this paper are complete.  An NFA $\cA$ is \emph{unambiguous} if for each string $w$, there is \emph{at most one accepting} run of $\cA$ on $w$.
For a string $w= a_1 \dots a_n$, we also use the notation $q_1 \xrightarrow[\cA]{w} q_{n+1}$ to denote the fact that there are $q_2,\dots, q_n \in Q$ such that for each $i \in [n]$, $(q_i, a_i, q_{i+1}) \in \delta$.  For an NFA $\cA=(Q, \delta, q_0, F)$ and $q, q' \in Q$, we use $\cA(q,q')$ to denote the NFA obtained from $\cA$ by changing the initial state to $q$ and the set of final states to $\{q'\}$. The \emph{size} of an NFA $\cA=(Q, \delta, q_0, F)$, denoted by $|\cA|$, is defined as $|Q|$, the number of states.

For convenience, we will also call an NFA without initial and final states, that is, a pair $(Q, \delta)$, as a \emph{transition graph}.
}

\FFA{} and \FA{} recognise precisely the same class of languages, i.e., 
\emph{regular languages}. The following result is standard and can be found in textbooks on automata theory
(e.g. \cite{Kozen-automata}). 

\begin{proposition}[\cite{Kozen-automata}]\label{prop-2nfa-nfa}
	Every \FFA{} $\Aut$ is equivalent to an \FA{} of size $2^{\bigO(|\Aut| \log |\Aut|)}$. 
%	Moreover, the equivalent
%    NFA can be computed in exponential time. 
    Moreover, every \FA{} can be transformed in polynomial time into an $\varepsilon$-free \FA{}, that is, an \FA{} where $\transrel\subseteq \controls \times \overline{\ialphabet} \times   \{\Right\} \times \controls $.
\end{proposition}

In the rest of this paper, \FA{}s refer to $\varepsilon$-free \FA{}s where directions in transitions are omitted. Moreover, for simplicity of notations, in \FA{}s, we omit the two end markers $\EndLeft, \EndRight$ and assume that $\transrel \subseteq \controls \times \ialphabet \times    \{\Right,\Stay\} \times \controls$. \zhilin{I assume that the two end markers are removed in FAs.}

\paragraph{Operations of \FA{}s.} For an FA $\Aut=(Q, q_0, F, \delta)$, $q \in Q$ and $P \subseteq Q$, we use $\Aut(q, P)$ to denote the FA $(Q, q, P, \delta)$, that is, the FA obtained from $\Aut$ by changing the initial state and the set of final states to $q$ and $P$ respectively. We use $q \xrightarrow[\Aut]{w} q'$ to denote the fact that a string $w$ is accepted by $\Aut(q, \{q'\})$. 


Given two \FA{}s $\Aut_1 = (Q_1, q_{0,1}, F_{1}, \delta_1)$ and $\Aut_2 = (Q_2, q_{0,2}, F_2, \delta_2)$, the \emph{product} of $\Aut_1$ and $\Aut_2$, denoted by $\Aut_1 \times \Aut_2$, is defined as $(Q_1 \times Q_2, (q_{0,1}, q_{0,2}), F_1 \times F_2, \delta_1 \times \delta_2)$, where $\delta_1 \times \delta_2$ is the set of tuples $((q_1,q_2), a, (q'_1, q'_2))$ such that $(q_1, a, q'_1) \in \delta_1$ and $(q_2, a, q'_2) \in \delta_2$. Evidently, we have $\Lang(\Aut_1 \times \Aut_2) = \Lang(\Aut_1) \cap \Lang(\Aut_2)$.


\anthony{I've removed the ``regular languages'' paragraph because it seems
unnecessary. I recommend that operation on sets can be defined in General
Notation above on general sets.}
%\tl{Other relevant models such as SST, when appropriate, will be put here.}

\zhilin{Let us fix the abbreviations here: \\
--\FFA{}(2FT): two-way finite-state automata(transducer),\\
--FA(FT): finite-state automata(transducer),\\
--2PT: two-way parametric transducer,\\
--PT: parametric transducer,\\
--2SA(2ST): two-way symbolic automata(transducer),\\
--SA(ST): symbolic automata(transducer) \\
--2SPT: two-way symbolic parametric transducer,\\
--SPT: symbolic parametric transducer 
}
\anthony{Can you please introduce macros?}

 
%===================================================================
 
\OMIT{
\paragraph{Regular Languages.}
Fix a finite \emph{alphabet} $\Sigma$. Elements in $\Sigma^*$ are called \emph{strings}. Let $\varepsilon$ denote the empty string and  $\Sigma^+ = \Sigma^* \setminus \{\varepsilon\}$. We will use $a,b,\cdots$ to denote letters from $\Sigma$ and $u, v, w, \cdots$ to denote strings from $\Sigma^*$. For a string $u \in \Sigma^*$, let $|u|$ denote the \emph{length} of $u$ (in particular, $|\varepsilon|=0$). A \emph{position} of a nonempty string $u$ of length $n$ is a number $i \in [n]$ (Note that the first position is $1$, instead of  0). In addition, for $i \in [|u|]$, let $u[i]$ denote the $i$-th letter of $u$.
For two strings $u_1, u_2$, we use $u_1 \cdot u_2$ to denote the \emph{concatenation} of $u_1$ and $u_2$, that is, the string $v$ such that $|v|= |u_1| + |u_2|$ and for each $i \in [|u_1|]$, $v[i]= u_1[i]$ and for each $i \in |u_2|$, $v[|u_1|+i]=u_2[i]$. Let $u, v$ be two strings. If $v = u \cdot v'$ for some string $v'$, then $u$ is said to be a \emph{prefix} of $v$. In addition, if $u \neq v$, then $u$ is said to be a \emph{strict} prefix of $v$. If $u$ is a prefix of $v$, that is, $v = u \cdot v'$ for some string $v'$, then
we use $u^{-1} v$ to denote $v'$. In particular, $\varepsilon^{-1} v = v$.

A \emph{language} over $\Sigma$ is a subset of $\Sigma^*$. We will use $L_1, L_2, \dots$ to denote languages. For two languages $L_1, L_2$, we use $L_1 \cup L_2$ to denote the union of $L_1$ and $L_2$, and $L_1 \cdot L_2$ to denote the concatenation of $L_1$ and $L_2$, that is, the language $\{u_1 \cdot u_2 \mid u_1 \in L_1, u_2 \in L_2\}$. For a language $L$ and $n \in \Nat$, we define $L^n$, the \emph{iteration} of $L$ for $n$ times, inductively as follows: $L^0=\{\varepsilon\}$ and $L^{n} =L \cdot L^{n-1}$ for $n > 0$. We also use $L^*$ to denote the iteration of $L$ for arbitrarily many times, that is, $L^* = \bigcup \limits_{n \in \Nat} L^n$. Moreover, let $L^+ = \bigcup \limits_{n \in \Nat \setminus \{0\}} L^n$.

\begin{definition}[Regular expressions $\regexp$]
	\[e \eqdef \emptyset \mid \varepsilon \mid a \mid e + e \mid e \concat e \mid e^*, \mbox{ where } a \in \Sigma. \]
	Since $+$ is associative and commutative, we also write $(e_1 + e_2) + e_3$ as $e_1 + e_2 + e_3$ for brevity. We use the abbreviation $e^+ \equiv e \concat e^*$. Moreover, for $\Gamma = \{a_1, \cdots, a_n\}\subseteq \Sigma$, we use the abbreviations $\Gamma \equiv a_1 + \cdots + a_n$ and $\Gamma^\ast \equiv (a_1 + \cdots + a_n)^\ast$.
\end{definition}
We define $\Ll(e)$ to be the language defined by $e$, that is, the set of strings that match $e$, inductively as follows: $\Ll(\emptyset) =\emptyset$,
%\begin{itemize}
%\item
$\Ll(\varepsilon) =\{\varepsilon\}$,
%
%\item
$\Ll(a)= \{a\}$,
%
%\item
$\Ll(e_1 + e_2) = \Ll(e_1) \cup \Ll(e_2)$,
%
%\item
$\Ll(e_1 \concat e_2) = \Ll(e_1) \cdot \Ll(e_2)$,
%
%\item
$\Ll(e_1^*)=(\Ll(e_1))^*$.
%\end{itemize}
In addition, we use $|e|$ to denote the number of symbols occurring in $e$.

%A \emph{nondeterministic finite automaton} (NFA) $\cA$ on $\Sigma$ is a tuple $(Q, \delta, q_0, F)$, where $Q$ is a finite set of \emph{states}, $q_0 \in Q$ is the \emph{initial} state, $F \subseteq Q$ is the set of \emph{final} states, and $\delta \subseteq Q \times \Sigma \times Q$ is the \emph{transition relation}. For a string $w = a_1 \dots a_n$, a \emph{run} of $\cA$ on $w$ is a state sequence $q_0 \dots q_n$ such that for each $i \in [n]$, $(q_{i-1}, a_i, q_i) \in \delta$. A run $q_0 \dots q_n$ is \emph{accepting} if $q_n \in F$. A string $w$ is \emph{accepted} by $\cA$ if there is an accepting run of $\cA$ on $w$. We use $\Ll(\cA)$ to denote the language defined by $\cA$, that is, the set of strings accepted by $\cA$. We will use $\cA, \cB, \cdots$ to denote NFAs.
%%An NFA $\cA$ is \emph{deterministic} if for each $(q, \sigma) \in Q \times \Sigma$, there is at most one $q' \in Q$ such that $(q, a, q') \in \delta$. An NFA $\cA$ is \emph{complete} if for each $(q, \sigma) \in Q \times \Sigma$, there is at least one $q' \in Q$ such that $(q, a, q') \in \delta$. We assume that all NFA considered in this paper are complete.  An NFA $\cA$ is \emph{unambiguous} if for each string $w$, there is \emph{at most one accepting} run of $\cA$ on $w$.
%For a string $w= a_1 \dots a_n$, we also use the notation $q_1 \xrightarrow[\cA]{w} q_{n+1}$ to denote the fact that there are $q_2,\dots, q_n \in Q$ such that for each $i \in [n]$, $(q_i, a_i, q_{i+1}) \in \delta$.  For an NFA $\cA=(Q, \delta, q_0, F)$ and $q, q' \in Q$, we use $\cA(q,q')$ to denote the NFA obtained from $\cA$ by changing the initial state to $q$ and the set of final states to $\{q'\}$. The \emph{size} of an NFA $\cA=(Q, \delta, q_0, F)$, denoted by $|\cA|$, is defined as $|Q|$, the number of states. For convenience, we will also call an NFA without initial and final states, that is, a pair $(Q, \delta)$, as a \emph{transition graph}.

It is well-known (e.g. see \cite{HU79}) that regular expressions and \FA{}s are
expressively equivalent, and generate precisely all \emph{regular languages}.
In particular, from a regular expression, an equivalent \FA{} can be constructed
in linear time. Moreover, regular languages are closed under Boolean
operations, i.e., union, intersection, and complementation.
In particular, given two \FA{} $\cA_1=(Q_1, \delta_1, q_{0,1}, F_1)$ and
$\cA_2=(Q_2, \delta_2, q_{0,2}, F_2)$ on $\Sigma$, the intersection $\Ll(\cA_1)
\cap \Ll(\cA_2)$ is recognised by the \emph{product automaton} $\cA_1 \times
\cA_2$ of $\cA_1$ and $\cA_2$ defined as $(Q_1 \times Q_2, \delta, (q_{0,1}, q_{0,2}), F_1 \times F_2)$, where $\delta$ comprises the transitions $((q_1, q_2), a, (q'_1, q'_2))$ such that $(q_1, a, q'_1) \in \delta_1$ and $(q_2, a, q'_2) \in \delta_2$.
}
%==========================================================================================


\section{The Framework: Transducers, Straight-Line Programs}
\label{sec:framework}

In this section we first review the framework of straight-line programs from
\cite{LB16} for analysing symbolic execution of programs with strings, which
involves capturing ``built-in'' functions using finite-state transductions.
We will then introduce the notion of parametric transducers, and show that
they can capture many other interesting string functions that cannot be captured
within the framework of \cite{LB16}. In particular, this allows us to capture
string functions $f: (\Sigma^*)^k \to \Sigma^*$ with multiple input strings
(e.g. the replaceall function), and the string reverse function. 
The framework of straight-line programs from \cite{LB16} can be adapted
to this new notion of transducers, whose path feasibility problem will be shown
to be decidable in Section \ref{sec:algo}.


\anthony{Some abbreviations: 
\begin{itemize}
    \item 2PT, two-way parametric transducers
    \item 1PT, one-way parametric transducers
    \item 2RPT, two-way reversal-bounded parametric transducers
    \item 2T, two-way non-parametric transducers
    \item 1T, one-way non-parametric transducers
\end{itemize}
}

\subsection{The ``straight-line'' framework}

As elegantly described in Bj\"{o}rner \emph{et al.} \cite{BTV09}, constraints 
from symbolic 
execution on string-manipulating programs can be viewed as the problem of \emph{path 
feasibility} over loopless string-manipulating programs $S$ with variable 
assignments and assertions, i.e., generated by the grammar
\begin{equation*}
    S ::= y := f(x_1,\ldots,x_n) \ |\ \text{\ASSERT{$g(x_1,\ldots,x_n)$}}\ |\ 
            S_1; S_2\ 
    %a ::= f(x_1,\ldots,x_n), \qquad b ::= g(x_1,\ldots,x_n) 
\end{equation*}
where $f: (\Sigma^*)^n \to \Sigma^*$ and $g: (\Sigma^*)^n \to \{0,1\}$ are
some string functions. 
That is, each symbolic execution $S$ is simply a 
straight-line
program with additional assertions on the way. The problem of path feasibility 
asks whether, for a given program $S$, there exist input strings that take
$S$ to the end of the program while satisfying all the assertions.
%Straight-line programs with assertions 
Notice that such straight-line programs can be viewed as constraints over the 
domain of strings by turning such programs 
into a Static Single Assignment (SSA) form (i.e. introduce a new variable 
on the left hand side of each assignment). The authors initiated the exploration
of what kind of string functions $f$ and $g$ for which the above path 
feasibility problem can be algorithmically solved. 

The question of incorporating some form of transducers into the definitions of 
the string functions $f$ and $g$ were briefly explored in \cite{BTV09}. In
summary, whenever all $f$ and $g$ in the program can be captured by synchronised
$k$-track finite automata (a.k.a.~\defn{automatic structures} \cite{BG04}), then
the path feasibility problem becomes decidable. The authors noted, however, that
this is too narrow for applications since string concatenation cannot be
captured in this framework. In a recent paper \cite{LB16}, Lin \& Barcelo 
proposed to allow concatenation, regular constraints (i.e. regular expression 
matching), and finite-state 
transductions (not
synchronised as in \cite{BTV09}, but only permits one input string and one 
output string) in straight-line programs. We will define finite-state
transducers in the following subsection, but roughly speaking
they are finite-state automata with an extra output (write-only) track, i.e.,
upon reading an input symbol, it may decide to output (or not output) a symbol 
on the output track. [In general, the input/output tracks of finite-state
transducers may be asynchronous, which means that they cannot be captured by
automatic structures of \cite{BG04}.] 
    Finite-state transducers are powerful for modelling many
different string functions including sanitisation functions (e.g. htmlescape and
backslash-escape) and implicit browser transductions in HTML5 applications (e.g.
innerHTML). Although they noted that this is
undecidable by the result of \cite{BFL13}, decidability can be obtained 
by disallowing \emph{string equality checks in the assertions $g$}, i.e., after
a simplification, we may assume that $g$ contains only regular constraints.
%In
%particular, an application of finite-state transducer and string equality checks
%are prohibited in $g$. 
The dubbed the
resulting class of string constraints \emph{the straight-line fragment}.
They showed that the straight-line fragment is sufficiently powerful in
interesting applications including analysis of cross-site scripting (XSS)
vulnerabilities. Decidable extensions of the straight-line fragment (including
length constraints and disequality checks) were given by the authors, which 
we will discuss in Section \ref{sec:extensions}.

Chen \emph{et al.} \cite{CCHLW18} recently noted that some XSS vulnerability
analysis involving web templating systems (e.g. \texttt{Mustache.js}
\cite{Mustache}) requires the use of the replace-all function that cannot be
captured by finite-state transducers. Essentially, since the ``holes'' to be 
filled in by data from a separate JSON file might be an arbitrary string, the
replacement string in the replace-all function could in fact be a string
variable. \anthony{Define the replace-all function later}. The authors
showed that the path feasibility problem where the assignments $f$ use the
replace-all function and the conditionals $g$ use the regular constraints 
is still decidable (in fact, with the same complexity as in \cite{LB16} 
EXPSPACE), so long as the pattern variable in the replace-all function is a
regular expression (i.e. \emph{not} a variable). Interestingly, such a usage of
the replace-all function can even capture the concatenation operator, thereby
providing another decidable subclass of the path feasibility problem involving
the replace-all function, the concatenation operator, and regular constraints.

%involves: (i) instantiating 
 
\OMIT{
The basic ``straight-line'' framework from Lin \& Barcelo \cite{LB16} involves: (i) instantiating 
the string functions in program assignments by either concatenation of string
variables/constants or an application of a one-way finite-state transducer
to a string variable/constant, and (ii) instantiating conditionals by regular
constraints.

One-way and two-way transducers. Mention what kind of closure/algorithmic
results.
}

\OMIT{
\begin{definition}[Recognisable relation]
	Given a finite alphabet $\Sigma$, a $k$-ary relation $R\subseteq \Sigma^*\times \cdots\times \Sigma^*$ is \emph{recognisable}  if $R=\bigcup_{i=1}^n L^{(i)}_1\times \cdots\times L^{(i)}_k$ where $L^{(i)}_j$ a regular for each $j\in [k]$ .
%
%	[One can certainly generalise this to $n$-ary relations. ]
\end{definition}
}

\OMIT{
\begin{definition}[Two-way Finite transducers]
  Nondeterministic two-way finite state \emph{transducers} (2NFTs) over $\Sigma$ and $\Gamma$ extend NFAs with a one-way left-to-right output tape. They are defined as 2NFAs except that the transition relation $\Delta$ is extended with outputs: $\Delta\subseteq Q \times \Sigma \times \Gamma \times \{-1, +1\} \times  Q $. If a transition $(q, a, b, q′, m)$ is enabled on a letter $a\in \Sigma$, the letter $b\in \Gamma$ is appended to the right of
	the output tape and the transducer goes to state $q'$.
\end{definition}
}

\subsection{Parametric transducers}

Introduce one-way and two-way parametric transducers. 

Give examples: htmlescape, replace-all, and reverse.

\subsection{The ``straight-line'' framework}

Give example of computing reverse of DNA string.


%!TEX root = main.tex

\section{Algorithmic results}
\label{sec:algo}

In this section, let us assume an $\straightline$ formula $\varphi \wedge \psi$.



\subsection{Two-way}

%\subsection{A generic decision procedure based on recognisable relations}

In this section, we will present a generic decision procedure for $\straightline[\twpt]$ based on recognisable relations.

\begin{definition}[Recognisable relation]
	Given a finite alphabet $\Sigma$, a $k$-ary relation $R\subseteq \Sigma^*\times \cdots\times \Sigma^*$ is \emph{recognisable}  if $R=\bigcup_{i=1}^n L^{(i)}_1\times \cdots\times L^{(i)}_k$ where $L^{(i)}_j$ is regular for each $j\in [k]$.
%
%	[One can certainly generalise this to $n$-ary relations. ]
\end{definition}

Let $\Transducer$ be a 2PT with $k$ parameters $y_1,\cdots, y_k$ and $\Aut$ be an NFA. Then the \emph{preimage} of $\Aut$ w.r.t. $\Transducer$, denoted by $\Pre_\Transducer(\Aut)$, is the set of tuples $(w, w_1,\cdots, w_k)$ such that there is an accepting run of $\Transducer$ on $w$, with the parameters $w_1,\cdots, w_k$, where the output $w'$ belongs to $\Lang(\Aut)$.


\begin{lemma}
Let $\Transducer$ be a 2PT and $\Aut$ be an NFA. Then $\Pre_\Transducer(\Aut)$ is a recognisable relation.
\end{lemma}

\begin{proof}
\end{proof}



decision procedure, straight-line fragment, complexity lower bound

\subsection{One-way}

expspace for the one-way transducers, or even reversal-bounded transducers.



\section{Extensions of integer constraints}
\label{sec:extensions}

In this section, we consider the language $\strline[\transet]$ extended with integer constraints.
%, character constraints, or $\indexof$ constraints, and show that each of such extensions leads to undecidability. 
%\mat{Should $\concat$ be removed?}\zhilin{agree, since the undecidability result is stronger if stated for $\strline[\replaceall]$}
We will use %variables of, in additional to the type $\str$, the Integer data type $\intnum$. The type $\str$ consists of the string variables as in the previous sections. A variable of type $\intnum$, usually referred to as an 
\emph{integer variable}, typically $\mathfrak{l}, \mathfrak{m}, \mathfrak{n}, \ldots$, to range over the set $\Nat$ of natural numbers. %Recall that, in previous sections, we have used $x, y, z, \ldots$ to denote the variables of $\str$ type.  Hereafter we typically use to denote the variables of $\intnum$. 
The
choice of omitting negative integers is for simplicity. Our
results can be easily extended to negative integers.

%We begin by defining the kinds of constraints we will use to extend $\strline[\replaceall]$.
First, we describe integer constraints, which express constraints on the length or number of occurrences of symbols in words. 


\begin{definition}[Integer constraints] \label{def:intconst} 
	An atomic integer constraint over $\Sigma$ is an expression of the form
	$a_1t_1+\cdots+a_nt_n\leq d$
	where $a_1, \cdots, a_n,d\in \mathbb{Z}$ are constant integers (represented in binary), and each \emph{term} $t_i$ is either 
	\begin{enumerate}
		\item an integer variable $\mathfrak{n}$;
		\item $|x|$ where $x$ is a  string variable; or 
		\item $|x|_a$ where $x$ is string variable and $a\in \Sigma$ is a constant letter.
	\end{enumerate}
	Here, $|x|$ and $|x|_a$ denote the length of $x$ and the number of occurrences of $a$ in $x$, respectively. 
	
	An \emph{integer constraint} over $\Sigma$ is a Boolean combination of atomic integer constraints over $\Sigma$.
\end{definition}
For $\transet$ being 1T, $\strline[1PT]$ with integer constraints has been thoroughly studied in \cite{LB16} and it was shown that the extension does not increase the complexity of satisfiability checking. For $\transet$ being 1PT, we notice that $\strline[\replaceall]$ extended with length constraints is undecidable \cite{CCHLW18}, and, since the replace-all functions can be represented by 1PT, the undecidability carries over to $\strline[1PT]$ with integer constraints. 

We will show that the extension of $\strline[2T]$ with integer constraints entails undecidability, by a reduction from (a variant of) the Hilbert's 10th problem, which is well-known to be undecidable \cite{Mat93}. 
Intuitively, we want to find a solution to $f(\mathfrak{n}_1, \cdots, \mathfrak{n}_n)=g(\mathfrak{n}_1, \cdots, \mathfrak{n}_n)$ in the natural numbers, where $f$ and $g$ are polynomials with positive coefficients.
We can use the length of string variables over a unary alphabet $\{a\}$ to represent integer variables, addition can be performed with concatenation, and multiplication of $\mathfrak{m}$ and $\mathfrak{n}$ with constraints of $\strline[2T]$.
The integer constraint $|\mathfrak{m}| = |\mathfrak{n}|$ asserts the equality of $f$ and $g$.

The crux of the reduction is, similar to the case for $\strline[\replaceall]$ \cite{CCHLW18}, is to encode the multiplication of two natural numbers. Assuming $\Sigma=\{a,b\}$. We consider the following transducers:
\begin{itemize}
	\item $\mathsf{copy}$ which, given a string $x$ in $a^*$ as input, copies $x$ and additionally outputs $b$ for multiple times, and 
	
	\item $\mathsf{remove_a}$ which, given a string in $(a^*b)^*$ as input, removes all $a$'s and outputs.  
	
	\item $\mathsf{remove_b}$ which, given a string in $(a^*b)^*$ as input, removes all $b$'s and outputs.
\end{itemize}
 Notice that $\mathsf{copy}$ falls into 2T (rather than 1T) and both $\mathsf{remove_a}$ and $\mathsf{remove_b}$ are obviously in 1T. 
 
Consider the constraints 
$$x'=\mathsf{copy}(x) \wedge x''=\mathsf{remove_a}(x') \wedge  z=\mathsf{remove_b}(x') \wedge |x''|=|y|$$

It is rather straightforward to verify that the integer constraint $|x''|=|y|$ forces that $x'=\underbrace{xb}_{|y|\text{ times}}$ %contains $|y|$ copies of $xb$ 
	and thus $|z|=|y|\cdot |x|$. By encoding two natural numbers in unary by $x,y\in a^*$, we have $z\in a^*$ encoding the multiplication of the two natural numbers. 
	

\begin{theorem}\label{thm-ext-int}
	For the extension of $\strline[2T]$ with \emph{integer constraints}, the satisfiability problem is undecidable, even if only a single integer constraint of the form $|x| = |y|$ or $|x|_a = |y|_a$ is used.
	%  \mat{Changed this in reponse to reviewer 4}
\end{theorem}

\OMIT{
Character constraints, on the other hand, allow to compare symbols from different strings. The formal definitions are given as follows. 

\begin{definition}[Character constraints]
	An \emph{atomic character constraint} over $\Sigma$ is an equation of the form $x[t_1]=y[t_2]$ where 
	\begin{itemize}
		\item $x$ and $y$ are either a string variable or a constant string in $\Sigma^*$, and 
		\item $t_1$ and $t_2$ are either integer variables or constant positive integers.
	\end{itemize} 
	Here, the interpretation of $x[t_1]$ is the $t_1$-th letter of $x$. In case that $x$ does not have the $t_1$-th letter \emph{or} $y$ does not have the $t_2$-th letter, the constraint $x[t_1] = y[t_2]$ is false by convention.  
	%Similarly for $y[t_2]$.
	%\mat{What if $x$ doesn't have a $t_1$th letter?}\zhilin{how about this ?}
	
	A \emph{character constraint} over $\Sigma$ is a Boolean combination of atomic character constraints over $\Sigma$. 
\end{definition}

We also consider the constraints involving the $\indexof$ function.

\begin{definition}[$\indexof$ Constraints]
	An atomic $\indexof$ constraint over $\Sigma$ is a formula of the form $t\ \mathfrak{o}\ \indexof(s_1, s_2)$, where 
	\begin{itemize}
		\item $t$ is an integer variable, or a positive integer (recall that here we assume that the first position of a string is $1$), or the value $0$ (denoting that there is no occurrence of $s_1$ in $s_2$), 
		\item $\mathfrak{o} \in \{\ge, \le\}$, and
		%
		\item  $s_1,s_2$ are either string variables or constant strings. 
	\end{itemize}
	We consider the \emph{first-occurrence} semantics of $\indexof$.  More specifically, $t \ge \indexof(s_1, s_2)$ holds if $t$ is no less than the first position in $s_2$ where $s_1$ occurs, similarly for $t \le \indexof(s_1, s_2)$.
	%\mat{What if $s_1$ does not appear in $s_2$?}\zhilin{if $s_1$ does not appear in $s_2$, then $\indexof(s_1, s_2)=0$, as defined above.}
	
	An $\indexof$ constraint over $\Sigma$ is a Boolean combination of atomic $\indexof$ constraints over $\Sigma$.
\end{definition}

%There are two natural semantics of $\indexof$, viz., the \emph{first-occurrence} semantics and the \emph{anywhere} semantics.  More specifically, $t \ge \indexof(s_1, s_2)$ holds 
%\begin{itemize}
%	\item under the \emph{first-occurrence} semantics, if $t$ is no less than the first position in $s_2$ where $s_1$ occurs;
%	
%	\item under the \emph{anywhere} semantics, if $t$ is no less than \emph{any} position in $s_2$ where $s_1$ occurs.
%\end{itemize}  


%One reason of introducing character constraints is, apart from the use of the JavaScript string method chatAt (which is used rather frequently in JavaScript according to the benchmark \cite{}), they can also be used to define $\indexof(w,x)$ for $w\in \Sigma^*$, which is the most standard usage of IndexOf method in practice. There are in general two natural semantics, viz., the \emph{first-occurrence} semantics and the \emph{anywhere} semantics. We write $u=\indexof(w,x)$ where $u$ is an integer variable, which holds
%\begin{itemize}
%	\item under  the \emph{first-occurrence} semantics, if $u$ is the first position in $x$ where $w$ occurs;
%	
%	\item the \emph{anywhere} semantics, if $u$ is \emph{any} position in $x$ where $w$ occurs;
%\end{itemize}  

 
 Note that the use of concatenation can be further dispensed since, by Proposition~\ref{prop-concat},   concatenation is expressible by $\replaceall$ at the price of a slightly extended alphabet.  
%\mat{Is this useful?}

%: Given two polynomials (aka Diophantine equations) $f(x_1, \cdots, x_n)$ and $g(x_1,\cdots, x_n)$ with positive integral coefficients over the same set of variables $x_1, \cdots, x_n$, decide whether $f(x_1, \cdots, x_n)=g(x_1, \cdots, x_n)$ has a solution in natural numbers. It is well-known that Hilbert's 10th problem is undecidable \cite{Mat93}.

%Recall the Hilbert 10th problem, which is, for any given Diophantine equation (a polynomial equation with integer coefficients and a finite number of unknowns), to decide whether the equation has a solution with all unknowns taking integer values. It is easy to observe that given two polynomials with positive integral coefficients over the same set of variables $x_1, \cdots, x_n$, it is \emph{undecidable} to check whether $f(x_1, \cdots, x_n)=g(x_1, \cdots, x_n)$ has a solution in natural numbers. 




%
%Notice, in the above proof, besides the $\strline[\replaceall]$ formula, only a \emph{single simple} atomic integer constraint $|y_f| = |y_g|$ is used. Therefore, the proof  shows that the extension of $\strline[\replaceall]$ with only one integer constraint of the form $|x| = |y|$ entails undecidability.
%On the other hand, by utilising a further result on Diophantine equations, we will show that for the extension of $\strline[\replaceall]$ with integer constraints, even if the $\strline[\replaceall]$ formulae are simple, in the sense that their dependency graphs are of depth at most one, the satisfiability problem is still undecidable (note that no restrictions are put on the integer constraints in this case).

Notice that the extension of $\strline[\replaceall]$ with only one integer constraint of the form $|x| = |y|$ entails undecidability. \zhilin{a remark for the Buchi and Senger's result about undecidability, please check} \tl{rephrased slightly}
We remark that the undecidability result here does \emph{not} follow from the undecidability result for the extension of word equations with the letter-counting modalities in \cite{buchi}, % does \emph{not} apply to the setting here 
since the formula by \cite{buchi} is not straight-line. 

By utilising a further result on Diophantine equations, we show that for the extension of $\strline[\replaceall]$ with integer constraints, even if the $\strline[\replaceall]$ formulae are simple (in the sense that their dependency graphs are of depth at most one), the satisfiability problem is still undecidable (note that no restrictions are put on the integer constraints in this case).

% The above proof essentially establishes a link between Diophantine equations and the extension of $\strline[\replaceall]$ with integer constraints over a unary alphabet. 
%
%With a further result from Hilbert 10'th problem, we can strengthen the undecidability results shown that satisfiability of even very simple string constraints (e.g., the $\replaceall$ function is unnested) will be undecidable in conjunction with length constraints. 



%Therefore, we get the following undecidability result.

\begin{theorem}\label{thm-ext-int-strong}
	For the extension of $\strline[\replaceall]$ with integer constraints, even if $\strline[\replaceall]$ formulae are restricted to those whose dependency graphs are of depth at most one, the satisfiability problem is still undecidable.
\end{theorem}

By essentially encoding $|x|=|y|$ with \emph{character} or \emph{$\indexof$ constraints}, we show:

\begin{proposition}\label{prop-ext-ch-index}
	For the extension of $\strline[\replaceall]$ with either the character constraints or the $\indexof$ constraints, the satisfiability problem is undecidable. 
\end{proposition}


}



%!TEX root = popl2018.tex

\section{Conclusion}

In this paper, we have investigated extensively the decidability boundary of the satisfiability problem for the string constraints involving the $\replaceall$ function and regular constraints. The $\replaceall$ functions are considered in their most general form, that is, $\replaceall(x, e, y)$, where $x,y$ can be string variables, and $e$ is either a string constant/variable, or a regular expression. As the satisfiability problem is undecidable in general, we focused on the straight-line fragment. We showed that while it remains to be undecidable if the second parameter of the $\replaceall$ function is a variable, it becomes decidable, more precisely in EXPSPACE, if the second parameter is a regular expression. The decision procedure was obtained by an automata-theoretic construction, which is modular and amenable to implementations. In addition, we proved that the decision procedure is in fact PSPACE-complete for several special cases that are meaningful in practice. Finally, we show that extending the decidable straight-line fragment with any of the integer constraints, character constraints, and constraints involving the $\indexof$ function, leads to the undecidability immediately. 
Our work clarified important fundamental issues surrounding the $\replaceall$ functions in string constraint solving and provided a novel decision procedure which paved a way to a string solver that is able to fully support the $\replaceall$ function. This would be the most direct future work. 


\section{Parametric transducers}

In this section, we introduce parametric transducers.

\section{String constraints} \label{sec-core}

In this section, we define a general string constraint language that supports
concatenation, transducers, and regular constraints.
\tl{for length constraints, let's decide later whether it should be put here}
Throughout this section, we fix an alphabet $\Sigma$.
We consider the String data type $\str$, and assume a countable set of variables
$x, y, z, \cdots$ of $\str$.


\begin{definition}[Relational and regular constraints]
	Relational constraints and regular constraints are defined by the following rules,
	\[
	\begin{array}{r c l cr}
	s &\eqdef & x \mid u & \ \ & \mbox{(string terms)}\\
%	p &\eqdef & x \mid e & \ \ & \mbox{(pattern terms)}\\
	%t &\eqdef & s \mid e & \ \ & \mbox{(terms)}\\
	\varphi &\eqdef & x = s \concat s  \mid  x = T(\vec{s}) \mid \varphi \wedge \varphi & \ \ & \mbox{(relational constraints)}\\
	\psi & \eqdef & x \in e \mid \psi \wedge \psi %\mid \psi \vee \psi \mid \neg \psi
	& \ \ & \mbox{(regular constraints)} \\
	\end{array}
	\]
	where $x$ is a string variable, $u \in \Sigma^\ast$ and $e$ is a regular expression over $\Sigma$.
\end{definition}
\tl{this is not optimal. For $T$ with multiple parameters, the concatenation is redundant}

For a formula $\varphi$ (resp. $\psi$), let $\vars(\varphi)$ (resp. $\vars(\psi)$) denote the set of variables occurring in $\varphi$ (resp. $\psi$). Given a relational constraint $\varphi$, a variable $x$ is called a \emph{source variable} of $\varphi$ if $\varphi$ \emph{does not} contain a conjunct of the form $x = s_1 \concat s_2$ or $x = T(\vec{s})$.

%We then notice that, with the $\replaceall$ function in its general form, the concatenation operation is in fact redundant.

%\begin{proposition}\label{prop-concat}
%	The concatenation  operation ($\concat$) can be simulated  by the $\replaceall$ function.
%\end{proposition}
%\begin{proof}
%	It is sufficient to observe that %the concatenation operator $s_1 \concat s_2$ is redundant in the sense that
%	a relational constraint $x = s_1 \concat s_2$ can be rewritten as
%	\[x' = \replaceall(ab, a, s_1) \wedge x = \replaceall(x', b, s_2),\] where $a,b$ are two fresh letters.
%\end{proof}

%In light of Proposition~\ref{prop-concat}, in the sequel, we will \emph{dispense the concatenation operator} mostly and focus on \textbf{the string constraints that involve  the $\replaceall$ function only}.

%Another example to show the power of the $\replaceall$ function is that it can simulate the extension of regular expressions with string variables, which is  supported by the mainstream scripting languages like Python, Javascript, and PHP. For instance, $x \in y^*$ can be expressed by $x =\replaceall(x', a, y) \wedge x' \in a^*$, where $x'$ is a fresh variable and $a$ is a fresh letter.



The generality of the constraint language makes it undecidable,
even in very simple cases. To retain decidability, we follow \cite{LB16} and focus on the ``straight-line fragment" of the language. This straight-line fragment captures the structure of straight-line string-manipulating
programs.

\begin{definition}[Straight-line relational constraints]
	A relational constraint $ \varphi$ with transducers is straight-line, if $\varphi \eqdef \bigwedge \limits_{1 \le i \le m} x_i = P_i$ such that
	\begin{itemize}
		\item $x_1,\dots, x_m$ are mutually distinct,
		\item for each $i \in [m]$, all the variables in $P_i$ are either source variables, or variables from $\{x_1,\dots, x_{i-1}\}$,
	\end{itemize}
	%Occasionally we refer to $x_m$ as the output variable.
\end{definition}
%Intuitively, in a straight-line relational constraint, the dependency graph (see Definition~\ref{def:dep-graph}) of the string variables is acyclic.
%\mat{forward reference!}

\begin{remark}
	Checking whether a relational constraint $\varphi$ is straight-line can be done in linear time.
\end{remark}

\begin{definition}[Straight-line string constraints]
	A straight-line string constraint $C$ with transducers (denoted by $\strline[T]$)  is defined as $ \varphi \wedge \psi$,  where
	\begin{itemize}
		\item $\varphi$ is a straight-line relational constraint with transducers,  and
		%
		\item $\psi$ is a regular constraint.
		%
	\end{itemize}
\end{definition}



 We first introduce a graphical representation of $\strline[T]$ formulae as follows.

 \begin{definition}[Dependency graph]
	\label{def:dep-graph}
	Suppose $C= \varphi \wedge \psi$ is an $\strline[\replaceall]$ formula where the pattern parameters of the $\replaceall$ terms are regular expressions. %Let $\vars(\varphi) = \{x_1,\dots, x_m, y_1, \dots, y_n\}$, where $y_1,\dots, y_n$ are  source variables.
	Define the \emph{dependency graph} of $C$ as $G_C= (\vars(\varphi), E_C)$, such that for each $i \in [m]$, if $x_i = \replaceall(z, e_i, z')$, then $(x_i, (\rpleft, e_i), z) \in E_C$ and $(x_i, (\rpright, e_i), z') \in E_C$. A final (resp. initial) vertex in $G_C$ is a vertex in $G_C$ without successors (resp. predecessors). The edges labelled by $(\rpleft, e_i)$ and $(\rpright, e_i)$ are called the $\rpleft$-edges and $\rpright$-edges respectively. The \emph{depth} of $G_C$ is the maximum length of the paths in $G_C$. In particular, if $\varphi$ is empty, then the depth of $G_C$ is zero.
	%The $\rpleft$-length of a path $\pi$, denoted by $\rpleftlen(\pi)$, is the number of $\rpleft$-edges on $\pi$. A path of $G_C$ is a sequence $z_1 \ell_1 z_2 \dots \ell_{k-1} z_k$ such that for each $i \in [k-1]$, $(z_i, \ell_i, z_{i+1}) \in E_C$. A path is initial (resp. final) if the path starts from an initial vertex (resp. stops at a final vertex).
	% e the $\src$-nesting-depth of $z$ in $G_C$, denoted by $\srcnd_{G_C}(z)$,  as the maximum number of $\src$-edges in paths from source variables to $z$.
 \end{definition}
 Note that $G_C$ is a DAG where the out-degree of each vertex is two or zero.

\subsection{The satisfiability problem} \label{sec-sat}
In this paper, we focus on the satisfiability problem of $\strline[T]$, which is formalised as follows.

%\smallskip

\begin{quote} \centering
	\framebox{Given an $\strline[T]$ constraint $C$, decide whether $C$ is satisfiable.}
\end{quote}
\smallskip

%To approach this problem, we identify several fragments of  $\strlineTT]$, depending on whether the pattern and the replacement parameters are constants or variables.  We shall investigate extensively the satisfiability problem of the fragments of $\strline[\replaceall]$. % (see Table~\ref{tab-sum}).  Note that for $x=\replaceall (y, p, z)$, $p$ is referred to as a \emph{pattern} and $z$ is referred to as a \emph{replacement}.

%=========================================================================================================

\begin{proposition}
undecidable in general
\end{proposition}

\section{A generic decision procedure}

\zhilin{results for two-way versus one-way, with parameters or without.}

In this section, we focus on the case where the transducer $T$ has only one parameter, i.e., the relational constraint is of form $x=T(y)$ for example. We will give both upper and low bounds.

\tl{A question: here there are two ways to present the result, one is to follow \cite{LB16} to encode $\concat$, and the other is to deal with $\concat$ explicitly. Which way do you think is better?}

\subsection{Upper-bound}
We first show that, when the 2-way transducer $T$ and an NFA


\subsection{Lower-bound}


We give a lower bound for the satisfiability problem for $\strline[T]$ where the transducers are two-way transducers.
In particular, we show the problem is non-elementary.
More specifically, we show that with $(\expheight+1)$ transducers, the problem is $\expheight$-EXPSPACE-hard.
For any $\expheight$, we proceed by reduction from a tiling problem that is hard for $\expheight$-EXPSPACE.
The reduction relies on the ability to manipulate large numbers that will be used to index the tiles in a solution to the tiling problem.
Similar encodings appear in the study of higher-order programs (e.g.~\cite{J01,CW07}).

\subsubsection{Tiling Problems}

\begin{definition}[Tiling Problem]
    A \emph{tiling problem} is a tuple
    $\tup{\tiles, \hrel, \vrel, \inittile, \fintile}$
    where
        $\tiles$ is a finite set of tiles,
        $\hrel \subseteq \tiles \times \tiles$ is a horizontal matching relation,
        $\vrel \subseteq \tiles \times \tiles$ is a vertical matching relation, and
        $\inittile, \fintile \in \tiles$ are initial and final tiles respectively.
\end{definition}

A solution to a tiling problem over a $\linlen$-width corridor is a sequence
\[
    \begin{array}{c}
        \tile^1_1 \ldots \tile^1_\linlen \\
        \tile^2_1 \ldots \tile^2_\linlen \\
        \ldots \\
        \tile^\tileheight_1 \ldots \tile^\tileheight_\linlen
    \end{array}
\]
where
$\tile^1_1 = \inittile$,
$\tile^\tileheight_\linlen = \fintile$,
and for all
$1 \leq i < \linlen$
and
$1 \leq j \leq \tileheight$
we have
$\tup{\tile^j_i, \tile^j_{i+1}} \in \hrel$
and for all
$1 \leq i \leq \linlen$
and
$1 \leq j < \tileheight$
we have
$\tup{\tile^j_i, \tile^{j+1}_i} \in \vrel$.
Note, we will assume that $\inittile$ and $\fintile$ can only appear at the beginning and end of the tiling respectively.

Tiling problems characterise many complexity classes~\cite{??}.
In particular, we will use the following facts.
\begin{itemize}
\item
    For any $\linlen$-space Turing machine, there exists a tiling problem of size polynomial in the size of the Turing machine, over a corridor of width $\linlen$, that has a solution iff the $\linlen$-space Turing machine has a terminating computation.

\item
    There is a fixed
    $\tup{\tiles, \hrel, \vrel, \inittile, \fintile}$
    such that for any width $\linlen$ there is a unique solution
    \[
        \begin{array}{c}
            \tile^1_1 \ldots \tile^1_\linlen \\
            \tile^2_1 \ldots \tile^2_\linlen \\
            \ldots \\
            \tile^\tileheight_1 \ldots \tile^\tileheight_\linlen
        \end{array}
    \]
    and moreover $\tileheight$ is exponential in $\linlen$.
    \mat{Not sure if this is folklore, but it can be seen by taking a $\linlen$ space Turing machine that deterministically increments from a tape with all 0 to all 1, then encoding this as a tiling problem.}
\end{itemize}

\subsubsection{Large Numbers}

The crux of the proof is encoding large numbers that can take values between $1$ and $\expheight$-fold exponential.

A linear-length binary number could be encoded simply as a sequence of bits
\[
    b_0 \ldots b_\linlen \in \set{0,1}^\linlen \ .
\]
To aid with later constructions we will take a more oblique approach.
Let
$\tup{\tilesnum{1}, \hrelnum{1}, \vrelnum{1}, \inittilenum{1}, \fintilenum{1}}$
be a copy of the fixed tiling problem from the previous section for which there is a unique solution, whose length must be exponential in the width.
In the future, we will need several copies of this problem, hence the indexing here.
Fix a width $\linlen$ and let $\nmax{1}$ be the corresponding corridor length.
A \emph{level-1} number can encode values from $1$ to $\nmax{1}$.
In particular, for $1 \leq i \leq \nmax{1}$ we define
\[
    \tenc{1}{i} = \tile^i_1 \ldots \tile^i_\linlen
\]
where
$\tile^i_1 \ldots \tile^i_\linlen$
is the tiling of the $i$th row of the unique solution to the tiling problem.

A \emph{level-2} number will be derived from tiling a corridor of width $\nmax{1}$, and thus the number of rows will be doubly-exponential.
For this, we require another copy
$\tup{\tilesnum{2}, \hrelnum{2}, \vrelnum{2}, \inittilenum{2}, \fintilenum{2}}$
of the above tiling problem.
Moreover, let $\nmax{2}$ be the length of the solution for a corridor of width $\nmax{1}$.
Then for any
$1 \leq i \leq \nmax{2}$
we define
\[
    \tenc{2}{i} =
        \tenc{1}{1} \tile^i_1
        \tenc{1}{2} \tile^i_2
        \ldots
        \tenc{1}{\nmax{1}} \tile^i_{\nmax{1}}
\]
where
$\tile^i_1 \ldots \tile^i_{\nmax{1}}$
is the tiling of the $i$th row of the unique solution to the tiling problem.
That is, the encoding indexes each tile with it's column number, where the column number is represented as a level-1 number.

In general, a \emph{level-$\expheight$} number is of length $\expheight$-fold exponential and can encode numbers $(\expheight+1)$-fold exponential in size.
We use a copy
$\tup{\tilesnum{\expheight},
      \hrelnum{\expheight},
      \vrelnum{\expheight},
      \inittilenum{\expheight},
      \fintilenum{\expheight}}$
of the above tiling problem and use a corridor of width
$\nmax{\expheight-1}$.
We define $\nmax{\expheight}$ as the length of the unique solution to this problem.
Then, for any $1 \leq i \leq \nmax{\expheight}$ we have
\[
    \tenc{\expheight}{i} =
        \tenc{(\expheight-1)}{1} \tile^i_1
        \tenc{(\expheight-1)}{2} \tile^i_2
        \ldots
        \tenc{(\expheight-1)}{\nmax{(\expheight-1)}} \tile^i_{\nmax{(\expheight-1)}}
\]
where
$\tile^i_1 \ldots \tile^i_{\nmax{\expheight-1}}$
is the tiling of the $i$th row of the unique solution to the tiling problem.

\subsubsection{Recognising Large Numbers}

We first define a useful formula
$\goodnums{\expheight}{x}$
with a single free-variable $x$ which can only be satisfied if $x$ is of the following form, where
$\numeq$, $\numplus$, and $\numsep$
are auxiliary symbols.
\[
    \begin{array}{c}
        \goodnums{\expheight}{x} \\
        \iff \\
        x \in \brac{
            \brac{\tenc{\expheight}{1} \brac{\numeq \tenc{\expheight}{1}}^\ast}
            \numplus
            \brac{\tenc{\expheight}{2} \brac{\numeq \tenc{\expheight}{2}}^\ast}
            \numplus
            \ldots
            \numplus
            \brac{
                \tenc{\expheight}{\nmax{\expheight}}
                    \brac{\numeq \tenc{\expheight}{\nmax{\expheight}}}^\ast
            }
            \numsep
        }^\ast
    \end{array}
\]
That is, the string must contain sequences of strings that count from $1$ to $\nmax{\expheight}$.
This counting may stutter and repeat a number several times before moving to the next.
A separator $\numeq$ indicates a stutter, while $\numplus$ indicates that the next number must add one to the current number.
Finally, a $\numsep$ ends a sequence and may start again from $\tenc{\expheight}{1}$.

\paragraph{Base case $\expheight = 1$.}

We define
\[
    \goodnums{1}{x} =
        (y = \ap{T}{x} \land y \in \top)
\]
where $\top$ is a character output by $T$ when $x$ is correctly encoded.
Otherwise $T$ outputs $\bot$.

We describe informally how $T$ operates.
\mat{We may or may not give the formal definition later.}
Because $T$ is two-way, it may perform several passes of the input $x$.
Recall that $T$ requires $x$ to contain a word of the form
\[
    \brac{
        \brac{\tenc{1}{1} \brac{\numeq \tenc{1}{1}}^\ast}
        \numplus
        \brac{\tenc{1}{2} \brac{\numeq \tenc{1}{2}}^\ast}
        \numplus
        \ldots
        \numplus
        \brac{
            \tenc{1}{\nmax{1}}
                \brac{\numeq \tenc{1}{\nmax{1}}}^\ast
        }
        \numsep
    }^\ast
\]
and each $\tenc{1}{i}$ is the $i$th row of the unique solution to the tiling problem of width $\linlen$.
The passes proceed as follows.
If a passes fails, the transducer outputs $\bot$ and terminates.
If all passes succeed, the transducer outputs $\top$ and terminates.
\begin{itemize}
\item
    During the first pass $T$ verifies that the input is of the form
    \[
        \brac{
            \tilesnum{1}^\linlen
            \brac{\set{\numeq,\numplus} \tilesnum{1}^\linlen}^\ast
            \numsep
        }^\ast \ .
    \]

\item
    During the second pass the transducer verifies that the first block of
    $\tilesnum{1}^\linlen$,
    and all blocks of
    $\tilesnum{1}^\linlen$
    immediately following a $\numsep$ have $\inittilenum{1}$ as the first tile.
    Simultaneously, it can verify that all blocks of
    $\tilesnum{1}^\linlen$
    immediately preceding a $\numsep$ finish with the tile $\fintilenum{1}$.
    Moreover, it checks that $\inittilenum{1}$ and $\fintilenum{1}$ do not appear elsewhere.

\item
    During the third pass $T$ verifies the horizontal tiling relation.
    That is, every contiguous pair of tiles $\tile, \tile'$ in $x$ must be such that
    $(\tile, \tile') \in \hrelnum{1}$.
    This can easily be done by storing the last character read into the states of $T$.

\item
    The vertical tiling relation and equality checks are verified using $\linlen$ more passes.
    During the $j$th pass, the $j$th column is tested.
    The transducer $T$ stores in its state the tile in the $j$th column of the first block of $\tilesnum{1}^\linlen$ or any block immediately following $\numsep$.
    (The transducer can count to $\linlen$ in its state.)
    It then moves to the $j$th column of the next block of $\tilesnum{1}^\linlen$, remembering whether the blocks were separated with $\numeq$, $\numplus$, or $\numsep$.
    If the separator was $\numeq$ the transducer checks that the $j$th tile of the current block matches the tile stored in the state (i.e.~is equal to the preceding block).
    If the separator was $\numplus$ the transducer checks that the $j$th tile of the current block is related by $\vrelnum{1}$ to the stored tile.
    In this case the current $j$th tile is stored and the previously stored tile forgotten.
    Finally, if the separator was $\numsep$ there is nothing to check.
    If any check fails, the pass will also fail.
\end{itemize}

If all passes succeed, we know that $x$ contains a word where blocks separated by $\numeq$ are equal
(since all positions are equal, as verified individually by the final $\linnum$ passes),
blocks separated by $\numplus$ satisfy $\vrelnum{1}$ in all positions,
the $\inittilenum{1}$ tile appears at the start of all sequences separated by $\numsep$ and each such sequence ends with $\fintilenum{1}$, and
finally the horizontal tiling relation is satisfied at all times.
Thus, $x$ must be of the form
\[
    \brac{
        \brac{\tenc{1}{1} \brac{\numeq \tenc{1}{1}}^\ast}
        \numplus
        \brac{\tenc{1}{2} \brac{\numeq \tenc{1}{2}}^\ast}
        \numplus
        \ldots
        \numplus
        \brac{
            \tenc{1}{\nmax{1}}
                \brac{\numeq \tenc{1}{\nmax{1}}}^\ast
        }
        \numsep
    }^\ast
\]
as required.


\paragraph{Inductive case $\expheight$.}

We define a formula
$\goodnums{\expheight}{x}$
such that
\[
    \begin{array}{c}
        \goodnums{\expheight}{x} \\
        \iff \\
        x \in \brac{
            \brac{\tenc{\expheight}{1} \brac{\numeq \tenc{\expheight}{1}}^\ast}
            \numplus
            \brac{\tenc{\expheight}{2} \brac{\numeq \tenc{\expheight}{2}}^\ast}
            \numplus
            \ldots
            \numplus
            \brac{
                \tenc{\expheight}{\nmax{\expheight}}
                    \brac{\numeq \tenc{\expheight}{\nmax{\expheight}}}^\ast
            }
            \numsep
        }^\ast \ .
    \end{array}
\]
Assume, by induction, we have a formula
$\goodnums{\expheight-1}{x}$
which already satisfies this property (for $\expheight-1$).
We define
\[
    \goodnums{\expheight}{x} =
    \brac{
        y = \ap{T}{x}
        \land
        \goodnums{\expheight-1}{y}
    }
\]
where $T$ is a transducer that behaves as described below.
\mat{We may or may not define it formally later.}
The transducer will perform several passes to make several checks.
If a check fails it will halt and output a symbol $\bot$, which means that $y$ can no longer satisfy
$\goodnums{\expheight-1}{y}$.
During normal execution $T$ will make checks that rely on level-$(\expheight-1)$ numbers appearing in the correct sequence or being equal.
To ensure these properties hold, $T$ will write these numbers to $y$ and rely on these properties then being verified by
$\goodnums{\expheight-1}{y}$.
The passes behave as follows.
\begin{itemize}
\item
    During the first pass $T$ verifies that $x$ belongs to the regular language
    \[
        \brac{
            \brac{
                \brac{
                    \brac{
                        \brac{\tilesnum{1}^\linlen \tilesnum{2}}^\ast \tilesnum{3}
                    }^\ast
                    \cdots
                }^\ast
                \tilesnum{\expheight}
            }^\ast
            \brac{
                \set{\numeq,\numplus}
                \brac{
                    \brac{
                        \brac{
                            \brac{\tilesnum{1}^\linlen \tilesnum{2}}^\ast \tilesnum{3}
                        }^\ast
                        \cdots
                    }^\ast
                    \tilesnum{\expheight}
                }^\ast
            }^\ast
            \numsep
         }^\ast
         \ .
    \]
    This can be done with a polynomial number of states.

\item
    During the second pass $T$ will verify that the first instance of
    $\tilesnum{\expheight}$
    appearing in the word or after a $\numsep$ is $\inittilenum{\expheight}$.
    Similarly, the final instance of any
    $\tilesnum{\expheight}$
    before any $\numsep$ is $\fintilenum{\expheight}$.
    Moreover, it checks that
    $\inittilenum{\expheight}$
    and
    $\fintilenum{\expheight}$
    do not appear elsewhere.

\item
    During the third pass $T$ will verify the horizontal tiling relation
    $\hrelnum{\expheight}$.
    Each block (separated by $\numeq$, $\numplus$, or $\numsep$) is checked in turn.
    There are two components to this.
    \begin{itemize}
    \item
        The indexing of the tiles must be correct.
        That is, the first tile of the block must be indexed
        $\tenc{\expheight-1}{1}$,
        the second
        $\tenc{\expheight-1}{2}$,
        through to
        $\tenc{\expheight-1}{\nmax{\expheight-1}}$.
        Thus, $T$ copies directly the instance of
        $\brac{
            \brac{
                \brac{\tilesnum{1}^\linlen \tilesnum{2}}^\ast \tilesnum{3}
            }^\ast
            \cdots
        }^\ast$
        preceding each
        $\tilesnum{\expheight}$
        to the output tape, followed immediately by
        $\numplus$
        as long as the character after
        $\tilesnum{\expheight}$
        is not a separator from
        $\set{\numeq,\numplus,\numsep}$.
        Otherwise, it is the end of the block and $\numsep$ is written.

        Hence,
        $\goodnums{\expheight-1}{y}$
        will verify that the output for each block is
        $\tenc{\expheight-1}{1}
         \numplus \cdots \numplus
         \tenc{\expheight-1}{\nmax{\expheight-1}}$
        which enforces that the indexing of the tiles is correct.

    \item
        Horizontally adjacent tiles must satisfy
        $\hrelnum{\expheight}$.
        This is done by simply storing the last read tile from
        $\tilesnum{\expheight}$
        in the state of $T$.
        Then whenever a new tile from
        $\tilesnum{\expheight}$
        is seen without a separator $\numeq$, $\numplus$, or $\numsep$, then it can be checked against the previous tile and
        $\hrelnum{\expheight}$.
    \end{itemize}

\item
    The transducer $T$ then performs a non-deterministic number of passes to check the vertical tiling relation.
    We will use
    $\goodnums{\expheight-1}{y}$
    to ensure that $T$ in fact performs
    $\nmax{\expheight-1}$
    passes, the first checking the first column of the tiling over
    $\tilesnum{\expheight}$,
    the second checking the second column, and so on up to the
    $\nmax{\expheight-1}$th column.

    Note, we know from the previous pass that each row of the tiling is indexed correctly.
    In the sequence, let us use the term ``session'' to refer to the sequences of characters separated by $\numsep$.

    Each pass of $T$ checks a single column (across all sessions).
    At the start of each session, $T$ moves non-deterministically to the start of some block
    $\brac{
        \brac{
            \brac{\tilesnum{1}^\linlen \tilesnum{2}}^\ast \tilesnum{3}
        }^\ast
        \cdots
    }^\ast
    \tilesnum{\expheight}$
    (without passing $\numeq$, $\numplus$, or $\numsep$).
    It then copies the tiles from
    $\brac{
        \brac{
            \brac{\tilesnum{1}^\linlen \tilesnum{2}}^\ast \tilesnum{3}
        }^\ast
        \cdots
    }^\ast$
    to $y$ and saves the tile from
    $\tilesnum{\expheight}$
    in its state before moving to the next separator from
    $\set{\numeq,\numplus,\numsep}$.
    In the case of $\numsep$ nothing needs to be checked and $T$ continues to the next session or finishes the pass if there are no more sessions.
    In the case of $\numeq$ or $\numplus$ the transducer remembers this separator and moves non-deterministically to the start of some block (without passing another $\numeq$, $\numplus$, or $\numsep$).
    It then writes $\numeq$ to $y$ as it is intended that $T$ choose the same column as before.
    This will be verified by
    $\goodnums{\expheight-1}{y}$.
    To aid with this $T$ copies the tiles from
    $\brac{
        \brac{
            \brac{\tilesnum{1}^\linlen \tilesnum{2}}^\ast \tilesnum{3}
        }^\ast
        \cdots
    }^\ast$
    to $y$.
    It can then check the tile from
    $\tilesnum{\expheight}$.
    If the remembered separator was $\numeq$ then this tile must match the saved one.
    If it was $\numplus$ then this tile must be related by
    $\vrelnum{\expheight}$
    to the saved one.
    If this succeeds , $T$ stores the new tile and forgets the old and continues to the next separator to continue checking
    $\vrelnum{\expheight}$.

    At the end of the pass (checking a single column from all sessions) then $T$ will either have failed and written $\bot$ or written a sequence of level-$(\expheight-1)$ numbers to $y$ separated by $\numeq$.
    That is
    \[
        \tenc{\expheight-1}{i_1}
        \numeq
        \cdots
        \numeq
        \tenc{\expheight-1}{i_{\alpha}}
    \]
    for some $\alpha$.
    Since, by induction,
    $\goodnums{\expheight-1}{y}$
    is correct, then the formula can only be satisfied if $T$ chose the same position in each row.
    That is
    $i_1 = \cdots = i_\alpha$.
    Thus, the vertical relation for the $i_1$th column has been verified.

    At this point $T$ can either write $\numsep$ and terminate or perform another pass (non-deterministically).
    In the latter case, it outputs $\numplus$, moves back to the beginning of the tape, and starts again.
    Thus, after a number of passes, $T$ will have written
    \[
        \brac{
            \tenc{\expheight-1}{i^1_1}
            \numeq
            \cdots
            \numeq
            \tenc{\expheight-1}{i^1_{\alpha_1}}
        }
        \numplus
        \cdots
        \numplus
        \brac{
            \tenc{\expheight-1}{i^\beta_1}
            \numeq
            \cdots
            \numeq
            \tenc{\expheight-1}{i^\beta_{\alpha_\beta}}
        }
        \numsep
    \]
    for some $\beta$, $\alpha_1$, \ldots, $\alpha_\beta$.
    Since
    $\goodnums{\expheight-1}{y}$
    will only accept such sequences of the form
    \[
        \brac{
            \tenc{\expheight-1}{1}
            \brac{
                \numeq \tenc{\expheight-1}{1}
            }^\ast
        }
        \numplus
        \cdots
        \numplus
        \brac{
            \tenc{\expheight-1}{\nmax{\expheight-1}}
            \brac{
                \numeq \tenc{\expheight-1}{\nmax{\expheight-1}}
            }^\ast
        }
        \numsep
    \]
    we know that $T$ must check each vertical column in turn, from $1$ to
    $\nmax{\expheight-1}$.
\end{itemize}

Thus, at the end of all passes, if $T$ has not output $\bot$ it has verified that $x$ is a correct encoding of a solution to
$\tup{\tilesnum{\expheight},
      \hrelnum{\expheight},
      \vrelnum{\expheight},
      \inittilenum{\expheight},
      \fintilenum{\expheight}}$.
That is, together with
$\goodnums{\expheight-1}{y}$
we know that
    the word is of the correct format,
    each row has a tile for each index and these indices appear in order,
    the horizontal relation is respected, and
    the vertical tiling relation is respected.
If $x$ is not a correct encoding then $T$ will not be able to produce a $y$ that satisfies
$\goodnums{\expheight-1}{y}$.


\subsubsection{Reducing from a Tiling Problem}

Now that we are able to encode large numbers, we can encode an $\expheight$-EXPSPACE-hard tiling problem as a satisfiability problem of $\strline[T]$ with two-way transducers.
In fact, most of the technical work has been done.

Thus, fix a tiling problem
$\tup{\tiles, \hrel, \vrel, \inittile, \fintile}$
that is $\expheight$-EXPSPACE-hard.
In particular, we allow a corridor $\nmax{\expheight}$ tiles wide.
We use the formula
\[
    \probfmla = \brac{y = \ap{T}{x} \land \goodnums{\expheight}{y}}
\]
where $T$ is defined exactly as in the inductive case of
$\goodnums{\expheight}{y}$
except the tiling problem used is
$\tup{\tiles, \hrel, \vrel, \inittile, \fintile}$
rather than
$\tup{\tilesnum{\expheight},
      \hrelnum{\expheight},
      \vrelnum{\expheight},
      \inittilenum{\expheight},
      \fintilenum{\expheight}}$.

A satisfying tiling
\[
    \begin{array}{c}
        \tile^1_1 \ldots \tile^1_{\nmax{\expheight}} \\
        \cdots \\
        \tile^\tileheight_1 \ldots \tile^\tileheight_{\nmax{\expheight}}
    \end{array}
\]
can be encoded
\[
    \tenc{\expheight}{1} \tile^1_1
    \cdots
    \tenc{\expheight}{\nmax{\expheight}} \tile^1_{\nmax{\expheight}}
    \numplus
    \cdots
    \numplus
    \tenc{\expheight}{1} \tile^\expheight_1
    \cdots
    \tenc{\expheight}{\nmax{\expheight}} \tile^\expheight_{\nmax{\expheight}}
    \numsep
\]
which will satisfy $\probfmla$ in the same way as a correct input to
$\goodnums{\expheight}{y}$.
To see this, note that
$\tenc{\expheight}{1} \tile^i_1
 \cdots
 \tenc{\expheight}{\nmax{\expheight}} \tile^i_{\nmax{\expheight}}$
acts like some
$\tenc{\expheight+1}{i}$.

In the opposite direction, assume some input satisfying $\probfmla$.
Arguing as in the encoding of large numbers, this input must be of the form
\[
    \brac{
        \brac{\tilerow_1 \brac{\numeq \tilerow_1}^\ast}
        \numplus
        \brac{\tilerow_2 \brac{\numeq \tilerow_2}^\ast}
        \numplus
        \ldots
        \numplus
        \brac{
            \tilerow_\tileheight
            \brac{\numeq \tilerow_\tileheight}^\ast
        }
        \numsep
    }^\ast
\]
where each $\tilerow_i$ is a row of a correct solution to the tiling problem.

Thus, with $\expheight+1$ transducers, we can encode a $\expheight$-EXPSPACE-hard problem.


%==========================================================================================

\section{Two-way transducers with length constraints}

This section is dynamical: we hope for the best of Anothy's result; in case it does not work, we have two possible backups: (1) reversal-bounded 2-way transducers; (2) using reversal-bounded counter machines to represent (both regular and length) constraints

%===========================================================================================

\section{One-way transducers with variables}

Not quite sure whether we need this section, it might be just a simple generalisation of the popl'18 paper, or be subsumed by the next section ; we will see.

%========================================================================================

\section{Two-way transducers with variables}

\subsection{Pre-image computation of 2-way transducers}

Let $\vec{y}=\{y_1, \cdots, y_m\}$.

The general idea is to encode a general string manipulating function $f(x, \vec{y})$ as a NFT $T$ over $\Sigma$ and $\Sigma\cup\{\vec{y}\}$. The question for the pre-image computation is formalised as follows:
\begin{itemize}
	\item INPUT: A NFT $T$, a regular language $\mathcal{A}$.
	\item OUTPUT: $(L^{(0)}_i, L^{(1)}_i, \cdots, L^{(m)}_i )_{i=1}^\ell$, such that
	\[\exists z\in\mathcal{A} \wedge z=f(x, \vec{y})\mbox{ iff }\exists k. x\in L^{(0)}_k \wedge y_i\in L^{(i)}_k \]
\end{itemize}

%===========================================================================================

%\section{Matt's pet :-), maybe another paper}
%
%\begin{definition}[multi-tape transducer, with k input tapes, and one output tape.]
%	\tl{Matt, please elaborate}
%	The
%	k input tapes follow a stack discipline: tape i can only move if the
%	head position of all tapes $j > i$ is 0
%\end{definition}


\section{Conclusion}

%==============================================================================================
\newpage

% Bibliography
\bibliographystyle{plain}
\bibliography{string}

\appendix

\end{document}
