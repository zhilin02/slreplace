%!TEX root = main.tex

\section{Proof of Theorem \ref{thm-generic-dec} in Section~\ref{sec:algo}}\label{app:algo}

In this proof, we will use the following proposition.
\begin{proposition}\label{prop-conj-fa-prod}
For a pair of conjunctive \FA{}s $\Aut_1 = ((\controls_1, \transrel_1), S_1)$ and $\Aut_2 = ((\controls_2, \transrel_2), S_2)$, a conjunctive \FA{} $\Aut = (\controls_1 \times \controls_2, \transrel_1 \times \transrel_2, S)$ such that $|S| = \max(|S_1|, |S_2|)$ can be constructed to recognise $\Lang(\Aut_1) \cap \Lang(\Aut_2)$. 
\end{proposition}
\begin{proof}
%\paragraph{Proof of Proposition~\ref{prop-conj-fa-prod}.}
Let $\Aut_1 = ((\controls_1, \transrel_1), S_1)$ and $\Aut_2 = ((\controls_2, \transrel_2), S_2)$ be two conjunctive \FA{}s such that $S_1 = \{(p_1, p'_1),\ldots, (p_m, p'_m)\}$ and $S_2 = \{(q_1, q'_1), \ldots, (q_n, q'_n)\}$ with $m \le n$. Then we construct a conjunctive \FA{} $\Aut = (\controls_1 \times \controls_2, \transrel_1 \times \transrel_2, S)$, where 
$$S= \{((p_i, q_i), (p'_i, q'_i)) \mid i \in [m]\} \cup \{((p_m, q_{m+i}), (p'_m, q_{m+i})) \mid i \in [n-m]\}.$$ 
Note $|S|  = \max(|S_1|, |S_2|)$. Evidently $\Lang(\Aut) = \Lang(\Aut_1) \cap \Lang(\Aut_2)$. 
\end{proof}

\paragraph{Spelling out the algorithm in full.}
To facilitate the complexity analysis, we will spell out the modified algorithm 
in detail. Let $S$ be a symbolic execution satisfying \prerec{} . 
We give a nondeterministic algorithm to decide the path feasibility of
$S$. 
%\tl{we need to unify the name, path feasibility or satisfiability?}\zhilin{path feasibility}
By definition, $S$ has $\rcdep(S)$ assignments.
%let $\cE_{\rcdep(S)}(x) = \cE(x)$.
First, for each variable $x \in \vars(S)$,
we compute a collection $\cE_{\rcdep(S)}(x)$ of conjunctive \FA{}
$\Aut$ appearing in a regular constraint $x \in \Aut$ as an assertion of $S$.
%conjunctive FAs over the alphabet $\Sigma$, 
%, by utilising the dependency graph $\cG(\varphi)$. 
% , in a top-down manner.
Note that at this stage each conjunctive \FA{} in $\cE_{\rcdep(S)}(x)$ is 
actually just a normal \FA{}. 
 %$\Aut$ $\left\{\Aut \mid x\in \Aut \text{ is a conjunct of }\psi \right\}$ for each variable $x$. 

%Starting from $\cG_0$ we repeat the following procedure until %we reach some $i$ where 
%$\cG_i$ becomes empty, i.e., a graph without edges.

\OMIT{
We start from the last assignment of $S$,  set $i:=\rcdep(S)$, and construct $\cE_{\rcdep(S)}(x)$ as follows: For each %conjunct 
%\mat{Can we use different terminology: we now have conjunctive NFA and conjuncts, which risks confusion}
$x \in \Aut$ appearing in an assertion of $S$, where $\Aut$ is an FA, 
%let $\Aut = (\controls, q_0, \transrel, q_F)$, 
we \emph{nondeterministically} select one state $q \in \finals$ and include $((\controls, \transrel), \{(q_0, q)\})$ into $\cE_{\rcdep(S)}(x)$. Then we iterate the following procedure to compute $\cE_{i}$ until $i=0$.  
}
%
%select (nondeterministically) a vertex $x$ of $\cG_i$ such that $x$ has no predecessors and has $k+1$ outgoing edges ($k\geq 0$) via edges $(x, (\Transducer, 0), y_0)$ and $(x, (\Transducer, j), y_j)$ with $j \in [k]$ in $\cG_i$. 
%(Intuitively this corresponds to a relational constraint $x=T(y_0, \vec{y})$ for $\vec{y}=(y_1, \ldots, y_k)$.
%Note that two different outgoing edges of $x$ in $\cG_i$ may correspond to the same variable.)

For the $i$-th iteration, suppose the $i$-th assignment of $S$ is $y:= f(x_1,
\cdots, x_\arity)$ with $\arity\geq 1$, and 
$\cE_i(y)=\{\Aut_1, \ldots, \Aut_n\}$, 
where $\Aut_j$ is a conjunctive FA for each $j \in [n]$.
%$(\Sigma, Q_j, q_{0,j}, F_j, \delta_j)$ 
By \prerec{}, we can enumerate a conjunctive representation
$(\Aut^{(\iota)}_{j, 1}, \ldots, \Aut^{(\iota)}_{j, \arity})$ of
each disjunct of
$f^{-1}(\Aut_j)$ in $\ell(|f|,|\Aut_j|)$ space.
%enumerate a representation of which, say 
%\tl{zhilin: why the index starts from 0? it is a $k$-ary relation, isn't it?} 
%can be computed effectively. 
Then $\cE_{i-1}(x)$ for $x \in  \{x_1,\cdots, x_\arity\}$ is constructed as follows: %and $\cG_{i+1}$ are computed as follows:
\begin{enumerate}
\item For each $j \in [n]$, nondeterministically select a disjunct 
    conjunctively represented by the tuple
    $\left(\Aut_{j, 1}, \ldots, \Aut_{j, \arity}\right)$.
%
\item For each $i \in \{1,\ldots, \arity\}$, let
\[
    \cE_{i-1}(x_i):= \cE_{i}(x_i) \cup \left\{\Aut_{j, i} \mid  j \in [n]
        \right\}.
\]
[For each $x{\notin} \{x_1,\ldots, x_\arity\}$, let $\cE_{i-1}(x) := 
        \cE_i(x)$.]
%We set $\cE_{i+1}(x) = \emptyset$.
%
%\item Let $\cG_{i+1}:= \cG_i \backslash \{(x, (\Transducer, j), y_j) \mid j \in \{0\} \cup [k]\}$.
\end{enumerate}
%For each iteration, $i$ is updated by  $i: = i+1$.
%\end{enumerate}
%
After all assignments of $S$ are scanned,  let $\cE(x):=\cE_{0}(x)$ for each variable $x$.
To decide the path feasibility of $S$, we simply show that each collection
$\cE(x)$ (for each $x \in \vars(S)$) of conjunctive \FA{} has a nonempty 
language intersection.


%%=================================================================
%%=================================================================
\hide{
%\begin{example}
\tl{tbh, I think the algo is clear enough and the example does not add too much.}\zhilin{maybe, but the example is at least more concrete. Moreover, the description of the algorithm is short and dry, maybe just use the example to let the reader spend a bit more time on the algorithm and digest, although with some redundancy. We may remove it if we run out of space. May decide this later.}
Let us use an example to help the reader understand the computation of $\cE_{i+1}$ from $\cE_i$.  Suppose that in $\cG_i$, to compute $\cE_{i+1}$, we select a variable $x$, which has no predecessors and three outgoing edges, say $(x, (\Transducer, 0), y)$, $(x, (\Transducer, 1), z)$, and $(x, (\Transducer, 2), y)$, where $y$ and $z$ are two distinct variables, moreover, $\cE_i(x) = \{\Aut_1, \Aut_2\}$. Let us also assume that  $\Pre_{\Transducer}(\Aut_1)$ (resp. $\Pre_{\Transducer}(\Aut_2)$) is represented by $(\Aut^{(j')}_{1, 0}, \Aut^{(j')}_{1, 1}, \Aut^{(j')}_{1, 2})_{ j'  \in [2]}$ (resp. $(\Aut^{(j')}_{2, 0}, \Aut^{(j')}_{2, 1}, \Aut^{(j')}_{2, 2})_{ j'  \in [3]}$). If for $j = 1$ (resp. $j=2$), $(\Aut^{(1)}_{1, 0}, \Aut^{(1)}_{1, 1}, \Aut^{(1)}_{1, 2})$  (resp. $\left(\Aut^{(3)}_{2, 0}, \Aut^{(3)}_{2, 1}, \Aut^{(3)}_{2, 2}\right)$)  is selected, then $\cE_{i+1}(y) = \cE_i(y) \cup \{\Aut^{(1)}_{1, 0}, \Aut^{(1)}_{1, 2}, \Aut^{(3)}_{2, 0} \cup \Aut^{(3)}_{2, 2}\}$ and $\cE_{i+1}(z) = \cE_{i}(z) \cup \{\Aut^{(1)}_{1, 1}, \Aut^{(3)}_{2, 1}\}$. 
\tl{I have not updated this part as likely the proof will go to appendix}
}
%%=================================================================
%%=================================================================

    \OMIT{
To decide the path feasibility of $S$, we have the following nondeterministic algorithm: first (nondeterministically) construct the sets $\cE(x)$ for $x \in \vars(S)$ as detailed above, then 
%guessing an accepting run of the product of NFAs 
checking the nonemptiness of the product of conjunctive \FA{}s in $\cE(x)$ for each $x \in \vars(S)$.
}


\paragraph{Detailed complexity analysis.}
For each $i$, 
%\begin{itemize}
%\item 
let $M_i$ be the maximum number of elements in $\cE_i(x)$ for $x  \in \vars(S)$,
%\item 
and $N_i$ be the maximum size of the conjunctive \FA{}s in $\bigcup \limits_{x \in \vars(S)} \cE_i(x)$.
%\end{itemize}
Then we have $M_{i-1} \le (\rcdim(S)+1)M_i $. Moreover, since each string function $f$ satisfies   $|f| \le \rcphi(S)$, we have that $N_{i-1} \le \ell(|f|, N_i) \le \ell(\rcphi(S), N_i)$ (note that we have assumed that $\ell$ is monotonic). Because for each $x \in \vars(S)$, $\cE_{\rcdep(S)}(x)$ contains at most $\rcreg(S)$ elements, and 
each conjunctive \FA{} in $\cE_{\rcdep(S)}(x)$ is of size bounded by $\rcpsi(S)$, we have that for each $x \in \vars(S)$, $\cE(x)$ contains at most $(\rcdim(S)+1)^{\rcdep(S)}\rcreg(S)$ elements, and each conjunctive \FA{} in $\cE(x)$ is of size bounded by $\ell^{\langle \rcdep(S) \rangle}(\rcphi(S), \rcpsi(S))$. 
We emphasise that, according to the \prerec{} assumption, the construction of these \FA{}s can be done in nondeterministic $\ell^{\langle \rcdep(S) \rangle}(\rcphi(S), \rcpsi(S))$ space as well. 
Therefore, according to Proposition~\ref{prop-conj-fa-prod}, for each $x \in \vars(S)$, the conjunctive product \FA{} $\Aut_x=((\controls_x, \transrel_x), S_x)$ of these conjunctive \FA{}s  in $\cE(x)$ is of size 
$$(\ell^{\langle \rcdep(S) \rangle}(\rcphi(S), \rcpsi(S)))^{(\rcdim(S)+1)^{\rcdep(S)} \rcreg(S)},$$
and $|S_x| \le (\ell^{\langle \rcdep(S) \rangle}(\rcphi(S), \rcpsi(S)))^2$. Therefore, the size of the (standard) \FA{} corresponding to $\Aut_x$ is 
$$(\ell^{\langle \rcdep(S) \rangle}(\rcphi(S), \rcpsi(S)))^{(\rcdim(S)+1)^{\rcdep(S)} \rcreg(S) (\ell^{\langle \rcdep(S) \rangle}(\rcphi(S), \rcpsi(S)))^2}.$$

Since the nonemptiness of an \FA{} can be solved in nondeterministic logarithmic space, we conclude that the nonemptiness of $\Aut_x$ can be solved in 
{\small
$$(\rcdim(S)+1)^{\rcdep(S)} \rcreg(S) (\ell^{\langle \rcdep(S) \rangle}(\rcphi(S), \rcpsi(S)))^2 \log \ell^{\langle \rcdep(S) \rangle}(\rcphi(S), \rcpsi(S))$$
}
space.

In summary, the aforementioned nondeterministic algorithm takes 
$$|\vars(S)|(\rcdim(S)+1)^{\rcdep(S)}  \rcreg(S) \left(\ell^{\langle \rcdep(S) \rangle}(\rcphi(S), \rcpsi(S)) \right)^c$$
 space for some constant $c > 0$.
