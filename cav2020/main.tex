 \documentclass{llncs}
\usepackage{minted}
%\documentclass[envcountsame, fleqn]{llncs}
  \usepackage{amsmath, amssymb, latexsym}
  \usepackage{graphicx}
\usepackage{epic,eepic}

%\pagestyle{plain}
\usepackage{listings}
\usepackage{psfrag}
\usepackage{rotating}

\usepackage{url}
\usepackage{amssymb,amsthm, epsfig,amstext}
\usepackage{txfonts}
\usepackage{algorithmic}
\usepackage[linesnumbered,ruled,noend]{algorithm2e}
\usepackage{graphicx}

\usepackage{mathrsfs}

\usepackage{color, colortbl}




\setminted{numbersep=5pt, xleftmargin=12pt}
\usemintedstyle{friendly} 

\usepackage{todonotes}
\usepackage{multirow}
\usepackage{float,color}
\usepackage{picinpar,color,xcolor,wrapfig}
\setcounter{secnumdepth}{4}
\sloppy

\pagestyle{plain}

%%%%%%%%%%%%%%%   MACROS  %%%%%%%%%%%%%%%%%%


%!TEX root = main.tex

\newcommand{\brac}[1]{\left( #1 \right)}
\newcommand{\tup}[1]{\left( #1 \right)}
\newcommand{\set}[1]{\{ #1 \}}
\newcommand{\sequence}[2]{(#1, \ldots, #2)}
\newcommand{\couple}[2]{(#1,#2)}
\newcommand{\pair}[2]{(#1,#2)}
\newcommand{\triple}[3]{(#1,#2,#3)}
\newcommand{\quadruple}[4]{(#1,#2,#3,#4)}
\newcommand{\tuple}[2]{(#1,\ldots,#2)}
\newcommand{\Nat}{\ensuremath{\mathbb{N}}}
\newcommand{\Rat}{\ensuremath{\mathbb{Q}}}
\newcommand{\Rea}{\ensuremath{\mathbb{R}}}
\newcommand{\Zed}{\ensuremath{\mathbb{Z}}}
%\newcommand{\true}{\top}
%\newcommand{\false}{\perp}
\newcommand{\bottom}{\perp}
%% \newcommand{\powerset}[1]{{\cal P}(#1)}
\newcommand{\npowerset}[2]{{\cal P}^{#1}(#2)}
\newcommand{\finitepowerset}[1]{{\cal P}_f(#1)}
\newcommand{\level}[2]{L_{#1}(#2)}
\newcommand{\card}[1]{\mbox{card}(#1)}
\newcommand{\range}[1]{\mathtt{ran}(#1)}
\newcommand{\astring}{s}

\newcommand{\Cc}{\mathcal{C}}


\newcommand {\notof}{\ensuremath{\neg}}
\newcommand {\myand}{\ensuremath{\wedge}}
\newcommand {\myor}{\ensuremath{\vee}}
\newcommand {\mynext}{\mbox{{\sf X}}}
\newcommand {\until}{\mbox{{\sf U}}}
\newcommand {\sometimes}{\mbox{{\sf F}}}
\newcommand {\previous}{\mynext^{-1}}
\newcommand {\since}{\mbox{{\sf S}}}
\newcommand {\fminusone}{\mbox{{\sf F}}^{-1}}
\newcommand {\everywhere}[1]{\mbox{{\sf Everywhere}}(#1)}



\newcommand{\aatomic}{{\rm A}}
\newcommand{\aset}{X}
\newcommand{\asetbis}{Y}
\newcommand{\asetter}{Z}

\newcommand{\avarprop}{p}
\newcommand{\avarpropbis}{q}
\newcommand{\avarpropter}{r}
\newcommand{\varprop}{{\rm PROP}} % Set of atomic propositions (for a given logic)

% formulae

\newcommand{\aformula}{\astateformula} % a formula
\newcommand{\aformulabis}{\astateformulabis} % another formula (when at least 2 are present)
\newcommand{\aformulater}{\astateformulater} % another formula (when at least 3 are present)
\newcommand{\asetformulae}{X}
\newcommand{\subf}[1]{sub(#1)}

\newcommand{\aautomaton}{{\mathbb A}}
\newcommand{\aautomatonbis}{{\mathbb B}}

\newcommand {\length}[1] {\ensuremath{|#1|}}



% Equivalences
\newcommand{\egdef}{\stackrel{\mbox{\begin{tiny}def\end{tiny}}}{=}} % =def=
\newcommand{\eqdef}{\stackrel{\mbox{\begin{tiny}def\end{tiny}}}{=}} % =def=
\newcommand{\equivdef}{\stackrel{\mbox{\begin{tiny}def\end{tiny}}}{\equivaut}} % <=def=>
\newcommand{\equivaut}{\;\Leftrightarrow\;}

\newcommand{\ainfword}{\sigma}

\newcommand{\amap}{\mathfrak{f}}
\newcommand{\amapbis}{\mathfrak{g}}

\newcommand{\step}[1]{\xrightarrow{\!\!#1\!\!}}
\newcommand{\backstep}[1]{\xleftarrow{\!\!#1\!\!}}

\newcommand {\aedge}[1] {\ensuremath{\stackrel{#1}{\longrightarrow}}}
\newcommand {\aedgeprime}[1] {\ensuremath{\stackrel{#1}{\longrightarrow'}}}
\newcommand {\afrac}[1] {\ensuremath{\mathit{frac}(#1)}}
\newcommand {\cl}[1] {\ensuremath{\mathit{cl}(#1)}}
\newcommand {\sfc}[1] {\ensuremath{\mathit{sfc}(#1)}}
\newcommand {\dunion} {\ensuremath{\uplus}}
\newcommand {\edge} {\ensuremath{\longrightarrow}}
\newcommand {\emptyword}{\ensuremath{\epsilon}}
\newcommand {\floor}[1] {\ensuremath{\lfloor #1 \rfloor}}
\newcommand {\intersection} {\ensuremath{\cap}}
\newcommand {\union} {\ensuremath{\cup}}
\newcommand {\vals}[2] {\ensuremath{\mathit{val}_{#2}(#1)}}



\newcommand {\pspace} {\textsc{pspace}}
\newcommand {\nlogspace} {\textsc{nlogspace}}
\newcommand {\logspace} {\textsc{logspace}}
\newcommand {\expspace} {\textsc{expspace}}
\newcommand {\np} {\textsc{np}}
\newcommand {\threeexptime} {\textsc{3exptime}}
\newcommand {\polytime} {\textsc{p}}
\newcommand{\twoexpspace}{\textsc{2expspace}}
\newcommand{\threeexpspace}{\textsc{3expspace}}
\newcommand {\nexptime} {\textsc{nexptime}}



\newcommand{\aalphabet}{\Sigma}     % an alphabet, A is already used for atoms
\newcommand{\aword}{\mathfrak{u}}
\newcommand{\awordbis}{\mathfrak{v}}



\newcommand{\aassertion}{P}
\newcommand{\aassertionbis}{Q}
\newcommand{\aexpression}{e}
\newcommand{\aexpressionbis}{f}
\newcommand{\avariable}{\mathtt{x}}
\newcommand{\uniquevar}{\mathtt{u}}
\newcommand{\uniquevarbis}{\mathtt{v}}
\newcommand{\avariablebis}{\mathtt{y}}
\newcommand{\avariableter}{\mathtt{z}}
\newcommand{\nullconstant}{\mathtt{null}}
\newcommand{\nilvalue}{nil}
\newcommand{\emptyconstant}{\mathtt{emp}}
\newcommand{\infheap}{\mathtt{inf}}
\newcommand{\saturated}{\mathtt{Saturated}}

\newcommand{\astateformula}{\phi}
\newcommand{\astateformulabis}{\psi}
\newcommand{\astateformulater}{\varphi}
%%
\newcommand{\separate}{\ast}
\newcommand{\sep}{\separate}
\newcommand{\size}{\mathtt{size}}
\newcommand{\sizeeq}[1]{\mathtt{size} \ = \ #1}
\newcommand{\alloc}[1]{\mathtt{alloc}(#1)}
\newcommand{\allocb}[2]{\mathtt{alloc}^{-1}[#2](#1)}
\newcommand{\isol}[1]{\mathtt{isoloc}(#1)}
\newcommand{\icell}{\mathtt{isocell}}
\newcommand{\malloc}{\mathtt{malloc}}
\newcommand{\cons}{\mathtt{cons}}
\newcommand{\new}{\mathtt{new}}
\newcommand{\free}[1]{\mathtt{free} #1}
\newcommand{\maxform}[1]{\mathtt{maxForms}(#1)}
\newcommand{\locations}[1]{\mathtt{loc}(#1)}
\newcommand{\values}{\mathtt{Val}}
\newcommand{\aheap}{\mathfrak{h}}
\newcommand{\avaluation}{\mathfrak{V}}
\newcommand{\heaps}{\mathcal{H}}
\newcommand{\astore}{\mathfrak{s}}
\newcommand{\stores}{\mathcal{S}}
\newcommand{\amodel}{\mathfrak{M}}
\newcommand{\alabel}{\ell}

\newcommand{\aprogram}{\mathtt{PROG}}
\newcommand{\programs}{\mathtt{P}}
\newcommand{\ctprograms}{\programs^{ct}}
\newcommand{\aninstruction}{\mathtt{instr}}
\newcommand{\ainstruction}{\mathtt{instr}}
\newcommand{\instructions}{\mathtt{I}}
\newcommand{\aguard}{\ensuremath{g}}
\newcommand{\guards}{\ensuremath{G}}
\newcommand{\domain}[1]{\mathtt{dom}(#1)}
\newcommand{\memory}{\stores\times\heaps}
\newcommand{\skipinstruction}{\mathtt{skip}}

\newcommand{\execution}{\mathtt{comp}}
\newcommand{\aux}{\mathtt{embd}}
\newcommand{\runof}{run}
\newcommand{\anexecution}{e}


\newcommand{\aletter}{\ensuremath{a}}
\newcommand{\aletterbis}{\ensuremath{b}}
\newcommand{\alocation}{\mathfrak{l}}

\newcommand{\pointsl}[1]{\stackrel{#1}{\hookrightarrow}}
\newcommand{\ppointsl}[1]{\stackrel{#1}{\mapsto}}
\newcommand{\ourhook}[1]{\stackrel{#1}{\hookrightarrow}}
\newcommand{\ltrue}{{\sf true}}
\newcommand{\lfalse}{{\sf false}}


\newcommand{\variables}{\mathtt{FVAR}}
\newcommand{\pvariables}{\mathtt{PVAR}}
\newcommand{\secvariables}{\mathtt{SVAR}}
\newcommand{\logique}[1]{\mathtt{FO}(#1)}



\newcommand{\atranslation}{\mathfrak{t}}
\newcommand{\nbpred}[1]{\widetilde{\sharp #1}}
\newcommand{\nbpredstar}[1]{\widetilde{\sharp #1}^{\star}}
\newcommand{\isolated}{\mathtt{isol}}
\newcommand{\stdmarks}{\mathtt{envir}}
\newcommand{\relation}[1]{\mathtt{relation}_{#1}}
\newcommand{\freevar}{\mathtt{FV}}
\newcommand{\notonmark}{\mathtt{notonenv}}
\newcommand{\InVal}[1]{\mathtt{InVal}\!\left(#1\right)}
\newcommand{\NotOnEnv}[1]{\mathtt{NotOnEnv}\!\left(#1\right)}
\newcommand{\PartOfVal}[1]{\mathtt{PartOfVal}\!\left(#1\right)}
%\newcommand{\nbpreds}[3]{\sharp #1 \geq #2}
\newcommand{\defstyle}[1]{{\emph{#1}}}

\newcommand{\cut}[1]{}
\newcommand{\interval}[2]{[#1,#2]}
\newcommand{\buniquevar}{\overline{\uniquevar}}
\newcommand{\bbuniquevar}{\overline{\overline{\uniquevar}}}
\newcommand{\magicwand}{\mathop{\mbox{$\mbox{$-~$}\!\!\!\!\ast$}}}
\newcommand{\wand}{\magicwand}
\newcommand{\septraction}{\stackrel{\hsize0pt \vbox to0pt{\vss\hbox to0pt{\hss\raisebox{-6pt}{\footnotesize$\lnot$}\hss}\vss}}{\magicwand}}
%% \newcommand{\reach}{\mathtt{reach}}
\mathchardef\mhyphen="2D % hyphen while in math mode

\newcommand{\adataword}{\mathfrak{dw}}
\newcommand{\adatum}{\mathfrak{d}}

\newcommand{\collectionknives}{\mathtt{ks}}
\newcommand{\collectionknivesfork}[1]{\mathtt{ksfs}_{=#1}}
\newcommand{\collectionknivesforks}{\mathtt{ksfs}}
\newcommand{\collectionkniveslargeforks}{\mathtt{kslfs}}


\newcommand{\acounter}{\mathtt{C}}

\newcommand{\fotwo}[3]{{\mbox{FO2}_{#1,#2}(#3)}}
\newcommand{\mtrans}[1]{t\!\left(#1\right)^{\Box}}
\newcommand{\mbtrans}[2]{\mtrans{#2}_{#1}}


\newcommand{\alogic}{\mathfrak{L}}


\newcommand{\semantics}[1]{\ensuremath{[ #1 ]}}


\newcommand{\adomino}{\mathfrak{d}}
\newcommand{\atile}{\mathfrak{d}}
\newcommand{\atiling}{\mathfrak{t}}

\newcommand{\hori}{\mathtt{h}}
\newcommand{\verti}{\mathtt{v}}
\newcommand{\domi}{\mathtt{d}}

\newcommand{\cpyrel}{\mathfrak{cp}}

\newcommand{\cntcmp}{\mathfrak{C}}

\newcommand{\heapdag}{\mathfrak{G}}

\newcommand{\onmainpath}{\mathtt{mp}}

\newcommand{\tree}{\mathtt{tree}}

%\newcommand{\tile}{\mathtt{tile}}

\newcommand{\type}{\mathtt{type}}

\newcommand{\ptype}{\mathtt{ptype}}

\newcommand{\exttype}{\mathtt{exttype}}

\newcommand{\anctypes}{\mathtt{AncTypes}}

\newcommand{\destypes}{\mathtt{DesTypes}}

\newcommand{\inctypes}{\mathtt{IncTypes}}

\newcommand{\treeic}{\mathtt{treeIC}}

\newcommand{\trs}{\mathfrak{trs}}


\newcommand{\nin}{\not \in}
\newcommand{\cupplus}{\uplus}
\newcommand{\aunarypred}{\mathtt{P}}


\newcommand{\hide}[1]{}

\newcommand{\eval}[2]{\llbracket#1\rrbracket_{#2}}
\newcommand\cur{\mathsf{cur}}
\newcommand\dom{\mathsf{dom}}
\newcommand\rng{\mathsf{rng}}

\newcommand\dd{\mathbb{D}}
\newcommand\nat{\mathbb{N}}


\newcommand\cA{\mathcal{A}}
\newcommand\cB{\mathcal{B}}
\newcommand\cC{\mathcal{C}}
\newcommand\cE{\mathcal{E}}
\newcommand\cG{\mathcal{G}}
\newcommand\cI{\mathcal{I}}
\newcommand\Ll{\mathcal{L}}
\newcommand\cM{\mathcal{M}}
\newcommand\cP{\mathcal{P}}
\newcommand\cR{\mathcal{R}}
\newcommand\cS{\mathcal{S}}
\newcommand\cT{\mathcal{T}}


\newcommand\vard{\mathfrak{d}}

\newcommand\replaceall{\mathsf{replaceAll}}
\newcommand\indexof{\mathsf{IndexOf}}



\newcommand\strline{\mathsf{SL}}

\newcommand\pstrline{\mathsf{SL_{pure}}}

\newcommand\search{\mathsf{search}}

\newcommand\verify{\mathsf{verify}}

\newcommand\searchleft{\mathsf{searchLeft}}

\newcommand\searchlong{\mathsf{searchLong}}


\newcommand\pref{\mathsf{Pref}}

\newcommand\wprof{\mathsf{WP}}

\newcommand\vars{\mathsf{Vars}}

\newcommand\dep{\mathsf{Dep}}
\newcommand\ptn{\mathsf{Ptn}}

\newcommand\src{\mathsf{src}}
\newcommand\strtorep{\mathsf{strToRep}}

\newcommand\rpleft{\mathsf{l}}
\newcommand\rpright{\mathsf{r}}


\newcommand\srcnd{\mathsf{srcND}}

\newcommand\ctxt{\mathsf{ctxt}}


\newcommand\ctxts{\mathsf{Ctxts}}

\newcommand\sprt{\mathsf{sprt}}

\newcommand\val{\mathsf{val}}

\newcommand\srclen{\mathsf{srcLen}}

\newcommand\rpleftlen{\mathsf{lLen}}


\newcommand\dfs{\mathsf{DFS}}

\newcommand\repr{\mathsf{rep}}

\newcommand\red{\mathsf{red}}

\newcommand\gfun{\mathcal{F}}


\newcommand{\leftmost}{{\sf leftmost}}
\newcommand{\longest}{{\sf longest}}

\newcommand{\arbidx}{{\sf Idx_{arb}}}
\newcommand{\dmdidx}{{\sf Idx_{dmd}}}
\newcommand{\lftlen}{{\sf Len_{lft}}}

\newcommand\rcdim{\mathsf{dim}}

\newcommand\rcdep{\mathsf{dep}}

\newcommand\tower{\mathrm{Tower}}


%\newtheorem{remark}[theorem]{Remark}

%%%%%%%%%%%%%%%%%%%%%%%%%%%%%%%%%%%%%%%
% Macros for two-way lower bound proof.

    \usepackage{unicode-math}

    \newcommand\ap[2]{{#1}\mathord{\brac{#2}}}

    % Tiling

    \newcommand\tiles{\Theta}
    \newcommand\hrel{H}
    \newcommand\vrel{V}
    \newcommand\tile{t}
    \newcommand\inittile{t_I}
    \newcommand\fintile{t_F}
    \newcommand\tileheight{h}

    % Large numbers

    \newcommand\expheight{n}
    \newcommand\linlen{m}
    \newcommand\tilesnum[1]{\Theta_{#1}}
    \newcommand\hrelnum[1]{H_{#1}}
    \newcommand\vrelnum[1]{V_{#1}}
    \newcommand\inittilenum[1]{\inittile^{#1}}
    \newcommand\fintilenum[1]{\fintile^{#1}}

    % \goodnums{n}{x} means x is good seqs of level-n nums
    \newcommand\goodnums[2]{\ap{\varphi_{#1}}{#2}}

    % \tenc{n}{val} = tile encoding of val in level-n
    \newcommand\tenc[2]{[#2]_{#1}}
    % \exptower{n}{m}  = tower of height n, to the m.
    \newcommand\nexp[2]{2 \uparrow_{#1} \brac{#2}}
    % \nmax{n} -- max number encodable at level n
    \newcommand\nmax[1]{\text{MAX}_{#1}}

    % nested alphabets, argument is level of nesting
    \newcommand\bone[1]{1_{#1}}
    \newcommand\bzero[1]{0_{#1}}
    \newcommand\nestednum[1]{c^{#1}}
    \newcommand\nestedalphabet[1]{\Sigma_{#1}}

    \newcommand\numeq{\circledequal}
    \newcommand\numplus{\oplus}
    \newcommand\numsep{\#}

    \newcommand\probfmla{\varphi}
    \newcommand\tilerow{r}

\newcommand{\ASSERT}[1]{\textbf{assert}(#1)}

\newcommand{\straightline}{\textsf{SL}}
\newcommand{\straightlinesym}{\textsf{SLS}}

\newcommand{\Pre}{\textsf{Pre}}

\newcommand{\twpt}{\textsf{2PT}}
\newcommand{\owpt}{\textsf{PT}}
\newcommand{\rbtwpt}{\textsf{RB2PT}}
\newcommand{\twspt}{\textsf{2SPT}}
\newcommand{\owspt}{\textsf{SPT}}

\newcommand{\transet}{\mathscr{T}}

\newcommand\theory{{\sf Th}}

\newcommand{\signature}{\mathcal{S}}

\newcommand{\sorts}{{\mathfrak{S}}}

\newcommand{\functions}{{\mathcal{F}}}

\newcommand{\predicates}{{\mathcal{P}}}

\newcommand\data{\mathbb{D}}

\newcommand{\interpretation}{\mathcal{I}}

\newcommand{\structure}{\mathcal{D}}

\newcommand{\issat}{\mathsf{isSat}}


%%%%%%%%%%%%%%%%%%%%%%%%%%%%%%%%%%%%%%%%%%%%

\newcommand{\zhilin}[1]{\color{cyan} {ZL: #1 :LZ} \color{black}}
\newcommand{\tl}[1]{\color{purple} {TL: #1 :LT} \color{black}}
\newcommand{\anthony}[1]{\color{blue} {AL: #1 :LA} \color{black}}
\newcommand{\matt}[1]{\color{blue} {MH: #1 :HM} \color{black}}
\newcommand{\philipp}[1]{\color{blue} {PR: #1 :RP} \color{black}}

%%%%%%%%%%%%%%%%%%%%%%%%%%%%%%%%%%%%%%%%%%%%


\title{A Decision Procedure for Path Feasibility of \\
String-Manipulating Programs\\
with Integer Data Type}


\author{}

\institute{ }

\begin{document}


%Regular Papers should not exceed 20 pages in LNCS format, not counting references and appendices.
%
%Regular papers at CAV 2020 will follow a full double blind review process, which means that author names and affiliations must be omitted from the submission. Additionally, if a submission refers to prior work done by the authors, the reference should be made in the third person. These are firm submission requirements, and any regular paper that does not conform to these requirements will be rejected without review.
%
%Authors can include a clearly marked appendix at the end of their submissions that is exempt from the page limit restrictions. However, the reviewers are not obliged to read the contents of these appendices. These papers should contain original research and sufficient detail to assess the merits and relevance of the contribution. Papers will be evaluated on basis of a combination of correctness, technical depth, significance, novelty, clarity, and elegance.

\maketitle

\begin{abstract}
Strings are widely used in programs, especially in Web programs, e.g. JavaScript and PHP. The integer data type naturally occurs in string-manipulating programs, typically being used to refer to length of, or position in, strings. 
%For instance, the inputs or outputs of widely used string operations including length, substring, and indexof involve the integer data type.
Analysis and testing of string-manipulating programs can be formulated as the path feasibility problem by symbolic execution, which is to decide whether a given execution path is feasible, namely, whether there exist assignments to the input variables so that a given path can execute to the end. %, which plays a central role in the static analysis and verification, e.g. symbolic execution, of programs. 
%Path feasibility problem of string-manipulating programs is very challenging and undecidable in general, especially when involving the integer data type. 
%Aiming at solving the path feasibility of string-manipulating programs in practice, 
%on the one hand, most of the 
State-of-the-art string constraint solvers usually provide strong support of complex string operations, in conjunction with limited support of integer data type. However, partially because this is a generally undecidable problem, %, and on the other hand, 
they mainly resort to heuristics without completeness guarantees. \\
In this paper, we propose a decision procedure of the path feasibility problem for a class of string-manipulating programs
which not only support complex string operations %involving only the string data type, e.g. 
such as concatenation, replaceAll, reverse, and finite state transducers, but also operations involving both the string and integer data type, e.g., length, substring, and indexof. To the best of our knowledge, this represents the most expressive string constraint language which is currently known to be decidable. 
Our decision procedure is automata-theoretic which is based on a variant of the cost register automata (Alur et al. LICS 2013). 
%To the best of our knowledge, %this is %the first time that 
%the first (complete) decision procedure which covers the most expressive string constraint language %have been achieved for the path feasibility problem of such an expressive class of string-manipulating programs. Furthermore, 
Different from most string constraint solving techniques which provide completeness guarantees, 
our decision procedure is amenable to implementation. 
%We implement the decision procedure on top of OSTRICH, resulting into a new solver called 
The implementation gives rise to OSTRICH+, which  is based on a recent solver OSTRICH which could not tackle integer data types. 
We utilise a wide range of benchmarks to evaluate the performance of OSTRICH+. The experimental results show that OSTRICH+ is competitive with some of the best state-of-the-art string constraint solvers.
\end{abstract}

%\section{Introduction}
%\label{intro}
%
%\section{Preliminaries}
%\label{prel}
%
%Definition for NFA, NFT.

%%%%%%%%%%%%%%%%%%%%%%%%%%%%%%%%%%%%%%%%%%%%%%%%%%%%%%%
\section{Introduction}

%20 pages, excluding references.
%
%Abstract+Conclusion: 1
%
%Introduction: 3
%
%Prelimilary: 2
%
%Logic: 2
%
%Decision procedure: 5-6
%
%Implementation: 3
%
%Evaluation:3


%!TEX root = main.tex

string manipulating programs

symbolic execution

string operations with integer data type

We use the following running example to illustrate the decision procedure in this paper.
%\begin{example}
{\small
\begin{minted}[linenos]{javascript}
function urlSimpleParse(url)
{
  var protocol='', host='';
  url = url.trim();
  var colonpos = url.indexof(':');
  if (colonpos >= 0) 
  {
    protocol = url.substr(0, colonpos).toLowerCase();
    if(/^http$|^https$/.test(protocol))
    {
      url = url.substr(colonpos+3);
      var slashpos = url.indexof('/');
      if (slashpos >= 0)  host = url.substr(0, slashpos); 
    }
    else protocol = '';
    return protocol, host; 
  }
}
\end{minted}
}
%\end{example}

% \in [\backslash w | \backslash x2E]^*$
We expect that host contains only the alphanumeric symbols as well as the dot symbol, but actually this is not the case. This question can be reduced to solving the path feasibility problem of the following program of the SSA (single static assignment) form.

{\small
\begin{minted}{javascript}
  protocol = ''; host = ''; url1 = url.trim(); 
  colonpos = url1.indexof(':'); assert(colonpos >= 0); 
  protocol1 = url1.substr(0, colonpos); 
  protocol2 = protocol1.toLowerCase();
  assert(/^http$|^https$/.test(protocol2));
  url2 = url1.substr(colonpos+3);
  slashpos = url2.indexof('/'); assert(slashpos >= 0);
  host1 = url2.substr(0, slashpos); assert(!/[\w|\x2E]*/.test(host1))
\end{minted}
}

state-of-the-art: heuristics

The contribution of this paper: decision procedure for string constraints involving integer data type

automata-theoretic, cost-enriched regular languages and recognisable relations, backward computation

implementation OSTRICH+, experimental results promising

first decision procedure for such an expressive class of string constraints involving so many different operations, natural extension of the decision procedure of OSTRICH, efficient implementation, extensive experiments, 


related work

SLENT: \cite{WC+18}

CVC4: \cite{cvc4}

TRAU, Z3-TRAU, TRAU+: \cite{Abdulla17,AbdullaA+19}

Z3-STR: \cite{Z3-str}

OSTRICH: \cite{CHL+19}

%%%%%%%%%%%%%%%%%%%%%%%%%%%%%%%%%%%%%%%%%%%%%%%%%%%%%%%
\section{Preliminaries}\label{sec:prel}

\section{Preliminaries}

Throughout the paper, $\Int^+$ denotes the set of positive integers, and  $\nat$ denotes the set of natural numbers. Furthermore, for $n\in \Int^+$, let $[n]:=\{1, \ldots, n\}$. 

\begin{definition}[Finite-state automata] \label{def:nfa}
	A \emph{(nondeterministic) finite-state automaton}
	(\FA{}) over a finite alphabet $\ialphabet$ is a tuple $\Aut =
	(\ialphabet, \controls, q_0, \finals, \transrel)$ where 
	$\controls$ is a finite set of 
	states, $q_0\in \controls$ is
	the initial state, $\finals\subseteq \controls$ is a set of final states, and 
	$\transrel\subseteq \controls \times 
	\ialphabet \times  \controls$ is the
	transition relation. 
\end{definition}

For an input string $w=a_1 \dots a_n$, a \emph{run} of $\Aut$ on $w$
%(with $a_0 = \EndLeft$ and $a_{n+1} = \EndRight$)
is a sequence of states $q_0, \ldots, q_n$ such that $(q_{j-1}, a_{j}, q_{j}) \in
\transrel$  for every $j \in [n]$.
The run is said to be \defn{accepting} if $q_n \in \finals$.
A string $w$ is \defn{accepted} by $\Aut$ if there is an accepting run of
$\Aut$ on $w$. In particular, the empty string $\varepsilon$ is accepted by $\Aut$ if $q_0 \in F$. The set of strings accepted by $\Aut$, i.e., the language \defn{recognised} by $\Aut$, is denoted by $\Lang(\Aut)$.
%Since we deal with computational complexity in the sequel, we define
The \defn{size} $|\Aut|$ of $\Aut$ is defined to be the cardinality of the set $Q$ of states, which will be 
used when the computational complexity is concerned.

For convenience, for $a \in \Sigma$, we use $\delta^{(a)}$ to denote the  relation $\{(q, q') \mid (q, a, q') \in \delta\}$.


  
%%%%%%%%%%%%%%%%%%%%%%%%%%%%%%%%%
 \subsection{Extended regular expressions}
%%%%%%%%%%%%%%%%%%%%%%%%%%%%%%%%%%  
An (extend) regular expression (with capturing group and back reference) is defined as follows.
  
\begin{definition}[Regular expressions with capturing group and back reference, $\regexp$]
  	\[e \eqdef \emptyset \mid \varepsilon \mid a \mid \$n \mid e + e \mid e \concat e \mid e^* \mid (e)  , \]
  	where $a \in \Sigma, n \in \Int^+$. 
  	%	Since $+$ is associative and commutative, we also write $(e_1 + e_2) + e_3$ as $e_1 + e_2 + e_3$ for brevity. 
  	%We use the abbreviation 
\end{definition}
  	
We use $e^+$ to abbreviate $e \concat e^*$. Moreover, for $\Gamma = \{a_1, \ldots, a_k\}\subseteq \Sigma$, we write $\Gamma$ for  $a_1 + \cdots + a_k$ and thus  $\Gamma^\ast \equiv (a_1 + \cdots + a_k)^\ast$. 

We assume that the parentheses in a regular expression are well matched. 
%
%Besides the common rules governing regular expressions, a regex obeys
%the following syntactic rule: 
Moreover, every back reference $\$ n$ is found to the right of the $n$-th pair of parentheses, where parentheses
are indexed according to the occurrence sequence of their left parenthesis.
\zhilin{some sanity conditions should be put to make the semantics of $\$ n$ well-defined.}
  
Note that standard regular expressions are those without $\$ n$ or $(e)$. Moreover, we use $\regexp[\sf CG]$ to denote the fragment of $\regexp$  excluding $\$ n$, and $\refexp$ to denote expressions generated by $e \eqdef \varepsilon \mid a \mid \$n \mid e \concat e$.
  %\tl{define the semantics here?}
  
  %\label{semantics:regex}
  
  
  
  %\subsection{Semantics of \regexp[\sf CG]}
  %In this section, we give one of the many semantics of \regexp[\sf CG], which we will utilize for $\replaceall$.
  
  \begin{definition}[Subexpression]
  	For any two $\regexp[\sf CG]$ $e$ and $r$, we say $r$ is a subexpression of $e$,
  	if either $r=e$ or
  	\begin{itemize}
  		\item If $e = e_1 e_2$ or $e_1 + e_2$ then $r$ is a subexpression of $e_1$
  		or $e_2$
  		
  		\item If $e = e_1^{\ast}$ or $(e_1)$ then $r$ is a subexpression of $e_1$
  	\end{itemize}
  	We use $S (e)$ to denote the set of all subexpressions of $e$.
  \end{definition}
  
  
  \begin{definition}[Match Tree]
  	A \tmtextbf{match tree} of $\regexp[\sf CG]$ $e$ is a finite directed and ordered
  	tree T, whose nodes are elements of $\Sigma^{\ast} \times S (e)$. A tree
  	is valid if the root is $(w, e)$ for some string w, and for any node $u =
  	(w, \alpha)$ in T, we have:
  	\begin{itemize}
  		\item If $\alpha = \alpha_1 \alpha_2$ then u has two children $(w_1,
  		\alpha_1)$ and $(w_2, \alpha_2)$ where $w = w_1 w_2$.
  		
  		\item If $\alpha = \alpha_1 + \alpha_2$ then u has a single child $(w,
  		\alpha_i)$ where $i \in \{ 1, 2 \}$.
  		
  		\item If $\alpha = \alpha_1^{\ast}$ then when $w = \varepsilon$, u is a
  		leaf otherwise there is $k \geqslant 1$ such that u has k children $(w_1,
  		\alpha_1), \ldots, (w_k, \alpha_1)$ where $w = w_1 \ldots w_k$ and for all
  		$i \in [k]$, $w_i \neq \varepsilon$, even if $\varepsilon \in L
  		(\alpha_1)$.
  		
  		\item If $\alpha = (\alpha_1)$ then u has a single child $(w, \alpha_1)$.
  		
  		\item If $\alpha = a$ (resp. $\alpha = \varepsilon$) then u is a leaf and
  		$w = a$ (resp. $w = \varepsilon$).
  	\end{itemize}
  	
  	Whenever unambiguous, we use a node u to represent the whole subtree
  	where u is the root. The notation $C(T)$ refers to all direct children of the root node of T
  	(and thus all direct subtrees).
  	
  	We also use $M (e)$ to denote all the valid match trees of e.
  \end{definition}
  
  \begin{definition}[Semantics of RegExp{[\sf CG]}]
  	\
  	
  	For any $\regexp[\sf CG]$ e, we recursively define a total order on $M (e)$, written $m
  	>_e n$ where $m, n \in M (e)$, as follows:
  	\begin{itemize}
  		\item $e = \varepsilon$ or $e = a$. There is only one match tree, thus the
  		order $>_e$ is empty.
  		
  		\item $e = (e_1)$. Suppose $C (m) = (w_1, e_1)$ and $C (n) = (w_2, e_1)$,
  		then $m >_e n$ iff $(w_1, e_1) >_{e_1} (w_2, e_1)$.
  		
  		\item $e = e_1 + e_2$.
  		\begin{itemize}
  			\item If $C (m) = (w, e_1)$ and $C (n) = (w', e_2)$, then $m >_e n$.
  			
  			\item If $C (m) = (w, e_i)$ and $C (n) = (w', e_i)$, where $i \in \{ 1,
  			2 \}$, then $m >_e n$ iff $(w, e_i) >_{e_i} (w', e_i)$.
  		\end{itemize}
  		\item $e = e_1 e_2$. Suppose $C (m) = (w_1, e_1) (w_2, e_2)$ and $C (n) =
  		(w_1', e_1) (w_2', e_2)$, then $m >_e n$ when either $(w_1, e_1) >_{e_1}
  		(w_1', e_1)$, or $w_1 = w_1'$ and $(w_2, e_2) >_{e_2} (w_2', e_2)$.
  		
  		\item $e = e_1^{\ast}$. If n is a leaf but m is not, then $m >_e n$.
  		Otherwise, suppose $C (m) = (w_1, e_1), \ldots, (w_k, e_1)$ and $C (n) =
  		(w_1', e_1), \ldots, (w_l', e_1)$, we have $m >_e n$ either when $C (n)$
  		is a proper prefix of $C (m)$, or for the first index j such that $w_j
  		\neq w_j'$, $(w_j, e_1) >_{e_1} (w_j', e_1)$.
  	\end{itemize}
  	
  	Let $e \in \regexp[\sf CG]$, and string $w \in L (e)$, then the accepting match of w
  	on e, denoted by $m_e (w)$, is the supremum of the set $\{ m \in M (e) \mid m = (w, e) \}$.
  	
  	For any subexpression $e'$ of e, suppose $(w_1, e') \ldots (w_m, e')$ are
  	all the nodes labeled by $(w', e')$, where $w' \neq \varepsilon$, in the
  	pre-order traversal of $m_e (w)$. Then the match result captured by $e'$, denoted
  	by $m_{e', e} (w)$, is the sequence of substring $w_1 \ldots w_m$.
  \end{definition}
  
  \zhilei{Give an example here}

%%%%%%%%%%%%%%%%%%%%%%%%%%%%%%%%%%%%%%%%%%%%%%%%%%%%%%%

\section{String-Manipulating Programs with Integer Data Type}\label{sec:logic}

%!TEX root = main.tex

\section{The string logic}\label{sec:logic}

In this section, we define the string-manipulating programming language $\strline$ considered in this paper.

%%%%%%%%%%%%%%%%%%%%%%%%%%%%%%%%%%%%%%%%


%\subsection{Prioritized Streaming String Transducer}
%Below, we introduce  a new class of prioritized transducer \cite{BM17} which combines the expressive power of streaming string transducer \cite{AC10,AD11}.
%
%\begin{definition}[Prioritized Streaming String Transducer]
%	A \emph{prioritized streaming string transducer} (PSST) is a tuple $\psst = (Q, \Sigma, X, E, \delta, q_0, F)$, where $Q$ a
%	finite set of states, $\Sigma$ is the input and output alphabet, and $X$ a finite set of variables. $E$ is a partial function from $Q \times \Sigma \times
%	Q$ to $X \rightarrow (X \cup \Sigma)^{\ast}$, i.e. the set of assignment,
%	$\delta \in Q \times \Sigma \rightarrow \overline{Q}$ and $F$ is a partial function
%	from $Q$ to $(X \cup \Sigma)^{\ast}$.
%\end{definition}
%
%A run of $\psst$ is the sequence $q_0 \sigma_1 s_1 q_1 \ldots \sigma_m s_m q_m$, where $F (q_m)$ is defined and for each $i \in [m], q_i \in \delta (q_{i-1}, \sigma_i)$ and $s_i = E (q_{i - 1}, \sigma_i, q_i)$. For any two runs on $w = \sigma_1 \ldots \sigma_m$, denoted by $p = q_0 \sigma_1 s_1 \ldots \sigma_m s_m q_m$ and $p' = q_0 \sigma_1
%s_1' \ldots \sigma_m s_m' q_m'$, we say that $p$ is of a higher priority over
%$p'$ if $p \neq p'$ and, for the smallest index $j$ with $q_j \neq q_j'$,
%$\delta (q_{j - 1}, \sigma_j) = \ldots q_j \ldots q_j' \ldots$
%
%The accepting run of $\psst$ on input $w$ is the run of the highest priority. The output of $T$ on w, denoted by $T(w)$, is defined as $\pi_m(F(q_m))$, where $\pi_0(x) = \varepsilon$ for each $x \in X$, and $\pi_{i}(x) = \pi_{i-1}(s_{i}(x))$ for $1 \le i \le m$ and $x \in X$. Note that here we abuse the notation  $\pi_m(F(q_m))$ and $\pi_{i-1}(s_{i}(x))$ by taking a function $\pi$ from $X$ to $\Sigma^*$ as a function from $(X \cup \Sigma)^*$ to $\Sigma^*$, which maps each $\sigma \in \Sigma$ to $\sigma$ and each $x \in X$ to $\pi(x)$.  
%
%%  $\tmop{Out} (r) =
%%  s_{\varepsilon} \circ s_1 \circ s_2 \ldots s_n \circ F (q_n)$ where
%%  $s_{\varepsilon}$ is the empty substitution which maps all variables to
%%  $\varepsilon$.
%
%\begin{definition}[pre-image]
%	For a string relation $R \subseteq \Sigma^* \times \Sigma^*$ and $L \subseteq \Sigma^*$, we define the \emph{pre-image} of $L$ under $R$ as $R^{-1}(L):=\{w \in \Sigma^* \mid \exists w'.\ w' \in L \mbox{ and } (w, w') \in R\}$. 
%\end{definition}
%
%\begin{theorem}[pre-image of \PSST{}]
%	\label{theorem:psst_preimage}
%	Given a \PSST{} $\psst = (Q_T, \Sigma$, $X, E, \delta_T, q_{0, T}, F_T)$ and \FA{} $A
%	= (Q_A, \Sigma, \delta_A, q_{0, A}, F_A)$, we can compute an \FA{} $B = (Q_B,
%	\Sigma, \delta_B, q_{0, B}, F_B)$ in exponential time  such that $\Lang(B) = \cR^{-1}_T(\Lang(\Aut))$.
%\end{theorem}
%
%\begin{proof}
%	Intuitively, $B$ simulates the run of $\psst$ on $w$, and, for each $x \in X$, records the set of state pairs $(p, q) \in Q_A \times Q_A$ such that starting from $p$, $A$ can reach $q$ after reading the string stored in $x$. Moreover, $B$ also records all the states accessible from a run with higher priority to ensure the current run is the accepting one of $\psst$.
%	
%	Formally, $Q_B = Q_T \times (\cP(Q_A \times Q_A ))^{X} \times \cP(Q_T)  $, $q_{0, B} = (q_{0, T}, \rho_{\varepsilon}, \emptyset)$ where $\rho_{\varepsilon} (x) = \{(q, q) \mid q \in Q\}$ for each $x \in X$, and $\delta_{B}$ comprises the tuples $((q, \rho, S), a, (q_i, \rho', S'))$ such that there exists $s \in \left((X \cup \Sigma\right)^*)^X$ satisfying
%	\begin{itemize}
%		\item $\delta_T (q, a) = (q_1 \ldots q_i \ldots q_m)$, 
%		\item $s = E(q,a,q_i)$.
%		\item $S' = \delta_T^{\ast} (S, a) \cup \{ q_1, \ldots, q_{i - 1} \}$, where $\delta_T^{\ast}(S,a) = \{q' \mid \exists q \in S, q' \in \delta_T(q,a)\}$.
%		\item and $\rho'$ is obtained from $\rho$ and $s$ as follows: for each $x \in X$, if $s(x) = \varepsilon$, then $\rho'(x) = \{(p, p) \mid p \in Q_A\}$, otherwise, let $s(x) = b_1 \cdots b_\ell$ with $b_i \in \Sigma \cup X$ for each $i \in [\ell]$, then $\rho'(x) = \theta_1 \circ \cdots \circ \theta_\ell$, where $\theta_i = \delta^{(b_i)}_A$ if $b_i \in \Sigma$, and $\theta_i = \rho(b_i)$ otherwise.
%		%
%		%$\rho'(x) = \theta_\ell$ such that $\theta_0 = \{(p,p) \mid p \in Q_A\}$, and for each $i \in [\ell]$, if $b_i \in \Sigma$, then $\theta_i = \{(p, p') \mid (p, p'') \in \theta_{i-1}, (p'', b_i, p') \in \delta_A \mbox{ for some } p''\}$, otherwise, $\theta_i = \theta_{i-1} \cdot \rho(x)$. 
%	\end{itemize}
%	
%	Moreover, $F_B$ is the set of states $(q, \rho, S) \in Q_B$ such that
%	\begin{enumerate}
%		\item $F_T (q)$ is defined,
%		\item For any $q' \in S$, $F_T (q')$ is not defined
%		
%		\item if $F_T(q) = \varepsilon$, then $q_{0, A}  \in F_A$, otherwise, 
%		let $F_T(q) = b_1 \cdots b_\ell$ with $b_i \in \Sigma \cup X$ for each $i \in [\ell]$, then $(\theta_1 \circ \cdots \circ \theta_\ell) \cap (\{q_{0,A}\} \times F_A) \neq \emptyset$, where for each $i \in [\ell]$, if $b_i \in \Sigma$, then $\theta_i = \delta^{(b_i)}_A$, otherwise, $\theta_i = \rho(b_i)$.
%	\end{enumerate}
%\end{proof}
%
%Note that the above construction  does not utilize the so-called \tmtextit{copyless} property \cite{AC10,AD11},
%thus it works for general, or \tmtextit{copyful} \PSST{} \cite{FR17}.

% Note that in the definition of \NSST, there is no \emph{copyless} restriction.



%%%%%%%%%%%%%%%%%%%%%%%%%%%%%%%%%%%%%%%%%%%%%%%%%%%%%


%We will use $\$ 1, \$2, \cdots$ to denote the references to capturing groups in regular expressions.

%We define the set of reference expressions as follows: 

%We consider the symbolic execution of the string-manipulating programs.

%\begin{definition}[The constraint language $\strline$] 
The $\strline$ language is defined by
\[
\begin{array}{l c l}
S &::= &  z:= x\ \concat\ y \mid y := \extract_{i, e}(x) \mid  \\ 
& &  
%y := \reverse(x) 
y := \replaceall_{\pat, \rep}(x)   \mid \\
%y := \Transducer(x)\  \mid\\
& & \text{$\ASSERT{x \in e}$} \mid S; S\
\label{eq:SL}
%a ::= f(x_1,\ldots,x_n), \qquad b ::= g(x_1,\ldots,x_n)
\end{array}
\]
%\tl{to avoid confusion, write  $\ASSERT{x \in A}$?} 
where 
\begin{itemize}
	\item $\concat$ is the string concatenation operation which concatenates two strings,
%
\item for the $\extract$ function, $i \in \Nat$, $e \in \regexp[\sf CG]$,
%
	\item  for the $\replaceall$ operation, $\pat\in \regexp[\sf CG]$, $\rep \in \refexp$, %$\replaceall$ is the replace-all function to be defined shortly,
%	\item $\reverse$ is the string function which reverses a string; 
%	\item $\Transducer$ is a \PSST,
%
	\item for assertions, $e \in \regexp$.
\end{itemize} 
%and $R$ is a recognisable relation represented by a collection of tuples of \FA{}s.
%\end{definition}
%
%\zhilei{Should we add NSST constraints? Basically NSST can express more than PSST.}
%\tl{maybe just use NSST to replace PSST?}
%
%\zhilei{PSST is needed for decision procedure. NSST can be decided too, but the algorithm is very similar, so maybe too tedious to add both }

%It is evident that the $\reverse$ function is subsumed by \PSST{}s.

%
%\begin{remark}
%	Zhilin mentioned that we might introduce a function which takes a string and a pattern with capturing groups, and does sort of pattern matching to extra substrings. This function can be captured by the transducer $T$. We will formalise this later.
%\end{remark}

The $\extract$ function models the regular-expression match function in programming languages, e.g. $\sf str.match(regexp)$ function in Javascript. Specifically, the $\extract_{i, e}(x)$ function extracts the match of the $i$-th capturing group in the match of $e$ in $x$, if the match of $e$ in $x$ exists (otherwise, the return value of the function is undefined). Note that if $i=0$, then $\extract_{i, e}(x)$ returns the match of $e$ in $x$. For instance, let $e = ((b^\ast) \cdot ((b \cdot a^\ast) | \varepsilon)) \cdot a^\ast$, then $\extract_{1, e}(baabba)=b$ and $\extract_{2, e}(baabba)=\varepsilon$, as shown in Example~\ref{exmp-regex-semantics}. 

The $\replaceall_{\pat, \rep}(x)$ function is indexed by the  %\emph{subject} string, the second parameter is a 
\emph{pattern} $\pat \in \regexp[\sf CG]$ and the \emph{replacement} string $\rep \in \refexp$. For a given input string $x$, the function identifies the first, second, $\dots$, match of $\pat$ in $x$ and replace them with the corresponding strings specified by the replacement string (where the references are replaced by the corresponding matches of the capturing groups).  For instance, let $\pat = ((b^\ast) \cdot ((b \cdot a^\ast) | \varepsilon)) \cdot a^\ast$ and $\rep = \#\$1$, then  $\replaceall_{\pat, \rep}(baabba) = \#b\#bb$. 

Without loss of generality, we assume that all the $\strline$ programs are in single static assignment (SSA) form, that is, each variable $x$ is assigned at most once, moreover, if it is assigned, then all its occurrences in the right-hand-sides of the assignment statements or in assertions are after the assignment statement of $x$.

For a $\strline$ program $S$, a variable $x$ occurring in $S$ is said to be an \emph{input} variable if $x$ does not occur in the left-hand-side of assignment statements. 
The \emph{path feasibility} problem of a $\strline$ program is to decide whether there are valuations of the input variables so that the program can execute to the end.


%
%For the semantics of $\replaceall$ function, in particular when the pattern is a regular expression, we adopt the \emph{leftmost and longest} matching. 




%For instance, $\replaceall(aababaab, (ab)^+, c) =ac\cdot \replaceall(aab, (ab)^+, c)= acac$, since the leftmost and longest matching of $(ab)^+$ in $aababaab$ is $abab$. Here we require that the language defined by the pattern parameter does \emph{not} contain the empty string, in order to avoid the troublesome definition of the semantics of the matching of the empty string. We refer the reader to \cite{CCHLW18} for the formal semantics of the $\replaceall$ function. To be consistent with the notation in this paper, for each regular expression $e$, we define
%the string function $\replaceall_e: \ialphabet^* \times \ialphabet^* \rightarrow \ialphabet^*$ such that for $u, v \in \ialphabet^*$, $\replaceall_e(u, v) = \replaceall(u, e, v)$, and we write $\replaceall(x, e, y)$ as $\replaceall_e(x,y)$.

It turns out that the path feasibility problem is undecidable, attributed to the the back references in assertion statements. 

\begin{proposition}\label{prop-und}
The path feasibility problem of $\strline$ is undecidable.
\end{proposition}

The proof of Proposition~\ref{prop-und} is obtained by an encoding of post correspondence problem (PCP).
Let $\Sigma$ be a finite alphabet such that $\# \not\in \Sigma$ and $[n] \cap \Sigma = \emptyset$, $(u_i, v_i)_{i \in [n]}$ be a PCP instance with $u_i, v_i \in \Sigma^\ast$. A solution of the PCP instance is a string $i_1 \cdots i_m$ with $i_j \in [n]$ for every $j \in [m]$ such that $u_{i_1} \cdots u_{i_m} = v_{i_1} \cdots v_{i_m}$. We will use $\replaceall$ to encode the generation of the strings $u_{i_1} \cdots u_{i_m}$ and $v_{i_1} \cdots v_{i_m}$ from $i_1 \cdots i_m$, then use a regular expression with  capturing groups and back references to verify the equality of $u_{i_1} \cdots u_{i_m}$ and $v_{i_1} \cdots v_{i_m}$. Specifically, suppose $\# \not \in \Sigma$, then the PCP instance is encoded by the following $\strline$ program,
\[
\begin{array}{l}
\ASSERT{x_0 \in \{1, \cdots, n\}^+}; \\
x_1 := \replaceall_{1, u_1}(x_0); \cdots; x_n:=\replaceall_{n, u_n}(x_{n-1}); \\
y_1:=\replaceall_{1, v_1}(x_0); \cdots; y_n:=\replaceall_{n, u_n}(y_{n-1});\\
z:= x_n \# y_n; \ASSERT{z \in (\Sigma^+)\#\$1}.
\end{array}
\]

We shall show that the path feasibility problem is decidable, if the uses of back references in assertion statements are forbidden, which turns out to be the situation in practice. In the sequel, we will use $\strline_{\sf reg}$ to denote the collection of $\strline$ programs where no back references occur in assertion statements. The decision procedure for $\strline_{\sf reg}$ utilizes a new model called prioritized streaming string transducers, which will be defined in the next section.


%%%%%%%%%%%%%%%%%%%%%%%%%%%%%%%%%%%%%%%%%%%%%%%%%%%%%%%%

%\section{Automata-Theoretic Foundations}\label{sec:cefa}

%%!TEX root = main.tex


%In this section, we introduce cost-enriched regular languages and recognisable relations, as the extensions of regular languages and recognisable relations, moreover, we investigate the decidability and complexity of a related decision problem, thus laying down the theoretical foundations of the decision procedure in the next section. 

\subsection{Cost-Enriched Regular Languages and Recognisable Relations} \label{sect:ce}

Let $k \in \Nat$ with $k > 0$. A \emph{$k$-cost-enriched string} is $(w, (n_1, \cdots, n_k))$ where $w$ is a string and $n_i \in \intnum$ for all $i \in [k]$. 
A \emph{$k$-cost-enriched language} $L$ is a subset of $\Sigma^* \times \intnum^k$. For our purpose, we identify a ``regular" fragment of cost-enriched languages as follows. 
%Note that all the cost-enriched strings in $L$ are associated with the same number of costs (i.e., $k$).

\begin{definition}[Cost-enriched regular languages]
Let $k \in \Nat$ with $k > 0$. A $k$-cost-enriched language is \emph{regular} (abbreviated as CERL) if it can be accepted by a \emph{cost-enriched finite automaton}. 

A cost-enriched finite automaton (CEFA) $\CEFA$ is a tuple $(Q, \Sigma, R, \delta, I, F)$ where 
\begin{itemize}
\item $Q, \Sigma, I, F$ are defined as in NFAs, 
%
\item $R=(r_1, \cdots, r_k)$ is a vector of (mutually distinct) \emph{cost registers}, 
%
\item $\delta$ is the transition relation which is a finite set of tuples $(q, a, q', \eta)$ where $q, q' \in Q$, $a \in \Sigma$, and %$\eta: R \rightarrow \intnum$ 
$\eta: R \rightarrow \Nat$
is a cost register update function. \\
For convenience, we usually write $(q, a, q', \eta) \in \Delta$ as $q \xrightarrow{a, \eta} q'$.
\end{itemize}
%
A \emph{run} of $\CEFA$ on a $k$-cost-enriched string $(a_1 \cdots a_m, (n_1, \cdots,n_k))$ is a  transition sequence $q_0 \xrightarrow{a_1, \eta_1} q_1 \cdots q_{m-1} \xrightarrow{a_m, \eta_m} q_m$ such that $q_0 \in I$ and $n_i = \sum \limits_{1\leq j\leq m}\eta_j(r_i)$ for each $i \in [k]$ (Note that the initial values of cost registers are zero). The run is \emph{accepting} if $q_m \in F$. A $k$-cost-enriched string $(w, (n_1, \cdots,n_k))$ is accepted by $\CEFA$ if there is an accepting run of $\CEFA$ on $(w, (n_1, \cdots,n_k))$. In particular, $(\varepsilon, n)$ is accepted by $\CEFA$ if $n=0$ and $I\cap F \neq \emptyset$.
The $k$-cost-enriched language defined by $\CEFA$, denoted by $\Lang(\CEFA)$, is the set of $k$-cost-enriched strings accepted by $\CEFA$. 
%A cost-enriched language $L \subseteq \Sigma^* \times \intnum^k$ is called a cost-enriched regular language (CERL) if there is a CEFA $\NFA$ such that $L = \Lang(\NFA)$.
\end{definition}
The \emph{size} of a CEFA $\CEFA=(Q, \Sigma, R, \delta, I, F)$, denoted by $|\CEFA|$, is defined as the sum of the sizes of its transitions, where the size of each transition $(q, a, q', \eta)$ is $\sum \limits_{r \in R} \lceil \log_2 (|\eta(r)|) \rceil +3$. Note here  the integer constants in $\CEFA$ are encoded in binary.

\begin{remark}
CEFAs can be seen as a variant of Cost Register Automata \cite{RLJ+13}, by admitting nondeterminism and discarding partial final cost functions. CEFAs are also closely related to the monotonic counter machine \cite{LB16}. The main difference is that CEFAs allow natural numbers in cost updates which are restricted to $0,1$ only in monotonic counter machine. Moreover, we explicit define CEFA as a language acceptor for an enriched language.
\end{remark}

\begin{example}[CEFA for $\length$]\label{exm:len}
The string function $\length$ can be captured by CEFA. For any NFA $\NFA=(Q, \Sigma,  \delta, I, F)$, it is not difficult to see that the cost-enriched language $\{(w, \length(w)) \mid w\in \Lang(\NFA)\}$ is accepted by a CEFA, i.e.
$(Q, \Sigma, (r_1), \delta', I, F)$ such that %$Q =I=F= \{q_0\}$, $R=(r_1)$, and 
such that for each $(q, a, q')\in \delta$, we let $(q, a, q', \eta)\in \delta'$, where $\eta(r_1) = 1$. For convenience, we use $\CEFA_{\rm len}$ to denote the CEFA $(Q, \Sigma, (r_1), \delta', I, F)$ for the special NFA $\NFA$ such that $\Lang(\NFA)=\Sigma^*$, specifically, $\NFA_{\rm len} = (\{q_0\}, \Sigma, (r_1), \{(q_0, a, q_0, \eta) \mid \eta(r_1) = 1\}, \{q_0\}, \{q_0\})$.
%\tl{it is a bit vague; you mean for any RE $e$,  is a cerl?}
\end{example}

We can show that the function $\indexof_v(\cdot, \cdot)$ can be captured by CEFA as well, in the sense that, for any NFA $\NFA$ and constant string $v$, we can construct a CEFA $\CEFA'$ such that $R(\CEFA')=(r_1,r_2)$ and $\Lang(\CEFA')=\{(w, (n, \indexof_v(w, n)))\mid w\in \Lang(\NFA) \}$. The construction is slightly technical and can be found in Appendix, Section~\ref{appendix:cefa_indexof}.
For convenience, we use $\CEFA_{\indexof_v}$ to denote the constructed CEFA $\CEFA'$ for the special NFA $\NFA$ such that $\Lang(\NFA)=\Sigma^*$. In the sequel, we will exemplify the construction of $\CEFA_{\indexof_v}$ for the special case that $v$ is a single character. 

\begin{example}\label{exm:indexof}
Let $a \in \Sigma$. Then  $\CEFA_{\indexof_a} = (\{(q_0, q_1, q_2)\}, \Sigma, (r_1,r_2), \delta_{\indexof_a}, \{q_0\}, \{q_2\})$, where $\delta_{\indexof_a}$ comprises the tuples
\begin{itemize}
\item $(q_0, b, q_0, \eta)$ such that $b \in \Sigma$, $\eta(r_1)=1$, $\eta(r_2)=1$,
%
\item $(q_0, b, q_1, \eta)$ such that $b \in \Sigma$, $\eta(r_1)=0$, $\eta(r_2) = 1$,
%
\item $(q_0, a, q_2, \eta)$ such that $\eta(r_1)=0$, $\eta(r_2) = 0$,
%
\item $(q_1, b, q_1, \eta)$ such that $b \in \Sigma \setminus \{a\}$, $\eta(r_1)=0$, $\eta(r_2)=1$,
%
\item $(q_1, a, q_2, \eta)$ such that $\eta(r_1)=0$, $\eta(r_2)=0$,
%
\item $(q_2, b, q_2, \eta)$ such that $b \in \Sigma$, $\eta(r_1)=0$, $\eta(r_2)=0$.
\end{itemize}
Intuitively, $r_1$ corresponds to the starting position $i$ of $\indexof_a(x, i)$, $r_2$ corresponds to the output of $\indexof_a(x, i)$, $q_0$ specifies that the current position is before $i$, $q_1$ specifies that the current position is after $i$, while $a$ has not occurred yet, and $q_2$ specifies that $a$ has occurred after $i$. 
\end{example}


Given two CEFAs $\CEFA_1 = ( Q_1, \Sigma, R_1, \delta_1, I_1, F_1)$ and $\CEFA_2 = (Q_2, \Sigma, \delta_2, R_2, I_2, F_2)$ with $R_1 \cap R_2 = \emptyset$ (where %the notation is abused a bit, viewing 
$R_1$ and $R_2$ are treated as sets), the product of $\CEFA_1$ and $\CEFA_2$, denoted by $\CEFA_1 \times \CEFA_2$, is defined as $(Q_1 \times Q_2, \Sigma, R_1 \cup R_2, \delta, I_1 \times I_2, F_1 \times F_2)$, where $\delta$ comprises the tuples $((q_1, q_2), \sigma, (q'_1, q'_2), \eta)$ such that $(q_1, \sigma, q'_1, \eta_1) \in \delta_1$, $(q_2, \sigma, q'_2, \eta_2) \in \delta_2$, and $\eta = \eta_1\cup \eta_2$.  %for some $\eta_1, \eta_2$.


For a CEFA $\CEFA$, we use $R(\CEFA)$ to denote the vector of cost registers occurring in $\CEFA$. %Note that cost registers of $\CEFA$ are simply integer variables to store costs in $\CEFA$.
%Moreover, for a CEFA $\NFA$ and 
Suppose $\CEFA$ is  CEFA with $R(\CEFA)=(r_1,\cdots, r_k)$ and $\vec{i} = (i_1,\cdots, i_k)$ with $R(\NFA) \cap \vec{i} = \emptyset$. We use $\CEFA[\vec{i}/R(\CEFA)]$ to denote the CEFA obtained from $\CEFA$ by simultaneously replacing $r_j$ with $i_j$ for $j \in [k]$. 

\smallskip

%Let $(k_1,\cdots, k_l) \in \Nat^l$ with $k_j > 0$ for every $j \in [l]$. A  $(k_1,\cdots, k_l)$-cost-enriched relation $\cR$ is a subset of $(\Sigma^* \times \intnum^{k_1}) \times \cdots (\Sigma^* \times \intnum^{k_l})$.

%%%%%%%%%%%%%%%%%%%%%%%%%%%%%%%%%%%%%%%%%%%%%%%%%%%%%%%%%%%%%%%%%%%%%%%%%%%%%%%%%%%%%%%%%%%%%%%
\begin{definition}[Cost-enriched recognisable relations]
Let $(k_1,\cdots, k_l) \in \Nat^l$ with $k_j > 0$ for every $j \in [l]$. A cost-enriched recognisable relation (CERR)  $\cR \subseteq (\Sigma^* \times \intnum^{k_1}) \times \cdots  \times (\Sigma^* \times \intnum^{k_l})$ is a finite union of products of CERLs. Formally,
	\[\cR = \bigcup \limits_{i=1}^n L_{i,1 } \times \cdots \times L_{i, l},\]
	where for every $j \in [l]$, $L_{i,j} \subseteq \Sigma^* \times \intnum^{k_j}$ is a CERL. 
	A CEFA representation of $\cR$ is a collection of CEFA tuples $(\CEFA_{i,1}, \cdots, \CEFA_{i,l})_{i \in [n]}$ such that $\Lang(\CEFA_{i,j}) = L_{i,j}$ for every $i \in [n]$ and $j \in [l]$.
\end{definition}

\begin{example}\label{exm:CERR}
The relation 
\[\cR=\left\{((w_1, |w_1|), (w_2, |w_2|)) \mid  w_1 \in \Lang((aa)^*), w_2 \in \Lang(b(bb)^*), |w_1|+|w_2| \ge 2\right\}\] 
is a CERR since 
$\cR = L_{1,1} \times L_{1,2} \cup L_{2,1} \times L_{2,2}$, where 
\begin{itemize}
\item $L_{1,1}=\{(w_1, |w_1|) \mid w_1 \in \Lang((aa)^*)\}$, 
\item $L_{1,2}=\{(w_2, |w_2|) \mid  w_2 \in \Lang(bbb(bb)^*)\}$ , 
\item $L_{2,1}=\{(w_1, |w_1|) \mid w_1 \in \Lang(aa(aa)^*)\}$,  and
\item $L_{2,2}=\{(w_2, |w_2|) \mid  w_2 \in \Lang(b(bb)^*)\}$. 
\end{itemize}
Note that $L_{i,j}$ for $i,j\in[2]$ are CERLs, with corresponding CEFAs $\CEFA_{i,j}$ by  Example~\ref{exm:len}. It follows that %Moreover, 
$(\CEFA_{i,1}, \CEFA_{i,2})_{i \in [2]}$ gives a CEFA representation of $\cR$. %, where 
%\tl{to be simplified.}
%\begin{itemize}
%\item  $\CEFA_{1,1} = (\{p_0, p_1\}, \{a,b\}, (r_{1}), \delta_{1,1}, \{p_0\}, \{p_0\})$ defines $L_{1,1}$, such that $\delta_{1,1}= \{(p_0, a, p_1,\eta_1)$, $(p_1, a, p_0, \eta_1)\}$, with $\eta_1(r_{1})=1$,
%%
%\item $\CEFA_{1,2}=(\{q_0,q_1,q_2\}, \{a,b\}, (r_{2}), \delta_{1,2}, \{q_0\}, \{q_1\})$ defines $L_{1,2}$, such that $\delta_{1,2}=\{(q_0, b, q_1,\eta_2), (q_1, b, q_2, \eta_2), (q_2, b, q_1,\eta_2)\}$, with $\eta_2(r_{2})=1$,
%%
%\item $\CEFA_{2,1} = (\{p_0, p_1,p_2, p_3\}, \{a,b\}, (r_{1}), \delta_{2,1}, \{p_0\}, \{p_2\})$ defines $L_{2,1}$, such that $\delta_{2,1} = \{(p_0, a, p_1, \eta_1), (p_1, a, p_2, \eta_1), (p_2, a, p_3, \eta_1), (p_3, a, p_2, \eta_1)\}$, with $\eta_1(r_{1})=1$, and
%%
%\item $\CEFA_{2,2}=(\{q_0,q_1,q_2\}, \{a,b\}, (r_{2}), \delta_{2,2}, \{q_1\})$ defines $L_{2,2}$, such that $\delta_{2,2}= (q_0, b, q_1,\eta_2), (q_1, b, q_2, \eta_2), (q_2, b, q_1,\eta_2)\}$, with $\eta_2(r_{2})=1$.
%\end{itemize}
\end{example}

%\subsection{Satisfiability of Linear Integer Arithmetic Formula with respect to CEFAs}
%
%In the sequel, we consider a decision problem for CEFAs which will be used for the decision procedure of the path feasibility problem for {\slint} in the next section.
%
%\begin{definition}[{\lasat} Problem]\label{def-la-sat-cefa}
%	The satisfiability problem of LIA formulas w.r.t. CEFAs (abbreviated as {\lasat} problem) is defined as follows.
%	
%	\textbf{Input}: a given quantifier-free LIA formula $\phi$ and CEFAs $\CEFA_1,\cdots,\CEFA_m$, such that $\CEFA_i=(Q_i, \Sigma, R_i, \delta_i, I_i, F_i)$ for every $i\in [m]$, 
%  $R_i \cap R_j = \emptyset$ for every $1 \le i < j \le m$, and
% the free variables of $\phi$ are from $\bigcup_{i\in [m]} R_i$, 
% 
%Decide whether %$\phi$ is satisfiable w.r.t. $\CEFA_1, \cdots, \CEFA_m$, namely, whether 
%there are an assignment function $\theta: \bigcup \limits_{i \in [m]} R_i \rightarrow \Int$ and strings $w_1, \cdots, w_m$  
%	such that  $\phi[(\theta(R_i)/R_i)_{i \in [m]}]$ hold and $(w_i, \theta(R_i)) \in \Lang(\NFA_i)$ for every $i \in [m]$.
%\end{definition}
%%Note that in Definition~\ref{def-la-sat-cefa}, registers in $\NFA_i$'s may intersect. 
%
%\begin{example}
%Let $\phi \equiv r_1 = r_2$ and $\CEFA_{1,1}, \CEFA_{1,2}$ be two CEFAs in Example~\ref{exm:CERR} defining $L_{1,1}, L_{1,2}$ respectively. (Recall that $R(\CEFA_{1,1})=(r_1)$ and $R(\CEFA_{1,2})=(r_2)$.) Then it is easy to see that $\phi$ is unsatisfiable w.r.t. $\CEFA_{1,1}$ and $\CEFA_{1,2}$, since for each $(w_1, n_1) \in L_{1,1}$, $n_1$ must be even, while for each $(w_2, n_2) \in L_{1,2}$, $n_2$ must be odd. Hence $n_1$ and $n_2$ cannot be equal.
%\end{example}
%
%\begin{theorem}\label{thm-la-sat-cefa}
%	The {\lasat} problem is NP-complete.
%\end{theorem}
%The proof is given in Appendix, Section~\ref{appendix:thm-la-sat-cefa}.

%%%%%%%%%%%%%%%%%%%%%%%%%%%%%%%%%%%%%%%%%%%%%%%%%%%%%%%%

\section{Decision Procedures for Path Feasibility}\label{sec:dec}

%!TEX root = main.tex

In this section, we will design a decision procedure for the path feasibility problem of {\slint} programs, based on the concepts of CERLs and CERRs introduced Section~\ref{sec:cefa}.

Before presenting the decision procedure, we introduce an additional concept, cost-enriched pre-images of CERLs under string operations. Moreover, we will show that the cost-enriched pre-images of CERLs under the string operations in {\slint}, namely, concatenation $\concat$, $\replaceall_{e,u}$, $\reverse$, NFTs $\NFT$, and $\substring$, are CERRs. 

To simplify the presentation of the decision procedure, in this section, we usually keep the string operations abstract by only mentioning the input and output data types, namely, we consider string operations $f: (\Sigma^* \times \Int^{k_1}) \times \cdots \times (\Sigma^* \times \Int^{k_l}) \rightarrow 2^{\Sigma^*}$ (if there is no integer input parameter, then $k_1,\cdots,k_l$ are zero), where each integer input parameter (if there is any) is assumed to be affiliated to a unique string input parameter. Note that  in general $f$ can be nondeterministic, namely, on one input, $f$ may output several  strings.


\subsection{Pre-images of CERLs under string operations}

\begin{definition}[Cost-enriched pre-images of CERLs]
Suppose that $f: (\Sigma^* \times \Int^{k_1}) \times \cdots \times (\Sigma^* \times \Int^{k_l}) \rightarrow 2^{\Sigma^*}$ is a string operation, $L \subseteq \Sigma^* \times \Int^{k_0}$ is a CERL defined by a CEFA $\CEFA=(Q, \Sigma, R, \delta, I, F)$ with $R= (r_1, \cdots, r_{k_0})$. Then the $R$-cost-enriched pre-image of $L$ under $f$, denoted by $f^{-1}_R(L)$, is a pair $(\cR, \vec{t})$ such that 
\begin{itemize}
\item $\cR \subseteq (\Sigma^* \times \Int^{k_1 + k_0}) \times \cdots \times (\Sigma^* \times \Int^{k_l + k_0})$,
\item $\vec{t} = (t_1, \cdots ,t_{k_0})$ is a vector of linear integer terms where for each $i \in [k_0]$, $t_i$ is a term whose variables are from $\{r^{(1)}_i, \cdots, r^{(l)}_i\}$ (intuitively, each cost register $r_i$ is split into $l$ cost registers $r^{(1)}_i, \cdots,r^{(l)}_i$, one for each string input parameter, and $t_i$ tells how to compute $r_i$ from $r^{(1)}_i, \cdots,r^{(l)}_i$), and
\item $L$ is equal to the language comprising the $k_0$-cost-enriched strings
%
\[\left(w_0, t_1\left[d^{(1)}_{1}/r^{(1)}_1, \cdots, d^{(l)}_{1}/r^{(l)}_1\right], \cdots, t_{k_0}\left[d^{(1)}_{k_0}/r^{(1)}_{k_0}, \cdots, d^{(l)}_{k_0}/r^{(l)}_{k_0}\right]
\right), \]
%
such that 
\[w_0 \in f\left((w_1, \vec{c_1}), \cdots, (w_l, \vec{c_l}\right)) \mbox{ for some } ((w_1, (\vec{c_1}, \vec{d_1})), \cdots, (w_l, (\vec{c_l}, \vec{d_l}))) \in \cR,\]
where $\vec{c_1} \in \Int^{k_1}$, $\cdots$, $\vec{c_l} \in \Int^{k_l}$, $\vec{d_1} = (d^{(1)}_{1}, \cdots, d^{(1)}_{k_0}) \in \Int^{k_0}$, $\cdots$, and $\vec{d_l} = (d^{(l)}_{1},\cdots, d^{(l)}_{k_0}) \in \Int^{k_0}$.
\end{itemize}
The $R$-cost-enriched pre-image of $L$ under $f$, say $f^{-1}_R(L)=(\cR, \vec{t})$, is said to be CERR-definable if $\cR$ is a CERR. A CEFA representation of a CERR-definable $f^{-1}_R(L)=(\cR, \vec{t})$ is a tuple $((\CEFA_{i,1}, \cdots, \CEFA_{i, l})_{i \in [n]}, \vec{t})$ such that $(\CEFA_{i,1}, \cdots, \CEFA_{i, l})_{i \in [n]}$ is a CEFA representation of $\cR$, where $R(\CEFA_{i,j})=(r'_{j,1}, \cdots, r'_{j,k_j}, r^{(j)}_1, \cdots,r^{(j)}_{k_0})$ for each $i \in [n]$ and $j \in [l]$. (The cost registers $r'_{1,1}, \cdots, r'_{1,k_1},\cdots, r'_{l,1}, \cdots, r'_{l,k_l}, r^{(1)}_1, \cdots,r^{(1)}_{k_0}, \cdots, r^{(l)}_1, \cdots,r^{(l)}_{k_0}$ are mutually distinct and fresh.) 
\end{definition}

\begin{example}\label{exm:pre-image}
Let $L = \{(w, |w|) \mid w \in \Lang((aa)^*b(bb)^*) \}$. Evidently $L$  is a CERL defined by a CEFA $\CEFA = (Q, \{a,b\}, R, \delta, I, F)$ with $R=(r_1)$. Since the concatenation operation $\concat$  is a string function from $\Sigma^* \times \Sigma^*$ to $\Sigma^*$, $\concat^{-1}_R(L)$, the $R$-cost-enriched pre-image of $L$ under concatenation $\concat$, is the pair $(\cR, t)$, where $t=r^{(1)}_1+r^{(2)}_1$ and 
\[\cR = L_{1,1} \times L_{1,2} \cup L_{2,1} \times L_{2,2} \cup L_{3,1} \times L_{3,2} \cup L_{4,1} \times L_{4,2} \cup L_{5,1} \times L_{5,2},\]
such that
\begin{itemize}
\item $L_{1,1} = \{(w_1, |w_1|) \mid w_1 \in \Lang((aa)^*)\}$ and $L_{1,2} = \{(w_2, |w_2|) \mid w_2 \in \Lang(b(bb)^*)\}$,
%
\item $L_{2,1} = \{(w_1, |w_1|) \mid w_1 \in \Lang((aa)^*)\}$ and $L_{2,2} = \{(w_2, |w_2|) \mid w_2 \in \Lang((aa)^*b(bb)^*)\}$,
%
\item $L_{3,1} = \{(w_1, |w_1|) \mid w_1 \in \Lang(a(aa)^*)\}$ and $L_{3,2} = \{(w_2, |w_2|) \mid w_2 \in \Lang(a(aa)^*b(bb)^*)\}$,
%
\item $L_{4,1} = \{(w_1, |w_1|) \mid w_1 \in \Lang((aa)^*b(bb)^*)\}$ and $L_{4,2} = \{(w_2, |w_2|) \mid w_2 \in \Lang((bb)^*)\}$,
%
\item $L_{5,1} = \{(w_1, |w_1|) \mid w_1 \in \Lang((aa)^*(bb)^*)\}$ and $L_{5,2} = \{(w_2, |w_2|) \mid w_2 \in \Lang(b(bb)^*)\}$.
\end{itemize}
It is easy to see that $\cR$ is a CERR. Thus $\concat^{-1}_R(L)$ is CERR-definable.
\end{example}

It turns out that for each string operation $f$ in {\slint}, the cost-enriched pre-images of CERLs under $f$ are CERR-definable.

\begin{proposition}\label{prop-pre-image}
Let $L$ be a CERL defined by a CEFA $\CEFA = (Q, \Sigma, R, \delta, I, F)$. Then for each string operation $f$ ranging over $\concat$, $\replaceall_{e,u}$, $\reverse$, NFTs $\NFT$, and $\substring$, $f^{-1}_R(L)$ is CERR-definable. In addition,
\begin{itemize}
\item a CEFA representation of $\concat^{-1}_R(L)$ can be computed in time $\bigO(|\CEFA|^2)$, 
%
\item a CEFA representation of $\reverse^{-1}_R(L)$ (resp. $\substring^{-1}_R(L)$) can be computed in time $\bigO(|\CEFA|)$,
%
\item a CEFA representation of  $(\Tran(\NFT))^{-1}_R(L)$ can be computed in time polynomial in $|\CEFA|$ and exponential in $|\NFT|$,
%
\item a CEFA representation of  $(\replaceall_{e,u})^{-1}_R(L)$ can be computed in time polynomial in $|\CEFA|$ and exponential in $|e|$ and $|u|$.
\end{itemize}
\end{proposition}

\begin{proof}
Let $\CEFA=(Q, \Sigma, R, \delta, I, F)$ be a CEFA with $R= (r_1, \cdots, r_k)$. We show how to construct a CEFA representation of $f^{-1}_R(L)$ for each function $f$ in {\slint}.

%%%%%%%%%%%%%%%%%%%%%%%%%%%%%%%%%%%%%%%%%%%%%%%%%%%%%%%%%%%%%%%%%%%%%%%%%%%%%%
\paragraph*{$\concat^{-1}_R(L)$.}
%
A CEFA representation of $\concat^{-1}_R(L)$ is given by $((\CEFA_{I, q}, \NFA_{q, F})_{q \in Q}, \vec{t})$, where 
\begin{itemize}
\item $\CEFA_{I, q}=(Q, \Sigma, R^{(1)}, \delta^{(1)}, I, \{q\})$ and  $\CEFA_{q, F}=(Q, \Sigma, R^{(2)}, \delta^{(2)}, \{q\}, F)$ such that 
\begin{itemize}
\item $R^{(1)} = (r^{(1)}_1, \cdots, r^{(1)}_k)$, $R^{(2)} = (r^{(2)}_1, \cdots, r^{(2)}_k)$, 
\item $\delta^{(1)}$ comprises the tuples $(q, a, q', \eta')$ satisfying that there exists $\eta$ such that $(q, a, q', \eta) \in \delta$ and for each $j \in [k]$, and $\eta'(r^{(1)}_j)=\eta(r_j)$,  similarly for $\delta^{(2)}$,
\end{itemize}
\item and $\vec{t} = (r^{(1)}_1 + r^{(2)}_1, \cdots, r^{(1)}_k + r^{(2)}_k)$.
\end{itemize}
Note that the size of $((\CEFA_{I, q}, \NFA_{q, F})_{q \in Q}, \vec{t})$ is $\bigO(|\CEFA|^2)$.
%
%%%%%%%%%%%%%%%%%%%%%%%%%%%%%%%%%%%%%%%%%%%%%%%%%%%%%%%%%%%%%%%%%%%%%%%%%%%%%%
%
\paragraph*{$\reverse^{-1}_R(L)$.} 
%
A CEFA representation of $\reverse^{-1}_R(L)$ is given by $(\CEFA^{(r)}, \vec{t})$, where 
\begin{itemize}
\item $\CEFA^{(r)}=(Q, \Sigma, R^{(1)}, \delta', F, I)$ such that 
\begin{itemize}
\item $R^{(1)}=(r^{(1)}_1,\cdots,r^{(1)}_k)$, and 
\item $\delta'$ comprises the tuples $(q', a, q, \eta')$ satisfying that there exists $\eta$ such that $(q, a, q', \eta) \in \delta$, and $\eta'(r^{(1)}_i) = \eta(r_i)$ for each $i \in [k]$,
\end{itemize}
%
\item and $\vec{t}=(r^{(1)}_1, \cdots, r^{(1)}_k)$. 
\end{itemize}
Note that $\Lang(\CEFA^{(r)}) = \{(w^{(r)}, \vec{n}) \mid (w, \vec{n}) \in \Lang(\CEFA)\}$, and the size of $(\CEFA^{(r)}, \vec{t})$ is $\bigO(|\CEFA|)$.

%%%%%%%%%%%%%%%%%%%%%%%%%%%%%%%%%%%%%%%%%%%%%%%%%%%%%%%%%%%%%%%%%%%%%%%%%%%%%%

\paragraph*{$\substring^{-1}_R(L)$.}
A CEFA representation of $\substring^{-1}_R(L)$ is given by $(\cB, \vec{t})$, where 
\begin{itemize}
\item $\cB = (Q', \Sigma, R', \delta', I', F')$ such that 
\begin{itemize}
\item $Q' = Q \times \{p_0, p_1, p_2\}$, (intuitively, $p_0$, $p_1$, and $p_2$ denote that the current position is before the starting position, between the starting position and ending position, and after the ending position respectively)
%
\item $R' = \left(r^{(1)}_1,\cdots, r^{(1)}_k, r'_1, r'_2 \right)$, (intuitively, $r'_1$ denotes the starting position, and $r'_2$ denotes the length of the substring)
%
\item $I'=I \times \{p_0\}$, $F'=F \times \{p_2\}$,
%
\item and $\delta'$ comprises 
\begin{itemize}
\item the tuples $((q, p_0), a, (q, p_0), \eta')$ such that $q \in I$, $a \in \Sigma$, and $\eta'$ satisfies that $\eta'(r^{(1)}_j)=0$ for each $j \in [k]$, $\eta'(r'_1)= 1$, and $\eta'(r'_2) = 0$,
%
\item the tuples $((q, p_0), a, (q', p_1), \eta')$ such that there exists $\eta$ satisfying that $(q, a, q', \eta) \in \delta$, $\eta'(r^{(1)}_j)=\eta(r_j)$ for each $j \in [k]$,  $\eta'(r'_1)=0$ (recall that the positions of strings start at $0$), and $\eta'(r'_2) = 1$,
%
\item the tuples $((q, p_0), a, (q', p_2), \eta')$ such that there exists $\eta$ satisfying that $(q, a, q', \eta) \in \delta$, $q' \in F$, $\eta'(r^{(1)}_j)=\eta(r_j)$ for each $j \in [k]$,  $\eta'(r'_1)=0$ (recall that the positions of strings start at $0$), and $\eta'(r'_2) = 1$,
%
\item the tuples $((q, p_1), a, (q', p_1), \eta')$ such that there exists $\eta$ satisfying that $(q, a, q', \eta) \in \delta$ and $\eta'(r^{(1)}_j)=\eta(r_j)$ for each $j \in [k]$, $\eta'(r'_1) = 0$, and $\eta'(r'_2) = 1$,
%
\item the tuples $((q, p_1), a, (q', p_2), \eta')$ such that there exists $\eta$ satisfying that $(q, a, q', \eta) \in \delta$, $q' \in F$, and $\eta'(r^{(1)}_j)=\eta(r_j)$ for each $j \in [k]$, $\eta'(r'_1) = 0$, and $\eta'(r'_2) = 1$,
%
%\item the tuples $((q, p_1), a, (q, p_2), \eta')$ such that $q \in F$ and $\eta'(r^{(1)}_j)=0$ for each $j \in [k]$, $\eta'(r'_1) = 0$, and $\eta'(r'_2) = 1$,
%
\item the tuples $((q, p_2), a, (q, p_2), \eta')$ such that $q \in F$ and $\eta'(r^{(1)}_j)=0$ for each $j \in [k]$, $\eta'(r'_1) = 0$, and $\eta'(r'_2) = 0$,
%
\end{itemize}
\end{itemize}
\item $\vec{t}=(r^{(1)}_1, \cdots, r^{(1)}_k)$.
\end{itemize}
Note that the size of $(\cB, \vec{t})$ is $\bigO(|\CEFA|)$.
%%%%%%%%%%%%%%%%%%%%%%%%%%%%%%%%%%%%%%%%%%%%%%%%%%%%%%%%%%%%%%%%%%%%%%%%%%%%%%
%
%
\paragraph*{$(\Tran(\NFT))^{-1}_R(L)$.}
%
Suppose $\NFT = (Q', \Sigma, \delta', I', F')$. Then a CEFA representation of $(\Tran(\NFT))^{-1}_R(L)$ is given by 
$(\cB, \vec{t})$, where 
\begin{itemize}
\item $\cB$ simulates the run of $\NFT$ on the input string, meanwhile, it simulates the run of $\CEFA$ on the output string of $\NFT$, formally, $\cB= (Q' \times Q, \Sigma, R^{(1)}, \delta'', I' \times I, F' \times F)$ such that 
\begin{itemize}
\item $R^{(1)}  = (r^{(1)}_1, \cdots, r^{(1)}_k)$, and
\item $\delta''$ comprises the tuples $((q'_1, q_1), a, (q'_2, q_2), \eta')$ satisfying one of the following conditions,
\begin{itemize}
\item there exist $u = a_1 \cdots a_n \in \Sigma^+$ and a transition sequence $p_0 \xrightarrow[\delta]{a_1, \eta_1} p_2 \cdots p_{n-1} \xrightarrow[\delta]{a_n, \eta_n} p_{n}$ in $\CEFA$ such that $(q'_1, a, q'_2, u) \in \delta'$, $p_0 = q_1$, $p_{n}= q_2$, and for each $j \in [k]$,  $\eta'(r^{(1)}_j) = \eta_1(r_j) + \cdots + \eta_n(r_j)$,
%
\item $(q'_1, a, q'_2, \varepsilon) \in \delta'$, $q_1 = q_2$, and $\eta'(r^{(1)}_j) =0$ for each $j \in [k]$,
\end{itemize}
\end{itemize}
%
\item $\vec{t}=(r^{(1)}_1, \cdots, r^{(1)}_k)$.
\end{itemize}
Note that the number of transitions of $\cB$ can be exponential in the worst case, since it summarises the updates of cost registers of $\CEFA$ on the output strings of the transitions of $\NFT$. More precisely,  let
\begin{itemize}
\item $\ell$ be the maximum length of the output strings of transitions of $\NFT$, 
\item $N$ be the maximum number of transitions between a given pair of states of $\CEFA$, and
\item  $C$ be the maximum absolute value of the integer constants occurring in $\CEFA$,
\end{itemize}
then $|\delta''|$, the cardinality of $\delta''$, is bounded by $|\delta'| \times |Q|^2 \times N^\ell $, and the integer constants occurring in each transition of $\delta''$ are bounded by $\ell C$. Therefore, 
the size of $(\cB, \vec{t})$ is 
\[
\bigO(|\delta'| \times |Q|^2 \times N^\ell \times k \log_2 (\ell C)).
\] 
Since $|\delta'|, \ell \le |\NFT|$, $|Q|, N, k \le |\CEFA|$, and $C \le 2^{|\CEFA|}$, we deduce that the size of $(\cB, \vec{t})$ is 
$
\bigO( |\NFT| \times  |\CEFA|^2 \times |\CEFA|^{|\NFT|} \times |\CEFA|^2 \log_2(|\NFT|))= |\CEFA|^{\bigO(|\NFT|)} |\NFT| \log_2(|\NFT|).$
%

%%%%%%%%%%%%%%%%%%%%%%%%%%%%%%%%%%%%%%%%%%%%%%%%%%%%%%%%%%%%%%%%%%%%%%%%%%%%%%
\paragraph*{$(\replaceall_{e,u})^{-1}_R(L)$.}
%
From the result in \cite{CCH+18}, we know that  a NFT $\NFT_{e,u}=(Q', \Sigma, \delta', I', F')$ can be constructed to capture $\replaceall_{e,u}$.  Moreover, 
\begin{itemize}
\item $|Q'|$, as well as $|\delta'|$, is $2^{\bigO(|e|)}$,
\item $\ell$, the maximum length of the output strings of transitions of $\NFT_{e,u}$, is $|u|$.
\end{itemize}
Then a CEFA representation of $(\replaceall_{e,u})^{-1}_R(L)$ can be constructed as that of $(\Tran(\NFT_{e,u}))^{-1}_R(L)$.
Let $N$ denote the maximum number of transitions between a given pair of states of $\CEFA$, and $C$ be the maximum absolute value of the integer constants occurring in $\CEFA$, which is bounded by $2^{|\CEFA|}$. Then the CEFA representation of $(\replaceall_{e,u})^{-1}_R(L)$ is of size 
\[
\bigO(|\delta'| \times |Q|^2 \times N^\ell \times k \log_2 (\ell C)) = 2^{\bigO(|e|)} |\CEFA|^2 |\CEFA|^{|u|} |\CEFA|^2 \log_2 |u|=2^{\bigO(|e|)} |\CEFA|^{\bigO(|u|)}.
\]

%
 according to the aforementioned discussion for NFTs.
% 
%
\end{proof}

\subsection{The Decision Procedure}

The decision procedure is nondeterministic and divided into four steps. 
\begin{description}
\item[Step I: Preprocessing.] 
%
\item[Step II: Removing the integer terms.]
%
\item [Step III: Removing the assignment statements]
%
\item[Step IV: Solving {\lasat} problem.]
\end{description}
preprossessing 

Without loss of generality, we assume that string operations only apply to string variables.

For each term $\length(x)$ occurring in $S$, introduce a \emph{fresh} integer variable $i$, replace every occurrence of $\length(x)$ by $i$, and add $\ASSERT{x \in \CEFA_{\rm len}[i/r_1]}$.  

For each term $\indexof_v(x, i)$ occurring in $S$, introduce two fresh integer variables $i_1$ and $i_2$, replace every occurrence of $\indexof_v(x, i)$ by $i_2$, and add $\ASSERT{x \in \CEFA_{\indexof_v}[i_1/r_1, i_2/r_2]}; \ASSERT{i_1 = i}$.  

Let $S':=S$ and $A':=A$. Moreover, let $A'':= \ltrue$. Then execute the following procedure to (partially) flatten the integer terms.
\begin{description}
\item[Step 1.] Recursively apply the following transformation until $S' \wedge A'$ contains no more occurrences of integer functions: Select an occurrence of integer functions, say $g(x_1, \vec{t_1}, \cdots, x_k, \vec{t_k})$, such that 
%it is a \emph{proper} subterm of the other integer term and 
{\it none} of $\vec{t_1}, \cdots, \vec{t_k}$ contains occurrences of integer functions, introduce a fresh integer variable $i$, let $S' \wedge A'$ be the formula obtained by replacing $g(x_1, \vec{t_1}, \cdots, x_k, \vec{t_k})$ with $i$, moreover, let $A'':= A'' \wedge i = g(x_1, \vec{t_1}, \cdots, x_k, \vec{t_k})$.
%
\item[Step 2.] It comprises the following two substeps. 
\begin{enumerate}
\item For each occurrence of string functions in $S'$, say $f(x_1, \vec{t_1}, \cdots, x_k, \vec{t_k})$, suppose $\vec{t_j} = (t_{j,1}, \cdots, t_{j, l_j})$ for each $j \in [k]$, introduce fresh integer variables $i_{j, j'}$ for $j \in [k]$ and $j' \in [l_j]$, replace $f(x_1, \vec{t_1}, \cdots, x_k, \vec{t_k})$ with $f(x_1, \vec{i_1}, \cdots, x_k, \vec{i_k})$ in $S'$, where $\vec{i_j} = (i_{j,1}, \cdots, i_{j, l_j})$ for each $j \in [k]$, and let $A':=A' \wedge \bigwedge \limits_{j \in [k], j' \in [l_j]} i_{j, j'} = t_{j, j'}$. 
\item For each occurrence of integer functions in $A''$, say $g(x_1, \vec{t_1}, \cdots, x_k, \vec{t_k})$, suppose $\vec{t_j} = (t_{j,1}, \cdots, t_{j, l_j})$ for each $j \in [k]$, introduce fresh integer variables $i_{j, j'}$ for $j \in [k]$ and $j' \in [l_j]$, replace $g(x_1, \vec{t_1}, \cdots, x_k, \vec{t_k})$ with $g(x_1, \vec{i_1}, \cdots, x_k, \vec{i_k})$ in $A''$, where $\vec{i_j} = (i_{j,1}, \cdots, i_{j, l_j})$ for each $j \in [k]$, and let $A':=A' \wedge \bigwedge \limits_{j \in [k], j' \in [l_j]} i_{j, j'} = t_{j, j'}$. 
\end{enumerate}
%
\item[Step 3.] Let $S:=S'$ and $A:=A'' \wedge A' $.
\end{description}
The aforementioned flattening procedure is a bit technical, for simplicity, we may assume that the integer terms are fully flattened, including the arithmetic operations.

Note that after the aforementioned flattening procedure, the resulting formula $S \wedge A$ satisfies the following property: 
\begin{quote}
The integer terms in all the occurrences of string and integer functions  are integer variables, moreover, each integer variable occurs at most once in these string and integer functions.  \hfill ($*$)
\end{quote}
Therefore, in the sequel, we assume that $S \wedge A$ satisfies the property ($*$).

\begin{theorem}\label{thm-sl-int-dec}
Path feasibility of {\slint} satisfying the semantic conditions is decidable.
\end{theorem}

\begin{proof}
In the following, we extend the generic decision procedure in \cite{CHL+19}, where NFA is replaced by CEFA.

Let $S \wedge A$ be an {\slint} formula (satisfying the property ($*$)).

For each occurrence of $i = g(x_1, \vec{i'_1}, \cdots, x_k, \vec{i'_k})$ in $A$ with $g$ an integer function, apply the following nondeterministic transformation to $A$: 
\begin{quote}
According to the 1st semantic condition, $g$ is a CERR linear integer function and a CEFA representation of $g$, say $((\NFA_{j,1}, \cdots, \NFA_{j, k})_{j \in [m]}, t)$, can be computed effectively from $g$. Consider $((\NFA'_{j,1}, \cdots, \NFA'_{j, k})_{j \in [m]}, t')$, where $\NFA'_{j,1}=\NFA_{j,1}[\vec{i'_1}/R(\NFA_{j,1})]$, $\cdots$, $\NFA'_{j,k}=\NFA_{j,k}[\vec{i'_k}/R(\NFA_{j,k})]$, and $t' = t[i^{(1)}/r^{(1)}, \cdots, i^{(k)}/r^{(k)}]$.
Nondeterministically choose $j \in [m]$, and replace $i = g(x_1, \vec{i'_1}, \cdots, x_k, \vec{i'_k})$ by $x_1 \in \NFA'_{j,1} \wedge \cdots \wedge x_k \in \NFA'_{j,k} \wedge i = t'$ in $A$.
\end{quote}
Note that after this transformation, $S \wedge A$ contains no occurrences of integer functions, moreover, as a result of the property ($*$), for every variable $x$, all the CEFAs to which $x$ belongs satisfy that their sets of registers are  mutually disjoint.

Then repeat the following procedure until $S$ becomes empty.
%
\begin{quote}
Suppose $y := f(x_1, \vec{i_1}, \cdots, x_k, \vec{i_k})$ is the last assignment of $S$. 
\\
Let $\rho := \{\NFA_1, \cdots, \NFA_s\}$ be the set of all CEFAs such that $y \in \NFA_j$ occurs in $A$ for each $j \in [s]$. Construct $\NFA = \NFA_1 \times \cdots \times \NFA_s$ (Recall that the sets of registers of $\NFA_1$, $\cdots$, $\NFA_s$ are mutually disjoint). Let  the vector of registers in $\NFA$ be $R = (r'_1, \cdots, r'_n)$. Then according to the 2nd semantic condition, 
a CEFA representation of the $R$-cost enriched pre-image of $\Lang(\NFA)$ under $f$, say $((\cB_{j, 1}, \cdots, \cB_{j, k})_{j \in [\ell]}, \vec{t})$, can be effectively computed from $\NFA$ and $f$. Consider $((\cB'_{j, 1}, \cdots, \cB'_{j, k})_{j \in [\ell]}, \vec{t'})$, where $\cB'_{j, 1} = \cB_{j, 1}[\vec{i_1}/R(\cB_{j,1}), \vec{(r')^{(1)}}/\vec{r^{(1)}}]$, $\cdots$, $\cB'_{j,k}=\cB_{j,k}[\vec{i_k}/R(\cB_{j,k}), \vec{(r')^{(k)}}/\vec{r^{(k)}}]$ (with $\vec{r^{(1)}}= (r^{1}_1, \cdots, r^{(1)}_n)$, similarly for $\vec{r^{(2)}}$ and so on), and $\vec{t'} = \vec{t}[\vec{r'_1}/\vec{r_1}, \cdots, \vec{r'_n}/\vec{r_n}]$ (with $\vec{r_1} = (r^{(1)}_1, \cdots, r^{(k)}_1)$, similarly for $\vec{r^{(2)}}$ and so on). 
\\
Nondeterministically choose $j \in [\ell]$ and let 
$$A:= A \wedge x_1 \in \cB'_{j, 1} \wedge \cdots \wedge x_k \in \cB'_{j, k}  \wedge \bigwedge \limits_{j' \in [n]} r'_{j'} = t'_{j'}.$$
%
Remove $y := f(x_1, \vec{i_1}, \cdots, x_k, \vec{i_k})$ from $S$.
\end{quote}

We would like to remark that if all the string functions $f$ in $S \wedge A$ are \emph{deterministic}, then the product of CEFAs before the pre-image computation can be avoided and the pre-image can be computed \emph{distributively} for CEFAs in $\rho$.

In the end, we get a formula $S \wedge A$ where $S$ is empty. Suppose $A = A_r \wedge A_i$, where $A_r$ is a conjunction of atomic formulae of the form $x \in \NFA$, and $A_i$ is linear arithmetic formula (containing no integer functions). By computing the product construction of CEFAs, $A_r$ can be rewritten as $x_1 \in \NFA_1 \wedge \cdots \wedge x_n \in \NFA_n$, where $x_1,\cdots, x_n$ are mutually distinct. Therefore, the path feasibility of $S \wedge A$ is exactly the satisfiability of $A_i$ w.r.t. the CEFAs $\NFA_1, \cdots, \NFA_n$. From Theorem~\ref{thm-incra-la-sat}, we conclude that the path feasibility of  {\slint} is decidable.
\qed
\end{proof}

\begin{corollary}
Path feasibility of {\cslint} is decidable.
\end{corollary}



%%%%%%%%%%%%%%%%%%%%%%%%%%%%%%%%%%%%%%%%%%%%%%%%
%%%%%%%%%%%%%%%%CERR linear integer functions removed%%%%%%%%%%%%%
%%%%%%%%%%%%%%%%CERR linear integer functions removed%%%%%%%%%%%%%
%%%%%%%%%%%%%%%%%%%%%%%%%%%%%%%%%%%%%%%%%%%%%%%%

\hide{
\begin{definition}[CERR linear integer functions]
An integer function $g: \Sigma^* \times \Int^{k_1} \times \Sigma^* \times \Int^{k_l} \rightarrow 2^\Int$ is  \emph{linear} if there is a pair $(\cR, t)$ such that $\cR \subseteq \Sigma^* \times \Int^{k_1+1} \times \Sigma^* \times \Int^{k_l+1}$ is a CERR and $t$ a linear integer term over $r^{(1)}, \cdots, r^{(l)}$ such that for all $\vec{c_1} \in \Int^{k_1}, \cdots, \vec{c_l} \in \Int^{k_l}$, and $d_1, \cdots, d_l \in \Int$, it holds that $(w_1, (\vec{c_1}, d_1), \cdots, w_l, (\vec{c_l}, d_l)) \in \cR$ iff $t[d_1/r^{(1)}, \cdots, d_l/r^{(l)}] \in g(w_1, \vec{c_1}, \cdots, w_l, \vec{c_l})$.  

For a CERR linear integer function $g$ witnessed by the pair $(\cR, t)$, a CEFA representation of $g$ is a tuple $((\NFA_{i,1}, \cdots, \NFA_{i, l})_{i \in [n]}, t)$, where $(\NFA_{i,1}, \cdots, \NFA_{i, l})_{i \in [n]}$ is a CEFA representation of $\cR$.

\end{definition}

\begin{example}
The string functions $\length$ and $\indexof_u$ are CERR linear integer functions, whose CEFA representations can be found in Section~\ref{sec-cslint}.
\end{example}
}

%%%%%%%%%%%%%%%%%%%%%%%%%%%%%%%%%%%%%%%%%%%%%%%%%%%%%%%%

\section{Implementation}

%!TEX root = main.tex

We have implemented the decision procedure on top of the string solver OSTRICH \cite{CHL+19}, resulting into a new solver OSTRICH+. OSTRICH is  written in Scala and based on the SMT solver Princess \cite{princess08}. 
OSTRICH+ reuses the parser of Princess. Moreover, it replaces NFAs in OSTRICH with CEFAs. Correspondingly, in OSTRICH+, the pre-image  (computation) operators for concatenation, $\replaceall$, $\reverse$, and finite-state transducers are reimplemented, and a new pre-image operator for $\substring$ is added. OSTRICH+ also adds CEFA constructions for $\length$ and $\indexof$.  

Similarly to OSTRICH, OSTRICH+ performs a depth-first exploration of the search tree resulting from repeatedly
splitting the disjunctions (or unions) in the cost-enriched recognisable pre-images of CERLs under string functions, as well as the case splits in the semantics of $\indexof$ and $\substring$.
The pseudo-code of Step II-III of the decision procedure in Section~\ref{sec:dec} is given by  the function $\mathit{checkSat}$ in Algorithm~\ref{alg:checksat}, which calls two functions $\mathit{indexofCaseSplit}$ and $\mathit{substringCaseSplit}$ for the case splits in the semantics of $\indexof_v$ and $\substring$. Due to space limit, the details of $\mathit{indexofCaseSplit}$ and $\mathit{substringCaseSplit}$ are omitted here and can be found in the appendix. Moreover,  $\mathit{checkSat}$ calls a recursive function  $\mathit{BackDfsExp}$ in Algorithm~\ref{alg:dfs} for the depth-first exploration (Step IV of the decision procedure), which in turn calls a function $\mathit{CheckCefaLIASat}$ to solve the {\lasat} problem (Step V). Note that Step I of the decision procedure is handled by the DPLL(T) procedure in Princess and omitted here. 

%%%%%%% the pseudo-code
%%%%%%% the pseudo-code
\begin{algorithm}[tbp]
\SetKw{Continue}{continue}
  \small
  \KwIn{$active$: set of CEFA constraints,  $arith$: arithmetic constraint,
%    $x \in L$,
    $\mathit{funApps}$: acyclic set of assignment statements. }
  \KwResult{$sat$ if the input constraints are satisfiable, and $unsat$ otherwise.\newline
   }
%  \Begin{
    	\For{each partition $(\mathcal{I}_l)_{l \in [5]}$ of the set of $\indexof_v(x, i)$ in $arith$ and \newline
	\hspace*{4mm} each partition $(\mathcal{J}_l)_{l \in [3]}$ of the set of $y:=\substring(x, i, j)$ in $\mathit{funApps}$ 
	}
	{
		\tcc{Case splits for semantics of $\indexof$ and $\substring$}
		$(active, arith, \mathit{funApps}) = \mathit{indexofCaseSplit}(active, arith, \mathit{funApps}, (\mathcal{I}_l)_{l \in [5]})$\; 
		$(active, arith, \mathit{funApps})= \mathit{substringCaseSplit}(active, arith, \mathit{funApps}, (\mathcal{J}_l)_{l \in [3]})$\; 
		\For{each $\length(x)$ occurring in $arith$}
		{
			choose a fresh integer variable $i$\;
			$active \leftarrow active \cup \{x \in \CEFA_{\rm len}[i/r_1]\}$; $arith\leftarrow arith[i/\length(x)]$;
		}
		\For{each $\indexof_v(x,i)$ occurring in $arith$}
		{
			choose fresh integer variables $i_1,i_2$\;
			$active \leftarrow active \cup \{x \in \CEFA_{\indexof_v}[i_1/r_1,i_2/r_2]\}$; $arith\leftarrow arith[i_2/\indexof_v(x,i)] \wedge i=i_1$;
		}
		\If{$\mathit{BackDfsExp}(active, \emptyset, arith, \mathit{funApps})$}
		{
			\Return{$sat$};}
%		{\Continue;}
	}
	\Return $unsat$; 		
%}
  \caption{Function $\mathit{checkSat}$
    for Step II-III} \label{alg:checksat} 
\end{algorithm}


\paragraph*{Optimisations for solving the {\lasat} problem.} From Proposition~\ref{prop:la-sat-cefa-inter}, a natural approach to solve the {\lasat} problem is to compute an existential LIA formula defining the Parikh image of products of CEFAs, then use SMT solvers, e.g. CVC4 or Z3, to decide the satisfiability of existential LIA formulas. Nevertheless, this approach faces the state explosion problem when computing the products of CEFAs.  We did implement this approach in the beginning. However, some preliminary experiments showed that this approach is not scalable. Therefore, in the implementation of the function $\mathit{CheckCefaLIASat}$ in Algorithm~\ref{alg:dfs},  we finally choose to utilise the symbolic model checker nuXmv \cite{nuxmv} to elevate the state explosion during the computation of products of CEFAs. The nuXmv tool is a well-known symbolic model checker that is capable of solving the model checking problem for both finite and infinite state systems. Our main idea is to encode the {\lasat} problem as instances of the model checking problem and then call nuXmv to solve it. Since  {\lasat} is a problem for quantifier-free LIA formulas and CEFAs that contain integer variables, the {\lasat} problem actually corresponds to the problem of model checking \emph{infinite state systems}. In the following, we use a simple example to illustrate how to encode the instances of  the {\lasat} problem into the inputs of nuXmv.

\begin{example}
Encoding of {\lasat} into nuXmv.
\end{example}


\begin{algorithm}[tbp]
\SetKw{Continue}{continue}
  \small
  \KwIn{$active, passive$: sets of CEFA constraints,  $arith$: arithmetic constraint,
%    $x \in L$,
    $\mathit{funApps}$: acyclic set of assignment statements. }
  \KwResult{$sat$ if the input constraints are satisfiable, and $unsat$ otherwise.\newline
   }

%  \Begin{
    \eIf{$\mathit{active} = \emptyset$}{
%      \tcc{use symbolic model checker {NuXmv} to check whether CEFA constraints are consistent with arithemtic Constraints}
      \tcc{Check whether the LIA constraint $arith$ is satisfiable with respect to the CEFA constraints in $passive$ (i.e. Step V).}
      \Return{$\mathit{CheckCefaLIASat}(passive, arith)$;} 
    }{
   	choose a CEFA constraint $x \in \CEFA$ in $active$ with $R(\CEFA)=(r_1,\cdots,r_k)$\;
	\eIf{there is an assignment~$x := f(y_1, \vec{i_1}, \ldots, y_l,\vec{i_l})$ defining $x$ in $\mathit{funApps}$ with \newline
	\hspace*{4mm} $\vec{i_j}=(i_{j,1},\cdots, i_{j, k_j})$ for $j \in [l]$
	}
	{
%	      	\tcc{Compute the $R(\CEFA)$-cost enriched pre-image of $\Lang(\CEFA)$ under $f$}
		compute $f^{-1}_{R(\CEFA)}(\Lang(\CEFA)) = \left((\CEFA^{(1)}_{j}, \cdots, \CEFA^{(l)}_{j})_{j \in [n]}, \vec{t}\right)$ where \newline 
		 $R(\CEFA^{(j')}_{j})=\left((r')^{(j', 1)}, \cdots,(r')^{(j', k_{j'})}, r^{(j')}_1,\cdots, r^{(j')}_k \right)$ for $j \in [n]$ and $j' \in [l]$\;
	        $active \leftarrow active \setminus \{x \in \CEFA\}$; $passive \leftarrow passive \cup \{x \in \CEFA\}$\;    
	        \For{$j \leftarrow 1$ \KwTo $n$}{
        		$active \leftarrow active \cup \{y_1 \in \CEFA^{(1)}_{j}, \ldots, y_l \in \CEFA^{(l)}_{j}\} $\;
        		$ arith \leftarrow arith \wedge \bigwedge_{j' \in [l], j'' \in [k_{j'}]} i_{j',j''} = (r')^{(j', j'')} \wedge \bigwedge_{j' \in [k]} r_{j'} = t_{j'}$\;
        		\eIf{$active \cup passive$ is inconsistent}{\Continue \tcc*{backtrack}}
		{
		          \Switch{$\mathit{BackDfsExp}(active, passive, arith, \mathit{funApps})$}{
				\lCase{$sat$}
					{\Return{$sat$}}
				\Case{$unsat$}
					{\Continue \tcc*{backtrack}}
          			}
        		}
	}
        	\Return{unsat}; 
      	}
      	{
        	\Return{$\mathit{BackDfsExp}(active \backslash \{x\in \CEFA\}, passive \cup \{x\in \CEFA\}, arith, \mathit{funApps})$;}
      	}
	} 
%}
  \caption{Function~$\mathit{BackDfsExp}$ for Step IV (depth-first exploration)}\label{alg:dfs}
\end{algorithm}
%%%%%%% the pseudo-code
%%%%%%% the pseudo-code




%%%%% Case splits of semantics %%%%
%%%%% Case splits of semantics %%%%

%reduce the number of registers
%\begin{itemize}
%\item $\substring(x, 0, i)$, $\indexof_v(x, 0)$, remove the input register,
%
%\item CEFAs without registers: product + minimization 
%
%CEFAs with one register updated with $+1$: product + minimization
%
%Other CEFAs: no optimization
%\end{itemize}


%The main bottleneck of the decision procedure is 
%
%$prefixOf(x, u), suffixOf(x,u), contains(x, u)$: transformed into regular constraints
%
%using nuxmv to avoid state explosion of the product operation.
%
%introduction to nuxmv
%
%introduction to the encoding into nuxmv instances
%
%\begin{example}
%An example for the nuxmv encoding.
%\end{example}
%
%start two threads, one guessing sat, another one guessing unsat, run concurrently
%
%three strategies: 
%
%product + parikh image
%
%product + nuxmv
%
%nuxmv




%%%%%%%%%%%%%%%%%%%%%%%%%%%%%%%%%%%%%%%%%%%%%%%%%%%%%%%%

\section{Evaluation}

%!TEX root = main.tex

We have compared OSTRICH+ with the state-of-the-art solvers on a wide range of benchmarks. The solvers we considered include CVC4, Z3-str3, two variants of Trau, namely Trau+ and Z3-Trau, and SLENT. 

We have implemented our decision procedure for path feasibility based on the tool OSTRICH, which is built on top of the SMT solver Princess \cite{}. 
%
%OSTRICH implements the optimised
%decision procedure for string functions as described in Section 5.1 (i.e. using distributivity of regular
%constraints across pre-images of functions) and has built-in support for concatenation, reverse, FFT,
%and replaceAll. Moreover, since the optimisation only requires that string operations are functional,
%we can also support additional functions that satisfy RegInvRel, such as replacee which replaces
%only the first (leftmost and longest) match of a regular expression. OSTRICH is extensible and new
%string functions can be added quite simply (Section 6.3).
%Our implementation adds a new theory solver for conjunctive formulas representing path
%feasibility problems to Princess (Section 6.1). This means that we support disjunction as well as
%conjunction in formulas, as long as every conjunction of literals fed to the theory solver corresponds
%to a path feasibility problem. OSTRICH also implements a number of optimisations to efficiently
%compute pre-images of relevant functions (Section 6.2). OSTRICH is entirely written in Scala and is
%open-source. We report on our experiments with OSTRICH in Section 6.4. The tool is available on
%GitHub6.  


\subsection{Benchmarks}
 
We have compared our tool with a number of existing solvers on a wide range of benchmarks. In
particular, we compared  with CVC4 1.6 \cite{}, SLOTH \cite{},
and Z3 configured to use the Z3-str3 string solver \cite{}. 

%We considered several sets
%of benchmarks which are described in the next sub-section. The results are given in Section 6.4.2.
%
%
%In [Holík et al. 2018] SLOTH was compared with S3P [Trinh et al. 2016] where inconsistent
%behaviour was reported. We contacted the S3P authors who report that the current code is unsupported;
%moreover, S3P is being integrated with Z3. Hence, we do not compare with this tool.
%
%
%The first set of benchmarks we call \textbf{Transducer}. It combines the benchmark
%sets of Stranger [Yu et al. 2010] and the mutation XSS benchmarks of [Lin and Barceló 2016]. The
%first (sub-)set appeared in [Holík et al. 2018] and contains instances manually derived from PHP
%programs taken from the website of Stranger [Yu et al. 2010]. It contains 10 formulae (5 sat, 5
%unsat) each testing for the absence of the vulnerability pattern .*<script.* in the program output.
%These formulae contain between 7 and 42 variables, with an average of 21. The number of atomic
%constraints ranges between 7 and 38 and averages 18. These examples use disjunction, conjunction,
%regular constraints, and concatenation, replaceAll. They also contain several one-way functional
%transducers (defined in SMTLIB in [Holík et al. 2018]) encoding functions such as addslashes and
%trim used by the programs. Note that transducers have been known for some time to be a good
%framework for specifying sanitisers and browser transductions (e.g., see the works by Minamide,
%Veanes, Saxena, and others [D’Antoni and Veanes 2013; Hooimeijer et al. 2011; Minamide 2005;
%Weinberger et al. 2011]), and a library of transducer specifications for such functions is available
%(e.g. see the language BEK [Hooimeijer et al. 2011]).
%
%The second (sub-)set was used by [Holík et al. 2018; Lin and Barceló 2016] and consists of 8
%formulae taken from [Kern 2014; Lin and Barceló 2016]. These examples explore mutation XSS
%vulnerabilities in JavaScript programs. They contain between 9 and 12 variables, averaging 9.75, and
%9-13 atomic constraints, with an average of 10.5. They use conjunctions, regular constraints, concatenation,
%replaceAll, and transducers providing functions such as htmlEscape and escapeString.
%Our next set of benchmarks, SLOG, came from the SLOG tool [Wang et al. 2016] and consist of
%3,392 instances. They are derived from the security analysis of real web applications and contain 1-
%211 string variables (average 6.5) and 1-182 atomic formula (average 5.8).We split these benchmarks
%into two sets SLOG (replace) and SLOG (replaceall). Each use conjunction, disjunction, regular
%constraints, and concatenation. The set SLOG (replace) contains 3,391 instances and uses replace.
%SLOG (replaceall) contains 120 instances using the replaceAll operation.
%
%
%Our next set of benchmarks Kaluza is the well-known set of Kaluza benchmarks [Saxena et al.
%2010] restricted to those instances which satisfy our semantic conditions (roughly 80\% of the
%benchmarks). Kaluza contains concatenation, regular constraints, and length constraints, most of
%which admit regular monadic decomposition. There are 37,090 such benchmarks (28 032 sat).
%We also considered the benchmark set of [Chen 2018a,b]. This contains 42 hand-crafted benchmarks
%using regular constraints, concatenation, and replaceAll with variables in both argument
%positions. The benchmarks contain 3-7 string variables and 3-9 atomic constraints.
% 

 

\paragraph*{Benchmark suites.} We consider the following four benchmark suites.

The first benchmark suite {\transducerbench+} is generated from the {\transducerbench} benchmark suite of OSTRICH \cite{CHL+19}.  The {\transducerbench} suite involves the following seven transducers: toUpperCase and its dual toLowerCase, htmlEscape\footnote{\url{https://github.com/google/closure-library/blob/master/closure/goog/string/string.js#L549}} and its dual htmlUnescape, escapeString\footnote{\url{https://github.com/google/closure-library/blob/master/closure/goog/string/string.js#L878}}, addslashes\footnote{\url{http://php.net/manual/en/function.addslashes.php}}, and trim\footnote{\url{https://www.php.net/manual/en/function.trim.php}}. The {\transducerbench+} suite is generated from these transducers by encoding the idempotence, duality, commutativity, and equivalence\footnote{These problems have been investigated for transducers in \cite{BEK}.} problems into the path feasibility of {\slint} programs as follows:
\begin{description}
\item[Idempotence.] For each FFT $\NFT$,  we encode the non-idempotence of $\NFT$, namely deciding whether $\exists x.\ \NFT(\NFT(x)) \neq \NFT(x)$, into the path feasibility of the {\slint} program $y:=\NFT(x); z:=\NFT(y); S_{y \neq z}$, where $y$ and $z$ are two fresh string variables, and $S_{y \neq z}$ is the {\slint} program encoding $y \neq z$, which uses $\length$ and $\charat$ (see Section~\ref{sec:logic} for the details of $S_{y \neq z}$).
%
\item[Duality.] For each pair of distinct FFTs $\NFT_1$ and $\NFT_2$, we encode the non-duality of $\NFT_1$ and $\NFT_2$, namely deciding whether $\exists x.\ \NFT_2(\NFT_1(x)) \neq x$, into the path feasibility of the {\slint} program $y:=\NFT_1(x); z: = \NFT_2(y); S_{x \neq z}$.
%
\item[Commutativity.] For each pair of distinct FFTs $\NFT_1$ and $\NFT_2$, we encode the non-commutativity of $\NFT_1$ and $\NFT_2$, namely deciding whether $\exists x.\ \NFT_2(\NFT_1(x)) \neq \NFT_1(\NFT_2(x))$, into the path feasibility of the {\slint} program $y:=\NFT_1(x); z: = \NFT_2(y); y':= \NFT_2(x); z': = \NFT_1(y'); S_{z \neq z'}$.
%
\item[Equivalence.] For each pair of distinct FFTs $\NFT_1$ and $\NFT_2$, we encode the non-equivalence of $\NFT_1$ and $\NFT_2$, namely deciding whether $\exists x.\ \NFT_1(x) \neq \NFT_2(x)$, into the path feasibility of the {\slint} program $y:=\NFT_1(x); z: = \NFT_2(x); S_{y \neq z}$.
%
\end{description}
In total, we get 70 ($7+21+21+21$) \zhilin{to be checked} instances for the {\transducerbench+} suite. 

The second benchmark suite is the {\pyexbench} suite from \cite{ReynoldsWBBLT17}, which comprises \zhilin{xxx} instances. 
This suite is derived from PyEx, a symbolic executor for Python programs. The {\pyexbench} suite was generated by the CVC4 group on four popular Python packages: httplib2, pip, pymongo, and requests. These instances use regular constraints, concatenation, $\length$, $\substring$, and $\indexof$ functions. Moreover, the {\pyexbench} suite is further divided into the three sub-suites: {\pyextdbench} comprising \zhilin{xxx} instances, {\pyexztbench} comprising \zhilin{xxx} instances, and {\pyexzzbench} comprising \zhilin{xxx} instances.

The third benchmark suite {\slogbench} is adapted from the SLOG benchmark suite used by the SLOG tool~\cite{fang-yu-circuits} and has 3,392 instances \zhilin{the number to be checked}. 
%The SLOG are derived from the security analysis of real web applications and contain 1-211 string variables (average 6.5) and 1-182 atomic formula (average 5.8).
The SLOG benchmark suite contains only the string data type and no integer data type.
In order to evaluate OSTRICH+ more faithfully, we adapt each SLOG instance into one containing the integer data type, by choosing an output string variable\footnote{A string variable is called the output variable if it occurs in the left-hand side of an assignment statement, but does not appear in the right-hand sides of the assignment statements}, say $x$, and adding the statement $\ASSERT{2\ \indexof_{a}(x, 0) < \length(x)}$ for some $a \in \Sigma$.
% and $c \in \Nat$ with $c > 1$.
We split the {\slogbench} benchmark suite into two sub-suites \slogbenchr\ and \slogbenchra, which comprises 3,391 instances \zhilin{the number to be checked} and 120 instances \zhilin{the number to be checked} respectively. Both  \slogbenchr\ and \slogbenchra\ use regular constraints and concatenation. The difference is in that \slogbenchr\ uses the $\replace$ function (replacing the first occurrence), while \slogbenchra\ uses the $\replaceall$ function (replacing all occurrences).
%Each use conjunction, disjunction, regular constraints, and concatenation.
%The \slogbenchr\ sub-suite contains 3,391 instances \zhilin{the number to be checked}, while \slogbenchra\ sub-suite contains 120 instances \zhilin{the number to be checked}.


The fourth benchmark suite {\kaluzabench} is the well-known \emph{Kaluza} benchmark suite~\cite{Berkeley-JavaScript}.
{\kaluzabench} contains 47,284 instances. They use regular constraints, concatenation, and $\length$ function.





\paragraph*{Experimental results.}




%$z_1:=\charat(x,i); z_2 := \charat(y,i); \ASSERT{\bigvee_{a \in \Sigma} (z_1 \in \NFA_a \wedge z_2 \in \NFA_{\Sigma^* \setminus a})}$, where $\NFA_a$ is the NFA accepting $\{a\}$ and $\NFA_{\Sigma^* \setminus a}$ is the NFA accepting $\Sigma^* \setminus \{a\}$.

comparison

CVC4

Z3-STR3

SLENT

TRAU+ or Z3-TRAU

\definecolor{Gray}{gray}{0.9}
%\newcolumntype{g}{>{\columncolor{Gray}}c}
\begin{table}[htbp]
\begin{center}
\begin{tabular}{|c|c|c|c|c|c|c|c|}
\hline
& &  CVC4 & Z3-str3 & SLENT & TRAU+ & Z3-TRAU & Ostrich+\\
\hline
\multirow{4}{*}{\transducerbench(xxx)} & \cellcolor{Gray} sat & \cellcolor{Gray} & \cellcolor{Gray} & \cellcolor{Gray} & \cellcolor{Gray} & \cellcolor{Gray} & \cellcolor{Gray}\\
\cline{2-8}
 & unsat &  &  &  &  & &\\
\cline{2-8}
 & \cellcolor{Gray}  timeout & \cellcolor{Gray} & \cellcolor{Gray} & \cellcolor{Gray} & \cellcolor{Gray} &\cellcolor{Gray} &\cellcolor{Gray} \\
\cline{2-8}
 & error/unknown &  &  &  &  & &\\
\hline
\multirow{4}{*}{\pyextdbench(5569)} & \cellcolor{Gray} sat & \cellcolor{Gray} & \cellcolor{Gray} & \cellcolor{Gray} & \cellcolor{Gray} & \cellcolor{Gray} & \cellcolor{Gray}\\
\cline{2-8}
 & unsat &  &  &  &  & &\\
\cline{2-8}
 & \cellcolor{Gray}  timeout & \cellcolor{Gray} & \cellcolor{Gray} & \cellcolor{Gray} & \cellcolor{Gray} &\cellcolor{Gray} &\cellcolor{Gray} \\
\cline{2-8}
 & error/unknown &  &  &  &  & &\\
\hline
\multirow{4}{*}{\pyexztbench(8414)} & \cellcolor{Gray} sat & \cellcolor{Gray} & \cellcolor{Gray} & \cellcolor{Gray} & \cellcolor{Gray} & \cellcolor{Gray} & \cellcolor{Gray}\\
\cline{2-8}
 & unsat &  &  &  &  & &\\
\cline{2-8}
 & \cellcolor{Gray}  timeout & \cellcolor{Gray} & \cellcolor{Gray} & \cellcolor{Gray} & \cellcolor{Gray} &\cellcolor{Gray} &\cellcolor{Gray} \\
\cline{2-8}
 & error/unknown &  &  &  &  & &\\
\hline
\multirow{4}{*}{\pyexzzbench(11438)} & \cellcolor{Gray} sat & \cellcolor{Gray} & \cellcolor{Gray} & \cellcolor{Gray} & \cellcolor{Gray} & \cellcolor{Gray} & \cellcolor{Gray}\\
\cline{2-8}
 & unsat &  &  &  &  & &\\
\cline{2-8}
 & \cellcolor{Gray}  timeout & \cellcolor{Gray} & \cellcolor{Gray} & \cellcolor{Gray} & \cellcolor{Gray} &\cellcolor{Gray} &\cellcolor{Gray} \\
\cline{2-8}
 & error/unknown &  &  &  &  & &\\
\hline
\multirow{4}{*}{\slogbenchr(3391)} & \cellcolor{Gray} sat & \cellcolor{Gray} & \cellcolor{Gray} & \cellcolor{Gray} & \cellcolor{Gray} & \cellcolor{Gray} & \cellcolor{Gray} \\
\cline{2-8}
 & unsat &  &  &  &  & &\\
\cline{2-8}
 & \cellcolor{Gray}  timeout & \cellcolor{Gray} & \cellcolor{Gray} & \cellcolor{Gray} & \cellcolor{Gray} &\cellcolor{Gray} &\cellcolor{Gray} \\
\cline{2-8}
 & error/unknown &  &  &  &  & &\\
\hline
\multirow{4}{*}{\slogbenchra(120)} & \cellcolor{Gray} sat & \cellcolor{Gray} & \cellcolor{Gray} & \cellcolor{Gray} & \cellcolor{Gray} & \cellcolor{Gray} & \cellcolor{Gray}\\
\cline{2-8}
 & unsat &  &  &  &  & &\\
\cline{2-8}
 & \cellcolor{Gray}  timeout & \cellcolor{Gray} & \cellcolor{Gray} & \cellcolor{Gray} & \cellcolor{Gray} &\cellcolor{Gray} &\cellcolor{Gray} \\
\cline{2-8}
 & error/unknown &  &  &  &  & &\\
\hline
\multirow{4}{*}{\kaluzabench(47284)} & \cellcolor{Gray} sat & \cellcolor{Gray} & \cellcolor{Gray} & \cellcolor{Gray} & \cellcolor{Gray} & \cellcolor{Gray} & \cellcolor{Gray}\\
\cline{2-8}
 & unsat &  &  &  &  & &\\
\cline{2-8}
 & \cellcolor{Gray}  timeout & \cellcolor{Gray} & \cellcolor{Gray} & \cellcolor{Gray} & \cellcolor{Gray} &\cellcolor{Gray} &\cellcolor{Gray} \\
\cline{2-8}
 & error/unknown &  &  &  &  & &\\
\hline
\multirow{2}{*}{Total(xxx)} & \cellcolor{Gray} solved & \cellcolor{Gray} & \cellcolor{Gray} & \cellcolor{Gray} & \cellcolor{Gray} & \cellcolor{Gray} & \cellcolor{Gray}\\
\cline{2-8}
 & \cellcolor{Gray}  unsolved & \cellcolor{Gray} & \cellcolor{Gray} & \cellcolor{Gray} & \cellcolor{Gray} &\cellcolor{Gray} &\cellcolor{Gray} \\
\hline
\end{tabular}
\end{center}
\caption{Experimental results}
\label{tab-experiment}
\end{table}%



32-core-Intel-Xeon-E5-2690-@2.90GHz
8G memory

unsat------------------

totaltimeout = 30

nmuxvtimout = 30

total num : 0-3320

total-time: 10460 s

averange-time: 3.15 s 

sat: $0 / 0\%$

unsat: $2840 / 85.54\%$

unknown: $351 / 10.57\%$

timeout: $127 / 3.83\%$

parser error: $2 / 0.06\%$

sat--------------------

totaltimeout = 30

nmuxvtimout = 30

register bound = 30

total num : 0-18852

total time: 63840 s

averange-time: 3.39 s 

sat: $17668 / 93.72\%$

unsat: $0 / 0\%$


unknown: $1172 / 6.22\%$

timeout: $11 / 0.06\%$

parser error: $2 / 0.01\%$

%%%%%%%%%%%%%%%%%%%%%%%%%%%%%%%%%%%%%%%%%%%%%%%%%%%%%%%%%
\section{Conclusion}

%%%%%%%%%%%%%%%%%%%%%%%%%%%%%%%%%%%%%%%%%%%%%%%%%%%%%%%%%%%%%%%%%%%%%%%%%%%%%%%

\newpage 
\bibliographystyle{abbrv}
\bibliography{string}

%\iffalse
\newpage
%\setcounter{page}{1}

\begin{appendix}
%!TEX root = popl2018.tex

\appendix

\begin{center}
{\huge Supplementary Material} \\
{\large ``What's Decidable About String Constraints with ReplaceAll Function?''} 
\end{center}

\bigskip

We provide below proofs and examples that were omitted from the main text due to space constraints.

%%%%%%%%%%%%%%%%%%%%%%%%%%%%%%%%%%%%%%%%%%%%%%%%%%%%%%%%%
%%%%%%%%%%%%%%%%%%%%%%%%%%%%%%%%%%%%%%%%%%%%%%%%%%%%%%%%%
\hide{
\noindent {\it Proposition~\ref{prop-num-path}}.
{\it Let $G=(V,E)$ be a DAG such that the out-degree of each vertex is at most two. Then there are $n^{O(\dmdidx(G))}$ different paths  in $G$.
}

\begin{proof}
\end{proof}
}
%%%%%%%%%%%%%%%%%%%%%%%%%%%%%%%%%%%%%%%%%%%%%%%%%%%%%%%%%
%%%%%%%%%%%%%%%%%%%%%%%%%%%%%%%%%%%%%%%%%%%%%%%%%%%%%%%%%
\def\refpropundpat{\ref{prop-und-pat-var}}

\section{Proof of Proposition~\protect\refpropundpat}
\label{sec:prop-und-pat-var-proof}

We recall Proposition~\ref{prop-und-pat-var} and then give its proof.

\medskip

\noindent \textsc{Proposition}~\ref{prop-und-pat-var}
{\em The satisfiability problem of $\strline[\replaceall]$ is undecidable, if the second parameters of the $\replaceall$ terms are allowed to be variables.
}

\begin{proof}
	We reduce from the Post Correspondence Problem (PCP). Recall that the input of the problem consists of two finite lists $\alpha_{1},\ldots ,\alpha_{N}$ and $\beta_1,\ldots ,\beta_N$ of nonempty strings over $\Sigma$. A solution to this problem is a sequence of indices $(i_{k})_{1\leq k\leq K}$ with $ K\geq 1$ and $ 1\leq i_{k}\leq N$ for all $k$, such that
	$	\alpha _{{i_{1}}}\ldots \alpha _{{i_{K}}}=\beta _{{i_{1}}}\ldots \beta _{{i_{K}}}.
	$
	The PCP problem is to decide whether such a solution exists or not.
	
	Without loss of generality, suppose $\Sigma \cap [N] = \emptyset$ and $\$ \not \in \Sigma \cup [N]$. Let $\Sigma' = \Sigma \cup [N] \cup \{\$\}$. We will construct an $\strline[\replaceall]$ formula $C$ over $\Sigma'$ such that the PCP instance has a solution iff $C$ is satisfiable. To this end, the formula $C$ utilises the capability that the second parameter of the $\replaceall$ terms may be variables.
	
	Let $x_1, \cdots, x_N, y_1, \cdots, y_N, z$ be mutually distinct string variables. Then the formula $C = \varphi \wedge \psi$, where 
	%
	$$
	\begin{array}{l c l}
	\varphi & = & \bigwedge \limits_{i \in [N]} (x_i = \replaceall(x_{i-1}, i, \alpha_i) \wedge y_i = \replaceall(y_{i-1}, i, \beta_i)) \wedge  z = \replaceall(x_N, y_N, \$), \\
	\psi & = & x_0 \in (1 + \cdots + N)^+ \wedge z \in \$.
	\end{array}
	$$
	
	It is not hard to see that $\varphi$ is a straight-line relational constraint, thus $C$ is an $\strline[\replaceall]$ formula. Note that in $\replaceall(x_N, y_N, \$)$, the second parameter is a variable. We show that $C$ is satisfiable iff the PCP instance has a solution: $C$ is satisfiable iff there is a string $i_1 \cdots i_K \in \Ll((1 + \cdots + N)^+)$ such that when $x_0$ is assigned with $i_1 \cdots i_K$, the value of $z$ is $\$$.
	Since $z = \replaceall(x_N, y_N, \$)$ and $x_N, y_N \in \Sigma^+$, we know that $z$ is $\$$ iff the values of $x_N$ and $y_N$ are the same. Therefore, $C$ is satisfiable iff there is a string $i_1 \cdots i_K \in \Ll((1 + \cdots + N)^+)$ such that when $x_0$ is assigned with $i_1 \cdots i_K$, the values of $x_N$ and $y_N$ are the same. Therefore, $C$ is satisfiable iff there is a sequence of indices $i_1 \cdots i_K$ such that $\alpha_{i_1} \cdots \alpha_{i_K} = \beta_{i_1} \cdots \beta_{i_K}$, that is, the PCP instance has a solution.
	%
	%
	%Suppose the PCP instance has a solution. Then there is a sequence of indices $i_1 \cdots i_K$ such that $\alpha _{{i_{1}}}\ldots \alpha _{{i_{K}}}=\beta _{{i_{1}}}\ldots \beta _{{i_{K}}}$. Let $x_0$ be $i_1 \cdots i_K$. Then from the construction of $C$, we know that the values of $x_N$ and $y_N$ are $\alpha _{{i_{1}}}\ldots \alpha _{{i_{K}}}$ and  $\beta _{{i_{1}}}\ldots \beta _{{i_{K}}}$ respectively. Thus the values of $x_N$ and $y_N$ are the same. Therefore, the value of $z=\replaceall(x_N, y_N, \$)$ is $\$$. The formula $C$ is satisfiable. 
	%
	%
	%Since $x_0 \in (1 + \cdots + N)^+$, we know that $x_N, y_N$ can only be strings over the alphabet $\Sigma$. Therefore, $z \in \$$ iff $x_N = y_N$.
	%
	%	
	%	We then introduce, for $i=1,\cdots, N$, 
	%	$x_{i+1}=\replaceall(x_0, \alpha_i, i)$ and $y_{i+1}=\replaceall(y_0, \beta_i, i)$, 
	%	$x_0'=\replaceall(x_0, \sharp, \epsilon)$ and $y_0'=\replaceall(y_0, \sharp, \epsilon)$
	%	
	%	$x_{N+1}=y_{N+1}$, $x_0'=y'_0$
	%	
	%	
	%	with regular constraints $x_0\in \sharp((\sum_{i=1}^N\alpha_i)\sharp)^*$ and $y_0\in \sharp((\sum_{i=1}^N\beta_i)\sharp)^*$,
	%	
	%	where $z=z'$ can be encoded by 
	%		$z''=\replaceall(z, z', \$)$ and $z''\in \$$. 
\end{proof}

\def\refsecreplaceallsl{\ref{sec:replaceallsl}}

\section{Section~\protect\refsecreplaceallsl: The Correctness of the decision procedure}
\label{sec:dp-sl-correctness}

We argue that the procedure in Section~\ref{sec:dp-sl-general} is correct.
Note that Proposition~\ref{prop-sat-sl-case} removed a single $\replaceall(-,-,-)$ to obtain only regular constraints.
Each step of our decision procedure effectively eliminates a $\replaceall(-,-,-)$.
Similar to Proposition~\ref{prop-sat-sl-case}, each step maintains the satisfiability from the preceding step.

In more detail, from each $G_i$ we can define a constraint $C_i$. This constraint is a conjunction of the following atomic constraints.
\begin{itemize}
\item For each variable $x$ such that $(x, (\rpleft, a), y)$ and $(x, (\rpright,a), z)$ are the edges in $G_i$, we assert in $C_i$ that $x = \replaceall(y, a, z)$.
\item In addition, for each variable $x$ such that $\cE_i(x)$ is not empty, moreover, \emph{either $x$ is a source variable in $G_C$ (not $G_i$) or there are (incoming or outgoing) edges connected to $x$ in $G_i$}, let $e_i(x)$ be the regular expression equivalent to the conjunction of all constraints in $\cE_i(x)$ (Note that the conjunction of multiple regular expressions still defines a regular language). We assert in $C_i$ that $x \in e_i(x)$. Note that if $x$ is not a source variable in $G_C$ and there are no edges connected to $x$ in $G_i$, then the regular constraints in $\cE_i(x)$ are not included into $C_i$.
\end{itemize}

%This constraint is a conjunction of the following clauses.
%For each variable $x$ such that $\cE_i(x)$ is not empty, we let $e_i(x)$ be the regular expression equivalent to the conjunction of all constraints in $\cE_i(x)$.
%Since this is the conjunction of multiple regular expressions (NFAs), it is regular.
%We assert in $C_i$ that $x \in e_i(x)$.

It is immediate that $C_0$ is equivalent to $C$.
We require the following proposition, which gives us the correctness of the decision procedure by induction.
Note that the final $C_i$ when exiting the loop will be a conjunction of regular constraints on the source variables.

\begin{proposition}
    For each $i$,  let the $\rpleft$-edge and the $\rpright$-edge from $x$ to $y$ and $z$ respectively be the two edges removed from $G_i$ to construct $G_{i+1}$. Then $C_i$ is satisfiable iff there are sets $T_{j, z}$ such that $C_{i+1}$ is satisfiable.
\end{proposition}

We can see the above proposition by observing that, in each step, $C_i$ is of the form
\[
    x = \replaceall(y, a, z) \wedge x \in e_i(x) \wedge y \in e_i(y) \wedge z \in e_i(z) \wedge C'
\]
where $C'$ does not contain $x$, and $C_{i+1}$ is of the form
\[
    y \in e_{i+1}(y) \wedge z \in e_{i+1}(z) \wedge C' \ .
\]
Note that $C'$ remains unchanged since only the two edges leaving $x$ are removed from $G_i$ and $\cE_{i+1}(x') = \cE_i(x')$ for all $x'$ distinct from $x$, $y$, and $z$.
First assume $y \neq z$.
Supposing $C_i$ is satisfiable, an argument similar to that of Proposition~\ref{prop-sat-sl-case} shows that there are sets $T_{j,z}$ such that the same values of $y$ and $z$ also satisfy $e_{i+1}(y)$ and $e_{i+1}(z)$.
Since $C'$ is unchanged, all $x'$ distinct from $x$, $y$, and $z$ can also keep the same value.
Thus, $C_{i+1}$ is also satisfiable.
In the other direction, suppose that there are sets $T_{j, z}$ such that $C_{i+1}$ is satisfiable. Take a satisfying assignment to $C_{i+1}$.
From the assignment to $y$ and $z$ we obtain as in Proposition~\ref{prop-sat-sl-case} an assignment to $x$ that satisfies $\replaceall(y, a, z) \wedge x \in e_i(x)$.
Furthermore, the assignments for $y$ and $z$ also satisfy $e_i(y)$ and $e_i(z)$ since $\cE_i(y)$ and $\cE_i(z)$ are subsets of $\cE_{i+1}(y)$ and $\cE_{i+1}(z)$.
Finally, since $C'$ is unchanged, the assignments to all other variables also transfer, giving us a satisfying assignment to $C_i$ as required.
In the case where $y = z$, the arguments proceed analogously to the case $y \neq z$.

\def\prodauttitle{$\cA_1 \times \cA_u$}
\def\defutitle{$u = 010$}
\section{The product automaton \protect\prodauttitle for \protect\defutitle}

In Figure~\ref{fig-cs-exmp} we give the product automaton $\cA_1 \times \cA_u$ for $u = 010$.
This is a straightforward product construction, but may be useful for reference when understanding Figure~\ref{fig-cs-exmp-2} which shows the automaton $\cB_{\cA_1, u, T_z}$ which is derived from the product.

\begin{figure}[htbp]
\begin{center}
\includegraphics[scale=0.65]{constant-string-example.pdf}
\end{center}
\caption{The NFA $\cA_1 \times \cA_u$ for $u = 010$}\label{fig-cs-exmp}
\end{figure}
%

\def\refsecreplaceallcs{\ref{sec:replaceallcs}}
\section{Complexity analysis in Section~\protect\refsecreplaceallcs}
\label{sec:cs-complexity-full}

We provide a more detailed analysis of the complexity of the algorithm for the constant string case, described in Section~\ref{sec:replaceallcs}.
A summary of this argument already appears in Section~\ref{sec:replaceallcs}.

When constructing $G_{i+1}$ from $G_i$, suppose the two edges from $x$ to $y$ and $z$ respectively are currently removed, let the labels of the two edges be $({\sf l}, u)$ and $({\sf r}, u)$ respectively, then each element $(\cT, \cP)$ of $\cE_i(x)$ may be transformed into an element $(\cT', \cP')$ of $\cE_{i+1}(y)$ such that $|\cT'| = O(|u||\cT|)$, meanwhile, it may also be transformed into an element $(\cT'', \cP'')$ of $\cE_{i+1}(z)$ such that $\cT''$ has the same state space as $\cT$. Thus, for each source variable $x$, $\cE(x)$ contains at most exponentially many elements, and each of them may have a state space of at most exponential size. For instance, for a path from $x'$ to $x$ where the constant strings $u_1,\cdots, u_n$ occur in the labels of edges, an element $(\cT,\cP) \in \cE_0(x')$ may induce an element $(\cT', \cP')$ of $\cE(x)$ such that $|\cT'| \le |\cT| |u_1| \cdots |u_n|$, which is exponential in the worst case. 
%
To solve the nonemptiness problem of the intersection of all these regular constraints, the exponential space is sufficient. Consequently, in this case, we still obtain an EXPSPACE upper-bound. 

Let us now consider the special situation that the $\rpleft$-length of $G_C$ is bounded by a constant $c$.
Since $\dmdidx(G_C) \le \lftlen(G_C)$, we know that $\dmdidx(G_C)$ is also bounded by $c$. Therefore, according to Proposition~\ref{prop-di}, there are at most polynomially different paths in $G_C$, we deduce that for each source variable $x$, $\cE(x)$ contains at most polynomially many elements. In addition, since the number of $\rpleft$-edges in each path is bounded by $c$, during the execution of the decision procedure, the number of times when $(\cT, \cP)$ of $\cE_i(x)$ may be transformed into an element $(\cT', \cP')$ of $\cE_{i+1}(y)$ such that $|\cT'| = O(|u||\cT|)$ is bounded by $c$.
Therefore, for each source variable $x$ and each element $(\cT'', \cP'')$ in $\cE(x)$,  $|\cT''|$ is at most polynomial in the size of $C$. We then conclude that for each source variable $x$, $\cE(x)$ corresponds to the intersection of polynomially many regular constraints such that each of them has a state space of polynomial size. Therefore, the nonemptiness of the intersection of all the regular constraints in $\cE(x)$ can be solved in polynomial space. In this situation, we obtain a PSPACE upper-bound.


\def\refsecreplaceallre{\ref{sec:replaceallre}}
\section{Complexity analysis in Section~\protect\refsecreplaceallre}
\label{sec:re-complexity-full}

We provide a more detailed analysis of the complexity of the algorithm for the regular-expression case, described in Section~\ref{sec:replaceallre}.
A summary of this argument already appears in Section~\ref{sec:replaceallre}.

In each step of the reduction, suppose the two edges out of $x$ are currently removed, let the two edges be from $x$ to $y$ and $z$ and labeled by $({\sf l}, e)$ and $({\sf r}, e)$ respectively, then each element of $(\cT, \cP)$ of $\cE_i(x)$ may be transformed into an element $(\cT',\cP')$ of $\cE_{i+1}(y)$ such that $|\cT'| = |\cT| \cdot 2^{O(p(|e|))}$, meanwhile, it may also be transformed into an element $(\cT'',\cP'')$ of $\cE_{i+1}(y)$ such that $\cT''$ has the same state space as $\cT$. Thus, after the reduction, for each source variable $x$, $\cE(x)$ may contain exponentially many elements, and each of them may have a state space of exponential size, more precisely, if we start from a vertex $x$ without predecessors, with an element $(\cT,\cP)$ in $\cE_0(x)$, and go to a source variable $y$ through a path where $k$ edges have been traversed and removed, let $e_1,\cdots, e_k$ be the regular expressions occurring in the labels of these edges, then the resulting element in $\cE(y)$ has a state space of size $|\cT| \cdot 2^{O(p(|e_1|))} \cdot 2^{O(p(|e_2|))} \cdot \cdots \cdot 2^{O(p(|e_k|))}$ in the worst case. To solve the nonemptiness problem of the intersection of all these regular constraints, the exponential space is sufficient. Consequently, for the most general case of regular expressions, we still obtain an EXPSPACE upper-bound. 

On the other hand, for the situation that the $\rpleft$-length of $G_C$ is at most one, we wan to show that the algorithm runs in polynomial space. Suppose the $\rpleft$-length of $G_C$ is at most one. Then the diamond index of $G_C$ is at most one as well. According to Proposition~\ref{prop-di}, there are only polynomially many paths in $G_C$. Nevertheless, for each source variable $x$, $\cE(x)$ may contain an element $(\cT,\cP)$ such that $|\cT|$ is exponential. Since $|\cP|$ may be exponential, $(\cT,\cP)$ may correspond to the intersection of exponentially many regular constraints. However, we can show that $|\cP|$ is at most polynomial, as a result of the fact that the $\rpleft$-length of $G_C$ is at most one. The arguments proceed as follows: Suppose two edges from $x$ to $y, z$ respectively are removed, and an element $(\cT', \cP')$ of $\cE_{i+1}(y)$ such that $|\cT'|$ is exponential and $|\cP'|$ is polynomial, is generated from an element of $(\cT, \cP)$ of $\cE_i(x)$. Then $y$ must be a source variable in $G_C$. Otherwise, there is an $\rpleft$-edge out of $y$ and the $\rpleft$-length of $G_C$ is at least two, a contradiction. Therefore, $y$ is a source variable in $G_C$, $(\cT', \cP')$  will not be used to generate the regular constraints for the other variables. In other words, $y$ is a source variable in $G_C$, and $(\cT', \cP') \in \cE(y)$ with $|\cP'|$ polynomial. We then conclude that for each source variable $x$, $|\cE(x)|$  is at most polynomial in the size of $C$ and for each element $(\cT, \cP) \in \cE(x)$, $|\cP|$ is polynomial in the size of $C$. Therefore, for each source variable $x$,  $\cE(x)$ corresponds to the intersection of polynomially many regular constraints, where each of them has a state space at most exponential size. To solve the nonemptiness of the intersection of these regular constraints, the polynomial space is sufficient. We obtain a PSPACE upper-bound for the situation that the $\rpleft$-length of $G_C$ is at most one.


\def\refsecreplaceallre{\ref{sec:replaceallre}}
\section{Examples in Section~\protect\refsecreplaceallre}

Due to space constraints, we did not provide examples of the decision procedure for the regular-expression case.
We provide some examples here.


\begin{example}\label{exmp-pa-re}
	Let $e_0 = 0^*0 1(1^* + 0^*)$. Then $\cA_{0}$ and $\cA_{e_0}$ are illustrated in Figure~\ref{fig-pa-re}, where ${\sf sleft}$ and ${\sf slong}$ are the abbreviations of $\searchleft$ and $\searchlong$ respectively. Let us use the state $(\{q_{0,1}\}\{q_{0,0}\}, {\sf sleft}, \emptyset)$ to illustrate the construction. Since $\big(\delta_0(\{q_{0,1}\}, 0) \cup \delta_0(\{q_{0,0}\}, 0)\big) \cap F_0 = \{q_{0,1}\} \cap F_0 = \emptyset$, $\delta_0(\emptyset, 0) \cap F_0 = \emptyset$, and $\red(\delta_0(\{q_{0,1}\}, 0) \delta_0(\{q_{0,0}\}, 0))=\{q_{0,1}\}$, we deduce that the transition 
\[
    ((\{q_{0,1}\}\{q_{0,0}\}, {\sf sleft}, \emptyset), 0, (\{q_{0,1}\} \{q_{0,0}\}, {\sf sleft}, \emptyset)) \in \delta_{e_0} \ .
\]
On the other hand, it is impossible to go from the state $(\{q_{0,1}\}\{q_{0,0}\}, {\sf sleft}, \emptyset)$ to the ``$\searchlong$'' mode. This is due to the fact that $\delta_0(\{q_{0,0}\}, 0)=\{q_{0,1}\} \subseteq \delta_0(\{q_{0,1}\},0)=\{q_{0,1}\}$. In addition, there are no $1$-transitions out of $(\{q_{0,1}\}\{q_{0,0}\}, {\sf sleft}, \emptyset)$. This is due to the fact that $\delta_0(\{q_{0,1}\}, 1) \cap F_0 = \{q_{0,2}, q_{0,3}\} \cap F_0 \neq \emptyset$.
	%
	\begin{figure}[htbp]
		\begin{center}
			\includegraphics[scale=0.7]{regular-expression-example.pdf}
		\end{center}
		\caption{The NFA $\cA_0$ and $\cA_{e_0}$ for $e_0 = 0^*0 1(1^* + 0^*)$}\label{fig-pa-re}
	\end{figure} 
\end{example}

\begin{example}
	Let $C \equiv x = \replaceall(y, e_0, z) \wedge x \in e_1 \wedge y \in e_2 \wedge z \in e_3$, where $e_1,e_2,e_3$ are as in Example~\ref{exmp-sl} (cf. Figure~\ref{fig-sl-exmp}) and $e_0$ is as in Example~\ref{exmp-pa-re} (cf. Figure~\ref{fig-pa-re}). Suppose $T_z = \{(q_0, q_0), (q_1, q_2)\}$. Then the NFA $\cB_{\cA_1, e_0, T_z}$ is as illustrated in Figure~\ref{fig-re-exmp}, where the thick edges denote the added transitions. Let us use the state $(q_1, (\{q_{0,0}\}, \searchleft, \emptyset))$ to exemplify the construction. The transition $((q_1, (\{q_{0,0}\}, \searchleft, \emptyset)), 1, (q_2, (\{q_{0,0}\}, \searchleft, \emptyset)))$ is  in $\cA_1 \times \cA_{e_0}$. Since $\delta_0(q_{0,0}, 1) \cap F_0 = \emptyset$, this transition is not removed and is thus in $\cB_{\cA_1, e_0, T_z}$. On the other hand, since there are no $0$-transitions out of $q_1$ in $\cA_1$, there are no $0$-transitions from $(q_1, (\{q_{0,0}\}, \searchleft, \emptyset))$ to some state from $Q_{\searchleft}$ in $\cB_{\cA_1, e_0, T_z}$. 
	Moreover, because $((\{q_{0,0}\}, \searchleft, \emptyset), 0, (\{q_{0,1}\}, \searchlong, \emptyset)) \in \delta_{e_0}$ and $(q_1, q_2) \in T_z$, the transition $((q_1, (\{q_{0,0}\}, \searchleft, \emptyset)), 0, (q_1, (\{q_{0,1}\}, \searchlong, \emptyset)))$ is added. 
	One may also note that there are no 0-transitions from $(q_2, (\{q_{0,0}\}, \searchleft, \emptyset))$ to the state $(q_2, (\{q_{0,1}\}, \searchlong, \emptyset))$, because there are no pairs $(q2,-) \in T_z$.
	It is not hard to see that $010101 \in \Ll(\cA_2) \cap \Ll(\cB_{\cA_1, e_0, T_z})$. In addition, $10 \in \Ll(\cA_3) \cap \Ll(\cA_1(q_0,q_0)) \cap \Ll(\cA_1(q_1,q_2))$. Let $y$ be $010101$ and $z$ be $10$. Then $x$ takes the value $\replaceall(010101, e_0, 10)=10 \cdot \replaceall(101, e_0, 10)=10110$, which is accepted by $\cA_1$. Therefore, $C$ is satisfiable.
	\begin{figure}[htbp]
		\begin{center}
			\includegraphics[scale=0.68]{regular-expression-example-2.pdf}
		\end{center}
		\caption{The NFA $\cB_{\cA_1, e_0, T_z}$}\label{fig-re-exmp}
	\end{figure} 
\end{example}



\def\refsecext{\ref{sec-ext}}
\section{Undecidability Proofs for Section~\protect\refsecext}
\label{sec:ext-undec-proofs}

We provide the proofs of the theorems and propositions in Section~\ref{sec-ext} which show the undecidability of various extensions of our string constraints.

\subsection{Proof of Theorem~\ref{thm-ext-int}}

We begin with the first Theorem, which is recalled below.

\medskip

\noindent \textsc{Proposition}~\ref{thm-ext-int}
{\em    
    For the extension of $\strline[\replaceall]$ with \emph{integer constraints}, the satisfiability problem is undecidable, even if only a single integer constraint $|x| = |y|$ is used.
}


\begin{proof}
	The basic idea of the reduction is to simulate the two polynomials $f(x_1,\cdots, x_n)$ and $g(x_1,\cdots, x_n)$, where $x_1,\cdots,x_n$ range over the set of natural numbers, with two $\strline[\concat,\replaceall]$ formulae $C_f, C_g$ over a unary alphabet $\{a\}$, with the output string variables $y_f, y_g$ respectively, and simulate the equality $f(x_1,\cdots, x_n) = g(x_1,\cdots, x_n)$ with the integer constraint $|y_f|=|y_g|$ (which is equivalent to $y_f = y_g$, since $y_f, y_g$ represent strings over the unary alphabet $\{a\}$). 
	
	A polynomial $f(x_1,\cdots, x_n)$ or $g(x_1,\cdots, x_n)$ where $x_1, \cdots, x_n$ range over the set of natural numbers, can be simulated by an $\strline[\concat,\replaceall]$ formula over an unary alphabet $\{a\}$ as follows: The natural numbers are represented by the strings over the alphabet $\{a\}$. A string variable is introduced for each subexpression of $f(x_1,\cdots, x_n)$. The numerical addition operator $+$ is simulated by the string operation $\concat$ 
	%\mat{$\concat$ is not part of $\strline[\replaceall]$, can it be simulated when the string alphabet is unary, or do we need two extra characters?}\zhilin{changed to $\strline[\concat,\replaceall]$.}
	and the multiplication operator $*$ is simulated by $\replaceall$. Since it is easy to figure out how the simulation proceeds, we will only use an example to illustrate it and omit the details here. Let us consider $f(x_1,x_2) = x_1^2 + 2 x_1 x_2 + 5$. By abusing the notation, we also use $x_1,x_2$ as string variables in the simulation. We will introduce a string variable for each subexpression in $f(x_1,x_2)$, namely the variables $y_{x_1^2}, y_{x_1x_2}, y_{2x_1x_2}, y_{x_1^2+2x_1x_2}, y_{f(x_1,x_2)}$. Then $f(x_1,x_2)$ is simulated by the $\strline[\concat,\replaceall]$ formula
	\[
	\begin{array} {l c l }
	C_f & \equiv & y_{x_1^2} = \replaceall(x_1,a, x_1)\ \wedge y_{x_1x_2} = \replaceall(x_1, a, x_2)\ \wedge \\
	& & y_{2x_1x_2} = \replaceall(aa, a, y_{x_1x_2})\ \wedge y_{x_1^2+2x_1x_2} = y_{x_1^2} \concat y_{2x_1x_2}\ \wedge  \\
	& & y_{f(x_1,x_2)}=y_{x_1^2+2x_1x_2} \concat a a a a a\ \wedge x_1 \in a^*\ \wedge x_2 \in a^*.
	\end{array}
	\]
	Then according to Proposition~\ref{prop-concat}, $C_f, C_g$ can be turned into equivalent $\strline[\replaceall]$ formula $C'_f, C'_g$ by introducing fresh letters.
	%\mat{But we may have to give up the unary alphabet?}\zhilin{yes, you are  right, it is fine.}
	
	Since $C'_f$ and $C'_g$ share only source variables $x_1,\cdots, x_n$, we know that $C'_f \wedge C'_g$ is still an $\strline[\replaceall]$ formula.
	From the construction of $C'_f, C'_g$, it is evident that for every pair of polynomials $f(x_1,\cdots, x_n)$ and $g(x_1,\cdots, x_n)$, $f(x_1,\cdots, x_n) = g(x_1,\cdots, x_n)$ has a solution in natural numbers iff $C'_f \wedge C'_g \wedge |y_f| = |y_g|$ is satisfiable. The proof is complete.
	%
	%%%%%%%%%%%%%%%%%%%%%%%%%%%%%%%%%%%%%%%%%%%%%%%%%%%%%%%%%%%
	%%%%%%%%%%%%%%%%%%%%%%%%%%%%%%%%%%%%%%%%%%%%%%%%%%%%%%%%%%%
	\hide{
		We shall reduce from the aforementioned version of the Hilbert tenth problem. For any polynomial with positive integral  $f(x_1, \cdots, x_n)$ where each coefficient is a positive, we can construct a (division-free) arithmetic circuit (AC) is a directed  acyclic graph with nodes labelled with constants from $\mathbb{Z}$, or with some indeterminates $X_1, \cdots, X_m$, or with the operators $+, -, *$. The nodes labelled with constants are called constant nodes, while those labelled with indeterminates are called input nodes. Both constant and input nodes do not have incoming edges. Internal nodes are those labelled with $+,-,*$. Output node is the one which does not have out-going edges. Without loss of generality we assume that each internal node has in-degree 2, and there is only one output node. Each node in the circuit represents a multivariate polynomial $\mathbb{Z}[X_1, \cdots, X_m]$. Vice verse, each polynomial $f\in \mathbb{Z}[X_1, \cdots, X_m]$ can be represented as an AC, and, if the polynomial has only positive (integral) coefficients, the corresponding AC does not contain nodes labelled by $-$ or negative constants.  
		
		We observe that, given an AC, one can construct an SL[$\concat, \replaceall$] formula over the alphabet $\Sigma=\{a\}$ as follows. Each node $n$ of the AC is associated with a string variable $x_n$. As a result, each input node of the AC labelled by $X_i$ (i.e., the indeterminate) corresponds to a  source variable.   
		\begin{itemize}
			\item For each internal node $n$ labelled by $+$, suppose that $n$ has two children nodes $n_l$ and $n_r$, we introduce a string constraint $x_n= x_{n_l}\concat x_{n_l}$.  
			
			\item For each internal node $n$ labelled by $*$, suppose that $n$ has two children nodes $n_l$ and $n_r$, we introduce a string constraint $x_n= \replaceall(x_{n_l}, a, x_{n_l})$.  		
		\end{itemize}
		Furthermore, we introduce, for each node $n$ labelled by a constant $c$, a regular constraint $x_n=a^c$. 
		
		It is straightforward to verify, according to the semantics of SL[$\concat, \replaceall$], that:
		\begin{itemize}
			\item for relational constraint $x_n= x_{n_l}\concat x_{n_l}$, $|x_n|= |x_{n_l}|+|x_{n_l}|$; 
			\item for relational constraint $x_n= \replaceall(x_{n_l}, a, x_{n_l})$,  $|x_n|= |x_{n_l}|\cdot |x_{n_l}|$; and 
			\item for regular $x_n=a^c$, $|x_n|=c$. 
		\end{itemize}
		
		It follows that for each polynomial $f(x_1, \cdots, x_m)$ with positive integral coefficients, we can construct a straight-line string constraint $\varphi_{f}\wedge\psi_g$ over $\Sigma=\{a\}$ with $y_f$ as the output variant and $y_1, \cdots, y_n$ as source variables such that
		$f(c_1, \cdots, c_m)=|y|$ and, for each $1\leq i\leq m$, $|y_i|= c_i$ (i.e., $y_i=a^{c_i}$).  
		
		Consequently, when given two polynomials $f(x_1, \cdots, x_m)$ and $g(x_1, \cdots, x_m)$, we have straight-line string constraints $\varphi_{f}\wedge \varphi_{g}\wedge \psi_{f}\wedge \psi_g$ with two distinguished two variables  $y_f$ and $y_g$ such that  
		\[\exists x_1, \cdots, x_m. f(x_1, \cdots, x_m)=g(x_1, \cdots, x_m)\mbox{ iff } |y_f|=|y_g|\wedge \varphi_{f}\wedge \varphi_{g}\wedge \psi_{f}\wedge \psi_g\mbox{ is satisfiable} \]
		
		Finally, note that any  SL[$\concat, \replaceall$] constraints can be transformed into SL[$\replaceall$] constraints, we obtain a reduction from the Hilbert's 10th problem to the satisfiability problem of  SL[$\replaceall$] with length constraints, which entail that the latter problem is undecidable. The proof is completed. 
	}
	%%%%%%%%%%%%%%%%%%%%%%%%%%%%%%%%%%%%%%%%%%%%%%%%%%%%%%%%%%%
	%%%%%%%%%%%%%%%%%%%%%%%%%%%%%%%%%%%%%%%%%%%%%%%%%%%%%%%%%%%
\end{proof}

\subsection{Undecidability of Depth-1 Dependency Graph}

We recall the undecidability of a depth-1 dependency graph before providing the proof below.

\medskip

\noindent\textsc{Theorem}\ref{thm-ext-int-strong}
{\em
	For the extension of $\strline[\replaceall]$ with integer constraints, even if $\strline[\replaceall]$ formulae are restricted to those whose dependency graphs are of depth at most one, the satisfiability problem is still undecidable.
}

\medskip

A \emph{linear polynomial} (resp.\ quadratic polynomial) is a polynomial with degree at most one (resp.\ with degree at most two) where each coefficient is an integer. %of the form $a_0 + a_1x_1 + \cdots + a_n x_n$ (resp. a polynomial with degree at most two) where each coefficient $a_i\in \mathbb{Z}$  for $0 \leq i \leq n$. A quadratic polynomial

\begin{theorem}[\cite{ID04}]\label{thm-quad-eq}
	%	There exists some (fixed) $k$ such that no algorithm can solve Diophantine systems in the following form
	%	\[y_1F_1=G_1, t_1H_1=I_1, \cdots, t_kF_k = G_k, t_kH_k = I_k,\] 
	%
	%	where $F_i, G_i, H_i, I_i$ for $1\leq i\leq k$ are nonnegative linear polynomials over natural number variables  $s_1, \cdots, s_m$.
	The following problem is undecidable: Determine whether a system of equations of the following form has a solution in natural numbers, 
	\[
	\begin{array} {l l }
	A_i = B_i, & i =1, \cdots, k,\\
	y_iF_i=G_i \wedge y_i H_i = I_i, & i =1, \cdots, m, 
	\end{array}
	\] 
	%
	where $A_i, B_i, F_i, G_i$ are linear polynomials on the variables $x_1,\cdots, x_n$ (Note that each variable $y_i$ occurs in exactly two quadratic equations).
\end{theorem}

We can get a reduction from the problem in Theorem~\ref{thm-quad-eq} to the satisfiability of the extension of $\strline[\replaceall]$ with integer constraints as follows: For each monomial $y_i x_j$ in the quadratic polynomials, we use an $\strline[\replaceall]$ formula $z_{y_i x_j} = \replaceall(y_i, a, x_j)$ to simulate $y_i x_j$, where $z_{y_i x_j}$ are freshly introduced string variables. Since each equation $y_iF_i=G_i$ or $y_i H_i = I_i$ can be seen as a linear combination of the terms $y_i x_j$ and $x_j$ for $i \in [m]$ and $j \in [n]$, we can replace each variable $x_j$ with $|x_j|$, and each term $y_ix_j$ with $|z_{y_i x_j}|$,  thus transform them into the (linear) integer constraints $F'_i = G'_i$ or $H'_i = I'_i$. Similarly, after replacing each variable $x_j$ with $|x_j|$, we transform each equation $A_i= B_i$ into an integer constraint $A'_i = B'_i$. Therefore, we get a formula 
$$
\begin{array}{l c l }
\bigwedge \limits_{i \in [m], j \in [n]} z_{y_i x_j} = \replaceall(y_i, a, x_j) \wedge \bigwedge \limits_{i \in [m]} y_i \in a^*\ \wedge  \bigwedge \limits_{j \in [n]} x_j \in a^* \  \wedge\\
\hspace{2cm} \bigwedge \limits_{i \in [k]} A'_i = B'_i \wedge \bigwedge \limits_{i \in [m]} (F'_i = G'_i \wedge H'_i = I'_i),
\end{array}
$$
where the dependency graph of the $\strline[\replaceall]$ subformula is of depth at most one.

%%%%%%%%%%%%%%%%%%%%%%%%%%%%%%%%%%%%%%%%%%%%%%%%%
%%%%%%%%%%%%%%%%%%%%%%%%%%%%%%%%%%%%%%%%%%%%%%%%%
\hide{
	From this class of quadratic Diophantine equations, we can introduce string variables $x_1, \cdots, x_k$ and $y_1, \cdots, y_m$, together with relational string constraints 
	\[z_{i,j}=\replaceall(x_i, a, y_j)\]
	for $1\leq i\leq k$ and $1\leq j\leq m$. Note that, for each $i$,  $t_i F_i=G_i$ can be written as
	\begin{equation} \label{eq:dio}
	t_i\cdot \left(a_0+\sum_{j=1}^s a_j s_j\right) =  b_0+\sum_{j=1}^s b_j s_j
	\end{equation}
	where $a$'s and $b$'s are all natural numbers. Moreover, \eqref{eq:dio} holds iff 
	\[a_0\cdot |y_i|+ \sum_{j=1}^s a_j |z_{i,j}| =  b_0+ \sum_{j=1}^s b_j |x_j| \] 
	which is an integer constraint defined in Definition~\ref{def:intconst}. This entails that
}
%%%%%%%%%%%%%%%%%%%%%%%%%%%%%%%%%%%%%%%%%%%%%%%%%
%%%%%%%%%%%%%%%%%%%%%%%%%%%%%%%%%%%%%%%%%%%%%%%%%

\subsection{Undecidability of the Character Constraints}

We provide part of the proof of Proposition~\ref{prop-ext-ch-index}, in particular, we show the undecidability of character constraints.

\begin{proposition}\label{prop-ext-char}
	For the extension of $\strline[\replaceall]$ with character constraints, the satisfiability problem is undecidable. 
\end{proposition}

The arguments for Proposition~\ref{prop-ext-char} proceed as follows. Recall that in the proof of Theorem~\ref{thm-ext-int}, we get a formula $C_f \wedge C_g \wedge |y_f| = |y_g|$ such that $f(x_1,\cdots, x_n) = g(x_1,\cdots, x_n)$ has a solution in natural numbers iff $C_f \wedge C_g \wedge |y_f| = |y_g|$ is satisfiable. Let $\$ \neq a$. Suppose  $z_f = y_f \concat \$$, and $z_g = y_g \concat \$$. Then $|y_f| = |y_g|$ can be captured by $z_f[\mathfrak{n}] = \$[1] \wedge  z_g[\mathfrak{n}] = \$[1]$, where $\mathfrak{n}$ is a variable of type $\intnum$. More precisely, 
%
we have 
\begin{quote}
	\centering
	$C_f \wedge C_g \wedge |y_f|= |y_g|$ is satisfiable \\
	%
	iff \\
	%
	$C_f \wedge C_g \wedge z_f = y_f \concat \$ \wedge z_g = y_g \concat \$ \wedge z_f[\mathfrak{n}] = \$[1] \wedge  z_g[\mathfrak{n}] = \$[1]$ is satisfiable. 
\end{quote}
Therefore, we get a reduction from Hilbert's tenth problem to the satisfiability problem for the extension of $\strline[\replaceall]$ with character constraints. 

%For any two string variables $x,y$ on the unary alphabet $\{a\}$, let $x' = x \concat \$$ and $y' = y \concat \$$, then $|x| = |y|$ iff .
%
% $|x|=|y|$ iff $\exists n. x[n]=y[n]=\$$. 
%
%
%\begin{lemma}
%	For any two strings $x,y\in a^*\$$, $|x|=|y|$ iff $\exists n. x[n]=y[n]=\$$. 
%\end{lemma}
%
%As SL[$\replaceall$] with length constraints is undecidable, we conclude that 
 

\subsection{Undecidability of the $\indexof$ Constraints}

We provide the final part of the proof of Proposition~\ref{prop-ext-ch-index}, in particular, we show the undecidability of $\indexof$ constraints.

\begin{proposition}\label{prop-indexof}
	For the extension of $\strline[\replaceall]$ with the $\indexof$ constraints, the satisfiability problem is undecidable. 
\end{proposition}

Proposition~\ref{prop-ext-char} follows from the following observation and Theorem~\ref{thm-ext-int}: For any two string variables $x,y$ over a unary alphabet, 
$1= \indexof(x,y)$ iff $x$ is a prefix of $y$. Therefore, $|x| = |y|$ iff $1=  \indexof(x,y) \wedge 1= \indexof(y,x)$. This implies that in the proof of Theorem~\ref{thm-ext-int}, we can replace $|y_f| = |y_g|$ with $1=\indexof(y_f, y_g) \wedge 1 = \indexof(y_g, y_f)$ and get a reduction from Hilbert's tenth problem to the satisfiability problem for the extension of $\strline[\replaceall]$ with the $\indexof$ constraints.
Note that $=$ can be simulated as a conjunction of $\leq$ and $\geq$.
 




\end{appendix}

%\fi

\end{document}
