%!TEX root = main.tex

In this section, we will design a decision procedure for the path feasibility problem of {\slint} programs, based on the concepts of CERLs and CERRs introduced Section~\ref{sec:cefa}.

Before presenting the decision procedure, we introduce an additional concept, cost-enriched pre-images of CERLs under string operations. Moreover, we will show that the cost-enriched pre-images of CERLs under the string operations in {\slint}, namely, concatenation $\concat$, $\replaceall_{e,u}$, $\reverse$, NFTs $\NFT$, and $\substring$, are CERRs. 

To simplify the presentation of the decision procedure, in this section, we usually keep the string operations abstract by only mentioning the input and output data types, namely, we consider string operations $f: (\Sigma^* \times \Int^{k_1}) \times \cdots \times (\Sigma^* \times \Int^{k_l}) \rightarrow 2^{\Sigma^*}$ (if there is no integer input parameter, then $k_1,\cdots,k_l$ are zero), where each integer input parameter (if there is any) is assumed to be affiliated to a unique string input parameter. Note that  in general $f$ can be nondeterministic, namely, on one input, $f$ may output several  strings.


\subsection{Pre-images of CERLs under string operations}

\begin{definition}[Cost-enriched pre-images of CERLs]
Suppose that $f: (\Sigma^* \times \Int^{k_1}) \times \cdots \times (\Sigma^* \times \Int^{k_l}) \rightarrow 2^{\Sigma^*}$ is a string operation, $L \subseteq \Sigma^* \times \Int^{k_0}$ is a CERL defined by a CEFA $\CEFA=(Q, \Sigma, R, \delta, I, F)$ with $R= (r_1, \cdots, r_{k_0})$. Then the $R$-cost-enriched pre-image of $L$ under $f$, denoted by $f^{-1}_R(L)$, is a pair $(\cR, \vec{t})$ such that 
\begin{itemize}
\item $\cR \subseteq (\Sigma^* \times \Int^{k_1 + k_0}) \times \cdots \times (\Sigma^* \times \Int^{k_l + k_0})$,
\item $\vec{t} = (t_1, \cdots ,t_{k_0})$ is a vector of linear integer terms where for each $i \in [k_0]$, $t_i$ is a term whose variables are from $\{r^{(1)}_i, \cdots, r^{(l)}_i\}$ (intuitively, each cost register $r_i$ is split into $l$ cost registers $r^{(1)}_i, \cdots,r^{(l)}_i$, one for each string input parameter, and $t_i$ tells how to compute $r_i$ from $r^{(1)}_i, \cdots,r^{(l)}_i$), and
\item $L$ is equal to the language comprising the $k_0$-cost-enriched strings
%
\[\left(w_0, t_1\left[d^{(1)}_{1}/r^{(1)}_1, \cdots, d^{(l)}_{1}/r^{(l)}_1\right], \cdots, t_{k_0}\left[d^{(1)}_{k_0}/r^{(1)}_{k_0}, \cdots, d^{(l)}_{k_0}/r^{(l)}_{k_0}\right]
\right), \]
%
such that 
\[w_0 \in f\left((w_1, \vec{c_1}), \cdots, (w_l, \vec{c_l}\right)) \mbox{ for some } ((w_1, (\vec{c_1}, \vec{d_1})), \cdots, (w_l, (\vec{c_l}, \vec{d_l}))) \in \cR,\]
where $\vec{c_1} \in \Int^{k_1}$, $\cdots$, $\vec{c_l} \in \Int^{k_l}$, $\vec{d_1} = (d^{(1)}_{1}, \cdots, d^{(1)}_{k_0}) \in \Int^{k_0}$, $\cdots$, and $\vec{d_l} = (d^{(l)}_{1},\cdots, d^{(l)}_{k_0}) \in \Int^{k_0}$.
\end{itemize}
The $R$-cost-enriched pre-image of $L$ under $f$, say $f^{-1}_R(L)=(\cR, \vec{t})$, is said to be CERR-definable if $\cR$ is a CERR. A CEFA representation of a CERR-definable $f^{-1}_R(L)=(\cR, \vec{t})$ is a tuple $((\CEFA_{i,1}, \cdots, \CEFA_{i, l})_{i \in [n]}, \vec{t})$ such that $(\CEFA_{i,1}, \cdots, \CEFA_{i, l})_{i \in [n]}$ is a CEFA representation of $\cR$. 
\end{definition}

\begin{example}\label{exm:pre-image}
Let $L = \{(w, |w|) \mid w \in \Lang((aa)^*b(bb)^*) \}$. Evidently $L$  is a CERL defined by a CEFA $\CEFA = (Q, \{a,b\}, R, \delta, I, F)$ with $R=(r_1)$. Since the concatenation operation $\concat$  is a string function from $\Sigma^* \times \Sigma^*$ to $\Sigma^*$, $\concat^{-1}_R(L)$, the $R$-cost-enriched pre-image of $L$ under concatenation $\concat$, is the pair $(\cR, t)$, where $t=r^{(1)}_1+r^{(2)}_1$ and 
\[\cR = L_{1,1} \times L_{1,2} \cup L_{2,1} \times L_{2,2} \cup L_{3,1} \times L_{3,2} \cup L_{4,1} \times L_{4,2} \cup L_{5,1} \times L_{5,2},\]
such that
\begin{itemize}
\item $L_{1,1} = \{(w_1, |w_1|) \mid w_1 \in \Lang((aa)^*)\}$ and $L_{1,2} = \{(w_2, |w_2|) \mid w_2 \in \Lang(b(bb)^*)\}$,
%
\item $L_{2,1} = \{(w_1, |w_1|) \mid w_1 \in \Lang((aa)^*)\}$ and $L_{2,2} = \{(w_2, |w_2|) \mid w_2 \in \Lang((aa)^*b(bb)^*)\}$,
%
\item $L_{3,1} = \{(w_1, |w_1|) \mid w_1 \in \Lang(a(aa)^*)\}$ and $L_{3,2} = \{(w_2, |w_2|) \mid w_2 \in \Lang(a(aa)^*b(bb)^*)\}$,
%
\item $L_{4,1} = \{(w_1, |w_1|) \mid w_1 \in \Lang((aa)^*b(bb)^*)\}$ and $L_{4,2} = \{(w_2, |w_2|) \mid w_2 \in \Lang((bb)^*)\}$,
%
\item $L_{5,1} = \{(w_1, |w_1|) \mid w_1 \in \Lang((aa)^*(bb)^*)\}$ and $L_{5,2} = \{(w_2, |w_2|) \mid w_2 \in \Lang(b(bb)^*)\}$.
\end{itemize}
It is easy to see that $\cR$ is a CERR. Thus $\concat^{-1}_R(L)$ is CERR-definable.
\end{example}

It turns out that for each string operation $f$ in {\slint}, the cost-enriched pre-images of CERLs under $f$ are CERR-definable.

\begin{proposition}\label{prop-pre-image}
Let $L$ be a CERL defined by a CEFA $\CEFA = (Q, \Sigma, R, \delta, I, F)$. Then for each string operation $f$ ranging over $\concat$, $\replaceall_{e,u}$, $\reverse$, NFTs $\NFT$, and $\substring$, $f^{-1}_R(L)$ is CERR-definable. In addition,
\begin{itemize}
\item a CEFA representation of $\concat^{-1}_R(L)$, (resp. $\reverse^{-1}_R(L)$ and $\substring^{-1}_R(L)$) can be computed in time $\bigO(|\CEFA|)$,
%
\item a CEFA representation of  $(\Tran(\NFT))^{-1}_R(L)$ can be computed in time polynomial in $|\CEFA|$ and exponential in $|\NFT|$,
%
\item a CEFA representation of  $(\replaceall_{e,u})^{-1}_R(L)$ can be computed in time polynomial in $|\CEFA|$ and exponential in $|e|$ and $|u|$.
\end{itemize}
\end{proposition}

\begin{proof}
to be done
\end{proof}

\zhilin{stopped here}

In this section, we show that the pre-images of the integer and string operations in {\slint} are representable by CEFAs. %satisfy the semantic conditions.

Let $\NFA=(Q, \Sigma, R, \delta, I, F)$ be a CEFA with $R= (r_1, \cdots, r_m)$. 

%%%%%%%%%%%%%%%%%%%%%%%%%%%%%%%%%%%%%%%%%%%%%%%%%%%%%%%%%%%%%%%%%%%%%%%%%%%%%%
\smallskip
\noindent \emph{Concatenation $x_1 \concat x_2$}.

\smallskip

Then $((\NFA_{I, q}, \NFA_{q, F})_{q \in Q}, \vec{t})$ is a CEFA representation of the $R$-cost enriched pre-image of $\Lang(\NFA)$ under $\concat$, where $\NFA_{I, q}=(Q, \Sigma, R^{(1)}, \delta^{(1)}, I, \{q\})$ and  $\NFA_{q, F}=(Q, \Sigma, R^{(2)}, \delta^{(2)}, \{q\}, F)$ such that 
\begin{itemize}
\item $R^{(1)} = (r^{(1)}_1, \cdots, r^{(1)}_m)$, $R^{(2)} = (r^{(2)}_1, \cdots, r^{(2)}_m)$, 
\item $\delta^{(1)}$ comprises the tuples $(q, \sigma, q', \eta')$ satisfying that $(q, \sigma, q', \eta) \in \delta$ and for each $j \in [m]$, $\eta'(r^{(1)}_j)=\eta(r_j)$,  similarly for $\delta^{(2)}$,
\end{itemize}
and $\vec{t} = (r^{(1)}_1 + r^{(2)}_1, \cdots, r^{(1)}_m + r^{(2)}_m)$.


%%%%%%%%%%%%%%%%%%%%%%%%%%%%%%%%%%%%%%%%%%%%%%%%%%%%%%%%%%%%%%%%%%%%%%%%%%%%%%
\smallskip 

\noindent \emph{Reverse $\reverse(x_1)$}. 

$(\NFA^{(r)}, (r^{(1)}_1, \cdots, r^{(1)}_m))$ is the CEFA representation of the $R$-cost enriched pre-image of $\Lang(\NFA)$ under $\reverse$, where $\NFA^{(r)}$ is $(Q, \Sigma, R, \delta', F, I)$ such that $\delta'$ comprises the set of tuples $(q', \sigma, q, \eta)$ with  $(q, \sigma, q', \eta) \in \delta$. Note that $\Lang(\NFA^{(r)}) = \{(w^{(r)}, \vec{c}) \mid (w, \vec{c}) \in \Lang(\NFA)\}$.


%%%%%%%%%%%%%%%%%%%%%%%%%%%%%%%%%%%%%%%%%%%%%%%%%%%%%%%%%%%%%%%%%%%%%%%%%%%%%%
\smallskip

\noindent \emph{Substring $\substring(x_1, i, j)$}.

%Intuitively, $\substring(x_1, i, j)$ returns the substring of $x_1$ starting from the position $i$ and ending at the position $j$ (assuming that $i  < j$), with the letter at the position $j$ excluded.

$(\cB, (r^{(1)}_1, \cdots, r^{(1)}_m))$ is the CEFA representation of the $R$-cost enriched pre-image of $\Lang(\NFA)$ under $\substring$, where $\cB = (Q \times \{p_0, p_1, p_2\}, \Sigma, R', \delta', I \times \{p_0\}, F \times \{p_2\})$ such that $R' = (i, j, r^{(1)}_1,\cdots, r^{(1)}_m)$ and $\delta'$ comprises 
\begin{itemize}
\item the tuples $((q, p_0), \sigma, (q, p_0), \eta')$ such that $q \in I$ and $\eta' = \eta_0 \cup \{i \rightarrow 1, j \rightarrow 1\}$, where $\eta_0(r^{(1)}_j)=0$ for each $j \in [m]$,
%
\item the tuples $((q, p_0), \sigma, (q', p_1), \eta')$ such that $(q, \sigma, q', \eta) \in \delta$ and $\eta' = \eta^{(1)} \cup \{i \rightarrow 1, j  \rightarrow 1\}$, where $\eta^{(1)}(r^{(1)}_j)=\eta(r_j)$ for each $j \in [m]$,
%
\item the tuples $((q, p_1), \sigma, (q', p_1), \eta')$ such that $(q, \sigma, q', \eta) \in \delta$ and $\eta' = \eta^{(1)} \cup \{i \rightarrow 0, j  \rightarrow 1\}$, where $\eta^{(1)}(r^{(1)}_j)=\eta(r_j)$ for each $j \in [m]$,
%
\item the tuples $((q, p_1), \sigma, (q, p_2), \eta')$ such that $q \in F$ and $\eta' = \eta_0 \cup \{i \rightarrow 0, j  \rightarrow 1\}$, where $\eta_0(r^{(1)}_j)=0$ for each $j \in [m]$,
%
\item the tuples $((q, p_2), \sigma, (q, p_2), \eta')$ such that $q \in F$ and $\eta' = \eta_0 \cup \{i \rightarrow 0, j  \rightarrow 0\}$, where $\eta_0(r^{(1)}_j)=0$ for each $j \in [m]$.
%
\end{itemize}
%

%%%%%%%%%%%%%%%%%%%%%%%%%%%%%%%%%%%%%%%%%%%%%%%%%%%%%%%%%%%%%%%%%%%%%%%%%%%%%%
\smallskip
\noindent \emph{FT $T(x_1)$}.

\smallskip

Let $T= (Q', \Sigma, \delta', I', F')$. Then $(\cB, (r^{(1)}, \cdots, r^{(1)}_m))$ is the CEFA representation of the $R$-cost enriched pre-image of $\Lang(\NFA)$ under $T$, where $ \cB= (Q \times Q', \Sigma, R^{(1)}, \delta'', I \times I', F \times F')$ such that $R^{(1)}  = (r^{(1)}, \cdots, r^{(1)}_m)$, $\delta''$ comprises the tuples $((q_1, q'_1), \sigma, (q_2, q'_2), \eta')$ satisfying that $(q'_1, \sigma, q'_2, u) \in \delta'$ with $u = \sigma_1 \cdots \sigma_i$, and in $\NFA$, we have $p_1 \xrightarrow{\sigma_1, \eta_1} p_2 \cdots \xrightarrow{\sigma_i, \eta_i} p_{i+1}$ with $p_1 = q_1$ and $p_{i+1}= q_2$, and for each $j \in [m]$,  $\eta'(r^{(1)}_j) = \eta_1(r_j) + \cdots + \eta_i(r_j)$.
%

%%%%%%%%%%%%%%%%%%%%%%%%%%%%%%%%%%%%%%%%%%%%%%%%%%%%%%%%%%%%%%%%%%%%%%%%%%%%%%
\smallskip 

\noindent \emph{ReplaceAll $\replaceall_{e,u}(x)$}.

\smallskip

Intuitively, $\replaceall_{e,u}(x)$ is the string obtained by replacing every occurrence of $e$ in $x$ with the constant string $u$.

From the results in \cite{CCH+18}, we know that  a FT $T_{e,u}$ can be constructed to simulate $\replaceall_{e,u}$. 
Therefore, a CEFA representation of the $R$-cost enriched pre-image of $\Lang(\NFA)$ under $T$ can be constructed as in the FT case.
% 

%%%%%%%%%%%%%%%%%%%%%%%%%%%%%%%%%%%%%%%%%%%%%%%%%%%%%%%%%%%%%%%%%%%%%%%%%%%%%%
\smallskip 

\noindent \emph{Length $\length(x_1)$}.

\smallskip

$(\cB, r^{(1)})$ is a CEFA representation of $\length$, where $\cB = (Q', \Sigma, R^{(1)}, \delta', I', F')$ such that $Q' = \{q'_0\}$, $I'=F'=\{q'_0\}$, $R^{(1)} = (r^{(1)})$, $\delta' = \{(q'_0, \sigma, q'_0, \eta) \mid \sigma \in \Sigma, \eta(r^{(1)}) = 1\}$.

%%%%%%%%%%%%%%%%%%%%%%%%%%%%%%%%%%%%%%%%%%%%%%%%%%%%%%%%%%%%%%%%%%%%%%%%%%%%%%
\smallskip 

\noindent \emph{IndexOf $\indexof_u(x_1, i)$}.

\smallskip 

Suppose $u = \sigma_1 \cdots \sigma_j$. We use the concept of window profiles of positions w.r.t. $u$, which are elements of $\{\bot, \top\}^{j-1}$, to recognise the first occurrence of $u$ after the position $i$. 

For $\pi \in \{\bot, \top\}^{j-1}$ and $\sigma' \in \Sigma$, $upt(\vec{\pi}, \sigma')$ is the updated window profile after reading the letter $\sigma'$, specifically, $upt(\vec{\pi}, \sigma') = \vec{\pi'}$ such that  
\begin{itemize}
\item $\pi'_1 = \top$ iff $\sigma' = \sigma_1$, 
%
\item for each $j' \in [j-2]$, $\pi'_{j'+1} = \top$ iff $\pi_{j'} = \top$ and $\sigma' = \sigma_{j'+1}$. 
\end{itemize}
The set of window profiles of $u$, denoted by $WP_u$, is computed by setting $WP_0 := \{\bot^{j-1}\}$ and iterating the following procedure, until $WP_i = WP_{i+1}$:
\[WP_{i+1}:=WP_i \cup \{upt(\vec{\pi}, \sigma') \mid \vec{\pi} \in WP_i, \sigma' \in \Sigma\}.\] 
From the results in \cite{CCH+18}, we know that $|WP_u| \le |u|$. Therefore, the aforementioned iteration terminates in at most $|u|$ steps.


Then $(\cB, r^{(1)})$ is a CEFA representation of $\indexof_u$, where 
$\cB= (Q', \Sigma, R', \delta', I', F')$ such that  $Q' = \{q'_0, q'_1\} \cup WP_u \cup WP_u \times [i]$, $R'=(i, r^{(1)})$, $I'=\{q'_0\}$, $F'=\{q'_1\}$, and $\delta'$ comprises 
\begin{itemize}
\item the tuples $(q'_0, \sigma, q'_0, \eta)$ such that $\sigma \in \Sigma$, $\eta(i)=1$, and $\eta(r^{(1)}) = 1$,
%
\item the tuples $(q'_0, \sigma, \vec{\pi}, \eta)$ such that $\sigma \in \Sigma$, $\vec{\pi} = \theta \bot^{j-2}$ where $\theta  = \top$ iff $\sigma = \sigma_1$, $\eta(i) = 1$, and $\eta(r^{(1)})= 1$,
% 
\item the tuples  $(\vec{\pi}, \sigma, upt(\vec{\pi}, \sigma), \eta)$ such that $\vec{\pi} \in WP_u$, $\sigma \in \Sigma$, $\pi_{j-1} = \bot$ or $\sigma \neq \sigma_{j}$, $\eta(i) = 0$, and $\eta(r^{(1)})= 1$,
%
\item the tuples $(\vec{\pi}, \sigma, (upt(\vec{\pi}, \sigma), 1), \eta)$ such that $\vec{\pi} \in WP_u$, $\sigma = \sigma_1$, $\pi_{j-1} = \bot$ or $\sigma \neq \sigma_{j}$, $\eta(i) = 0$, and $\eta(r^{(1)})= 1$,
%
\item the tuples $((\vec{\pi}, j'),  \sigma, (upt(\vec{\pi}, \sigma), j'+1), \eta)$ such that $\vec{\pi} \in WP_u$, $j' \in [j-2]$, $\sigma = \sigma_{j'+1}$, $\pi_{j-1} = \bot$ or $\sigma \neq \sigma_{j}$, $\eta(i) = 0$, and $\eta(r^{(1)})= 0$,
%
\item the tuples $((\vec{\pi}, j-1),  \sigma, q'_1, \eta)$ such that $\vec{\pi} \in WP_u$, $\sigma = \sigma_{j}$, $\eta(i) =0$, and $\eta(r^{(1)})= 0$,
%
\item the tuples  $(q'_1, \sigma, q'_1, \eta)$ such that $\sigma \in \Sigma$, $\eta(i) = 0$, and $\eta(r^{(1)})= 0$.
\end{itemize}



\subsection{Decision procedures}

Let $S':=S$ and $A':=A$. Moreover, let $A'':= \ltrue$. Then execute the following procedure to (partially) flatten the integer terms.
\begin{description}
\item[Step 1.] Recursively apply the following transformation until $S' \wedge A'$ contains no more occurrences of integer functions: Select an occurrence of integer functions, say $g(x_1, \vec{t_1}, \cdots, x_k, \vec{t_k})$, such that 
%it is a \emph{proper} subterm of the other integer term and 
{\it none} of $\vec{t_1}, \cdots, \vec{t_k}$ contains occurrences of integer functions, introduce a fresh integer variable $i$, let $S' \wedge A'$ be the formula obtained by replacing $g(x_1, \vec{t_1}, \cdots, x_k, \vec{t_k})$ with $i$, moreover, let $A'':= A'' \wedge i = g(x_1, \vec{t_1}, \cdots, x_k, \vec{t_k})$.
%
\item[Step 2.] It comprises the following two substeps. 
\begin{enumerate}
\item For each occurrence of string functions in $S'$, say $f(x_1, \vec{t_1}, \cdots, x_k, \vec{t_k})$, suppose $\vec{t_j} = (t_{j,1}, \cdots, t_{j, l_j})$ for each $j \in [k]$, introduce fresh integer variables $i_{j, j'}$ for $j \in [k]$ and $j' \in [l_j]$, replace $f(x_1, \vec{t_1}, \cdots, x_k, \vec{t_k})$ with $f(x_1, \vec{i_1}, \cdots, x_k, \vec{i_k})$ in $S'$, where $\vec{i_j} = (i_{j,1}, \cdots, i_{j, l_j})$ for each $j \in [k]$, and let $A':=A' \wedge \bigwedge \limits_{j \in [k], j' \in [l_j]} i_{j, j'} = t_{j, j'}$. 
\item For each occurrence of integer functions in $A''$, say $g(x_1, \vec{t_1}, \cdots, x_k, \vec{t_k})$, suppose $\vec{t_j} = (t_{j,1}, \cdots, t_{j, l_j})$ for each $j \in [k]$, introduce fresh integer variables $i_{j, j'}$ for $j \in [k]$ and $j' \in [l_j]$, replace $g(x_1, \vec{t_1}, \cdots, x_k, \vec{t_k})$ with $g(x_1, \vec{i_1}, \cdots, x_k, \vec{i_k})$ in $A''$, where $\vec{i_j} = (i_{j,1}, \cdots, i_{j, l_j})$ for each $j \in [k]$, and let $A':=A' \wedge \bigwedge \limits_{j \in [k], j' \in [l_j]} i_{j, j'} = t_{j, j'}$. 
\end{enumerate}
%
\item[Step 3.] Let $S:=S'$ and $A:=A'' \wedge A' $.
\end{description}
The aforementioned flattening procedure is a bit technical, for simplicity, we may assume that the integer terms are fully flattened, including the arithmetic operations.

Note that after the aforementioned flattening procedure, the resulting formula $S \wedge A$ satisfies the following property: 
\begin{quote}
The integer terms in all the occurrences of string and integer functions  are integer variables, moreover, each integer variable occurs at most once in these string and integer functions.  \hfill ($*$)
\end{quote}
Therefore, in the sequel, we assume that $S \wedge A$ satisfies the property ($*$).

\begin{theorem}\label{thm-sl-int-dec}
Path feasibility of {\slint} satisfying the semantic conditions is decidable.
\end{theorem}

\begin{proof}
In the following, we extend the generic decision procedure in \cite{CHL+19}, where NFA is replaced by CEFA.

Let $S \wedge A$ be an {\slint} formula (satisfying the property ($*$)).

For each occurrence of $i = g(x_1, \vec{i'_1}, \cdots, x_k, \vec{i'_k})$ in $A$ with $g$ an integer function, apply the following nondeterministic transformation to $A$: 
\begin{quote}
According to the 1st semantic condition, $g$ is a CERR linear integer function and a CEFA representation of $g$, say $((\NFA_{j,1}, \cdots, \NFA_{j, k})_{j \in [m]}, t)$, can be computed effectively from $g$. Consider $((\NFA'_{j,1}, \cdots, \NFA'_{j, k})_{j \in [m]}, t')$, where $\NFA'_{j,1}=\NFA_{j,1}[\vec{i'_1}/R(\NFA_{j,1})]$, $\cdots$, $\NFA'_{j,k}=\NFA_{j,k}[\vec{i'_k}/R(\NFA_{j,k})]$, and $t' = t[i^{(1)}/r^{(1)}, \cdots, i^{(k)}/r^{(k)}]$.
Nondeterministically choose $j \in [m]$, and replace $i = g(x_1, \vec{i'_1}, \cdots, x_k, \vec{i'_k})$ by $x_1 \in \NFA'_{j,1} \wedge \cdots \wedge x_k \in \NFA'_{j,k} \wedge i = t'$ in $A$.
\end{quote}
Note that after this transformation, $S \wedge A$ contains no occurrences of integer functions, moreover, as a result of the property ($*$), for every variable $x$, all the CEFAs to which $x$ belongs satisfy that their sets of registers are  mutually disjoint.

Then repeat the following procedure until $S$ becomes empty.
%
\begin{quote}
Suppose $y := f(x_1, \vec{i_1}, \cdots, x_k, \vec{i_k})$ is the last assignment of $S$. 
\\
Let $\rho := \{\NFA_1, \cdots, \NFA_s\}$ be the set of all CEFAs such that $y \in \NFA_j$ occurs in $A$ for each $j \in [s]$. Construct $\NFA = \NFA_1 \times \cdots \times \NFA_s$ (Recall that the sets of registers of $\NFA_1$, $\cdots$, $\NFA_s$ are mutually disjoint). Let  the vector of registers in $\NFA$ be $R = (r'_1, \cdots, r'_n)$. Then according to the 2nd semantic condition, 
a CEFA representation of the $R$-cost enriched pre-image of $\Lang(\NFA)$ under $f$, say $((\cB_{j, 1}, \cdots, \cB_{j, k})_{j \in [\ell]}, \vec{t})$, can be effectively computed from $\NFA$ and $f$. Consider $((\cB'_{j, 1}, \cdots, \cB'_{j, k})_{j \in [\ell]}, \vec{t'})$, where $\cB'_{j, 1} = \cB_{j, 1}[\vec{i_1}/R(\cB_{j,1}), \vec{(r')^{(1)}}/\vec{r^{(1)}}]$, $\cdots$, $\cB'_{j,k}=\cB_{j,k}[\vec{i_k}/R(\cB_{j,k}), \vec{(r')^{(k)}}/\vec{r^{(k)}}]$ (with $\vec{r^{(1)}}= (r^{1}_1, \cdots, r^{(1)}_n)$, similarly for $\vec{r^{(2)}}$ and so on), and $\vec{t'} = \vec{t}[\vec{r'_1}/\vec{r_1}, \cdots, \vec{r'_n}/\vec{r_n}]$ (with $\vec{r_1} = (r^{(1)}_1, \cdots, r^{(k)}_1)$, similarly for $\vec{r^{(2)}}$ and so on). 
\\
Nondeterministically choose $j \in [\ell]$ and let 
$$A:= A \wedge x_1 \in \cB'_{j, 1} \wedge \cdots \wedge x_k \in \cB'_{j, k}  \wedge \bigwedge \limits_{j' \in [n]} r'_{j'} = t'_{j'}.$$
%
Remove $y := f(x_1, \vec{i_1}, \cdots, x_k, \vec{i_k})$ from $S$.
\end{quote}

We would like to remark that if all the string functions $f$ in $S \wedge A$ are \emph{deterministic}, then the product of CEFAs before the pre-image computation can be avoided and the pre-image can be computed \emph{distributively} for CEFAs in $\rho$.

In the end, we get a formula $S \wedge A$ where $S$ is empty. Suppose $A = A_r \wedge A_i$, where $A_r$ is a conjunction of atomic formulae of the form $x \in \NFA$, and $A_i$ is linear arithmetic formula (containing no integer functions). By computing the product construction of CEFAs, $A_r$ can be rewritten as $x_1 \in \NFA_1 \wedge \cdots \wedge x_n \in \NFA_n$, where $x_1,\cdots, x_n$ are mutually distinct. Therefore, the path feasibility of $S \wedge A$ is exactly the satisfiability of $A_i$ w.r.t. the CEFAs $\NFA_1, \cdots, \NFA_n$. From Theorem~\ref{thm-incra-la-sat}, we conclude that the path feasibility of  {\slint} is decidable.
\qed
\end{proof}

\begin{corollary}
Path feasibility of {\cslint} is decidable.
\end{corollary}



%%%%%%%%%%%%%%%%%%%%%%%%%%%%%%%%%%%%%%%%%%%%%%%%
%%%%%%%%%%%%%%%%CERR linear integer functions removed%%%%%%%%%%%%%
%%%%%%%%%%%%%%%%CERR linear integer functions removed%%%%%%%%%%%%%
%%%%%%%%%%%%%%%%%%%%%%%%%%%%%%%%%%%%%%%%%%%%%%%%

\hide{
\begin{definition}[CERR linear integer functions]
An integer function $g: \Sigma^* \times \Int^{k_1} \times \Sigma^* \times \Int^{k_l} \rightarrow 2^\Int$ is  \emph{linear} if there is a pair $(\cR, t)$ such that $\cR \subseteq \Sigma^* \times \Int^{k_1+1} \times \Sigma^* \times \Int^{k_l+1}$ is a CERR and $t$ a linear integer term over $r^{(1)}, \cdots, r^{(l)}$ such that for all $\vec{c_1} \in \Int^{k_1}, \cdots, \vec{c_l} \in \Int^{k_l}$, and $d_1, \cdots, d_l \in \Int$, it holds that $(w_1, (\vec{c_1}, d_1), \cdots, w_l, (\vec{c_l}, d_l)) \in \cR$ iff $t[d_1/r^{(1)}, \cdots, d_l/r^{(l)}] \in g(w_1, \vec{c_1}, \cdots, w_l, \vec{c_l})$.  

For a CERR linear integer function $g$ witnessed by the pair $(\cR, t)$, a CEFA representation of $g$ is a tuple $((\NFA_{i,1}, \cdots, \NFA_{i, l})_{i \in [n]}, t)$, where $(\NFA_{i,1}, \cdots, \NFA_{i, l})_{i \in [n]}$ is a CEFA representation of $\cR$.

\end{definition}

\begin{example}
The string functions $\length$ and $\indexof_u$ are CERR linear integer functions, whose CEFA representations can be found in Section~\ref{sec-cslint}.
\end{example}
}