%!TEX root = main.tex

We write $\nat$ for the set of natural numbers. For $n \in \nat$ with $n \ge 1$, we use $[n]$ to denote $\{1, \cdots, n\}$. Throughout the paper we use $\Sigma$ to denote a  finite alphabet. 

A string over $\Sigma$ is a (possibly empty) sequence of elements from $\Sigma$. An empty string is usually denoted by $\varepsilon$. We write $\Sigma^*$ (resp. $\Sigma^+$) for the set of all (resp. nonempty) strings over $\Sigma$. Let $w=a_1\cdots a_n$ be a string. The reverse of $w$, denoted by $w^{(r)}$, is $a_n \cdots a_1$. 

We consider two data types, i.e., the string data type and the integer data type. We will use $c, d,\dots$ to denote integer constants, $u, v, \dots$ to denote string constants,  $i, j, \dots$ to denote the  integer variables, and $x, y, \dots$ to denote the string variables.

\noindent\textbf{Automata models.} A finite automaton (FA) $\Aut$ is a tuple $(Q, \Sigma, \delta, I, F)$, where $Q$ is a finite set of states, $\Sigma$ is a finite alphabet, $\delta \subseteq Q \times \Sigma \times Q$ is the transition relation, $I,F \subseteq Q$ are the set of initial and final states respectively. 

A string $w=\sigma_1 \cdots \sigma_n$ is accepted by $\Aut$ if there is a state sequence $q_0 \cdots q_n$ such that $q_0 \in I$, $q_n \in F$, and $(q_{i-1}, \sigma_i, q_i) \in \delta$ for each $i \in [n]$. In particular, an empty string $\varepsilon$ is accepted by $\Aut$ if $I \cap F \neq \emptyset$. The language defined by $\Aut$, denoted by $\Lang(\Aut)$, is defined as the set of strings accepted by $\Aut$.

 
%Fix a finite \emph{alphabet} $\Sigma$. Elements in $\Sigma^*$ are called \emph{strings}. Let $\varepsilon$ denote the empty string and  $\Sigma^+ = \Sigma^* \setminus \{\varepsilon\}$. We will use $a,b,\ldots$ to denote letters from $\Sigma$ and $u, v, w, \ldots$ to denote strings from $\Sigma^*$. For a string $u \in \Sigma^*$, let $|u|$ denote the \emph{length} of $u$ (in particular, $|\varepsilon|=0$), moreover, 
For $a \in \Sigma$, let $|u|_a$ denote the number of occurrences of $a$ in $u$. A \emph{position} of a nonempty string $u$ of length $n$ is a number $i \in [n]$ (Note that the first position is $1$, instead of  0). In addition, for $i \in [|u|]$, let $u[i]$ denote the $i$-th letter of $u$. For a string $u \in \Sigma^*$, we use $u^R$ to denote the reverse of $u$, that is, if $u = a_1 \cdots a_n$, then $u^R= a_n \cdots a_1$.
For two strings $u_1, u_2$, we use $u_1 \cdot u_2$ to denote the \emph{concatenation} of $u_1$ and $u_2$, that is, the string $v$ such that $|v|= |u_1| + |u_2|$ and for each $i \in [|u_1|]$, $v[i]= u_1[i]$, and for each $i \in |u_2|$, $v[|u_1|+i]=u_2[i]$. Let $u, v$ be two strings. If $v = u \cdot v'$ for some string $v'$, then $u$ is said to be a \emph{prefix} of $v$. In addition, if $u \neq v$, then $u$ is said to be a \emph{strict} prefix of $v$. If $u$ is a prefix of $v$, that is, $v = u \cdot v'$ for some string $v'$, then 
we use $u^{-1} v$ to denote $v'$. In particular, $\varepsilon^{-1} v = v$.

A \emph{language} over $\Sigma$ is a subset of $\Sigma^*$. We will use $L_1, L_2, \dots$ to denote languages. For two languages $L_1, L_2$, we use $L_1 \cup L_2$ to denote the union of $L_1$ and $L_2$, and $L_1 \cdot L_2$ to denote the concatenation of $L_1$ and $L_2$, that is, the language $\{u_1 \cdot u_2 \mid u_1 \in L_1, u_2 \in L_2\}$. For a language $L$ and $n \in \Nat$, we define $L^n$, the \emph{iteration} of $L$ for $n$ times, inductively as follows: $L^0=\{\varepsilon\}$ and $L^{n} =L \cdot L^{n-1}$ for $n > 0$. We also use $L^*$ to denote an arbitrary number of iterations of $L$, that is, $L^* = \bigcup \limits_{n \in \Nat} L^n$. Moreover, let $L^+ = \bigcup \limits_{n \in \Nat \setminus \{0\}} L^n$.

A (nondeterministic) finite transducer (FT) $T$ is a tuple $(Q, \Sigma, \delta, I, F)$, where $Q$ is a finite set of states, $\Sigma$ is a finite alphabet, $\delta$ is the transition relation, which is a finite subset of $Q \times \Sigma \times Q \times \Sigma^*$, $I,F \subseteq Q$ are the set of initial and final states respectively. For readability, we write a transition $(q, \sigma, q', u)$ as $q \xrightarrow{\sigma, u} q'$. 

A run of $T$ over a string $w=\sigma_1 \cdots \sigma_n$ is a state sequence of transitions $q_0 \xrightarrow{\sigma_1, u_1} q_1 \cdots q_n \xrightarrow{\sigma_n, u_n} q_n$. The run is accepting if $q_0 \in I$ and $q_n \in F$. The string $u_1 \cdots u_n$ is called the output of the run. We use $\cT(T)$ to denote the set of pairs $(w, u)$ such that there is an accepting run of $T$ on $w$, with the output $u$.

We remark that an FT usually defines a relation.

\noindent\textbf{Linear arithmetic.}  A linear arithmetic formula $\phi$ is defined by the rules: $\phi::= t \ o \ t \mid \neg \phi \mid \phi \vee \phi \mid \exists i.\ \phi$, where $o \in \{=, \neq, \le, \ge, <, >\}$ and $t$ is defined by  $t::= i \mid c \mid ct \mid t + t $.  
For a quantifier-free linear integer arithmetic formula $\phi$ that contains the free variables $i_1, \cdots, i_k$, we use $\cM(\phi)$ to denote the set of models of $\phi$, namely, $\left\{(c_1, \cdots ,c_k) \mid \phi[c_1/i_1, \cdots, c_k/i_k]\right\}$. An existential linear arithmetic formula is a linear arithmetic formula where all the existential quantifiers are under the scope of even number of negation symbols.
