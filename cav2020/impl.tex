%!TEX root = main.tex

%%%%%%% the pseudo-code
%%%%%%% the pseudo-code
\begin{algorithm}[tbp]
	\SetKw{Continue}{continue}
	\small
	\KwIn{$active$: set of CEFA constraints,  $arith$: arithmetic constraints,
		%    $x \in L$,
		$\mathit{funApps}$: acyclic set of assignment statements. }
	\KwResult{$\mathit{sat}$ if the input constraints are satisfiable, and $\mathit{unsat}$ otherwise.\newline
	}
	%  \Begin{
	\For{each partition $(\mathcal{I}_l)_{l \in [5]}$ of the set of $\indexof_v(x, i)$ in $\mathit{arith}$ and \newline
		\hspace*{4mm} each partition $(\mathcal{J}_l)_{l \in [3]}$ of the set of $\substring(x, i, j)$ in $\mathit{funApps}$ 
		\tcc{the partitions refer to (1)-(5) for $\indexof_v(x, i)$ and (1)-(3) for $\substring(x, i, j)$ in Step II of Section~\ref{sec:dc}}
	}
	{
		\tcc{Case splits for semantics of $\indexof$ and $\substring$}
		$(\mathit{active}, \mathit{arith}, \mathit{funApps}) = \mathit{indexofCaseSplit}(\mathit{active}, \mathit{arith}, \mathit{funApps}, (\mathcal{I}_l)_{l \in [5]})$\; 
		$(\mathit{active}, \mathit{arith}, \mathit{funApps})= \mathit{substringCaseSplit}(\mathit{active}, \mathit{arith}, \mathit{funApps}, (\mathcal{J}_l)_{l \in [3]})$\; 
		\For{each $\length(x)$ occurring in $\mathit{arith}$}
		{
			choose a fresh integer variable $i$\;
			$\mathit{active} \leftarrow \mathit{active} \cup \{x \in \CEFA_{\rm len}[i/r_1]\}$; $\mathit{arith}\leftarrow \mathit{arith}[i/\length(x)]$;
		}
		\For{each $\indexof_v(x,i)$ occurring in $\mathit{arith}$}
		{
			choose fresh integer variables $i_1,i_2$\;
			$\mathit{active} \leftarrow \mathit{active} \cup \{x \in \CEFA_{\indexof_v}[i_1/r_1,i_2/r_2]\}$; $\mathit{arith}\leftarrow \mathit{arith}[i_2/\indexof_v(x,i)] \wedge i=i_1$;
		}
		\If{$\mathit{BackDfsExp}(\mathit{active}, \emptyset, \mathit{arith}, \mathit{funApps})$}
		{
			\Return{$sat$};}
		%		{\Continue;}
	}
	\Return $\mathit{unsat}$; 		
	%}
	\caption{Function $\mathit{checkSat}$
		for Step II-III} \label{alg:checksat} 
\end{algorithm}

We have implemented the decision procedure based on the recent string constraint solver OSTRICH \cite{CHL+19}, resulting in a new solver OSTRICH+. OSTRICH is  written in Scala and based on the SMT solver Princess \cite{princess08}. OSTRICH+ reuses the parser of Princess, but replaces the NFAs from OSTRICH with CEFAs. Correspondingly, in OSTRICH+, the pre-image  computation for concatenation, $\replaceall$, $\reverse$, and finite transducers is reimplemented, and a new pre-image operator for $\substring$ is added. OSTRICH+ also implements CEFA constructions for $\length$ and $\indexof$.  

%Similarly to OSTRICH, 
OSTRICH+ performs a depth-first exploration of the search tree resulting from repeatedly
splitting the disjunctions (or unions) in the cost-enriched recognisable pre-images of CERLs under string functions, as well as the case splits in the semantics of $\indexof$ and $\substring$.
The pseudo-code of Step II-III of the decision procedure in Section~\ref{sec:dec} is given by  the function $\mathit{checkSat}$ in Algorithm~\ref{alg:checksat}, which calls two functions $\mathit{indexofCaseSplit}$ and $\mathit{substringCaseSplit}$ for the case splits in the semantics of $\indexof_v$ and $\substring$ respectively.
%\philipp{I believe we need to comment on the partitions $(\mathcal{I}_l)_{l \in [5]}$ here; at least I don't really understand this notation, and the meaning of the partitions.} 
%\tl{I add a comment; not sure whether it helps though}\zhilin{Fine with me.}
%
(The details of $\mathit{indexofCaseSplit}$ and $\mathit{substringCaseSplit}$ can be found in Appendix, Section~\ref{app:case-split-semantics}.) Moreover,  $\mathit{checkSat}$ calls a recursive function  $\mathit{BackDfsExp}$ in Algorithm~\ref{alg:dfs} for the depth-first exploration (Step IV of the decision procedure), which in turn calls a function $\mathit{CheckCefaLIASat}$ to solve the {\lasat} problem (Step V). Note that Step I of the decision procedure is handled by the DPLL(T) procedure in Princess and is omitted here. 


\begin{algorithm}%[tbp]
\SetKw{Continue}{continue}
  \small
  \KwIn{$\mathit{active}, \mathit{passive}$: sets of CEFA constraints,  $\mathit{arith}$: arithmetic constraints,
%    $x \in L$,
    $\mathit{funApps}$: acyclic set of assignment statements. }
  \KwResult{$\mathit{sat}$ if the input constraints are satisfiable, and $\mathit{unsat}$ otherwise.\newline
   }
%
%  \Begin{
    \eIf{$\mathit{active} = \emptyset$}{
%      \tcc{use symbolic model checker {NuXmv} to check whether CEFA constraints are consistent with arithemtic Constraints}
      \tcc{Check whether the LIA constraint $\mathit{arith}$ is satisfiable with respect to the CEFA constraints in $\mathit{passive}$ (i.e. Step V).}
      \Return{$\mathit{CheckCefaLIASat}(\mathit{passive}, \mathit{arith})$;} 
    }{
   	choose a CEFA constraint $x \in \CEFA$ in $active$ with $R(\CEFA)=(r_1,\cdots,r_k)$\;
	\eIf{there is an assignment~$x := f(y_1, \vec{i_1}, \ldots, y_l,\vec{i_l})$ defining $x$ in $\mathit{funApps}$ with \newline
	\hspace*{4mm} $\vec{i_j}=(i_{j,1},\cdots, i_{j, k_j})$ for $j \in [l]$
	}
	{
%	      	\tcc{Compute the $R(\CEFA)$-cost enriched pre-image of $\Lang(\CEFA)$ under $f$}
		compute $f^{-1}_{R(\CEFA)}(\Lang(\CEFA)) = \left((\CEFA^{(1)}_{j}, \cdots, \CEFA^{(l)}_{j})_{j \in [n]}, \vec{t}\right)$ where \newline 
		 $R\left(\CEFA^{(j')}_{j}\right)=\left((r')^{(j', 1)}, \cdots,(r')^{(j', k_{j'})}, r^{(j')}_1,\cdots, r^{(j')}_k \right)$ for $j \in [n]$ and $j' \in [l]$\;
	        $\mathit{active} \leftarrow \mathit{active} \setminus \{x \in \CEFA\}$; $\mathit{passive} \leftarrow \mathit{passive} \cup \{x \in \CEFA\}$\;    
	        \For{$j \leftarrow 1$ \KwTo $n$}{
        		$\mathit{active} \leftarrow \mathit{active} \cup \{y_1 \in \CEFA^{(1)}_{j}, \ldots, y_l \in \CEFA^{(l)}_{j}\} $\;
        		$\mathit{arith} \leftarrow \mathit{arith} \wedge \bigwedge_{j' \in [l], j'' \in [k_{j'}]} i_{j',j''} = (r')^{(j', j'')} \wedge \bigwedge_{j' \in [k]} r_{j'} = t_{j'}$\;
        		\eIf{$\mathit{active} \cup \mathit{passive}$ is inconsistent}{\Continue \tcc*{backtrack}}
		{
		          \Switch{$\mathit{BackDfsExp}(\mathit{active}, \mathit{passive}, \mathit{arith}, \mathit{funApps})$}{
				\lCase{$sat$}
					{\Return{$\mathit{sat}$}}
				\Case{$\mathit{unsat}$}
					{\Continue \tcc*{backtrack}}
          			}
        		}
	}
        	\Return{$\mathit{unsat}$}; 
      	}
      	{
        	\Return{$\mathit{BackDfsExp}(\mathit{active} \backslash \{x\in \CEFA\}, \mathit{passive} \cup \{x\in \CEFA\}, \mathit{arith}, \mathit{funApps})$;}
      	}
	} 
%}
  \caption{Function~$\mathit{BackDfsExp}$ for Step IV (depth-first exploration)}\label{alg:dfs}
\end{algorithm}
%\vspace{-5cm}

%%%%%%% the pseudo-code
%%%%%%% the pseudo-code

%\vspace{-mm}

\paragraph*{Optimisations for solving the {\lasat} problem.} From Proposition~\ref{prop:la-sat-cefa-inter}, a natural approach to solve the {\lasat} problem is to compute an existential LIA formula defining the Parikh image of products of CEFAs, and then use off-the-shelf SMT solvers (e.g. CVC4 or Z3) to decide the satisfiability of the existential LIA formula. %Nevertheless, 
However, our preliminary experiments show that this approach suffers from a scalability issue, in particular, %faces 
the state-space explosion when computing products of CEFAs.  %We did implement this approach in the beginning. However, some preliminary experiments showed that this approach is not scalable. Therefore, 
In the implementation of the function $\mathit{CheckCefaLIASat}$ in Algorithm~\ref{alg:dfs},  we %finally choose 
opt to utilise the symbolic model checker nuXmv \cite{nuxmv} to mitigate the state-space explosion during the computation of products of CEFAs. The nuXmv tool is a well-known symbolic model checker that is capable of analysing both finite and infinite state systems. Our technique is to encode {\lasat} as an instance of the model checking problem, which can  be solved by %and then call 
nuXmv. Since  {\lasat} is a problem for quantifier-free LIA formulas and CEFAs that contain integer variables, the {\lasat} problem actually corresponds to the problem of model checking \emph{infinite state systems}. 
%
%In the following, we use a simple example to illustrate how to encode the instances of  the {\lasat} problem into the inputs of nuXmv.
%
%\begin{example}
%Encoding of {\lasat} into nuXmv.
%\end{example}


%%%%% Case splits of semantics %%%%
%%%%% Case splits of semantics %%%%

%reduce the number of registers
%\begin{itemize}
%\item $\substring(x, 0, i)$, $\indexof_v(x, 0)$, remove the input register,
%
%\item CEFAs without registers: product + minimization 
%
%CEFAs with one register updated with $+1$: product + minimization
%
%Other CEFAs: no optimization
%\end{itemize}


%The main bottleneck of the decision procedure is 
%
%$prefixOf(x, u), suffixOf(x,u), contains(x, u)$: transformed into regular constraints
%
%using nuxmv to avoid state explosion of the product operation.
%
%introduction to nuxmv
%
%introduction to the encoding into nuxmv instances
%
%\begin{example}
%An example for the nuxmv encoding.
%\end{example}
%
%start two threads, one guessing sat, another one guessing unsat, run concurrently
%
%three strategies: 
%
%product + parikh image
%
%product + nuxmv
%
%nuxmv


