%% For double-blind review submission
%\documentclass[acmsmall,10pt,review,anonymous]{acmart}\settopmatter{printfolios=true}
%% For single-blind review submission
%\documentclass[acmsmall,10pt,review]{acmart}\settopmatter{printfolios=true}
%% For final camera-ready submission
%\documentclass[acmsmall,10pt]{acmart}\settopmatter{}
\documentclass[acmsmall,screen]{acmart}
%% Note: Authors migrating a paper from PACMPL format to traditional
%% SIGPLAN proceedings format should change 'acmsmall' to
%% 'sigplan'.


%% Some recommended packages.
\usepackage{booktabs}   %% For formal tables:
                        %% http://ctan.org/pkg/booktabs
\usepackage{subcaption} %% For complex figures with subfigures/subcaptions
                        %% http://ctan.org/pkg/subcaption
\usepackage{latexsym}
\usepackage{setspace}
\usepackage{cancel}
\usepackage{listings}
\usepackage{graphicx}
\usepackage{appendix}
\usepackage{amssymb}
\usepackage{stmaryrd}
\usepackage{amsmath}
\usepackage{leftidx}
\usepackage{mathtools}
\usepackage{paralist}
\usepackage{color}
\usepackage{mathrsfs}
\usepackage{tikz}
\usepackage[draft]{minted}
\usetikzlibrary{shapes}
\usepackage[linesnumbered,ruled]{algorithm2e}

%==========================================================
%!TEX root = main.tex

\newcommand{\brac}[1]{\left( #1 \right)}
\newcommand{\tup}[1]{\left( #1 \right)}
\newcommand{\set}[1]{\{ #1 \}}
\newcommand{\sequence}[2]{(#1, \ldots, #2)}
\newcommand{\couple}[2]{(#1,#2)}
\newcommand{\pair}[2]{(#1,#2)}
\newcommand{\triple}[3]{(#1,#2,#3)}
\newcommand{\quadruple}[4]{(#1,#2,#3,#4)}
\newcommand{\tuple}[2]{(#1,\ldots,#2)}
\newcommand{\Nat}{\ensuremath{\mathbb{N}}}
\newcommand{\Rat}{\ensuremath{\mathbb{Q}}}
\newcommand{\Rea}{\ensuremath{\mathbb{R}}}
\newcommand{\Zed}{\ensuremath{\mathbb{Z}}}
%\newcommand{\true}{\top}
%\newcommand{\false}{\perp}
\newcommand{\bottom}{\perp}
%% \newcommand{\powerset}[1]{{\cal P}(#1)}
\newcommand{\npowerset}[2]{{\cal P}^{#1}(#2)}
\newcommand{\finitepowerset}[1]{{\cal P}_f(#1)}
\newcommand{\level}[2]{L_{#1}(#2)}
\newcommand{\card}[1]{\mbox{card}(#1)}
\newcommand{\range}[1]{\mathtt{ran}(#1)}
\newcommand{\astring}{s}

\newcommand{\Cc}{\mathcal{C}}


\newcommand {\notof}{\ensuremath{\neg}}
\newcommand {\myand}{\ensuremath{\wedge}}
\newcommand {\myor}{\ensuremath{\vee}}
\newcommand {\mynext}{\mbox{{\sf X}}}
\newcommand {\until}{\mbox{{\sf U}}}
\newcommand {\sometimes}{\mbox{{\sf F}}}
\newcommand {\previous}{\mynext^{-1}}
\newcommand {\since}{\mbox{{\sf S}}}
\newcommand {\fminusone}{\mbox{{\sf F}}^{-1}}
\newcommand {\everywhere}[1]{\mbox{{\sf Everywhere}}(#1)}



\newcommand{\aatomic}{{\rm A}}
\newcommand{\aset}{X}
\newcommand{\asetbis}{Y}
\newcommand{\asetter}{Z}

\newcommand{\avarprop}{p}
\newcommand{\avarpropbis}{q}
\newcommand{\avarpropter}{r}
\newcommand{\varprop}{{\rm PROP}} % Set of atomic propositions (for a given logic)

% formulae

\newcommand{\aformula}{\astateformula} % a formula
\newcommand{\aformulabis}{\astateformulabis} % another formula (when at least 2 are present)
\newcommand{\aformulater}{\astateformulater} % another formula (when at least 3 are present)
\newcommand{\asetformulae}{X}
\newcommand{\subf}[1]{sub(#1)}

\newcommand{\aautomaton}{{\mathbb A}}
\newcommand{\aautomatonbis}{{\mathbb B}}

\newcommand {\length}[1] {\ensuremath{|#1|}}



% Equivalences
\newcommand{\egdef}{\stackrel{\mbox{\begin{tiny}def\end{tiny}}}{=}} % =def=
\newcommand{\eqdef}{\stackrel{\mbox{\begin{tiny}def\end{tiny}}}{=}} % =def=
\newcommand{\equivdef}{\stackrel{\mbox{\begin{tiny}def\end{tiny}}}{\equivaut}} % <=def=>
\newcommand{\equivaut}{\;\Leftrightarrow\;}

\newcommand{\ainfword}{\sigma}

\newcommand{\amap}{\mathfrak{f}}
\newcommand{\amapbis}{\mathfrak{g}}

\newcommand{\step}[1]{\xrightarrow{\!\!#1\!\!}}
\newcommand{\backstep}[1]{\xleftarrow{\!\!#1\!\!}}

\newcommand {\aedge}[1] {\ensuremath{\stackrel{#1}{\longrightarrow}}}
\newcommand {\aedgeprime}[1] {\ensuremath{\stackrel{#1}{\longrightarrow'}}}
\newcommand {\afrac}[1] {\ensuremath{\mathit{frac}(#1)}}
\newcommand {\cl}[1] {\ensuremath{\mathit{cl}(#1)}}
\newcommand {\sfc}[1] {\ensuremath{\mathit{sfc}(#1)}}
\newcommand {\dunion} {\ensuremath{\uplus}}
\newcommand {\edge} {\ensuremath{\longrightarrow}}
\newcommand {\emptyword}{\ensuremath{\epsilon}}
\newcommand {\floor}[1] {\ensuremath{\lfloor #1 \rfloor}}
\newcommand {\intersection} {\ensuremath{\cap}}
\newcommand {\union} {\ensuremath{\cup}}
\newcommand {\vals}[2] {\ensuremath{\mathit{val}_{#2}(#1)}}



\newcommand {\pspace} {\textsc{pspace}}
\newcommand {\nlogspace} {\textsc{nlogspace}}
\newcommand {\logspace} {\textsc{logspace}}
\newcommand {\expspace} {\textsc{expspace}}
\newcommand {\np} {\textsc{np}}
\newcommand {\threeexptime} {\textsc{3exptime}}
\newcommand {\polytime} {\textsc{p}}
\newcommand{\twoexpspace}{\textsc{2expspace}}
\newcommand{\threeexpspace}{\textsc{3expspace}}
\newcommand {\nexptime} {\textsc{nexptime}}



\newcommand{\aalphabet}{\Sigma}     % an alphabet, A is already used for atoms
\newcommand{\aword}{\mathfrak{u}}
\newcommand{\awordbis}{\mathfrak{v}}



\newcommand{\aassertion}{P}
\newcommand{\aassertionbis}{Q}
\newcommand{\aexpression}{e}
\newcommand{\aexpressionbis}{f}
\newcommand{\avariable}{\mathtt{x}}
\newcommand{\uniquevar}{\mathtt{u}}
\newcommand{\uniquevarbis}{\mathtt{v}}
\newcommand{\avariablebis}{\mathtt{y}}
\newcommand{\avariableter}{\mathtt{z}}
\newcommand{\nullconstant}{\mathtt{null}}
\newcommand{\nilvalue}{nil}
\newcommand{\emptyconstant}{\mathtt{emp}}
\newcommand{\infheap}{\mathtt{inf}}
\newcommand{\saturated}{\mathtt{Saturated}}

\newcommand{\astateformula}{\phi}
\newcommand{\astateformulabis}{\psi}
\newcommand{\astateformulater}{\varphi}
%%
\newcommand{\separate}{\ast}
\newcommand{\sep}{\separate}
\newcommand{\size}{\mathtt{size}}
\newcommand{\sizeeq}[1]{\mathtt{size} \ = \ #1}
\newcommand{\alloc}[1]{\mathtt{alloc}(#1)}
\newcommand{\allocb}[2]{\mathtt{alloc}^{-1}[#2](#1)}
\newcommand{\isol}[1]{\mathtt{isoloc}(#1)}
\newcommand{\icell}{\mathtt{isocell}}
\newcommand{\malloc}{\mathtt{malloc}}
\newcommand{\cons}{\mathtt{cons}}
\newcommand{\new}{\mathtt{new}}
\newcommand{\free}[1]{\mathtt{free} #1}
\newcommand{\maxform}[1]{\mathtt{maxForms}(#1)}
\newcommand{\locations}[1]{\mathtt{loc}(#1)}
\newcommand{\values}{\mathtt{Val}}
\newcommand{\aheap}{\mathfrak{h}}
\newcommand{\avaluation}{\mathfrak{V}}
\newcommand{\heaps}{\mathcal{H}}
\newcommand{\astore}{\mathfrak{s}}
\newcommand{\stores}{\mathcal{S}}
\newcommand{\amodel}{\mathfrak{M}}
\newcommand{\alabel}{\ell}

\newcommand{\aprogram}{\mathtt{PROG}}
\newcommand{\programs}{\mathtt{P}}
\newcommand{\ctprograms}{\programs^{ct}}
\newcommand{\aninstruction}{\mathtt{instr}}
\newcommand{\ainstruction}{\mathtt{instr}}
\newcommand{\instructions}{\mathtt{I}}
\newcommand{\aguard}{\ensuremath{g}}
\newcommand{\guards}{\ensuremath{G}}
\newcommand{\domain}[1]{\mathtt{dom}(#1)}
\newcommand{\memory}{\stores\times\heaps}
\newcommand{\skipinstruction}{\mathtt{skip}}

\newcommand{\execution}{\mathtt{comp}}
\newcommand{\aux}{\mathtt{embd}}
\newcommand{\runof}{run}
\newcommand{\anexecution}{e}


\newcommand{\aletter}{\ensuremath{a}}
\newcommand{\aletterbis}{\ensuremath{b}}
\newcommand{\alocation}{\mathfrak{l}}

\newcommand{\pointsl}[1]{\stackrel{#1}{\hookrightarrow}}
\newcommand{\ppointsl}[1]{\stackrel{#1}{\mapsto}}
\newcommand{\ourhook}[1]{\stackrel{#1}{\hookrightarrow}}
\newcommand{\ltrue}{{\sf true}}
\newcommand{\lfalse}{{\sf false}}


\newcommand{\variables}{\mathtt{FVAR}}
\newcommand{\pvariables}{\mathtt{PVAR}}
\newcommand{\secvariables}{\mathtt{SVAR}}
\newcommand{\logique}[1]{\mathtt{FO}(#1)}



\newcommand{\atranslation}{\mathfrak{t}}
\newcommand{\nbpred}[1]{\widetilde{\sharp #1}}
\newcommand{\nbpredstar}[1]{\widetilde{\sharp #1}^{\star}}
\newcommand{\isolated}{\mathtt{isol}}
\newcommand{\stdmarks}{\mathtt{envir}}
\newcommand{\relation}[1]{\mathtt{relation}_{#1}}
\newcommand{\freevar}{\mathtt{FV}}
\newcommand{\notonmark}{\mathtt{notonenv}}
\newcommand{\InVal}[1]{\mathtt{InVal}\!\left(#1\right)}
\newcommand{\NotOnEnv}[1]{\mathtt{NotOnEnv}\!\left(#1\right)}
\newcommand{\PartOfVal}[1]{\mathtt{PartOfVal}\!\left(#1\right)}
%\newcommand{\nbpreds}[3]{\sharp #1 \geq #2}
\newcommand{\defstyle}[1]{{\emph{#1}}}

\newcommand{\cut}[1]{}
\newcommand{\interval}[2]{[#1,#2]}
\newcommand{\buniquevar}{\overline{\uniquevar}}
\newcommand{\bbuniquevar}{\overline{\overline{\uniquevar}}}
\newcommand{\magicwand}{\mathop{\mbox{$\mbox{$-~$}\!\!\!\!\ast$}}}
\newcommand{\wand}{\magicwand}
\newcommand{\septraction}{\stackrel{\hsize0pt \vbox to0pt{\vss\hbox to0pt{\hss\raisebox{-6pt}{\footnotesize$\lnot$}\hss}\vss}}{\magicwand}}
%% \newcommand{\reach}{\mathtt{reach}}
\mathchardef\mhyphen="2D % hyphen while in math mode

\newcommand{\adataword}{\mathfrak{dw}}
\newcommand{\adatum}{\mathfrak{d}}

\newcommand{\collectionknives}{\mathtt{ks}}
\newcommand{\collectionknivesfork}[1]{\mathtt{ksfs}_{=#1}}
\newcommand{\collectionknivesforks}{\mathtt{ksfs}}
\newcommand{\collectionkniveslargeforks}{\mathtt{kslfs}}


\newcommand{\acounter}{\mathtt{C}}

\newcommand{\fotwo}[3]{{\mbox{FO2}_{#1,#2}(#3)}}
\newcommand{\mtrans}[1]{t\!\left(#1\right)^{\Box}}
\newcommand{\mbtrans}[2]{\mtrans{#2}_{#1}}


\newcommand{\alogic}{\mathfrak{L}}


\newcommand{\semantics}[1]{\ensuremath{[ #1 ]}}


\newcommand{\adomino}{\mathfrak{d}}
\newcommand{\atile}{\mathfrak{d}}
\newcommand{\atiling}{\mathfrak{t}}

\newcommand{\hori}{\mathtt{h}}
\newcommand{\verti}{\mathtt{v}}
\newcommand{\domi}{\mathtt{d}}

\newcommand{\cpyrel}{\mathfrak{cp}}

\newcommand{\cntcmp}{\mathfrak{C}}

\newcommand{\heapdag}{\mathfrak{G}}

\newcommand{\onmainpath}{\mathtt{mp}}

\newcommand{\tree}{\mathtt{tree}}

%\newcommand{\tile}{\mathtt{tile}}

\newcommand{\type}{\mathtt{type}}

\newcommand{\ptype}{\mathtt{ptype}}

\newcommand{\exttype}{\mathtt{exttype}}

\newcommand{\anctypes}{\mathtt{AncTypes}}

\newcommand{\destypes}{\mathtt{DesTypes}}

\newcommand{\inctypes}{\mathtt{IncTypes}}

\newcommand{\treeic}{\mathtt{treeIC}}

\newcommand{\trs}{\mathfrak{trs}}


\newcommand{\nin}{\not \in}
\newcommand{\cupplus}{\uplus}
\newcommand{\aunarypred}{\mathtt{P}}


\newcommand{\hide}[1]{}

\newcommand{\eval}[2]{\llbracket#1\rrbracket_{#2}}
\newcommand\cur{\mathsf{cur}}
\newcommand\dom{\mathsf{dom}}
\newcommand\rng{\mathsf{rng}}

\newcommand\dd{\mathbb{D}}
\newcommand\nat{\mathbb{N}}


\newcommand\cA{\mathcal{A}}
\newcommand\cB{\mathcal{B}}
\newcommand\cC{\mathcal{C}}
\newcommand\cE{\mathcal{E}}
\newcommand\cG{\mathcal{G}}
\newcommand\cI{\mathcal{I}}
\newcommand\Ll{\mathcal{L}}
\newcommand\cM{\mathcal{M}}
\newcommand\cP{\mathcal{P}}
\newcommand\cR{\mathcal{R}}
\newcommand\cS{\mathcal{S}}
\newcommand\cT{\mathcal{T}}


\newcommand\vard{\mathfrak{d}}

\newcommand\replaceall{\mathsf{replaceAll}}
\newcommand\indexof{\mathsf{IndexOf}}



\newcommand\strline{\mathsf{SL}}

\newcommand\pstrline{\mathsf{SL_{pure}}}

\newcommand\search{\mathsf{search}}

\newcommand\verify{\mathsf{verify}}

\newcommand\searchleft{\mathsf{searchLeft}}

\newcommand\searchlong{\mathsf{searchLong}}


\newcommand\pref{\mathsf{Pref}}

\newcommand\wprof{\mathsf{WP}}

\newcommand\vars{\mathsf{Vars}}

\newcommand\dep{\mathsf{Dep}}
\newcommand\ptn{\mathsf{Ptn}}

\newcommand\src{\mathsf{src}}
\newcommand\strtorep{\mathsf{strToRep}}

\newcommand\rpleft{\mathsf{l}}
\newcommand\rpright{\mathsf{r}}


\newcommand\srcnd{\mathsf{srcND}}

\newcommand\ctxt{\mathsf{ctxt}}


\newcommand\ctxts{\mathsf{Ctxts}}

\newcommand\sprt{\mathsf{sprt}}

\newcommand\val{\mathsf{val}}

\newcommand\srclen{\mathsf{srcLen}}

\newcommand\rpleftlen{\mathsf{lLen}}


\newcommand\dfs{\mathsf{DFS}}

\newcommand\repr{\mathsf{rep}}

\newcommand\red{\mathsf{red}}

\newcommand\gfun{\mathcal{F}}


\newcommand{\leftmost}{{\sf leftmost}}
\newcommand{\longest}{{\sf longest}}

\newcommand{\arbidx}{{\sf Idx_{arb}}}
\newcommand{\dmdidx}{{\sf Idx_{dmd}}}
\newcommand{\lftlen}{{\sf Len_{lft}}}

\newcommand\rcdim{\mathsf{dim}}

\newcommand\rcdep{\mathsf{dep}}

\newcommand\tower{\mathrm{Tower}}


%\newtheorem{remark}[theorem]{Remark}

%%%%%%%%%%%%%%%%%%%%%%%%%%%%%%%%%%%%%%%
% Macros for two-way lower bound proof.

    \usepackage{unicode-math}

    \newcommand\ap[2]{{#1}\mathord{\brac{#2}}}

    % Tiling

    \newcommand\tiles{\Theta}
    \newcommand\hrel{H}
    \newcommand\vrel{V}
    \newcommand\tile{t}
    \newcommand\inittile{t_I}
    \newcommand\fintile{t_F}
    \newcommand\tileheight{h}

    % Large numbers

    \newcommand\expheight{n}
    \newcommand\linlen{m}
    \newcommand\tilesnum[1]{\Theta_{#1}}
    \newcommand\hrelnum[1]{H_{#1}}
    \newcommand\vrelnum[1]{V_{#1}}
    \newcommand\inittilenum[1]{\inittile^{#1}}
    \newcommand\fintilenum[1]{\fintile^{#1}}

    % \goodnums{n}{x} means x is good seqs of level-n nums
    \newcommand\goodnums[2]{\ap{\varphi_{#1}}{#2}}

    % \tenc{n}{val} = tile encoding of val in level-n
    \newcommand\tenc[2]{[#2]_{#1}}
    % \exptower{n}{m}  = tower of height n, to the m.
    \newcommand\nexp[2]{2 \uparrow_{#1} \brac{#2}}
    % \nmax{n} -- max number encodable at level n
    \newcommand\nmax[1]{\text{MAX}_{#1}}

    % nested alphabets, argument is level of nesting
    \newcommand\bone[1]{1_{#1}}
    \newcommand\bzero[1]{0_{#1}}
    \newcommand\nestednum[1]{c^{#1}}
    \newcommand\nestedalphabet[1]{\Sigma_{#1}}

    \newcommand\numeq{\circledequal}
    \newcommand\numplus{\oplus}
    \newcommand\numsep{\#}

    \newcommand\probfmla{\varphi}
    \newcommand\tilerow{r}

\newcommand{\ASSERT}[1]{\textbf{assert}(#1)}

\newcommand{\straightline}{\textsf{SL}}
\newcommand{\straightlinesym}{\textsf{SLS}}

\newcommand{\Pre}{\textsf{Pre}}

\newcommand{\twpt}{\textsf{2PT}}
\newcommand{\owpt}{\textsf{PT}}
\newcommand{\rbtwpt}{\textsf{RB2PT}}
\newcommand{\twspt}{\textsf{2SPT}}
\newcommand{\owspt}{\textsf{SPT}}

\newcommand{\transet}{\mathscr{T}}

\newcommand\theory{{\sf Th}}

\newcommand{\signature}{\mathcal{S}}

\newcommand{\sorts}{{\mathfrak{S}}}

\newcommand{\functions}{{\mathcal{F}}}

\newcommand{\predicates}{{\mathcal{P}}}

\newcommand\data{\mathbb{D}}

\newcommand{\interpretation}{\mathcal{I}}

\newcommand{\structure}{\mathcal{D}}

\newcommand{\issat}{\mathsf{isSat}}

\newcommand{\OMIT}[1]{}

\newcommand{\Left}{\ensuremath{\leftarrow}}
\newcommand{\Right}{\ensuremath{\rightarrow}}
\newcommand{\Stay}{\ensuremath{\text{\scshape S}}}

\newcommand{\Aut}{\ensuremath{\mathcal{A}}}
\newcommand{\AutB}{\ensuremath{\mathcal{B}}}
\newcommand{\Transducer}{\ensuremath{T}}
\newcommand{\controls}{\ensuremath{Q}}
\newcommand{\finals}{\ensuremath{F}}
\newcommand{\transrel}{\ensuremath{\delta}}

\newcommand{\Lang}{\mathcal{L}}
\newcommand{\ialphabet}{\Sigma}

\newcommand{\EndLeft}{\ensuremath{\triangleright}}
\newcommand{\EndRight}{\ensuremath{\triangleleft}}

%==========================================================

%\newcommand\shortlong[2]{#2}
\newcommand\shortlong[2]{#1}

%\newif\ifdraft\drafttrue
\newif\ifdraft\draftfalse
\ifdraft
\newcommand{\anthony}[1]{\color{red} {YA: #1 :AY} \color{black}}
\newcommand{\zhilin}[1]{\color{brown} {ZL: #1 :LZ} \color{black}}
\newcommand{\tl}[1]{\color{blue} {TL: #1 :LT} \color{black}}
\newcommand{\mat}[1]{\color{cyan} {MH: #1 :HM} \color{black}}
\else
\newcommand{\anthony}[1]{}
\newcommand{\zhilin}[1]{}
\newcommand{\tl}[1]{}
\newcommand{\mat}[1]{}
\fi

\newcommand{\concat} {\circ}
\newcommand{\replace} {{\sf replace}}
\newcommand{\str} {{\sf Str}}
\newcommand{\intnum} {{\sf Int}}
\newcommand{\regexp} {{\sf RegExp}}
\newcommand{\strarr} {{\sf StringArray}}
\newcommand{\dtypes} {{\sf DataTypes}}
\newcommand{\anarr} {{\mathbb{A}}}

%============================================================


%% Bibliography style
\bibliographystyle{ACM-Reference-Format}
%% Citation style
%% Note: author/year citations are required for papers published as an
%% issue of PACMPL.
\citestyle{acmauthoryear}   %% For author/year citations

%%% If you see 'ACMUNKNOWN' in the 'setcopyright' statement below,
%%% please first submit your publishing-rights agreement with ACM (follow link on submission page).
%%% Then please update our instructions page and copy-and-paste the NEW commands into your article.
%%% Please contact us in case of questions; allow up to 10 min for the system to propagate the information.
%%%
%%% The following is specific to POPL'18 and the paper
%%% 'What's Decidable about String Constraints with ReplaceAll Function?'
%%% by Taolue Chen, Yan Chen, Matthew Hague, Anthony W. Lin, and Zhilin Wu.
%%%
\setcopyright{ACMUNKNOWN}
\acmPrice{}
\acmDOI{10.1145/3158091}
\acmYear{2018}
\copyrightyear{2018}
\acmJournal{PACMPL}
\acmVolume{2}
\acmNumber{POPL}
\acmArticle{3}
\acmMonth{1}

\begin{document}
\newtheorem{remark}[theorem]{Remark}

%% Title information
\title[What Is Decidable about String Constraints with the ReplaceAll Function]{What Is Decidable about String Constraints with the ReplaceAll Function}         %% [Short Title] is optional;
                                        %% when present, will be used in
                                        %% header instead of Full Title.
%\titlenote{}             %% \titlenote is optional;
                                        %% can be repeated if necessary;
                                        %% contents suppressed with 'anonymous'
%\subtitle{}                     %% \subtitle is optional
%\subtitlenote{}       %% \subtitlenote is optional;
                                        %% can be repeated if necessary;
                                        %% contents suppressed with 'anonymous'


%% Author information
%% Contents and number of authors suppressed with 'anonymous'.
%% Each author should be introduced by \author, followed by
%% \authornote (optional), \orcid (optional), \affiliation, and
%% \email.
%% An author may have multiple affiliations and/or emails; repeat the
%% appropriate command.
%% Many elements are not rendered, but should be provided for metadata
%% extraction tools.

\author{Taolue Chen}
\orcid{0000-0002-5993-1665}
\affiliation{
  %\position{Position1}
  \department{Department of Computer Science and Information Systems}              %% \department is recommended
  \institution{Birkbeck, University of London}            %% \institution is required
  \streetaddress{Malet Street}
  \city{London}
  %\state{State1}
  \postcode{WC1E 7HX}
  \country{United Kingdom}
}
\email{taolue@dcs.bbk.ac.uk}          %% \email is recommended

\author{Yan Chen}
%\orcid{0000-0002-5993-1665}
\affiliation{
  %\position{Position1}
%  \department{State Key Laboratory of Computer Science} %% \department is recommended
  \institution{State Key Laboratory of Computer Science, Institute of Software, Chinese Academy of Sciences} %% \institution is required
  %\streetaddress{Street1 Address1}
  %\city{City1}
  %\state{State1}
  %\postcode{Post-Code1}
  \country{China}
}
\affiliation{
  %\position{Position1}
%  \department{State Key Laboratory of Computer Science} %% \department is recommended
  \institution{University of Chinese Academy of Sciences} %% \institution is required
  %\streetaddress{Street1 Address1}
  %\city{City1}
  %\state{State1}
  %\postcode{Post-Code1}
  \country{China}
}
%\email{first1.last1@inst1.edu}          %% \email is recommended

\author{Matthew Hague}
\orcid{0000-0003-4913-3800}
\affiliation{
  %\position{Position1}
  \department{Department of Computer Science} %% \department is recommended
  \institution{Royal Holloway, University of London} %% \institution is required
  \streetaddress{Egham Hill}
  \city{Egham}
  \state{Surrey}
  \postcode{TW20 0EX}
  \country{United Kingdom}
}
\email{matthew.hague@rhul.ac.uk}          %% \email is recommended

\author{Anthony W. Lin}
\orcid{0000-0003-4715-5096}
\affiliation{
  %\position{Position1}
  \department{Department of Computer Science}              %% \department is recommended
  \institution{University of Oxford}            %% \institution is required
  \streetaddress{Wolfson Buildin, Parks Road}
  \city{Oxford}
  %\state{State1}
  \postcode{OX1 3QD}
  \country{United Kingdom}
}
\email{anthony.lin@cs.ox.ac.uk}          %% \email is recommended

\author{Zhilin Wu}
%\orcid{0000-0002-5993-1665}
\affiliation{
  %\position{Position1}
%  \department{State Key Laboratory of Computer Science} %% \department is recommended
  \institution{State Key Laboratory of Computer Science, Institute of Software, Chinese Academy of Sciences} %% \institution is required
  %\streetaddress{Street1 Address1}
  %\city{City1}
  %\state{State1}
  %\postcode{Post-Code1}
  \country{China}
}
%\email{first1.last1@inst1.edu}          %% \email is recommended




%% Paper note
%% The \thanks command may be used to create a "paper note" ---
%% similar to a title note or an author note, but not explicitly
%% associated with a particular element.  It will appear immediately
%% above the permission/copyright statement.
%\thanks{with paper note}                %% \thanks is optional
                                        %% can be repeated if necesary
                                        %% contents suppressed with 'anonymous'


%% Abstract
%% Note: \begin{abstract}...\end{abstract} environment must come
%% before \maketitle command


%% 2012 ACM Computing Classification System (CSS) concepts
%% Generate at 'http://dl.acm.org/ccs/ccs.cfm'.
\begin{CCSXML}
<ccs2012>
<concept>
<concept_id>10003752.10003790.10003794</concept_id>
<concept_desc>Theory of computation~Automated reasoning</concept_desc>
<concept_significance>500</concept_significance>
</concept>
<concept>
<concept_id>10003752.10003790.10011192</concept_id>
<concept_desc>Theory of computation~Verification by model checking</concept_desc>
<concept_significance>500</concept_significance>
</concept>
<concept>
<concept_id>10003752.10010124.10010138.10010142</concept_id>
<concept_desc>Theory of computation~Program verification</concept_desc>
<concept_significance>500</concept_significance>
</concept>
<concept>
<concept_id>10003752.10010124.10010138.10010143</concept_id>
<concept_desc>Theory of computation~Program analysis</concept_desc>
<concept_significance>500</concept_significance>
</concept>
<concept>
<concept_id>10003752.10003790.10002990</concept_id>
<concept_desc>Theory of computation~Logic and verification</concept_desc>
<concept_significance>300</concept_significance>
</concept>
<concept>
<concept_id>10003752.10003777.10003778</concept_id>
<concept_desc>Theory of computation~Complexity classes</concept_desc>
<concept_significance>100</concept_significance>
</concept>
</ccs2012>
\end{CCSXML}

\ccsdesc[500]{Theory of computation~Automated reasoning}
\ccsdesc[500]{Theory of computation~Verification by model checking}
\ccsdesc[500]{Theory of computation~Program verification}
\ccsdesc[500]{Theory of computation~Program analysis}
\ccsdesc[300]{Theory of computation~Logic and verification}
\ccsdesc[100]{Theory of computation~Complexity classes}
%% End of generated code


%% Keywords
%% comma separated list
\keywords{String Constraints, ReplaceAll, Decision Procedures, Constraint Solving, Straight-Line Programs}

%!TEX root = main.tex

\begin{abstract}
%% some background on regular expressions
Regular expressions (RE) are a classical concept in formal language theory.
%, which are expressions built from characters by the operators of concatenation, union, and Kleene star. 
Real-world regular expressions (RWRE) in programming languages differ from REs 
in the non-standard semantics of operators (e.g. non-commutative union and 
greedy/lazy Kleene star), as well as  the additional features like capturing
groups and backreferences. 
%% symbolic execution requires faithful encoding of regex semantics
%String constraint solvers are one of the cornerstones of the analysis and verification of string manipulating programs. Faithful encoding of the semantics of real-world regular expressions in string constraint solvers facilitates more precise program analysis and verification.
%% state of the art string constraint solvers
%The semantics of real-word regular expressions are tricky and vary in different programming languages. It is challenging for string constraint solvers to support real-world regular expressions.
While REs are supported by every state-of-the-art string constraint solver, 
RWREs are thus far unsupported. Recent works have suggested that 
the mismatch between REs and RWREs
 %in string constraint 
%solvers and RWREs in programming languages 
%hinders the precision and efficiency 
makes it difficult for a symbolic execution engine to deal with RWREs,
which are hitherto approximated by a CEGAR-based approach,\philipp{this should be rephrased, it sounds as if CEGAR is the standard approach to deal with RWREs. It is not, there is only one symex tool doing that.} which performs
many satisfiability checks of string constraints with only REs.
%; hitherto,
%RWREs are either approximated away
%
%by resorting to approximate 
%encoding of RWREs and counter-example guided abstraction refinements (CEGAR).
%
%The semantics of real-world regular expressions are better to be encoded as faithfully as possible. For instance, in dynamic symbolic execution of string manipulating programs, in order to generate the inputs for execution paths and improve coverage,  the semantics of real-world regular expressions are better to be encoded as faithfully as possible in string constraint solvers. 
%% contribution
In this paper, we propose an approach of natively supporting RWREs in string
constraint solving. 
The key idea of our approach is to introduce a new automata model, called \emph{prioritized streaming string transducers} (PSST), 
 to model the string functions involving RWREs.
PSSTs combine \emph{priorities,} which have previously been introduced
in prioritized finite-state automata to capture greedy/non-greediness,
with \emph{string variables} as in streaming string transducers to
model capture groups.
%
%non-standard semantics of regular expression operators can be modeled by priorities and new features of capturing groups and back references can be modeled by string variables. 
%the non-standard semantics of regular expression operators can be modeled by priorities and new features of capturing groups and back references can be modeled by string variables. 
Based on PSSTs, we design a decision procedure for string constraints with 
RWREs and provide its implementation.
%% implementation and experiments
%We implement the decision procedure 
%and do extensive experiments to evaluate its performance. 
We evaluate its performance on over 160,000 string constraints generated 
from RWREs in open-source programs.
Our approach dramatically improves the CEGAR-based approach for RWREs in both 
precision and efficiency. 
%To the best of our knowledge, this work represents the first string constraint solver supporting RWREs.
\end{abstract}



%% \maketitle
%% Note: \maketitle command must come after title commands, author
%% commands, abstract environment, Computing Classification System
%% environment and commands, and keywords command.
\maketitle

%!TEX root = main.tex

string manipulating programs

symbolic execution

string operations with integer data type

We use the following running example to illustrate the decision procedure in this paper.
%\begin{example}
{\small
\begin{minted}[linenos]{javascript}
function urlSimpleParse(url)
{
  var protocol='', host='';
  url = url.trim();
  var colonpos = url.indexof(':');
  if (colonpos >= 0) 
  {
    protocol = url.substr(0, colonpos).toLowerCase();
    if(/^http$|^https$/.test(protocol))
    {
      url = url.substr(colonpos+3);
      var slashpos = url.indexof('/');
      if (slashpos >= 0)  host = url.substr(0, slashpos); 
    }
    else protocol = '';
    return protocol, host; 
  }
}
\end{minted}
}
%\end{example}

% \in [\backslash w | \backslash x2E]^*$
We expect that host contains only the alphanumeric symbols as well as the dot symbol, but actually this is not the case. This question can be reduced to solving the path feasibility problem of the following program of the SSA (single static assignment) form.

{\small
\begin{minted}{javascript}
  protocol = ''; host = ''; url1 = url.trim(); 
  colonpos = url1.indexof(':'); assert(colonpos >= 0); 
  protocol1 = url1.substr(0, colonpos); 
  protocol2 = protocol1.toLowerCase();
  assert(/^http$|^https$/.test(protocol2));
  url2 = url1.substr(colonpos+3);
  slashpos = url2.indexof('/'); assert(slashpos >= 0);
  host1 = url2.substr(0, slashpos); assert(!/[\w|\x2E]*/.test(host1))
\end{minted}
}

state-of-the-art: heuristics

The contribution of this paper: decision procedure for string constraints involving integer data type

automata-theoretic, cost-enriched regular languages and recognisable relations, backward computation

implementation OSTRICH+, experimental results promising

first decision procedure for such an expressive class of string constraints involving so many different operations, natural extension of the decision procedure of OSTRICH, efficient implementation, extensive experiments, 


related work

SLENT: \cite{WC+18}

CVC4: \cite{cvc4}

TRAU, Z3-TRAU, TRAU+: \cite{Abdulla17,AbdullaA+19}

Z3-STR: \cite{Z3-str}

OSTRICH: \cite{CHL+19}

\section{Preliminaries}

Throughout the paper, $\Int^+$ denotes the set of positive integers, and  $\nat$ denotes the set of natural numbers. Furthermore, for $n\in \Int^+$, let $[n]:=\{1, \ldots, n\}$. 

\begin{definition}[Finite-state automata] \label{def:nfa}
	A \emph{(nondeterministic) finite-state automaton}
	(\FA{}) over a finite alphabet $\ialphabet$ is a tuple $\Aut =
	(\ialphabet, \controls, q_0, \finals, \transrel)$ where 
	$\controls$ is a finite set of 
	states, $q_0\in \controls$ is
	the initial state, $\finals\subseteq \controls$ is a set of final states, and 
	$\transrel\subseteq \controls \times 
	\ialphabet \times  \controls$ is the
	transition relation. 
\end{definition}

For an input string $w=a_1 \dots a_n$, a \emph{run} of $\Aut$ on $w$
%(with $a_0 = \EndLeft$ and $a_{n+1} = \EndRight$)
is a sequence of states $q_0, \ldots, q_n$ such that $(q_{j-1}, a_{j}, q_{j}) \in
\transrel$  for every $j \in [n]$.
The run is said to be \defn{accepting} if $q_n \in \finals$.
A string $w$ is \defn{accepted} by $\Aut$ if there is an accepting run of
$\Aut$ on $w$. In particular, the empty string $\varepsilon$ is accepted by $\Aut$ if $q_0 \in F$. The set of strings accepted by $\Aut$, i.e., the language \defn{recognised} by $\Aut$, is denoted by $\Lang(\Aut)$.
%Since we deal with computational complexity in the sequel, we define
The \defn{size} $|\Aut|$ of $\Aut$ is defined to be the cardinality of the set $Q$ of states, which will be 
used when the computational complexity is concerned.

For convenience, for $a \in \Sigma$, we use $\delta^{(a)}$ to denote the  relation $\{(q, q') \mid (q, a, q') \in \delta\}$.


  
%%%%%%%%%%%%%%%%%%%%%%%%%%%%%%%%%
 \subsection{Extended regular expressions}
%%%%%%%%%%%%%%%%%%%%%%%%%%%%%%%%%%  
An (extend) regular expression (with capturing group and back reference) is defined as follows.
  
\begin{definition}[Regular expressions with capturing group and back reference, $\regexp$]
  	\[e \eqdef \emptyset \mid \varepsilon \mid a \mid \$n \mid e + e \mid e \concat e \mid e^* \mid (e)  , \]
  	where $a \in \Sigma, n \in \Int^+$. 
  	%	Since $+$ is associative and commutative, we also write $(e_1 + e_2) + e_3$ as $e_1 + e_2 + e_3$ for brevity. 
  	%We use the abbreviation 
\end{definition}
  	
We use $e^+$ to abbreviate $e \concat e^*$. Moreover, for $\Gamma = \{a_1, \ldots, a_k\}\subseteq \Sigma$, we write $\Gamma$ for  $a_1 + \cdots + a_k$ and thus  $\Gamma^\ast \equiv (a_1 + \cdots + a_k)^\ast$. 

We assume that the parentheses in a regular expression are well matched. 
%
%Besides the common rules governing regular expressions, a regex obeys
%the following syntactic rule: 
Moreover, every back reference $\$ n$ is found to the right of the $n$-th pair of parentheses, where parentheses
are indexed according to the occurrence sequence of their left parenthesis.
\zhilin{some sanity conditions should be put to make the semantics of $\$ n$ well-defined.}
  
Note that standard regular expressions are those without $\$ n$ or $(e)$. Moreover, we use $\regexp[\sf CG]$ to denote the fragment of $\regexp$  excluding $\$ n$, and $\refexp$ to denote expressions generated by $e \eqdef \varepsilon \mid a \mid \$n \mid e \concat e$.
  %\tl{define the semantics here?}
  
  %\label{semantics:regex}
  
  
  
  %\subsection{Semantics of \regexp[\sf CG]}
  %In this section, we give one of the many semantics of \regexp[\sf CG], which we will utilize for $\replaceall$.
  
  \begin{definition}[Subexpression]
  	For any two $\regexp[\sf CG]$ $e$ and $r$, we say $r$ is a subexpression of $e$,
  	if either $r=e$ or
  	\begin{itemize}
  		\item If $e = e_1 e_2$ or $e_1 + e_2$ then $r$ is a subexpression of $e_1$
  		or $e_2$
  		
  		\item If $e = e_1^{\ast}$ or $(e_1)$ then $r$ is a subexpression of $e_1$
  	\end{itemize}
  	We use $S (e)$ to denote the set of all subexpressions of $e$.
  \end{definition}
  
  
  \begin{definition}[Match Tree]
  	A \tmtextbf{match tree} of $\regexp[\sf CG]$ $e$ is a finite directed and ordered
  	tree T, whose nodes are elements of $\Sigma^{\ast} \times S (e)$. A tree
  	is valid if the root is $(w, e)$ for some string w, and for any node $u =
  	(w, \alpha)$ in T, we have:
  	\begin{itemize}
  		\item If $\alpha = \alpha_1 \alpha_2$ then u has two children $(w_1,
  		\alpha_1)$ and $(w_2, \alpha_2)$ where $w = w_1 w_2$.
  		
  		\item If $\alpha = \alpha_1 + \alpha_2$ then u has a single child $(w,
  		\alpha_i)$ where $i \in \{ 1, 2 \}$.
  		
  		\item If $\alpha = \alpha_1^{\ast}$ then when $w = \varepsilon$, u is a
  		leaf otherwise there is $k \geqslant 1$ such that u has k children $(w_1,
  		\alpha_1), \ldots, (w_k, \alpha_1)$ where $w = w_1 \ldots w_k$ and for all
  		$i \in [k]$, $w_i \neq \varepsilon$, even if $\varepsilon \in L
  		(\alpha_1)$.
  		
  		\item If $\alpha = (\alpha_1)$ then u has a single child $(w, \alpha_1)$.
  		
  		\item If $\alpha = a$ (resp. $\alpha = \varepsilon$) then u is a leaf and
  		$w = a$ (resp. $w = \varepsilon$).
  	\end{itemize}
  	
  	Whenever unambiguous, we use a node u to represent the whole subtree
  	where u is the root. The notation $C(T)$ refers to all direct children of the root node of T
  	(and thus all direct subtrees).
  	
  	We also use $M (e)$ to denote all the valid match trees of e.
  \end{definition}
  
  \begin{definition}[Semantics of RegExp{[\sf CG]}]
  	\
  	
  	For any $\regexp[\sf CG]$ e, we recursively define a total order on $M (e)$, written $m
  	>_e n$ where $m, n \in M (e)$, as follows:
  	\begin{itemize}
  		\item $e = \varepsilon$ or $e = a$. There is only one match tree, thus the
  		order $>_e$ is empty.
  		
  		\item $e = (e_1)$. Suppose $C (m) = (w_1, e_1)$ and $C (n) = (w_2, e_1)$,
  		then $m >_e n$ iff $(w_1, e_1) >_{e_1} (w_2, e_1)$.
  		
  		\item $e = e_1 + e_2$.
  		\begin{itemize}
  			\item If $C (m) = (w, e_1)$ and $C (n) = (w', e_2)$, then $m >_e n$.
  			
  			\item If $C (m) = (w, e_i)$ and $C (n) = (w', e_i)$, where $i \in \{ 1,
  			2 \}$, then $m >_e n$ iff $(w, e_i) >_{e_i} (w', e_i)$.
  		\end{itemize}
  		\item $e = e_1 e_2$. Suppose $C (m) = (w_1, e_1) (w_2, e_2)$ and $C (n) =
  		(w_1', e_1) (w_2', e_2)$, then $m >_e n$ when either $(w_1, e_1) >_{e_1}
  		(w_1', e_1)$, or $w_1 = w_1'$ and $(w_2, e_2) >_{e_2} (w_2', e_2)$.
  		
  		\item $e = e_1^{\ast}$. If n is a leaf but m is not, then $m >_e n$.
  		Otherwise, suppose $C (m) = (w_1, e_1), \ldots, (w_k, e_1)$ and $C (n) =
  		(w_1', e_1), \ldots, (w_l', e_1)$, we have $m >_e n$ either when $C (n)$
  		is a proper prefix of $C (m)$, or for the first index j such that $w_j
  		\neq w_j'$, $(w_j, e_1) >_{e_1} (w_j', e_1)$.
  	\end{itemize}
  	
  	Let $e \in \regexp[\sf CG]$, and string $w \in L (e)$, then the accepting match of w
  	on e, denoted by $m_e (w)$, is the supremum of the set $\{ m \in M (e) \mid m = (w, e) \}$.
  	
  	For any subexpression $e'$ of e, suppose $(w_1, e') \ldots (w_m, e')$ are
  	all the nodes labeled by $(w', e')$, where $w' \neq \varepsilon$, in the
  	pre-order traversal of $m_e (w)$. Then the match result captured by $e'$, denoted
  	by $m_{e', e} (w)$, is the sequence of substring $w_1 \ldots w_m$.
  \end{definition}
  
  \zhilei{Give an example here}


\section{Outline of Decision Procedures}

We describe our decision procedure across three sections (Section~\ref{sec:replaceallsl}--Section~\ref{sec:replaceallre}).
This means the ideas can be introduced in a step-by-step fashion, which we hope helps the reader.
In addition, by presenting separate algorithms, we can give the fine-grained complexity analysis required to show Corollary~\ref{cor-pspace}.
We first outline the main ideas needed by our approach.

We will use automata-theoretic techniques.
That is, we make use of the fact that regular expressions can be represented as NFAs.
We can then consider a very simple string expression, which is a single regular constraint $x \in e$.
It is well-known that an NFA $\cA$ can be constructed that is equivalent to $e$.
We can also test in LOGSPACE whether there is some word $w$ accepted by $\cA$.
If this is the case, then this word can be assigned to $x$, giving a satisfying assignment to the constraint.
If this is not the case, then there is no satisfying assignment.

A more complex case is a conjunction of several constraints of the form $x \in e$.
If the constraints apply to different variables, they can be treated independently to find satisfying assignments.
If the constraints apply to the same variable, then they can be merged into a single NFA.\@
Intuitively, take $x \in e_1 \land x \in e_2$ and $\cA_1$ and $\cA_2$ equivalent to $e_1$ and $e_2$ respectively.
We can use the fact that NFA are closed under intersection a check if there is a word accepted by $\cA_1 \times \cA_2$.
If this is the case, we can construct a satisfying assignment to $x$ from an accepting run of $\cA_1 \times \cA_2$.


In the general case, however, variables are not independent, but may be related by a use of $\replaceall$.
In this case, we perform a kind of \emph{$\replaceall$ elimination}.
That is, we successively remove instances of $\replaceall$ from the constraint, building up an expanded set of regular constraints (represented as automata).
Once there are no more instances of $\replaceall$ we can solve the regular constraints as above.
Briefly, we identify some $x = \replaceall(y, e, z)$ where $x$ does not appear as an argument to any other use of $\replaceall$.
We then transform any regular constraints on $x$ into additional constraints on $y$ and $z$.
This allows us to remove the variable $x$ since the extended constraints on $y$ and $z$ are sufficient for determining satisfiability.
Moreover, from a satisfying assignment to $y$ and $z$ we can construct a satisfying assignment to $x$ as well.
This is the technical part of our decision procedure and is explained in detail in the following sections, for increasingly complex uses of $\replaceall$.





%!TEX root = popl2018.tex


\section{Decision procedure for $\strline[\replaceall]$: The single-letter case} \label{sec:replaceallsl}





%%%%%%%%%%%%%%%%%%%%%%%%%%%%%%%%%
%%%%%%%%%%%%%%%%%%%%%%%%%%%%%%%%%
\hide{
\subsection{Encode concatenation by replaceall}

We will transform a given constraint $\varphi$ into a constraint $\varphi'$ which is concatenation free. As the first step, we extend the original alphabet with two fresh letters $a,b$.

For each $x=yz$, we introduce a new variable $x'$ and replace $x=yz$ by two new constraints
$x'=\replaceall(ab, a, x)$ and $x=\replaceall(x', b, z)$.

\begin{proposition}
	$\varphi$ and $\varphi'$ are equisatisfiable.
\end{proposition}
}
%%%%%%%%%%%%%%%%%%%%%%%%%%%%%%%%%
%%%%%%%%%%%%%%%%%%%%%%%%%%%%%%%%%


%\subsection{The single-letter case}

In this section, we consider the single-letter case, that is, for the $\strline[\replaceall]$ formula $C = \varphi \wedge \psi$, every term of the form $\replaceall(z, e, z')$ in $\varphi$ satisfies that $e=a$ for $a \in \Sigma$.
We begin by explaining the idea of the decision procedure in the case where there is a single use of a $\replaceall(\cdots)$ term.
Then we describe the decision procedure in full details.

%We first assume that the regular constraint $\psi = \bigwedge \limits_{x \in \vars(\varphi)} x \in e_x$, that is, there is exactly one atomic regular constraint for each variable.

%We first explain the basic idea of the decision procedure.

\subsection{A single use of $\replaceall(\cdots)$}

Let us start with the simple case that
\[C \equiv x = \replaceall(y, a, z) \wedge x \in e_1 \wedge y \in e_2 \wedge z \in e_3,\]
where, for $i =1, 2, 3$, we suppose  $\cA_i = (Q_i, \delta_i, q_{0,i}, F_i)$
%\cA_y = (Q_y, \delta_y, q_{0, y}, F_y), \cA_z = (Q_z, \delta_z, q_{0, z}, F_z)$
is the NFA corresponding to the regular expression $e_i$.

From the semantics, $C$ is satisfiable if and only if $x, y, z$ can be assigned with strings $u, v, w$ so that: (1) $u$ is obtained from $v$ by replacing all the occurrences of $a$ in $v$ with $w$, and (2) $u, v, w$ are accepted by $\cA_1, \cA_2, \cA_3$ respectively. Let $u,v,w$ be the strings satisfying these two constraints. As $u$ is accepted by $\cA_1$,  there must be an accepting run of $\cA_1$ on $u$. Let $v = v_1 a v_2 a \cdots a v_k$ such that for each $i \in [k]$, $v_i \in (\Sigma \setminus \{a\})^*$. Then $u = v_1 w v_2 w \cdots w v_k$ and there are states $q_1, q'_1, \cdots, q_{k-1}, q'_{k-1}, q_k$  such that
%
$$
q_{0,1} \xrightarrow[\cA_1]{v_1} q_1 \xrightarrow[\cA_1]{w} q'_1 \xrightarrow[\cA_1]{v_2} q_2 \xrightarrow[\cA_1]{w} q'_2 \cdots q_{k-1} \xrightarrow[\cA_1]{w} q'_{k-1} \xrightarrow[\cA_1]{v_k} q_k
$$
%
 and $q_k \in F_{1}$. Let $T_z$ denote $\left\{(q_i, q'_i) \mid i \in [k-1] \right\}$. Then $w \in \Ll(\cA_3)\ \cap\ \bigcap \limits_{(q, q') \in T_z} \Ll(\cA_1(q, q'))$. In addition, let  $\cB_{\cA_1,a,T_z}$ be the NFA obtained from $\cA_1$ by removing all the $a$-transitions first and then adding the $a$-transitions $(q, a, q')$ for $(q, q') \in T_z$. Then
 %
$$
q_{0,1} \xrightarrow[\cB_{\cA_1,a,T_z}]{v_1} q_1 \xrightarrow[\cB_{\cA_1,a,T_z}]{a} q'_1 \xrightarrow[\cB_{\cA_1,a,T_z}]{v_2} q_2 \xrightarrow[\cB_{\cA_1,a,T_z}]{a} q'_2 \cdots q_{k-1} \xrightarrow[\cB_{\cA_1,a,T_z}]{a} q'_{k-1} \xrightarrow[\cB_{\cA_1,a,T_z}]{v_k} q_k.
$$
%
Therefore,
$v \in \Ll(\cA_2) \cap \Ll(\cB_{\cA_1,a,T_z})$. We deduce that there is $T_z \subseteq Q_1 \times Q_1$ such that $\Ll(\cA_3)\ \cap\ \bigcap \limits_{(q, q') \in T_z} \Ll(\cA_1(q, q')) \neq \emptyset$ and $ \Ll(\cA_2) \cap \Ll(\cB_{\cA_1,a,T_z}) \neq \emptyset$. In addition, it is not hard to see that this condition is also sufficient for the satisfiability of $C$. The arguments proceed as follows: Let $v \in  \Ll(\cA_2) \cap \Ll(\cB_{\cA_1,a,T_z})$ and $w \in \Ll(\cA_3)\ \cap\ \bigcap \limits_{(q, q') \in T_z} \Ll(\cA_1(q, q'))$. From $v \in \Ll(\cB_{\cA_1,a,T_z})$, we know that there is an accepting run of $\cB_{\cA_1,a,T_z}$ on $v$. Recall that $\cB_{\cA_1,a,T_z}$ is obtained from $\cA_1$ by first removing all the $a$-transitions, then adding all the transitions $(q,a,q')$ for $(q,q') \in T_z$.  Suppose $v = v_1 a v_2 \cdots a v_k$ such that $v_i \in (\Sigma \setminus \{a\})^*$ for each $i \in [k]$ and
$$
q_{0,1} \xrightarrow[\cB_{\cA_1,a,T_z}]{v_1} q_1 \xrightarrow[\cB_{\cA_1,a,T_z}]{a} q'_1 \xrightarrow[\cB_{\cA_1,a,T_z}]{v_2} q_2 \xrightarrow[\cB_{\cA_1,a,T_z}]{a} q'_2 \cdots q_{k-1} \xrightarrow[\cB_{\cA_1,a,T_z}]{a} q'_{k-1} \xrightarrow[\cB_{\cA_1,a,T_z}]{v_k} q_k
$$
is an accepting run of $\cB_{\cA_1,a,T_z}$ on $v$. Then $q_{0,1} \xrightarrow[\cA_1]{v_1} q_1$, and for each $i \in [k-1]$ we have $(q_i, q'_i) \in T_z$ and $q'_i \xrightarrow[\cA_1]{v_{i+1}} q'_{i+1}$; moreover, $q_k \in F_1$.
Let $u = \replaceall(v, a, w)=v_1 w v_2 \cdots w v_k$. Since $w \in \bigcap \limits_{(q, q') \in T_z} \Ll(\cA_1(q, q'))$,  we infer that
$$
q_{0,1} \xrightarrow[\cA_1]{v_1} q_1 \xrightarrow[\cA_1]{w} q'_1 \xrightarrow[\cA_1]{v_2} q_2 \xrightarrow[\cA_1]{w} q'_2 \cdots q_{k-1} \xrightarrow[\cA_1]{w} q'_{k-1} \xrightarrow[\cA_1]{v_k} q_k
$$
is an accepting run of $\cA_1$ on $u$. Therefore, $u$ is accepted by $\cA_1$ and $C$ is satisfiable.

\begin{proposition}\label{prop-sat-sl-case}
Let $C \equiv x = \replaceall(y, a, z) \wedge x \in e_1 \wedge y \in e_2 \wedge z \in e_3$. Then $C$ is satisfiable iff there is a set $T_{z} \subseteq Q_1 \times Q_1$ such that $\Ll(\cA_3)\ \cap\ \bigcap \limits_{(q, q') \in T_z} \Ll(\cA_1(q, q')) \neq \emptyset$ and $ \Ll(\cA_2) \cap \Ll(\cB_{\cA_1,a,T_z}) \neq \emptyset$.
\end{proposition}

From Proposition~\ref{prop-sat-sl-case}, we can decide the satisfiability of $C$ in polynomial space as follows:
\begin{description}
\item[Step I.] Guess a set $T_{z} \subseteq Q_1 \times Q_1$.
%
\item[Step II.] Guess an accepting run of the product automaton of $\cA_3$ and $\cA_1(q, q')$ for $(q,q') \in T_{z}$.
%
\item[Step III.] Guess an accepting run of the product automaton of $\cA_2$ and $\cB_{\cA_1, a,  T_{z}}$.
\end{description}
During Step II and III, it is sufficient to record $T_z$ and a state of the product automaton, which occupies only a polynomial space.

The above decision procedure can be easily generalised to the case that there are multiple atomic regular constraints for $x$. For instance, let $x \in e_{1,1} \wedge x \in e_{1,2}$ and for $j = 1, 2$, $\cA_{1,j} = (Q_{1,j}, \delta_{1, j}, q_{0,1, j}, F_{1,j})$ be
the NFA corresponding to $e_{1,j}$. Then in Step I, two sets $T_{1,z} \subseteq Q_{1,1} \times Q_{1,1}$ and $T_{2,z} \subseteq Q_{1,2} \times Q_{1,2}$ are guessed, moreover, Step II and III are adjusted accordingly.

\begin{example}\label{exmp-sl}
Let $C \equiv x= \replaceall(y, 0, z) \wedge x \in e_1 \wedge y \in e_2 \wedge z \in e_3$, where $e_1=(0+1)^* (00(0+1)^* + 11(0+1)^*)$,  $e_2= (01)^*$,  and $e_3 = (10)^*$. The NFA $\cA_{1}, \cA_{2}, \cA_{3}$ corresponding to $e_1, e_2, e_3$ respectively are illustrated in Figure~\ref{fig-sl-exmp}. Let $T_z = \{(q_0, q_0), (q_1, q_2)\}$.  Then
$$
\begin{array}{l c l}
 \Ll(\cA_3)\ \cap\ \bigcap \limits_{(q, q') \in T_z} \Ll(\cA_1(q, q'))  & = & \Ll(\cA_3)\ \cap \Ll(\cA_1(q_0, q_0)) \cap \Ll(\cA_1(q_1,q_2)) \\
& = & \Ll((10)^*) \cap \Ll((0+1)^*) \cap \Ll(1(0+1)^*) \\
& \neq & \emptyset.
\end{array}
$$
In addition, $\cB_{\cA_1, 0, T_z}$ (also illustrated in Figure~\ref{fig-sl-exmp}) is obtained from $\cA_1$ by removing all the $0$-transitions, then adding the transitions $(q_0, 0, q_0)$ and $(q_1, 0, q_2)$. Then
$$
 \Ll(\cA_2) \cap \Ll(\cB_{\cA_1,0,T_z})   =  \Ll((01)^*) \cap \Ll((0+1)^* 10 1^*) \neq \emptyset.
$$
We can choose $z$ to be a string from $\Ll(\cA_3) \cap \ \bigcap \limits_{(q, q') \in T_z} \Ll(\cA_1(q, q'))=\Ll((10)^*) \cap \Ll((0+1)^*) \cap \Ll(1(0+1)^*)$, say $10$, and $y$ to be a string from $ \Ll(\cA_2) \cap \Ll(\cB_{\cA_1,0,T_z})   =  \Ll((01)^*) \cap \Ll((0+1)^* 10 1^*)$, say $0101$,  then we set $x$ to $\replaceall(0101, 0, 10)=101101$, which is in $\Ll(\cA_1)$. Thus, $C$ is satisfiable.\qed
%
\begin{figure}[htbp]
\begin{center}
\includegraphics[scale=0.65]{single-letter-example.pdf}
\end{center}
\caption{An example for the single-letter case: One $\replaceall$}\label{fig-sl-exmp}
\end{figure}
\end{example}


\subsection{The general case}

Let us now consider the general case where $C$ contains multiple occurrences of $\replaceall(\cdots)$ terms.
Then the satisfiability of $C$ is decided by the following two-step procedure.

\smallskip

\noindent{\bf Step I.} We utilise the dependency graph $C$ and compute nondeterministically a collection of atomic regular constraints $\cE(x)$ for each variable $x$, in a top-down manner.

%During the computation, for each variable $x$, the collection of atomic regular constraints
Notice that $\cE(x)$ is represented succinctly as a set of pairs $(\cT, \cP)$, where $\cT=(Q, \delta)$ is a transition graph and $\cP \subseteq Q \times Q$. The intention of $(\cT, \cP)$ is to represent succinctly the collection of the atomic regular constraints containing $(Q, \delta, q, \{q'\})$ for each $(q,q') \in \cP$, where $q$ is the initial state and $\{q'\}$ is the set of final states.

%\tl{you want to say that all regular constraints $\cE(x)$ share the same "underlying" graph, but differ in the initial and final states.} \tl{I rephrased a bit-- the original was commented below. Check whether this is what you mean}

Initially, let $G_0:= G_C$.  In addition, for each variable $x$ with the conjunct $x \in e$ in $\psi$ we define $\cE_0(x)$ as follows.  Let $\cA=(Q, \delta, q_0, F)$ be the NFA corresponding to $e$. We nondeterministically choose $q \in F$ and set $\cE_0(x):=  \{((Q, \delta), \{(q_0, q)\})\}$.


%In addition, for each variable $x$, compute $\cE_0(x)$ as follows: Initially, let $\cE_0(x) := \emptyset$.
%For each conjunct $x \in e$ in $\psi$, let $\cA=(Q, \delta, q_0, F)$ be the NFA corresponding to $e$, then nondeterministically choose $q \in F$ and let $\cE_0(x):= \cE_0(x) \cup \{((Q, \delta), \{(q_0, q)\})\}$.
%

We begin with $i:= 0$ and repeat the following procedure until we reach some $i$ where $G_i$ is an empty graph, i.e. a graph without edges.
Note that $G_0$ was defined above.
\begin{enumerate}
\item Select a vertex $x$ of $G_i$ such that $x$ has no predecessors  and has two successors via edges $(x, (\rpleft, a), y)$ and $(x, (\rpright,a), z)$ in $G_i$.  Suppose $\cE_i(x)=\{(\cT_1,\cP_1), \cdots, (\cT_k, \cP_k)\}$, where for each $j \in [k]$, $\cT_j = (Q_j, \delta_j)$. Then $\cE_{i+1}(z)$ and  $\cE_{i+1}(y)$ and $G_{i+1}$ are computed as follows:
\begin{enumerate}
\item For each $j \in [k]$, guess a set $T_{j, z} \subseteq Q_j \times Q_j$.
%
\item If $y \neq z$, then let
$$\cE_{i+1}(z):= \cE_{i}(z) \cup \left\{(\cT_j, T_{j,z}) \mid j \in [k] \right\}, \text{ and}$$
$$\cE_{i+1}(y): = \cE_{i}(y) \cup \left\{(\cT_{\cT_j, a, T_{j,z}}, \cP_j) \mid j \in [k] \right\},$$
where $\cT_{\cT_j, a, T_{j,z}}$ is obtained from $\cT_j$ by first removing all the $a$-transitions, then adding all the transitions $(q, a, q')$ for $(q,q') \in T_{j,z}$.

Otherwise, let
$$\cE_{i+1}(z):= \cE_{i}(z) \cup \left\{(\cT_j, T_{j,z}) \mid j \in [k] \right\} \cup \left\{(\cT_{\cT_j, a, T_{j,z}}, \cP_j) \mid j \in [k] \right\}.$$
In addition, for each vertex $x'$ distinct from $x, y, z$, let $\cE_{i+1}(x') := \cE_i(x')$.
We set $\cE_{i+1}(x) = \emptyset$.
%
\item Let $G_{i+1}:= G_i \setminus \{(x, (\rpleft, a), y), (x, (\rpright,a), z)\}$.
\end{enumerate}
%
\item Let $i: = i+1$.
\end{enumerate}

For each variable $x$, let $\cE(x)$ denote the set $\cE_i(x)$ after exiting the above loop.

\smallskip

\noindent {\bf Step II.}
For each source variable $x$, guess an accepting run of the product of all the NFA in $\cE(x)$.

\smallskip

%Note that in the first sight, the cardinalities of the sets $\cE(x)$ seem exponential w.r.t. the size of $C$ in the worst case and the nondeterministic algorithm may take exponential space.
%By a refined analysis, we can see that the cardinalities of the sets $\cE(x)$ are in fact polynomial.

It remains to argue the correctness and complexity of the above procedure.
Correctness follows a similar argument to Proposition~\ref{prop-sat-sl-case} and is presented in Appendix~\ref{sec:dp-dl-correctness}.
\mat{How do we reference the appendix?}
Intuitively, Proposition~\ref{prop-sat-sl-case} shows our procedure correctly eliminates occurrences of $\replaceall$ until only regular constraints remain.
We now give an example then argue the complexity.

\begin{example}
Suppose $C \equiv x= \replaceall(y, 0, z) \wedge y = \replaceall(y', 1, z') \wedge x \in e_1 \wedge y \in e_2 \wedge z \in e_3 \wedge y' \in e_4  \wedge z' \in e_5$, where $e_1, e_2, e_3$ are as in Example~\ref{exmp-sl}, $e_4=0^* 1^* 0^* 1^*$, and $e_5=0^*1^*$. Let $\cA_4,\cA_5$ be the NFA corresponding to $e_4$ and $e_5$ respectively (see Figure~\ref{fig-sl-exmp-nested}). The dependency graph $G_C$ of $C$ is illustrated in Figure~\ref{fig-sl-exmp-nested}. Let $\cT_1,\cdots, \cT_5$ be the transition graph of $\cA_1,\cdots, \cA_5$ respectively. Then the collection of regular constraints $\cE(\cdot)$ are computed as follows.
\begin{itemize}
\item Let $G_0=G_C$. We nondeterministically choose $\cE_0(x) = \{(\cT_1, \{(q_0, q_2)\})\}$, $\cE_0(y) = \{(\cT_2, \{(q^\prime_0, q^\prime_0)\})\}$, $\cE_0(z) = \{(\cT_3, \{(q^{\prime\prime}_0, q^{\prime\prime}_0)\})\}$, $\cE_0(y^\prime) = \{(\cT_4, \{(p_0, p_1)\})\}$, $\cE_0(z^\prime) = \{(\cT_5, \{(p^{\prime}_0, p^{\prime}_1)\})\}$.
%
\item Select the vertex $x$ in $G_0$, construct $\cE_1(y)$ and $\cE_1(z)$ as in Example~\ref{exmp-sl}, that is, nondeterministically choose $T_z =\{(q_0, q_0), (q_1,q_2)\}$, let $\cE_1(z) = \{(\cT_3, \{(q^{\prime\prime}_0, q^{\prime\prime}_0)\}), (\cT_1, \{(q_0,q_0), (q_1, q_2)\})\}$ and $\cE_1(y)=\{(\cT_2, \{(q^\prime_0, q^\prime_0)\}), (\cT_{\cT_1, 0, T_z}, \{(q_0,q_2)\})\}$, where $\cT_{\cT_1, 0, T_z}$ is the transition graph of $\cB_{\cA_1, 0, T_z}$ illustrated in Figure~\ref{fig-sl-exmp}. In addition, $\cE_1(x)=\cE_0(x)$, $\cE_1(y')=\cE_0(y')$ and $\cE_1(z')=\cE_0(z')$. Finally, $G_1$ is obtained from $G_0$ by removing the two edges out of $x$.
%
\item Select the vertex $y$ in $G_1$, construct $\cE_2(y')$ and $\cE_2(z')$ as follows: Nondeterministically choose $T_{1,z'} = \{(q^\prime_0, q^\prime_0)\}$  for $\cT_2$ and $T_{2,z'}=\{(q_0,q_1),(q_1,q_2)\}$ for $\cT_{\cT_1, 0, T_z}$, let
$$\cE_2(z')=\left\{(\cT_5, \{(p^{\prime}_0, p^{\prime}_1)\}), (\cT_2, \{(q^\prime_0, q^\prime_0)\}), (\cT_{\cT_1, 0, T_z}, \{(q_0,q_1),(q_1,q_2)\}) \right\}, \text{ and}$$
$$\cE_2(y')=\left\{ (\cT_4, \{(p_0, p_1)\}), (\cT_{\cT_2, 1, T_{1,z'}}, \{(q^\prime_0, q^\prime_0)\}), (\cT_{\cT_{\cT_1, 0, T_z}, 1, T_{2,z'}}, \{(q_0,q_2)\}) \right\},$$
where $\cT_{\cT_2, 1, T_{1,z'}}$ and $\cT_{\cT_{\cT_1, 0, T_z}, 1, T_{2,z'}}$ are shown in Figure~\ref{fig-sl-exmp-nested-2}. In addition, $\cE_2(x)=\cE_1(x)$, $\cE_2(y)=\cE_1(y)$, and $\cE_2(z)=\cE_1(z)$. Finally, $G_2$ is obtained from $G_1$ by removing the two edges out of $y$.
%
\end{itemize}
Since $G_2$ contains no edges, we have $\cE(x)=\cE_2(x)$, similarly for $\cE(y)$, $\cE(z)$, $\cE(y')$, and $\cE(z')$.
For the three source variables $y', z', z$, it is not hard to check that $01$ belongs to the intersection of the regular constraints in $\cE(z')$, $11$ belongs to the intersection of the regular constraints in $\cE(y')$, and $10$ belongs to the intersection of the regular constraints in $\cE(z)$. Then $y$ takes the value $\replaceall(11, 1, 01)=0101 \in \Ll(e_2)$, and $x$ takes the value $\replaceall(0101, 0, 10)=101101 \in \Ll(e_1)$. Therefore, $C$ is satisfiable. \qed
\begin{figure}[htbp]
\begin{center}
\includegraphics[scale=0.7]{single-letter-example-nested.pdf}
\end{center}
\caption{An example for the single-letter case: Multiple $\replaceall$}\label{fig-sl-exmp-nested}
\end{figure}

\begin{figure}[htbp]
\begin{center}
\includegraphics[scale=0.7]{single-letter-example-nested-2.pdf}
\end{center}
\caption{$\cT_{\cT_2, 1, T_{1,z'}}$ and $\cT_{\cT_{\cT_1, 0, T_z}, 1, T_{2,z'}}$}\label{fig-sl-exmp-nested-2}
\end{figure}
\end{example}




\subsubsection{Complexity}

To show our decision procedure works in exponential space, it is sufficient to show that the cardinalities of the sets $\cE(x)$ are exponential w.r.t. the size of $C$.

\begin{proposition}\label{prop-sl-comp}
The cardinalities of $\cE(x)$ for the variables $x$ in $G_C$ are at most exponential in $\dmdidx(G_C)$, the diamond index of $G_C$.
\end{proposition}
Therefore, according to Proposition~\ref{prop-sl-comp}, if the diamond index of $G_C$ is bounded by a constant $c$, then the cardinalities of $\cE(x)$ become \emph{polynomial} in the size of $C$ and we obtain a polynomial space decision procedure. In this case, we conclude that the satisfiability problem is PSPACE-complete.

\begin{proof}[Proof of Proposition~\ref{prop-sl-comp}]
Let $K$ be the maximum of $|\cE_0(x)|$ for $x \in \vars(\varphi)$. 
For each variable $x$ in $G_C$, all the regular constraints in $\cE(x)$ are either from $\cE_0(x)$, or are generated from some regular constraints from $\cE_0(x')$ for the ancestors $x'$ of $x$. Let $x'$ be an ancestor of $x$. Then for each $(\cT, \cP) \in \cE_0(x')$, according to Step I in the decision procedure, by an induction on the maximum length of the paths in from $x'$ to $x$, we can show that the number of elements in $\cE(x)$ that are generated from $(\cT, \cP)$ is at most the number of different paths from $x'$ to $x$. 
%
From Proposition~\ref{prop-di}, we know that there are at most $(|\vars(\varphi)||E_C|)^{O(\dmdidx(G_C))}$ different paths from $x'$ to $x$. Since there are at most $|\vars(\varphi)|$ ancestors of $x$, we deduce that $|\cE(x)| \le |\vars(\varphi)|\ K\ (|\vars(\varphi)||E_C|)^{O(\dmdidx(G_C))}$.
\end{proof}

%%%%%%%%%%%%%%%%%%%%%%%%the original complexity analysis%%%%%%%%%%%%%%%%%%%%%%%%
%%%%%%%%%%%%%%%%%%%%%%%%the original complexity analysis%%%%%%%%%%%%%%%%%%%%%%%%
%%%%%%%%%%%%%%%%%%%%%%%%the original complexity analysis%%%%%%%%%%%%%%%%%%%%%%%%
%
\hide{
\begin{proposition}
The cardinalities of $\cE(x)$ for the variables $x$ in $G_C$ are at most exponential  in the size of $C$ and become polynomial for $\strline_{\sf ss}[\replaceall]$ formulae.
\end{proposition}

\begin{proof}
%The transition graphs $\cT=(Q, \delta)$ and $\cT'=(Q', \delta')$ are said to be \emph{isomorphic} if there is a bijection $\pi: Q \rightarrow Q'$ such that for each $q, q' \in Q$ and $a \in \Sigma$, $(q, a, q') \in \delta$ iff $(\pi(q), a, \pi(q')) \in \delta'$.
%
Let $K$ denote the maximum cardinality of $\cE_0(x)$ for vertices $x$ in $G_C$.
%In addition, for each node $x$ in $G_C$, let $\#^{\sf anc}$
%In addition, let $M$ denote the maximum size (number of states) of the NFA in $\cE_0(x)$ for nodes $x$ in $G_C$.
For each $i$ and each vertex $x$ in $G_C$, let  $\#^{\sf anc}_i(x)$ denote the number of ancestors of $x$ in $G_i$.
%, in addition, let ${\sf Idx}^{\sf arb}_i(x)$ denote the maximum number of join vertices in a path starting from $x$ in $G_i$.

%the number of edges of $G_0 \setminus G_i$ (i.e. the subgraph of $G_0$ comprising all the edges not in $G_i$) that are co-reachable from $x$ in  $G_0 \setminus G_i$.

% and $\#_{\sf lft}(x)$ denote the number of $\rpleft$-edges that are co-reachable from $x$ in $G_C$.

%Then the proposition follows from the following claim, since for each node $x$, $|\cE(x)| \le |\cG(\cE(x))|M^2$.

We first prove the following claim.

\smallskip

\noindent {\bf Claim}. For each $i$ and each vertex $x$ in $G_C$,
$|\cE_i(x)| \le 3^i K$. In addition, if $C \in \strline_{\sf ss}[\replaceall]$, then for each non-source variable $x$ in $G_C$, $|\cE_i(x)| \le (\#^{\sf anc}_0(x)-\#^{\sf anc}_i(x)+1) K$.
%In addition, if $G_C$ is a tree, then $|\cE_i(x)| \le (\#^{\sf anc}_0(x)-\#^{\sf anc}_i(x)+1) K$.

\smallskip

We prove the claim by an induction on $i$.

\smallskip


\noindent {\it Induction base}: $i=0$. Evidently $|\cE_0(x)| \le K = 3^0 K$. Moreover, if $C \in \strline_{\sf ss}[\replaceall]$, then for each non-source variable $x$, $|\cE_0(x)| \le K = (\#^{\sf anc}_0(x)-\#^{\sf anc}_0(x)+1) K$.
% = (\#^{\sf anc}_0(x)-\#^{\sf anc}_0(x)+1) K$.

\smallskip

\noindent {\it Induction step}:
Suppose $i > 0$.

Let $x$ be the vertex without predecessors and with successors in $G_{i-1}$ that is used to construct $G_{i}$. In addition, let $(x, (\rpleft, a), y)$ and $(x, (\rpright, a), z)$ be the two edges out of $x$ in $G_{i-1}$.

Let us first assume $y \neq z$.
Then $|\cE_{i}(z)| \le |\cE_{i-1}(z)| + |\cE_{i-1}(x)|$ and $|\cE_{i}(y)| \le |\cE_{i-1}(y)| + |\cE_{i-1}(x)|$.  By the induction hypothesis, $|\cE_{i-1}(x)| \le  3^{i-1} K$, $|\cE_{i-1}(y)| \le 3^{i-1} K$, and $|\cE_{i-1}(z)| \le 3^{i-1} K$. Therefore, $|\cE_{i}(z)|  \le |\cE_{i-1}(z)| + |\cE_{i-1}(x)| \le 3^{i-1} K + 3^{i-1} K \le 3^i K$. Similarly, $|\cE_{i}(y)| \le 3^i K$.

Next, we assume $y = z$. Then $|\cE_{i}(z)| \le |\cE_{i-1}(z)| + |\cE_{i-1}(x)| + |\cE_{i-1}(x)| \le 3* 3^{i-1} K = 3^i K$.

Let us assume that $C \in \strline_{\sf ss}[\replaceall]$, moreover, either $y$ or $z$ is not a source variable. Then $y \neq z$. Because otherwise, the in-degree of $y=z$ in $G_C$ is more than one, from the fact that $G_C$ is source-sharing, we deduce that $y=z$ has to be a source variable, a contradiction. Let us first assume that $z$ is not a source variable. Then the in-degree of $z$ is one, that is, the edge from $x$ to $z$ is the only incoming edge of $z$ in $G_C$. From this, we deduce that  $\cE_{i-1}(z) = \cE_0(z)$. Therefore, $|\cE_{i}(z)| \le |\cE_{i-1}(z)| + |\cE_{i-1}(x)| \le K + |\cE_{i-1}(x)|$. By the induction hypothesis, $|\cE_{i-1}(x)| \le (\#^{\sf anc}_{0}(x) - \#^{\sf anc}_{i-1}(x)+1)K$. We deduce that $|\cE_{i}(z)| \le K+ (\#^{\sf anc}_{0}(x) - \#^{\sf anc}_{i-1}(x)+1)K$. Moreover, since $\#^{\sf anc}_{0}(z) = \#^{\sf anc}_{0}(x)+1$ and $\#^{\sf anc}_{i-1}(x)=\#^{\sf anc}_{i}(x)= \#^{\sf anc}_{i}(z)=0$, we have $|\cE_{i}(z)| \le K+ (\#^{\sf anc}_{0}(z)-1 - \#^{\sf anc}_{i}(z)+1)K = (\#^{\sf anc}_{0}(z) - \#^{\sf anc}_{i}(z)+1)K$.
Similarly, if $y$ is not a source variable, we have $|\cE_{i}(y)| \le |\cE_{i-1}(y)| + |\cE_{i-1}(x)| = |\cE_{0}(y)| + |\cE_{i-1}(x)| \le K+|\cE_{i-1}(x)| \le K+ (\#^{\sf anc}_{0}(x) - \#^{\sf anc}_{i-1}(x)+1)K \le K+ (\#^{\sf anc}_{0}(y)-1 - \#^{\sf anc}_{i}(y)+1)K = (\#^{\sf anc}_{0}(y) - \#^{\sf anc}_{i}(y)+1)K$.

The proof of the claim is complete.

To complete the proof of the proposition, let $H$ be the maximum length of the paths in $G_C$. From the claim, we deduce that for each non-source variable $x$, $|\cE(x)| \le (H-1)K$. In addition, for each source variable $y$ in $G_C$, suppose that the in-degree of $y$ in $G_C$ is $m$, then $|\cE(y)| \le K + \sum \limits_{x: \mbox{ \small predecessor of } y} |\cE(x)| \le  K+m(H-1)K=(mH-m+1) K$.
%
Therefore, we conclude that if $C \in \strline_{\sf ss}[\replaceall]$, then the cardinalities of $\cE(x)$ become polynomial in the size of $C$.
\end{proof}
}
%%%%%%%%%%%%%%%%%%%%%%%%%%%%%%%%%%%%%%%%%%%%%%%%%%%%%%%%%%%%%%
%%%%%%%%%%%%%%%%%%%%%%%%%%%%%%%%%%%%%%%%%%%%%%%%%%%%%%%%%%%%%%
%%%%%%%%%%%%%%%%%%%%%%%%%%%%%%%%%%%%%%%%%%%%%%%%%%%%%%%%%%%%%%

%\begin{quote}
%\it Let $x$ be a node in the graph without predecessors and $x = \replaceall(y, a, z)$ be a conjunct of $\varphi$ and $\cA_1, \dots, \cA_m$ be a collection of regular constraints for $x$. Then remove the two edges out of $x$, guess $T_{i, z} \subseteq Q_i \times Q_i$ for $i \in [m]$, and add additional regular constraints to $y$ and $z$ as specified above.
%\end{quote}

%For doing this, we need deal with the typical situation that $x = \replaceall(y, a, z)$  and the regular constraints for $x$ are specified by

%We first introduce a concept of dependency graphs.




%!TEX root = popl2018.tex

\section{Decision procedure for $\strline[\replaceall]$: The constant-string case}

In this section, we consider the constant-string special case, that is, given an $\strline[\replaceall]$ formula $C = \varphi \wedge \psi$, it holds that every term of the form $\replaceall(z, e, z')$ in $\varphi$ satisfies that $e=u$ for $u \in \Sigma^+$.

Let us start with the simple case that $C \equiv x = \replaceall(y, u, z) \wedge \bigwedge \limits_{i \in [k]} x \in e_{i}$ such that $|u| \ge 2$.  In addition, for $i \in [k]$, suppose $\cA_i = (Q_i, \delta_i, q_{0, i}, F_i)$ 
%\cA_y = (Q_y, \delta_y, q_{0, y}, F_y), \cA_z = (Q_z, \delta_z, q_{0, z}, F_z)$ 
is the NFA corresponding to $e_i$. 
\begin{enumerate}
\item For each $i \in [k]$, we will guess a set $T_{i, z} \subseteq Q_i \times Q_i$, which intuitively means that in $\cA_i$, for each state pair $(q, q') \in T_{i, z}$, starting from $q$, after reading $z$, the state $q' $ can be reached. The functions $(T_{i, z})_{i \in [k]}$ induce regular constraints $\Ll(\cA_i(q,q'))$ on $z$, where $i \in [k]$ and $(q,q') \in T_{i, z}$.

\item For each $i \in [k]$, we construct an NFA $\cB_{\cA_i, u,  T_{i, z}}$, which specifies some additional regular constraint on $y$. Intuitively, a string $v$ is accepted by $\cB_{\cA_i, u,  T_{i, z}}$ iff either $v \not \in \Sigma^\ast u \Sigma^\ast$ and $v$ is accepted by $\cA_i$, or otherwise, let $v = v'_1 u v'_2 u \dots v'_{k-1} u v'_{k}$ such that $v'_i u' \not \in \Sigma^\ast u \Sigma^\ast$ for each $i \in [k-1]$ and each strict prefix $u'$ of $u$ and $v'_k \not \in \Sigma^\ast u \Sigma^\ast$, then $q_{i,0} \xrightarrow{v'_1} q_1 \xrightarrow{T_{i,z}} q'_1 \xrightarrow{v'_2} q_2 \xrightarrow{T_{i,z}} q'_2 \dots \xrightarrow{v'_{k-1}} q_{k-1} \xrightarrow{T_{i,z}} q'_{k-1} \xrightarrow{v'_k} q_k$ for states $q_1, q'_1, \dots, q_{k-1}, q'_{k-1}, q_k \in Q_i$ with $q_k \in F_{i,z}$. 
\end{enumerate}

In order to construct the NFA $\cB_{\cA_i, u,  T_{i, z}}$, we introduce concepts of window profiles and parsing automata defined below.

Let $u \in \Sigma^+$ and $k=|u| \ge 2$.


%If $u = \sigma$ for $\sigma \in \Sigma$, then $\cA_u=(Q_u, \delta_u, q_{0,u}, F_u)$, where $Q_u = \{(q_0, \bot), (q_0, \top) \}$, $\delta_u = \{(q_0, \bot) \xrightarrow{\sigma} (q_0, \top), (q_0, \bot) \xrightarrow{\Sigma \setminus \sigma} (q_0, \bot), (q_0, \top) \xrightarrow{ \sigma} (q_0, \top), (q_0, \top) \xrightarrow{\Sigma \setminus \sigma} (q_0, \bot)\}$, $q_{0,u} = (q_0, \bot)$, and $F_u = Q_u$.

%In the following, we assume that $|u| = k \ge 2$. Let $u = u_1 \dots u_k$, where each $u_i \in \Sigma$.

%We construct an NFA $\cA_u=(Q_u, \delta_u, q_{0,u}, F_u)$ which over a string $v$, parses $v \in \Sigma^\ast u \Sigma^\ast$ into $v_1 u v_2 u \dots v_l u v_{l+1}$ such that $v_j u_1 \dots u_{k-1} \not \in \Sigma^\ast u \Sigma^\ast$ for each $1 \le j \le l$, and $v_{l+1} \not \in \Sigma^\ast u \Sigma^\ast$. 
%Let $u = u_1\dots u_k$ such that $u_i \in \Sigma$ for each $i \in [k]$.  

\begin{definition}[$k$-window profiles w.r.t. $u$]
A $k$-\emph{window profile $\overrightarrow{W}$ w.r.t. $u$} is an element of $\{\bot,\top\}^{k-1}$. Intuitively, in the position $i$ of a string $v$, $\overrightarrow{W}$ is an abstraction of the substring $v[i-k+2] \dots v[i]$ such that for each $j \in [k-1]$, $\overrightarrow{W}(j) = \top$ iff $v[i-j+1] \dots v[i] = u[1] \dots u[j]$. Let $\wprof_{u, k}$ denote the set of $k$-window profiles w.r.t. $u$. 
%In particular, if $k = 1$, then $\wprof_{u, k} = \emptyset$. 
\end{definition}

%\zhilin{the following fact is noticed by Yan Chen.} 
\begin{proposition}
The number of $k$-window profiles w.r.t. $u$ is polynomial in the length of $u$. 
\end{proposition}
\begin{proof}
The arguments for this fact proceed as follows: For each profile $\overrightarrow{W}$, let $v$ be a string and $i$ be a position of $v$ such that for each $j \in [k-1]$, $\overrightarrow{W}(j) = \top$ iff $v[i-j+1] \dots v[i] = u[1] \dots u[j]$. Define ${\sf idx}_{\overrightarrow{W}}$ as the maximum index $j \in [k-1]$ such that $\overrightarrow{W}(j)=\top$. Then 
\begin{itemize}
	\item for each $j': j < j' < k$, $\overrightarrow{W}(j')=\bot$, 
	\item in addition, since $v[i-{\sf idx}_{\overrightarrow{W}}+1] \dots v[i] = u[1] \dots u[{\sf idx}_{\overrightarrow{W}}]$, the values of $\overrightarrow{W}(1),\dots, \overrightarrow{W}({\sf idx}_{\overrightarrow{W}})$ are completely determined by $u[1] \dots u[{\sf idx}_{\overrightarrow{W}}]$.
\end{itemize}
From the above arguments, we can  conclude that the number of $k$-window profile $\vec{W}$ w.r.t. $u$ is actually at most $k$.
\end{proof}

From $u$, we will construct a parsing automaton $\cA_u=(Q_u, \delta_u, q_{0,u}, F_u)$ which parses a string $v \in \Sigma^\ast u \Sigma^\ast$ into $v_1 u v_2 u \dots v_l u v_{l+1}$ such that $v_j u[1] \dots u[k-1] \not \in \Sigma^\ast u \Sigma^\ast$ for each $1 \le j \le l$, in addition, $v_{l+1} \not \in \Sigma^\ast u \Sigma^\ast$. 
The concept of $k$-window profiles w.r.t. $u$ are used to check that a substring of $v$ is \emph{not} in $\Sigma^\ast u \Sigma^\ast$.
Specifically, the NFA $\cA_u$ is constructed as follows.
\begin{itemize}
	\item  $Q_u =\{q_0\} \cup \{(\search, \overrightarrow{W}) \mid \overrightarrow{W} \in \wprof_{u, k}\} \cup \{(\verify, j, \overrightarrow{W}) \mid j \in [k-1], \overrightarrow{W} \in \wprof_{u,k}\}$, where $q_0$ is a distinguished state whose purpose will become clear later on,  $\search$ and $\verify$ are used to denote whether $\cA_u$ is in the ``search''-mode to search the next occurrence of $u$, or in the ``verify'' mode to verify that the current position is a part of an occurrence of $u$.
	%
	\item $q_{0,u}=q_0$.
	
	\item $\delta_{u}$ comprises the following transitions,
	%guesses over each position, one of the following holds, the substring comprising the next $k$-symbols (including the current one) is $u$ or not.
	\begin{itemize}
		\item $q_0 \xrightarrow{\sigma} (\search, \overrightarrow{W})$, where $\overrightarrow{W}(1)=\top$ iff $\sigma = u[1]$, and for each $i: 2 \le i \le k-1$, $\overrightarrow{W}(i) = \bot$,
		%
		\item for each state $(\search, \overrightarrow{W})$ and $\sigma \in \Sigma$ such that $\overrightarrow{W}(k-1) = \bot$ or $\sigma \neq u[k]$,
		\begin{itemize}
			\item $(\search, \overrightarrow{W}) \xrightarrow{\sigma} (\search, \overrightarrow{W}')$, where $\vec{W}'(1) = \top$ iff $\sigma = u[1]$, and for each $i: 2 \le i \le k-1$, $\vec{W}'(i) =\top$ iff $\overrightarrow{W}({i-1}) = \top$ and $\sigma = u[i]$,
			%
			\item if $\sigma = u[1]$, then $(\search, \overrightarrow{W}) \xrightarrow{\sigma} (\verify, 1, \overrightarrow{W}')$,  where $\overrightarrow{W}'(1)=\top$,  and for each $i: 2 \le i \le k-1$, $\overrightarrow{W}'(i) =\top$ iff $\overrightarrow{W}({i-1}) = \top$ and $\sigma = u[i]$,
			%
		\end{itemize}
		%
		\item for each state $(\verify, i-1, \overrightarrow{W})$ such that
		\begin{itemize}
			\item $2 \le i \le k-1$,
			\item $\overrightarrow{W}(i-1)=\top$, $\sigma = u[i]$, and
			\item either $\overrightarrow{W}(k-1)=\bot$ or $\sigma \neq u[k]$, 
		\end{itemize}
		we have $(\verify, i-1, \overrightarrow{W}) \xrightarrow{\sigma} (\verify, i, \overrightarrow{W}')$, where for each $j: 2 \le j \le k-1$, $\overrightarrow{W}'(j) = \top$ iff $\overrightarrow{W}(j-1)=\top$ and $\sigma = u[j]$, 
		%
		\item for each state $(\verify, k-1, \overrightarrow{W})$ such that $\overrightarrow{W}(k-1)=\top$, we have $(\verify, k-1, \overrightarrow{W}) \xrightarrow{u[k]} q_0$.
		%where $\bot^k$ in $(\search, \bot^k)$ is used to \emph{reinitialise} the $k$-window profile w.r.t. $u$.
		%
	\end{itemize}
	Note that the constraint $\vec{W}(k-1) = \bot$ or $\sigma \neq u[k]$ is used to guarantee that when parsing a string $v$ into $v_1 u v_2 u \dots v_{l} u v_{l+1}$, $v_j u[1] \dots u[k-1] \not \in \Sigma^\ast u \Sigma^\ast$ for each $j \in [l]$, in addition, $v_{l+1} \not \in  \Sigma^\ast u \Sigma^\ast$.
	%
	\item $F_u=\{q_0\} \cup \{(\search, \overrightarrow{W}) \mid \overrightarrow{W} \in \wprof_{u, k} \} $. Note that the states $(\verify, j, \overrightarrow{W})$ are not final states, since when in these states, the verification of the next occurrence of $u$ has not yet been complete.
\end{itemize}
In an accepting run $r$ of $\cA_u$ on a string $v = v_1 u v_2 u \dots v_l u v_{l+1}$, the state sequence in the run is of the form 
$$q_0\ r_1\ q_0\ r_2\ q_0\ \dots\ r_l\ q_0\ r_{l+1}$$ 
such that  for each $j \in [l]$, $r_j \in (Q_{\search})^+ Q_{\verify, 1}  \dots  Q_{\verify, k-1}$, and $r_{l+1} \in (Q_{\search})^+$, where $Q_{\search}  = \{(\search, \overrightarrow{W}) \in \mid \overrightarrow{W} \in \wprof_{u,k}\}$ and $Q_{\verify, i} = \{(\verify, i, \overrightarrow{W}) \mid \overrightarrow{W} \in \wprof_{u,k}\}$ for $i \in [k-1]$. Intuitively, each occurence of $q_0$, except the first one, witnesses the \emph{first} occurrence of $u$ after its previous occurrence or starting from the beginning.

%The parsing automaton $\cA_u$ constructed above is \emph{unambiguous} in the sense that for each string $v \in \Sigma^+$, there is \emph{exactly one accepting run} of $\cA_u$ on $v$.

\begin{example}
	An example for $\cA_u$.
\end{example}

We are ready to present the construction of $\cB_{\cA_i, u,  T_{i, z}}$. $\cB_{\cA_i, u,  T_{i, z}}$ is constructed by the following two-step procedure.
\begin{enumerate}
\item Construct the product of $\cA_i$ and $\cA_u$. Then remove all the states $(q, (\verify, j, \overrightarrow{W}))$ as well as the transitions associated with them.

\item For each pair $(q,q') \in T_{i,z}$ and each sequence of transitions in $\cA_u$ of the form  
$$
\begin{array}{l}
((\search, \overrightarrow{W}), u[1], (\verify, 1, \overrightarrow{W'_1})), ((\verify, 1, \overrightarrow{W'_1}), u[2], 
 (\verify, 2, \overrightarrow{W'_2})), \\
 \hspace{3cm} \dots, ((\verify, k-1, \overrightarrow{W'_{k-1}}), u[k], q_0),
\end{array}
$$ 
add the transitions 
$$
\begin{array}{l}
((q, (\search, \overrightarrow{W})), u[1], (q, (\verify, 1, \overrightarrow{W'_1}))), ((q, (\verify, 1, \overrightarrow{W'_1})), u[2], (q, (\verify, 2, \overrightarrow{W'_2}))), \dots,  \\
((q, (\verify, k-2, \overrightarrow{W'_{k-2}})), u[k-1], (q, (\verify, k-1, \overrightarrow{W'_{k-1}}))), ((q, (\verify, k-1, \overrightarrow{W'_{k-1}})), u[k], (q', q_0)).
\end{array}
$$
\end{enumerate}

%Similarly to the single-letter case, we can define the dependency graph $G_C$. In addition, we can adapt $\dfs(z, z', a, f)$ into a procedure $\dfs(z, z', u, f)$, which integrates the automata $\cA_{u'}$ into the computation of the functions $f_{z', \cA_z}$, where $u,u'$ are constant strings occurring in the edge-labels in $G_C$.

%!TEX root = popl2018.tex

\section{Decision procedure for $\strline[\replaceall]$: The regular-expression case} \label{sec:replaceallre}

Let us consider the case that the second parameter of the $\replaceall$ function may be a regular expression. 

As in previous sections, let us still start with the simple situation that $C \equiv x = \replaceall(y, e_0, z) \wedge x \in e_1 \wedge y \in e_2 \wedge z \in e_3$. For $i=0,1,2,3$, let $\cA_i = (Q_i, \delta_i, q_{0,i}, F_i)$ be the NFA corresponding to $e_i$. 

Let us first assume $\varepsilon \in \Ll(e_0)$. Then according to the semantics, for each string $u = a_1 \cdots a_n$, $\replaceall(u, e_0, v) = v a_1 v \cdots v a_n v$. Under this assumption, we can solve the satisfiability of $C$ as follows: 
\begin{enumerate}
\item Guess a set $T_z \subseteq Q_1 \times Q_1$. 
%
\item Construct $\cB_{\cA_1, \varepsilon, T_z}$ from $\cA_1$ and $T_z$ as follows: For each $(q,q') \in T_z$, add to $\cA_1$ a transition $(q, \varepsilon, q')$. Then transform the resulting NFA into one without $\varepsilon$-transitions (which can be done in polynomial time).
%
\item  Decide the nonemptiness of $\Ll(\cA_2) \cap \Ll(\cB_{\cA_1, \varepsilon, T_z})$ and $\Ll(\cA_3) \cap \bigcap \limits_{(q,q') \in T_z} \Ll(\cA_1(q,q'))$.
\end{enumerate}

Next, let us assume $\varepsilon \not \in \Ll(e_0)$.

To solve the satisfiability of $C$, similar to the constant-string case, we construct a parsing automaton $\cA_{e_0}$ that parses a string $v \in \Sigma^\ast e_0 \Sigma^\ast$ into $v_1 u_1 v_2 u_2 \dots v_l u_l v_{l+1}$ such that 
\begin{itemize}
	\item for each $j \in [l]$, $u_j$ is the leftmost and longest matching of $e_0$ in $(v_1 u_1 \dots v_{j-1} u_{j-1})^{-1} v$,
	%
	%\item $v_j u_j[1] \dots u_j[|u_j|-1] \not \in \Sigma^\ast e \Sigma^\ast$ for each $1 \le j \le l$, in addition, $v_{l+1} \not \in \Sigma^\ast e \Sigma^\ast$,
	\item $v_{l+1} \not \in \Sigma^\ast e_0 \Sigma^\ast$.
\end{itemize}

\zhilin{I stopped here}

%
Intuitively, in order to search for the leftmost and longest matching of $e_0$, 
\begin{itemize}
\item $\cA_{e_0}$ has two modes, ``$\searchleft$'' and ``$\searchlong$'', which intuitively means search for the leftmost and longest matching respectively,
 %
	\item when in the ``$\searchleft$'' mode, $\cA_{e_0}$ starts a new thread of $\cA_0$ in each position and keeps a vector of states of these threads, in addition, it nondeterministically makes a ``leftmost'' guessing, that is, guesses that the current position is the first position of the leftmost matching, if it makes such a guessing, then it enters the ``$\searchlong$'' mode, it runs the thread started in the current position and search for the longest matching, moreover, it continues running all the threads that were started before to make sure that the final states will not be reached (thus, the guessing is valid),
	%
	\item when in the ``$\searchlong$'' mode, $\cA_{e_0}$ runs a thread to search for the longest matching, if the thread enters a final state, then $\cA_{e_0}$ nondeterministically makes a ``longest'' guessing, that is, guesses that the current position is the last position of the leftmost and longest matching, if it makes such a guessing, then it resets the states and starts a new round of leftmost and longest matching,
	%
%	\item when a thread $i$ enters a final state, a matching of $e$ is found, $\cB_e$ nondeterministically guesses whether this matching is the leftmost matching or the longest matching, 
	%
%	\item if $\cB_e$ makes a ``leftmost and longest'' guessing, then $\cB_e$ forgets all the other threads that were started later than the thread $i$, and continues running the thread $i$ and all the threads that were started earlier than the thread $i$ to make sure that final states will not be reached and the ``leftmost and longest'' guessing is correct,
%
%	\item if $\cB_e$ makes a ``leftmost and non-longest'' guessing, then $\cB_e$ forgets all the other threads that were started later than the thread $i$, and continues running all the threads that were started earlier than the thread $i$ to make sure that final states will not be reached and the ``leftmost'' guessing is correct, in addition, it continues running the thread $i$ and searching for the longest matching,
%
%	\item if $\cB_e$ makes a ``non-leftmost'' guessing, then $\cB_e$ forgets the thread $i$ and all the other threads that were started later than the thread $i$, and continues running all the threads that were started earlier than the thread $i$ and searching for the leftmost matching,
	%
	\item moreover, in order to keep the length of the vectors of states of threads \emph{bounded}, the following trick is applied: for two threads starting at the position $i$ and $j$ respectively such that $i < j$, if the current states of the two threads are the same, then the thread $j$ is removed.
\end{itemize}
%
Formally, 
\begin{itemize}
	\item the state set $Q'_e$ of $\cB_e$ comprises 
	\begin{itemize}
		\item the tuples $((q_{0,e}), \searchleft, S)$ such that $S \subseteq Q_e$,
		%
		\item $(\rho, \searchleft, S)$ such that  $\rho$ is a nonempty vector of \emph{pairwise distinct} states of $\cA_e$, $\rho \neq (q_{0,e})$, and $S \subseteq Q_e \setminus F_e$, 
		%
		% \item the tuples $(q_{0,e}, \longest, S)$ such that  $S \subseteq Q_e$,
		%
		\item the tuples $(q, \searchlong, S)$ such that $q \in Q_e$ and $S \subseteq Q_e \setminus F_e$;
	\end{itemize}
	%
	\item $q'_{0,e}= ((q_{0,e}), \searchleft, \emptyset)$,
	%
	\item $F'_{e}$ comprises the states of the form $(-, \searchleft, -) \in Q'_e$,
	%
	\item $\delta'_e$ comprises the following tuples: 
	\begin{itemize}
		%\item $((q_{0,e}, \leftmost, S), a, ((\delta_e(q_{0,e},a), q_{0,e}), \leftmost, \delta_e(S,a)))$, where $\delta_e(S,a) = \{\delta_e(q,a) \mid q \in S \}$,
		%
		\item suppose $\rho = q_1 \dots q_m$ (where $q_m = q_{0,e}$),  $a \in \Sigma$ and $\delta_e(S,a) \cap F_e = \emptyset$, then 
		$$((\rho, \searchleft, S), a, (\red(\delta_e(\rho,a))q_{0,e}, \searchleft, \delta_e(S,a))) \in \delta'_e,$$ 
		%
		\item suppose $\rho = q_1 \dots q_m$ (where $q_m = q_{0,e}$),  $a \in \Sigma$, and $\delta_e(S,a) \cap F_e = \emptyset$, then 
		$$((\rho, \searchleft, S), a, (\delta_e(q_m, a), \searchlong, \delta_e(S,a) \cup \{\delta_e(q_j, a) \mid j \in [m-1]\})) \in \delta'_e,$$ 
		%
		\item suppose $\rho = q_1 \dots q_m$ (where $q_m = q_{0,e}$),  $a \in \Sigma$, $\delta_e(q_m, a) \in F_e$, and $\delta_e(S,a) \cap F_e = \emptyset$, then 
		$$((\rho, \searchleft, S), a, ((q_{0,e}), \searchleft, \delta_e(S,a) \cup \{\delta_e(q_j, a) \mid j \in [m]\})) \in \delta'_e,$$ 
		%
%%%%%%%%%%%%%%%%%%%%%%%%%%%%%%%%%%%%%%%%%%
%%%%%%%%%%%%%%%%%%%%%%%%%%%%%%%%%%%%%%%%%%
\hide{
		\item suppose $\rho = q_1 \dots q_m$,  $a \in \Sigma$, $\delta_e(\rho, a) = \delta_e(q_1,a) \dots \delta_e(q_m, a)$ contains \emph{no} states from $F_e$, $\delta_e(S,a) = \{\delta_e(q,a) \mid q \in S \}$, and $\delta_e(S,a) \cap F_e = \emptyset$, then 
		$$((\rho, \searchleft, S), a, (\red(\delta_e(\rho,a))q_{0,e}, \searchleft, \delta_e(S,a))) \in \delta'_e,$$ 
		%
		\item {\bf ``leftmost and longest'' guessing}: \\
		suppose $\rho = q_1 \dots q_m$, $a \in \Sigma$, $\delta_e(\rho, a) = \delta_e(q_1,a) \dots \delta_e(q_m, a)$, $\delta_e(\rho,a)$ contains at least one state from $F_e$, $i \in [m]$ is the smallest index such that $\delta_e(q_i, a) \in F_e$, and $\delta_e(S,a) \cap F_e = \emptyset$, then 
		$$((\rho, \searchleft, S), a, ((q_{0,e}), \searchleft, \delta_e(S,a) \cup \{\delta_e(q_j, a) \mid 1\le j \le i\})) \in  \delta'_e,$$ 
		%
%		intuitively, $\cB_e$ makes a ``leftmost and longest'' guessing and continues running all the threads of indices no greater than $i$,
		\item {\bf ``leftmost and non-longest'' guessing}: \\
		suppose $\rho = q_1 \dots q_m$, $a \in \Sigma$, $\delta_e(\rho, a) = \delta_e(q_1,a) \dots \delta_e(q_m, a)$ contains at least one state from $F_e$, $i \in [m]$ is the smallest index such that $\delta_e(q_i, a) \in F_e$,   and $\delta_e(S,a) \cap F_e = \emptyset$, then 		
		$$((\rho, \searchleft, S), a, (\delta_e(q_i, a), \searchlong, \delta_e(S,a) \cup \{\delta_e(q_j, a) \mid 1 \le j \le i-1\})) \in  \delta'_e,$$ 
		%
		\item {\bf ``non-leftmost'' guessing}: \\
		suppose $\rho = q_1 \dots q_m$, $a \in \Sigma$, $\delta_e(\rho, a) = \delta_e(q_1,a) \dots \delta_e(q_m, a)$ contains at least one state from $F_e$, $i \in [m]$ is the smallest index such that $\delta_e(q_i, a) \in F_e$, $i > 1$,  and $\delta_e(S,a) \cap F_e = \emptyset$, then 		
		$$((\rho, \searchleft, S), a, (\red(\delta_e(q_1,a) \dots \delta_e(q_{i-1},a)), \searchleft, \delta_e(S,a)) \in  \delta'_e,$$ 
%
}
%%%%%%%%%%%%%%%%%%%%%%%%%%%%%%%%%%%%%%%%%%
%%%%%%%%%%%%%%%%%%%%%%%%%%%%%%%%%%%%%%%%%%
		\item suppose $\delta_e(S,a) \cap F_e = \emptyset$, then 
		$$((q, \searchlong, S), a, (\delta_e(q,a), \searchlong, \delta_e(S,a)) \in \delta'_e,$$
		%
		\item suppose $\delta_e(q,a) \in F_e$ and $\delta_e(S,a) \cap F_e = \emptyset$, then 
		$$((q, \searchlong, S), a, ((q_{0,e}), \searchleft, \delta_e(S,a) \cup \{\delta_e(q,a)\}) \in \delta'_e.$$
	\end{itemize}
\end{itemize}

\begin{example}
$\cA_{e_0}$
\end{example}


Similarly to the constant-string case, the main technical difficulty is to construct $\cB_{\cA_i, e,  T_{i, z}}$. The NFA $\cB_{\cA_i, u,  T_{i, z}}$ is constructed by the following two-step procedure.
\begin{enumerate}
\item Construct the product of $\cA_i$ and $\cB_e$. Then remove all the states $(q, (-, \searchlong, -))$ as well as the transitions associated with them, in addition, remove all the transitions entering the states $(q, ((q_{0,e}), \searchleft,-))$.

\item For each pair $(q,q') \in T_{i,z}$, do the following,
\begin{itemize}
\item for each transition
$$((\rho, \searchleft, S), a, (\delta_e(q_m, a), \searchlong, \delta_e(S,a) \cup \{\delta_e(q_j, a) \mid j \in [m-1]\})) \in \delta'_e,$$
add a transition
$$((q, (\rho, \searchleft, S)), a, (q, (\delta_e(q_m, a), \searchlong, \delta_e(S,a) \cup \{\delta_e(q_j, a) \mid j \in [m-1]\}))),$$
%
\item for each transition 
$$((q'', \searchlong, S), a, (\delta_e(q'', a), \searchlong, \delta_e(S,a))) \in \delta'_e,$$  
add a transition 
$$((q, (q'', \searchlong, S)), a, (q, (\delta_e(q'', a), \searchlong, \delta_e(S,a)))),$$
%
\item for each transition
$$((q'', \searchlong, S), a, ((q_{0,e}), \searchleft, \delta_e(S,a) \cup \{\delta_e(q'',a)\}) \in \delta'_e,$$
add a transition
$$((q, (q'', \searchlong, S)), a, (q', ((q_{0,e}), \searchleft, \delta_e(S,a) \cup \{\delta_e(q,a)\}))),$$
%
\item for each transition
$$((\rho, \searchleft, S), a, ((q_{0,e}), \searchleft, \delta_e(S,a) \cup \{\delta_e(q_j, a) \mid j \in [m]\})) \in \delta'_e,$$
add a transition
$$((q, (\rho, \searchleft, S)), a, ((q', (q_{0,e}), \searchleft, \delta_e(S,a) \cup \{\delta_e(q_j, a) \mid j \in [m]\}))).$$
\end{itemize}
\end{enumerate}

\begin{example}
$\cB_{\cA_1, e_0,  T_{z}}$
\end{example}

The more general case that there are multiple $\replaceall(\cdots)$ terms.

%\subsection{A decision procedure for $\strline[\replaceall]$}




%!TEX root = popl2018.tex

\section{Undecidable extensions}\label{sec-ext}

In this section, we consider the language $\strline[\replaceall]$ extended with either integer constraints, character constraints, or $\indexof$ constraints and show that each such an extension leads to undecidability. 
%\mat{Should $\concat$ be removed?}\zhilin{agree, since the undecidability result is stronger if stated for $\strline[\replaceall]$}
We will use variables of, in additional to the type $\str$, the Integer data type $\intnum$. The type $\str$ consists of the string variables as in the previous sections. A variable of type $\intnum$, usually referred to as an \emph{integer variable}, ranges over the set $\Nat$ of all natural numbers. Recall that, in previous sections, we have used $x, y, z, \cdots$ to denote the variables of $\str$ type.  Hereafter we typically use $\mathfrak{l}, \mathfrak{m}, \mathfrak{n}, \cdots$ to denote the variables of $\intnum$. The
choice of omitting negative integers is for simplicity. Our
results can be easily extended to the case where $\intnum$ includes negative integers.

We begin by defining the kinds of constraints we will use to extend $\strline[\concat,\replaceall]$.
First, we describe integer constraints, which express constraints on the length or number of occurrences of symbols in words. 


\begin{definition}[Integer constraints] \label{def:intconst} 
	An atomic integer constraint over $\Sigma$ is an expression of the form
	\[a_1t_1+\cdots+a_nt_n\leq d\]
where $a_1, \cdots, a_n,d\in \mathbb{Z}$ are constant integers (represented in binary), and each \emph{term} $t_i$ is either 
	\begin{enumerate}
		\item an integer variable $\mathfrak{n}$;
		\item $|x|$ where $x$ is a  string variable; or 
		\item $|x|_a$ where $x$ is string variable and $a\in \Sigma$ is a constant letter.
	\end{enumerate}
Here, $|x|$ and $|x|_a$ denote the length of $x$ and the number of occurrences of $a$ in $x$, respectively. 

An \emph{integer constraint} over $\Sigma$ is a Boolean combination of atomic integer constraints over $\Sigma$.
\end{definition}

Character constraints, on the other hand, allow to compare symbols from different strings. The formal definitions are given as follows. 

\begin{definition}[Character constraints]
	An \emph{atomic character constraint} over $\Sigma$ is an equation of the form $x[t_1]=y[t_2]$ where 
	\begin{itemize}
		\item $x$ and $y$ are either a string variable or a constant string in $\Sigma^*$, and 
		\item $t_1$ and $t_2$ are either integer variables or constant positive integers.
	\end{itemize} 
Here, the interpretation of $x[t_1]$ is the $t_1$-th letter of $x$. In case that $x$ does not have the $t_1$-th letter \emph{or} $y$ does not have the $t_2$-th letter, the constraint $x[t_1] = y[t_2]$ is false by convention.  
%Similarly for $y[t_2]$.
%\mat{What if $x$ doesn't have a $t_1$th letter?}\zhilin{how about this ?}
	
A \emph{character constraint} over $\Sigma$ is a Boolean combination of atomic character constraints over $\Sigma$. 
\end{definition}

We also consider the constraints involving the $\indexof$ function.

\begin{definition}[$\indexof$ Constraints]
An atomic $\indexof$ constraint over $\Sigma$ is a formula of the form $t\ \mathfrak{o}\ \indexof(s_1, s_2)$, where 
\begin{itemize}
\item $t$ is an integer variable, or a positive integer (recall that here we assume that the first position of a string is $1$), or the value $0$ (denoting that there is no occurrence of $s_1$ in $s_2$), 
\item $\mathfrak{o} \in \{\ge, \le\}$, and
%
\item  $s_1,s_2$ are either string variables or constant strings. 
\end{itemize}
We consider the \emph{first-occurrence} semantics of $\indexof$.  More specifically, $t \ge \indexof(s_1, s_2)$ holds if $t$ is no less than the first position in $s_2$ where $s_1$ occurs, similarly for $t \le \indexof(s_1, s_2)$.
%\mat{What if $s_1$ does not appear in $s_2$?}\zhilin{if $s_1$ does not appear in $s_2$, then $\indexof(s_1, s_2)=0$, as defined above.}

An $\indexof$ constraint over $\Sigma$ is a Boolean combination of atomic $\indexof$ constraints over $\Sigma$.
\end{definition}

%There are two natural semantics of $\indexof$, viz., the \emph{first-occurrence} semantics and the \emph{anywhere} semantics.  More specifically, $t \ge \indexof(s_1, s_2)$ holds 
%\begin{itemize}
%	\item under the \emph{first-occurrence} semantics, if $t$ is no less than the first position in $s_2$ where $s_1$ occurs;
%	
%	\item under the \emph{anywhere} semantics, if $t$ is no less than \emph{any} position in $s_2$ where $s_1$ occurs.
%\end{itemize}  


%One reason of introducing character constraints is, apart from the use of the JavaScript string method chatAt (which is used rather frequently in JavaScript according to the benchmark \cite{}), they can also be used to define $\indexof(w,x)$ for $w\in \Sigma^*$, which is the most standard usage of IndexOf method in practice. There are in general two natural semantics, viz., the \emph{first-occurrence} semantics and the \emph{anywhere} semantics. We write $u=\indexof(w,x)$ where $u$ is an integer variable, which holds
%\begin{itemize}
%	\item under  the \emph{first-occurrence} semantics, if $u$ is the first position in $x$ where $w$ occurs;
%	
%	\item the \emph{anywhere} semantics, if $u$ is \emph{any} position in $x$ where $w$ occurs;
%\end{itemize}  
 


\subsection{Undecidability of the extensions}

We will show that the extension of $\strline[\replaceall]$ with integer constraints entails undecidability, by a reduction from (a variant of) the Hilbert's 10th problem, which is well-known to be undecidable \cite{Mat93}. 
For space reasons, all proofs appear in Appendix~\ref{sec:ext-undec-proofs}.
Intuitively, we want to find a solution to $f(x_1, \cdots, x_n)=g(x_1, \cdots, x_n)$ in the natural numbers, where $f$ and $g$ are polynomials with positive coefficients.
We can use the length of string variables over a unary alphabet $\{a\}$ to represent integer variables, addition can be performed with concatenation, and multiplication of $x$ and $y$ with $\replaceall(x, a, y)$.
The integer constraint $|x| = |y|$ asserts the equality of $f$ and $g$.
\mat{Is this useful?}

%: Given two polynomials (aka Diophantine equations) $f(x_1, \cdots, x_n)$ and $g(x_1,\cdots, x_n)$ with positive integral coefficients over the same set of variables $x_1, \cdots, x_n$, decide whether $f(x_1, \cdots, x_n)=g(x_1, \cdots, x_n)$ has a solution in natural numbers. It is well-known that Hilbert's 10th problem is undecidable \cite{Mat93}.

%Recall the Hilbert 10th problem, which is, for any given Diophantine equation (a polynomial equation with integer coefficients and a finite number of unknowns), to decide whether the equation has a solution with all unknowns taking integer values. It is easy to observe that given two polynomials with positive integral coefficients over the same set of variables $x_1, \cdots, x_n$, it is \emph{undecidable} to check whether $f(x_1, \cdots, x_n)=g(x_1, \cdots, x_n)$ has a solution in natural numbers. 

\begin{theorem}\label{thm-ext-int}
	For the extension of $\strline[\replaceall]$ with \emph{integer constraints}, the satisfiability problem is undecidable, even if only a single integer constraint $|x| = |y|$ is used.
\end{theorem}


%
%Notice, in the above proof, besides the $\strline[\replaceall]$ formula, only a \emph{single simple} atomic integer constraint $|y_f| = |y_g|$ is used. Therefore, the proof  shows that the extension of $\strline[\replaceall]$ with only one integer constraint of the form $|x| = |y|$ entails undecidability.
%On the other hand, by utilising a further result on Diophantine equations, we will show that for the extension of $\strline[\replaceall]$ with integer constraints, even if the $\strline[\replaceall]$ formulae are simple, in the sense that their dependency graphs are of depth at most one, the satisfiability problem is still undecidable (note that no restrictions are put on the integer constraints in this case).

Notice that the extension of $\strline[\replaceall]$ with only one integer constraint of the form $|x| = |y|$ entails undecidability. By utilising a further result on Diophantine equations, we will show that for the extension of $\strline[\replaceall]$ with integer constraints, even if the $\strline[\replaceall]$ formulae are simple, in the sense that their dependency graphs are of depth at most one, the satisfiability problem is still undecidable (note that no restrictions are put on the integer constraints in this case).

% The above proof essentially establishes a link between Diophantine equations and the extension of $\strline[\replaceall]$ with integer constraints over a unary alphabet. 
%
%With a further result from Hilbert 10'th problem, we can strengthen the undecidability results shown that satisfiability of even very simple string constraints (e.g., the $\replaceall$ function is unnested) will be undecidable in conjunction with length constraints. 
 


%Therefore, we get the following undecidability result.

\begin{theorem}\label{thm-ext-int-strong}
	For the extension of $\strline[\replaceall]$ with integer constraints, even if $\strline[\replaceall]$ formulae are restricted to those whose dependency graphs are of depth at most one, the satisfiability problem is still undecidable.
\end{theorem}

By essentially encoding $|x|=|y|$ by character or $\indexof$ Constraints, we show:

\begin{proposition}
	For the extension of $\strline[\replaceall]$ with either the character constraints or the $\indexof$ constraints, the satisfiability problem is undecidable. 
\end{proposition}



%!TEX root = popl2018.tex

\section{Related work}

 
In this section, we discuss some related work. 

\paragraph{Word equations} Makanin's and Plandowski's  on the decidability
and complexity of satisfiability for word equations, i.e., a conjunction of equations of $v=w$, where $v, w$ are concatenation of string constants and variable. 

In addition, it is still a long-standing open problem whether word equations with length constraints is decidable, though it is known that letter-counting (i.e., counting the number of occurrences of 0s and 1s separately) yields undeciability.  

\subsection*{Heuristics and string solver implementation}

There is a large amount of work in the past years on developing practical string solvers
\tl{plz help to put reference here}. String solvers that support concatenations and the replace-all operator are available. \cite{BTV09, TCJ14, YABI14,TCJ16}


\cite{BTV09} We discuss the problem of path feasibility for programs manipulating strings using a collection of standard string library functions. We prove results on the complexity of this problem, including its undecidability in the general case and decidability of some special cases. In the context of test-case generation, we are interested in an efficient finite model finding method for string constraints. To this end we develop a two-tier finite model finding procedure. First, an integer abstraction of string constraints are passed to an SMT (Satisfiability Modulo Theories) solver. The abstraction is either unsatisfiable, or the solver produces a model that fixes lengths of enough strings to reduce the entire problem to be finite domain. The resulting fixed-length string constraints are then solved in a second phase. We implemented the procedure in a symbolic execution framework, report on the encouraging results and discuss directions for improving the method further.


\cite{YABI14} %Verifying string manipulating programs is a crucial problem in computer security. String operations are used extensively within web applications to manipulate user input, and their erroneous use is the most common cause of security vulnerabilities in web applications. We 
The authors present an automata-based approach for symbolic analysis of string manipulating programs. 

For the string operations, the authors focus on two common ones: concatenation and replacement. For the latter, two semantics are considered, i.e., longest match and first match. The replacement operation provides an over-approximation of such more restricted replace semantics. 

\cite{SMV12} translating regular expression matching into transducer. 

They use deterministic finite automata (DFAs) to represent possible values of string variables. Using forward reachability analysis we compute an over-approximation of all possible values that string variables can take at each program point. They also implemented Stranger, an automata-based string analysis tool, with experiments. 

%Intersecting these with a given attack pattern yields the potential attack strings if the program is vulnerable. Based on the presented techniques, we have implemented Stranger, an automata-based string analysis tool for detecting string-related security vulnerabilities in PHP applications. We evaluated Stranger on several open-source Web applications including one with 350,000+ lines of code. Stranger is able to detect known/unknown vulnerabilities, and, after inserting proper sanitization routines, prove the absence of vulnerabilities with respect to given attack patterns.

%Motivated by the vulnerability analysis of web programs which
work on string inputs, we 
\cite{TCJ14}, the authors present S3 (which stands for Symbolic String Solver), a new symbolic string solver.
%Our solver employs a new algorithm for a constraint language that
%is expressive enough for widespread applicability. Specifically, our
The constraint language covers all the main string operations including the replace-all function. The authors 
provided algorithms which make use of a symbolic representation so that membership in a set defined by a regular expression can be encoded as string equations. 

%To amplify this point, let us now state some statistics from a comprehensive
%study of practical JavaScript applications [28]. Constraints
%arising from the applications have an average (per benchmark
%query) of 63 JavaScript string operations, while the remaining
%are boolean, logical and arithmetic constraints. The largest fraction
%are for operations like indexOf, length (78%). A significant
%fraction of the operations, including substring (5%), replace
%(8%), and split, match (1%). Of the match, split and
%replace operations, 31% are based on regular expressions. Operations
%such as replace and split give rise to new strings
%from the original ones, thereby giving rise to constraints involving
%multiple string variables.



%. The algorithm first makes use of a symbolic representation
%so that membership in a set defined by a regular expression
%can be encoded as string equations. Secondly, there is a constraint based
%generation of instances from these symbolic expressions so
%that the total number of instances can be limited. 
%
%We evaluate S3 on a well-known set of practical benchmarks, demonstrating both
%its robustness (more definitive answers) and its efficiency (about 20
%times faster) against the state-of-the-art.



%Progressive Reasoning over Recursively-Defined Strings
\cite{TCJ16} 
Trinh et al considered %the problem of reasoning over an expressive constraint language for unbounded strings. 
%In particular, they considered 
recursively defined string functions, a very expressive way to define functions manipulating strings. This includes a recursive definition of the replace-all function considered in this paper\footnote{\cite{TCJ16} used the notation \textbf{replace}}. The authors argue that ``the difficulty comes from ``recursively defined" functions such as replace, making state-of-the-art algorithms non-terminating." They proposed a progressive search algorithm, %to not only mitigate the problem of non-terminating reasoning but also guide the search towards a “minimal solution” when the input formula is in fact satisfiable. We have 
implemented within the state-of-the-art Z3 framework, with experimental evaluations. The algorithm is genetic and  applicable to all recursively defined string functions, but it is doomed to be incomplete as reasoning about unbounded strings defined recursively is in general an undecidable problem.   

%Importantly, we have enabled conflict clause learning for string theory so that our solver can be used effectively in the setting of program verification. Finally, our experimental evaluation shows leadership in a large benchmark suite, and a first deployment for another benchmark suite which requires reasoning about string formulas of a class that has not been solved before.

%A Decision Procedure for String Logic with Equations, Regular Membership and Length Constraints 
\cite{L16}, the author considered the satisfiability problem for string logic with equations, regular membership and Presburger constraints over length functions. %The difficulty comes from multiple occurrences of string variables making state-of-the-art algorithms non-terminating. Our main contribution is to 
He show that the satisfiability problem in a fragment where no string variable occurs more than twice in an equation is decidable. In particular, he proposed a semi-decision procedure for arbitrary string formulae with word equations, regular membership and length functions, and showed that the algorithm always terminates for the aforementioned decidable fragment, with a complexity analysis. 
This fragment is largely incomparable to ours, as no replace-all functions are addressed therein. (However, it allows more expressive Presburger constraints over length functions.)
%The essence of our procedure is an algorithm to enumerate an equivalent set of solvable disjuncts for the formula. We further show that the algorithm always terminates for the aforementioned decidable fragment. Finally, we provide a complexity analysis of our decision procedure to prove that it runs, in the worst case, in factorial time.

The focus of our work is on the fundamental issue of decidability, and this is complementary to the work. Our result may be considered a completeness guarantee for existing string solver. 

The $\replaceall$ function is a special yet expressive string transformation function, aka. string transducer. With this viewpoint, 
the $\replaceall$ function is also related to two recently introduced transducer models: streaming string transducers \cite{AC10} and symbolic transducers \cite{symbolic-transducer}. 

A streaming string transducer is a finite state machine where  a finite set of string variables are used to store the intermediate results for output. The $\replaceall(x, e, y)$ term can be modelled by an extension of streaming string transducers with parameters, that is, a streaming string transducer which reads an input string (interpreted as the value of $x$), uses $y$ as a free string variable which is presumed to be read-only, and updates a string variable $z$, which stores the computation result, by a string term which may involve $y$. Nevertheless, to the best of our knowledge, this extension of streaming string transducers has not been investigated so far. 

Symbolic transducers is an extension of Mealy machine to infinite alphabets by using a variable $cur$ to represent the symbol in the current position, and replacing the input and output letters in transitions with unary predicates $\varphi(cur)$ and terms involving $cur$ respectively. It is an interesting future work to consider an extension of the $\replaceall$ function to sequences of numerical values, by following the idea of symbolic transducers. For instance, one may consider the term $\replaceall(x, cur \equiv 0 \bmod 2, y)$ which replaces every even number in $x$ with $y$.

In \cite{DHK16}, Daca et al. considered an extension of the quantifier-free theory of integer arrays, called array folds logic, to express counting. The main feature of the logic is the \emph{fold} terms, borrowed from the folding concept in functional languages. Intuitively, a fold term applies a function to every element of the array to compute an output. If strings are taken as arrays where the elements are from a finite domain (the alphabet), the $\replaceall$ function can be seen as a fold term on arrays. Nevertheless, the $\replaceall$ function goes beyond the fold terms in \cite{DHK16}, since it outputs a string (an array), instead of an integer. Therefore, the results in \cite{DHK16} cannot be applied here.


%!TEX root = popl2018.tex

\section{Conclusion}

In this paper, we have investigated extensively the decidability boundary of the satisfiability problem for the string constraints involving the $\replaceall$ function and regular constraints. The $\replaceall$ functions are considered in their most general form, that is, $\replaceall(x, e, y)$, where $x,y$ can be string variables, and $e$ is either a string constant/variable, or a regular expression. As the satisfiability problem is undecidable in general, we focused on the straight-line fragment. We showed that while it remains to be undecidable if the second parameter of the $\replaceall$ function is a variable, it becomes decidable, more precisely in EXPSPACE, if the second parameter is a regular expression. The decision procedure was obtained by an automata-theoretic construction, which is modular and amenable to implementations. In addition, we proved that the decision procedure is in fact PSPACE-complete for several special cases that are meaningful in practice. Finally, we show that extending the decidable straight-line fragment with any of the integer constraints, character constraints, and constraints involving the $\indexof$ function, leads to the undecidability immediately. 
Our work clarified important fundamental issues surrounding the $\replaceall$ functions in string constraint solving and provided a novel decision procedure which paved a way to a string solver that is able to fully support the $\replaceall$ function. This would be the most direct future work. 

% Acknowledgments
\begin{acks}
 %% acks environment is optional
 %% contents suppressed with 'anonymous'
 %% Commands \grantsponsor{<sponsorID>}{<name>}{<url>} and
 %% \grantnum[<url>]{<sponsorID>}{<number>} should be used to
 %% acknowledge financial support and will be used by metadata
 %% extraction tools.
 % This material is based upon work supported by the
 % \grantsponsor{GS100000001}{National Science
 %   Foundation}{http://dx.doi.org/10.13039/100000001} under Grant
 % No.~\grantnum{GS100000001}{nnnnnnn} and Grant
 % No.~\grantnum{GS100000001}{mmmmmmm}.  Any opinions, findings, and
 % conclusions or recommendations expressed in this material are those
 % of the author and do not necessarily reflect the views of the
 % National Science Foundation.
    %This work was supported by the
T. Chen is supported by the
\grantsponsor{}{Australian Research Council }{} under Grant No.~\grantnum{}{DP160101652}
and the
\grantsponsor{}{Engineering and Physical Sciences Research Council}{} under Grant No.~\grantnum{}{EP/P00430X/1}.
    %
    M. Hague is supported by the
    \grantsponsor{GS501100000266}{Engineering and Physical Sciences Research Council}
                 {http://dx.doi.org/10.13039/501100000266}
    under Grant No.~\grantnum{GS501100000266}{EP/K009907/1}.
    A.~Lin is supported by the European Research Council (ERC) under the European
    Union's Horizon 2020 research and innovation programme (grant agreement no
    759969).
    %
    Z. Wu is supported by the
    \grantsponsor{}{National Natural Science Foundation of China}{}
    under Grant No.~\grantnum{}{61472474} and Grant No. ~\grantnum{}{61572478},
    \grantsponsor{}{the INRIA-CAS joint research project ``Verification, Interaction, and Proofs''}{},
    \mat{Add your sponsors here!}\zhilin{I suggest ordering the grants according to the order of the authors}
\end{acks}


%% Bibliography
%\bibliography{bibfile}

\newpage

\bibliography{string}

\shortlong{}{

\newpage

% Appendix
%!TEX root = popl2018.tex

\appendix

\begin{center}
{\huge Supplementary Material} \\
{\large ``What's Decidable About String Constraints with ReplaceAll Function?''} 
\end{center}

\bigskip

We provide below proofs and examples that were omitted from the main text due to space constraints.

%%%%%%%%%%%%%%%%%%%%%%%%%%%%%%%%%%%%%%%%%%%%%%%%%%%%%%%%%
%%%%%%%%%%%%%%%%%%%%%%%%%%%%%%%%%%%%%%%%%%%%%%%%%%%%%%%%%
\hide{
\noindent {\it Proposition~\ref{prop-num-path}}.
{\it Let $G=(V,E)$ be a DAG such that the out-degree of each vertex is at most two. Then there are $n^{O(\dmdidx(G))}$ different paths  in $G$.
}

\begin{proof}
\end{proof}
}
%%%%%%%%%%%%%%%%%%%%%%%%%%%%%%%%%%%%%%%%%%%%%%%%%%%%%%%%%
%%%%%%%%%%%%%%%%%%%%%%%%%%%%%%%%%%%%%%%%%%%%%%%%%%%%%%%%%
\def\refpropundpat{\ref{prop-und-pat-var}}

\section{Proof of Proposition~\protect\refpropundpat}
\label{sec:prop-und-pat-var-proof}

We recall Proposition~\ref{prop-und-pat-var} and then give its proof.

\medskip

\noindent \textsc{Proposition}~\ref{prop-und-pat-var}
{\em The satisfiability problem of $\strline[\replaceall]$ is undecidable, if the second parameters of the $\replaceall$ terms are allowed to be variables.
}

\begin{proof}
	We reduce from the Post Correspondence Problem (PCP). Recall that the input of the problem consists of two finite lists $\alpha_{1},\ldots ,\alpha_{N}$ and $\beta_1,\ldots ,\beta_N$ of nonempty strings over $\Sigma$. A solution to this problem is a sequence of indices $(i_{k})_{1\leq k\leq K}$ with $ K\geq 1$ and $ 1\leq i_{k}\leq N$ for all $k$, such that
	$	\alpha _{{i_{1}}}\ldots \alpha _{{i_{K}}}=\beta _{{i_{1}}}\ldots \beta _{{i_{K}}}.
	$
	The PCP problem is to decide whether such a solution exists or not.
	
	Without loss of generality, suppose $\Sigma \cap [N] = \emptyset$ and $\$ \not \in \Sigma \cup [N]$. Let $\Sigma' = \Sigma \cup [N] \cup \{\$\}$. We will construct an $\strline[\replaceall]$ formula $C$ over $\Sigma'$ such that the PCP instance has a solution iff $C$ is satisfiable. To this end, the formula $C$ utilises the capability that the second parameter of the $\replaceall$ terms may be variables.
	
	Let $x_1, \cdots, x_N, y_1, \cdots, y_N, z$ be mutually distinct string variables. Then the formula $C = \varphi \wedge \psi$, where 
	%
	$$
	\begin{array}{l c l}
	\varphi & = & \bigwedge \limits_{i \in [N]} (x_i = \replaceall(x_{i-1}, i, \alpha_i) \wedge y_i = \replaceall(y_{i-1}, i, \beta_i)) \wedge  z = \replaceall(x_N, y_N, \$), \\
	\psi & = & x_0 \in (1 + \cdots + N)^+ \wedge z \in \$.
	\end{array}
	$$
	
	It is not hard to see that $\varphi$ is a straight-line relational constraint, thus $C$ is an $\strline[\replaceall]$ formula. Note that in $\replaceall(x_N, y_N, \$)$, the second parameter is a variable. We show that $C$ is satisfiable iff the PCP instance has a solution: $C$ is satisfiable iff there is a string $i_1 \cdots i_K \in \Ll((1 + \cdots + N)^+)$ such that when $x_0$ is assigned with $i_1 \cdots i_K$, the value of $z$ is $\$$.
	Since $z = \replaceall(x_N, y_N, \$)$ and $x_N, y_N \in \Sigma^+$, we know that $z$ is $\$$ iff the values of $x_N$ and $y_N$ are the same. Therefore, $C$ is satisfiable iff there is a string $i_1 \cdots i_K \in \Ll((1 + \cdots + N)^+)$ such that when $x_0$ is assigned with $i_1 \cdots i_K$, the values of $x_N$ and $y_N$ are the same. Therefore, $C$ is satisfiable iff there is a sequence of indices $i_1 \cdots i_K$ such that $\alpha_{i_1} \cdots \alpha_{i_K} = \beta_{i_1} \cdots \beta_{i_K}$, that is, the PCP instance has a solution.
	%
	%
	%Suppose the PCP instance has a solution. Then there is a sequence of indices $i_1 \cdots i_K$ such that $\alpha _{{i_{1}}}\ldots \alpha _{{i_{K}}}=\beta _{{i_{1}}}\ldots \beta _{{i_{K}}}$. Let $x_0$ be $i_1 \cdots i_K$. Then from the construction of $C$, we know that the values of $x_N$ and $y_N$ are $\alpha _{{i_{1}}}\ldots \alpha _{{i_{K}}}$ and  $\beta _{{i_{1}}}\ldots \beta _{{i_{K}}}$ respectively. Thus the values of $x_N$ and $y_N$ are the same. Therefore, the value of $z=\replaceall(x_N, y_N, \$)$ is $\$$. The formula $C$ is satisfiable. 
	%
	%
	%Since $x_0 \in (1 + \cdots + N)^+$, we know that $x_N, y_N$ can only be strings over the alphabet $\Sigma$. Therefore, $z \in \$$ iff $x_N = y_N$.
	%
	%	
	%	We then introduce, for $i=1,\cdots, N$, 
	%	$x_{i+1}=\replaceall(x_0, \alpha_i, i)$ and $y_{i+1}=\replaceall(y_0, \beta_i, i)$, 
	%	$x_0'=\replaceall(x_0, \sharp, \epsilon)$ and $y_0'=\replaceall(y_0, \sharp, \epsilon)$
	%	
	%	$x_{N+1}=y_{N+1}$, $x_0'=y'_0$
	%	
	%	
	%	with regular constraints $x_0\in \sharp((\sum_{i=1}^N\alpha_i)\sharp)^*$ and $y_0\in \sharp((\sum_{i=1}^N\beta_i)\sharp)^*$,
	%	
	%	where $z=z'$ can be encoded by 
	%		$z''=\replaceall(z, z', \$)$ and $z''\in \$$. 
\end{proof}

\def\refsecreplaceallsl{\ref{sec:replaceallsl}}

\section{Section~\protect\refsecreplaceallsl: The Correctness of the decision procedure}
\label{sec:dp-sl-correctness}

We argue that the procedure in Section~\ref{sec:dp-sl-general} is correct.
Note that Proposition~\ref{prop-sat-sl-case} removed a single $\replaceall(-,-,-)$ to obtain only regular constraints.
Each step of our decision procedure effectively eliminates a $\replaceall(-,-,-)$.
Similar to Proposition~\ref{prop-sat-sl-case}, each step maintains the satisfiability from the preceding step.

In more detail, from each $G_i$ we can define a constraint $C_i$. This constraint is a conjunction of the following atomic constraints.
\begin{itemize}
\item For each variable $x$ such that $(x, (\rpleft, a), y)$ and $(x, (\rpright,a), z)$ are the edges in $G_i$, we assert in $C_i$ that $x = \replaceall(y, a, z)$.
\item In addition, for each variable $x$ such that $\cE_i(x)$ is not empty, moreover, \emph{either $x$ is a source variable in $G_C$ (not $G_i$) or there are (incoming or outgoing) edges connected to $x$ in $G_i$}, let $e_i(x)$ be the regular expression equivalent to the conjunction of all constraints in $\cE_i(x)$ (Note that the conjunction of multiple regular expressions still defines a regular language). We assert in $C_i$ that $x \in e_i(x)$. Note that if $x$ is not a source variable in $G_C$ and there are no edges connected to $x$ in $G_i$, then the regular constraints in $\cE_i(x)$ are not included into $C_i$.
\end{itemize}

%This constraint is a conjunction of the following clauses.
%For each variable $x$ such that $\cE_i(x)$ is not empty, we let $e_i(x)$ be the regular expression equivalent to the conjunction of all constraints in $\cE_i(x)$.
%Since this is the conjunction of multiple regular expressions (NFAs), it is regular.
%We assert in $C_i$ that $x \in e_i(x)$.

It is immediate that $C_0$ is equivalent to $C$.
We require the following proposition, which gives us the correctness of the decision procedure by induction.
Note that the final $C_i$ when exiting the loop will be a conjunction of regular constraints on the source variables.

\begin{proposition}
    For each $i$,  let the $\rpleft$-edge and the $\rpright$-edge from $x$ to $y$ and $z$ respectively be the two edges removed from $G_i$ to construct $G_{i+1}$. Then $C_i$ is satisfiable iff there are sets $T_{j, z}$ such that $C_{i+1}$ is satisfiable.
\end{proposition}

We can see the above proposition by observing that, in each step, $C_i$ is of the form
\[
    x = \replaceall(y, a, z) \wedge x \in e_i(x) \wedge y \in e_i(y) \wedge z \in e_i(z) \wedge C'
\]
where $C'$ does not contain $x$, and $C_{i+1}$ is of the form
\[
    y \in e_{i+1}(y) \wedge z \in e_{i+1}(z) \wedge C' \ .
\]
Note that $C'$ remains unchanged since only the two edges leaving $x$ are removed from $G_i$ and $\cE_{i+1}(x') = \cE_i(x')$ for all $x'$ distinct from $x$, $y$, and $z$.
First assume $y \neq z$.
Supposing $C_i$ is satisfiable, an argument similar to that of Proposition~\ref{prop-sat-sl-case} shows that there are sets $T_{j,z}$ such that the same values of $y$ and $z$ also satisfy $e_{i+1}(y)$ and $e_{i+1}(z)$.
Since $C'$ is unchanged, all $x'$ distinct from $x$, $y$, and $z$ can also keep the same value.
Thus, $C_{i+1}$ is also satisfiable.
In the other direction, suppose that there are sets $T_{j, z}$ such that $C_{i+1}$ is satisfiable. Take a satisfying assignment to $C_{i+1}$.
From the assignment to $y$ and $z$ we obtain as in Proposition~\ref{prop-sat-sl-case} an assignment to $x$ that satisfies $\replaceall(y, a, z) \wedge x \in e_i(x)$.
Furthermore, the assignments for $y$ and $z$ also satisfy $e_i(y)$ and $e_i(z)$ since $\cE_i(y)$ and $\cE_i(z)$ are subsets of $\cE_{i+1}(y)$ and $\cE_{i+1}(z)$.
Finally, since $C'$ is unchanged, the assignments to all other variables also transfer, giving us a satisfying assignment to $C_i$ as required.
In the case where $y = z$, the arguments proceed analogously to the case $y \neq z$.

\def\prodauttitle{$\cA_1 \times \cA_u$}
\def\defutitle{$u = 010$}
\section{The product automaton \protect\prodauttitle for \protect\defutitle}

In Figure~\ref{fig-cs-exmp} we give the product automaton $\cA_1 \times \cA_u$ for $u = 010$.
This is a straightforward product construction, but may be useful for reference when understanding Figure~\ref{fig-cs-exmp-2} which shows the automaton $\cB_{\cA_1, u, T_z}$ which is derived from the product.

\begin{figure}[htbp]
\begin{center}
\includegraphics[scale=0.65]{constant-string-example.pdf}
\end{center}
\caption{The NFA $\cA_1 \times \cA_u$ for $u = 010$}\label{fig-cs-exmp}
\end{figure}
%

\def\refsecreplaceallcs{\ref{sec:replaceallcs}}
\section{Complexity analysis in Section~\protect\refsecreplaceallcs}
\label{sec:cs-complexity-full}

We provide a more detailed analysis of the complexity of the algorithm for the constant string case, described in Section~\ref{sec:replaceallcs}.
A summary of this argument already appears in Section~\ref{sec:replaceallcs}.

When constructing $G_{i+1}$ from $G_i$, suppose the two edges from $x$ to $y$ and $z$ respectively are currently removed, let the labels of the two edges be $({\sf l}, u)$ and $({\sf r}, u)$ respectively, then each element $(\cT, \cP)$ of $\cE_i(x)$ may be transformed into an element $(\cT', \cP')$ of $\cE_{i+1}(y)$ such that $|\cT'| = O(|u||\cT|)$, meanwhile, it may also be transformed into an element $(\cT'', \cP'')$ of $\cE_{i+1}(z)$ such that $\cT''$ has the same state space as $\cT$. Thus, for each source variable $x$, $\cE(x)$ contains at most exponentially many elements, and each of them may have a state space of at most exponential size. For instance, for a path from $x'$ to $x$ where the constant strings $u_1,\cdots, u_n$ occur in the labels of edges, an element $(\cT,\cP) \in \cE_0(x')$ may induce an element $(\cT', \cP')$ of $\cE(x)$ such that $|\cT'| \le |\cT| |u_1| \cdots |u_n|$, which is exponential in the worst case. 
%
To solve the nonemptiness problem of the intersection of all these regular constraints, the exponential space is sufficient. Consequently, in this case, we still obtain an EXPSPACE upper-bound. 

Let us now consider the special situation that the $\rpleft$-length of $G_C$ is bounded by a constant $c$.
Since $\dmdidx(G_C) \le \lftlen(G_C)$, we know that $\dmdidx(G_C)$ is also bounded by $c$. Therefore, according to Proposition~\ref{prop-di}, there are at most polynomially different paths in $G_C$, we deduce that for each source variable $x$, $\cE(x)$ contains at most polynomially many elements. In addition, since the number of $\rpleft$-edges in each path is bounded by $c$, during the execution of the decision procedure, the number of times when $(\cT, \cP)$ of $\cE_i(x)$ may be transformed into an element $(\cT', \cP')$ of $\cE_{i+1}(y)$ such that $|\cT'| = O(|u||\cT|)$ is bounded by $c$.
Therefore, for each source variable $x$ and each element $(\cT'', \cP'')$ in $\cE(x)$,  $|\cT''|$ is at most polynomial in the size of $C$. We then conclude that for each source variable $x$, $\cE(x)$ corresponds to the intersection of polynomially many regular constraints such that each of them has a state space of polynomial size. Therefore, the nonemptiness of the intersection of all the regular constraints in $\cE(x)$ can be solved in polynomial space. In this situation, we obtain a PSPACE upper-bound.


\def\refsecreplaceallre{\ref{sec:replaceallre}}
\section{Complexity analysis in Section~\protect\refsecreplaceallre}
\label{sec:re-complexity-full}

We provide a more detailed analysis of the complexity of the algorithm for the regular-expression case, described in Section~\ref{sec:replaceallre}.
A summary of this argument already appears in Section~\ref{sec:replaceallre}.

In each step of the reduction, suppose the two edges out of $x$ are currently removed, let the two edges be from $x$ to $y$ and $z$ and labeled by $({\sf l}, e)$ and $({\sf r}, e)$ respectively, then each element of $(\cT, \cP)$ of $\cE_i(x)$ may be transformed into an element $(\cT',\cP')$ of $\cE_{i+1}(y)$ such that $|\cT'| = |\cT| \cdot 2^{O(p(|e|))}$, meanwhile, it may also be transformed into an element $(\cT'',\cP'')$ of $\cE_{i+1}(y)$ such that $\cT''$ has the same state space as $\cT$. Thus, after the reduction, for each source variable $x$, $\cE(x)$ may contain exponentially many elements, and each of them may have a state space of exponential size, more precisely, if we start from a vertex $x$ without predecessors, with an element $(\cT,\cP)$ in $\cE_0(x)$, and go to a source variable $y$ through a path where $k$ edges have been traversed and removed, let $e_1,\cdots, e_k$ be the regular expressions occurring in the labels of these edges, then the resulting element in $\cE(y)$ has a state space of size $|\cT| \cdot 2^{O(p(|e_1|))} \cdot 2^{O(p(|e_2|))} \cdot \cdots \cdot 2^{O(p(|e_k|))}$ in the worst case. To solve the nonemptiness problem of the intersection of all these regular constraints, the exponential space is sufficient. Consequently, for the most general case of regular expressions, we still obtain an EXPSPACE upper-bound. 

On the other hand, for the situation that the $\rpleft$-length of $G_C$ is at most one, we wan to show that the algorithm runs in polynomial space. Suppose the $\rpleft$-length of $G_C$ is at most one. Then the diamond index of $G_C$ is at most one as well. According to Proposition~\ref{prop-di}, there are only polynomially many paths in $G_C$. Nevertheless, for each source variable $x$, $\cE(x)$ may contain an element $(\cT,\cP)$ such that $|\cT|$ is exponential. Since $|\cP|$ may be exponential, $(\cT,\cP)$ may correspond to the intersection of exponentially many regular constraints. However, we can show that $|\cP|$ is at most polynomial, as a result of the fact that the $\rpleft$-length of $G_C$ is at most one. The arguments proceed as follows: Suppose two edges from $x$ to $y, z$ respectively are removed, and an element $(\cT', \cP')$ of $\cE_{i+1}(y)$ such that $|\cT'|$ is exponential and $|\cP'|$ is polynomial, is generated from an element of $(\cT, \cP)$ of $\cE_i(x)$. Then $y$ must be a source variable in $G_C$. Otherwise, there is an $\rpleft$-edge out of $y$ and the $\rpleft$-length of $G_C$ is at least two, a contradiction. Therefore, $y$ is a source variable in $G_C$, $(\cT', \cP')$  will not be used to generate the regular constraints for the other variables. In other words, $y$ is a source variable in $G_C$, and $(\cT', \cP') \in \cE(y)$ with $|\cP'|$ polynomial. We then conclude that for each source variable $x$, $|\cE(x)|$  is at most polynomial in the size of $C$ and for each element $(\cT, \cP) \in \cE(x)$, $|\cP|$ is polynomial in the size of $C$. Therefore, for each source variable $x$,  $\cE(x)$ corresponds to the intersection of polynomially many regular constraints, where each of them has a state space at most exponential size. To solve the nonemptiness of the intersection of these regular constraints, the polynomial space is sufficient. We obtain a PSPACE upper-bound for the situation that the $\rpleft$-length of $G_C$ is at most one.


\def\refsecreplaceallre{\ref{sec:replaceallre}}
\section{Examples in Section~\protect\refsecreplaceallre}

Due to space constraints, we did not provide examples of the decision procedure for the regular-expression case.
We provide some examples here.


\begin{example}\label{exmp-pa-re}
	Let $e_0 = 0^*0 1(1^* + 0^*)$. Then $\cA_{0}$ and $\cA_{e_0}$ are illustrated in Figure~\ref{fig-pa-re}, where ${\sf sleft}$ and ${\sf slong}$ are the abbreviations of $\searchleft$ and $\searchlong$ respectively. Let us use the state $(\{q_{0,1}\}\{q_{0,0}\}, {\sf sleft}, \emptyset)$ to illustrate the construction. Since $\big(\delta_0(\{q_{0,1}\}, 0) \cup \delta_0(\{q_{0,0}\}, 0)\big) \cap F_0 = \{q_{0,1}\} \cap F_0 = \emptyset$, $\delta_0(\emptyset, 0) \cap F_0 = \emptyset$, and $\red(\delta_0(\{q_{0,1}\}, 0) \delta_0(\{q_{0,0}\}, 0))=\{q_{0,1}\}$, we deduce that the transition 
\[
    ((\{q_{0,1}\}\{q_{0,0}\}, {\sf sleft}, \emptyset), 0, (\{q_{0,1}\} \{q_{0,0}\}, {\sf sleft}, \emptyset)) \in \delta_{e_0} \ .
\]
On the other hand, it is impossible to go from the state $(\{q_{0,1}\}\{q_{0,0}\}, {\sf sleft}, \emptyset)$ to the ``$\searchlong$'' mode. This is due to the fact that $\delta_0(\{q_{0,0}\}, 0)=\{q_{0,1}\} \subseteq \delta_0(\{q_{0,1}\},0)=\{q_{0,1}\}$. In addition, there are no $1$-transitions out of $(\{q_{0,1}\}\{q_{0,0}\}, {\sf sleft}, \emptyset)$. This is due to the fact that $\delta_0(\{q_{0,1}\}, 1) \cap F_0 = \{q_{0,2}, q_{0,3}\} \cap F_0 \neq \emptyset$.
	%
	\begin{figure}[htbp]
		\begin{center}
			\includegraphics[scale=0.7]{regular-expression-example.pdf}
		\end{center}
		\caption{The NFA $\cA_0$ and $\cA_{e_0}$ for $e_0 = 0^*0 1(1^* + 0^*)$}\label{fig-pa-re}
	\end{figure} 
\end{example}

\begin{example}
	Let $C \equiv x = \replaceall(y, e_0, z) \wedge x \in e_1 \wedge y \in e_2 \wedge z \in e_3$, where $e_1,e_2,e_3$ are as in Example~\ref{exmp-sl} (cf. Figure~\ref{fig-sl-exmp}) and $e_0$ is as in Example~\ref{exmp-pa-re} (cf. Figure~\ref{fig-pa-re}). Suppose $T_z = \{(q_0, q_0), (q_1, q_2)\}$. Then the NFA $\cB_{\cA_1, e_0, T_z}$ is as illustrated in Figure~\ref{fig-re-exmp}, where the thick edges denote the added transitions. Let us use the state $(q_1, (\{q_{0,0}\}, \searchleft, \emptyset))$ to exemplify the construction. The transition $((q_1, (\{q_{0,0}\}, \searchleft, \emptyset)), 1, (q_2, (\{q_{0,0}\}, \searchleft, \emptyset)))$ is  in $\cA_1 \times \cA_{e_0}$. Since $\delta_0(q_{0,0}, 1) \cap F_0 = \emptyset$, this transition is not removed and is thus in $\cB_{\cA_1, e_0, T_z}$. On the other hand, since there are no $0$-transitions out of $q_1$ in $\cA_1$, there are no $0$-transitions from $(q_1, (\{q_{0,0}\}, \searchleft, \emptyset))$ to some state from $Q_{\searchleft}$ in $\cB_{\cA_1, e_0, T_z}$. 
	Moreover, because $((\{q_{0,0}\}, \searchleft, \emptyset), 0, (\{q_{0,1}\}, \searchlong, \emptyset)) \in \delta_{e_0}$ and $(q_1, q_2) \in T_z$, the transition $((q_1, (\{q_{0,0}\}, \searchleft, \emptyset)), 0, (q_1, (\{q_{0,1}\}, \searchlong, \emptyset)))$ is added. 
	One may also note that there are no 0-transitions from $(q_2, (\{q_{0,0}\}, \searchleft, \emptyset))$ to the state $(q_2, (\{q_{0,1}\}, \searchlong, \emptyset))$, because there are no pairs $(q2,-) \in T_z$.
	It is not hard to see that $010101 \in \Ll(\cA_2) \cap \Ll(\cB_{\cA_1, e_0, T_z})$. In addition, $10 \in \Ll(\cA_3) \cap \Ll(\cA_1(q_0,q_0)) \cap \Ll(\cA_1(q_1,q_2))$. Let $y$ be $010101$ and $z$ be $10$. Then $x$ takes the value $\replaceall(010101, e_0, 10)=10 \cdot \replaceall(101, e_0, 10)=10110$, which is accepted by $\cA_1$. Therefore, $C$ is satisfiable.
	\begin{figure}[htbp]
		\begin{center}
			\includegraphics[scale=0.68]{regular-expression-example-2.pdf}
		\end{center}
		\caption{The NFA $\cB_{\cA_1, e_0, T_z}$}\label{fig-re-exmp}
	\end{figure} 
\end{example}



\def\refsecext{\ref{sec-ext}}
\section{Undecidability Proofs for Section~\protect\refsecext}
\label{sec:ext-undec-proofs}

We provide the proofs of the theorems and propositions in Section~\ref{sec-ext} which show the undecidability of various extensions of our string constraints.

\subsection{Proof of Theorem~\ref{thm-ext-int}}

We begin with the first Theorem, which is recalled below.

\medskip

\noindent \textsc{Proposition}~\ref{thm-ext-int}
{\em    
    For the extension of $\strline[\replaceall]$ with \emph{integer constraints}, the satisfiability problem is undecidable, even if only a single integer constraint $|x| = |y|$ is used.
}


\begin{proof}
	The basic idea of the reduction is to simulate the two polynomials $f(x_1,\cdots, x_n)$ and $g(x_1,\cdots, x_n)$, where $x_1,\cdots,x_n$ range over the set of natural numbers, with two $\strline[\concat,\replaceall]$ formulae $C_f, C_g$ over a unary alphabet $\{a\}$, with the output string variables $y_f, y_g$ respectively, and simulate the equality $f(x_1,\cdots, x_n) = g(x_1,\cdots, x_n)$ with the integer constraint $|y_f|=|y_g|$ (which is equivalent to $y_f = y_g$, since $y_f, y_g$ represent strings over the unary alphabet $\{a\}$). 
	
	A polynomial $f(x_1,\cdots, x_n)$ or $g(x_1,\cdots, x_n)$ where $x_1, \cdots, x_n$ range over the set of natural numbers, can be simulated by an $\strline[\concat,\replaceall]$ formula over an unary alphabet $\{a\}$ as follows: The natural numbers are represented by the strings over the alphabet $\{a\}$. A string variable is introduced for each subexpression of $f(x_1,\cdots, x_n)$. The numerical addition operator $+$ is simulated by the string operation $\concat$ 
	%\mat{$\concat$ is not part of $\strline[\replaceall]$, can it be simulated when the string alphabet is unary, or do we need two extra characters?}\zhilin{changed to $\strline[\concat,\replaceall]$.}
	and the multiplication operator $*$ is simulated by $\replaceall$. Since it is easy to figure out how the simulation proceeds, we will only use an example to illustrate it and omit the details here. Let us consider $f(x_1,x_2) = x_1^2 + 2 x_1 x_2 + 5$. By abusing the notation, we also use $x_1,x_2$ as string variables in the simulation. We will introduce a string variable for each subexpression in $f(x_1,x_2)$, namely the variables $y_{x_1^2}, y_{x_1x_2}, y_{2x_1x_2}, y_{x_1^2+2x_1x_2}, y_{f(x_1,x_2)}$. Then $f(x_1,x_2)$ is simulated by the $\strline[\concat,\replaceall]$ formula
	\[
	\begin{array} {l c l }
	C_f & \equiv & y_{x_1^2} = \replaceall(x_1,a, x_1)\ \wedge y_{x_1x_2} = \replaceall(x_1, a, x_2)\ \wedge \\
	& & y_{2x_1x_2} = \replaceall(aa, a, y_{x_1x_2})\ \wedge y_{x_1^2+2x_1x_2} = y_{x_1^2} \concat y_{2x_1x_2}\ \wedge  \\
	& & y_{f(x_1,x_2)}=y_{x_1^2+2x_1x_2} \concat a a a a a\ \wedge x_1 \in a^*\ \wedge x_2 \in a^*.
	\end{array}
	\]
	Then according to Proposition~\ref{prop-concat}, $C_f, C_g$ can be turned into equivalent $\strline[\replaceall]$ formula $C'_f, C'_g$ by introducing fresh letters.
	%\mat{But we may have to give up the unary alphabet?}\zhilin{yes, you are  right, it is fine.}
	
	Since $C'_f$ and $C'_g$ share only source variables $x_1,\cdots, x_n$, we know that $C'_f \wedge C'_g$ is still an $\strline[\replaceall]$ formula.
	From the construction of $C'_f, C'_g$, it is evident that for every pair of polynomials $f(x_1,\cdots, x_n)$ and $g(x_1,\cdots, x_n)$, $f(x_1,\cdots, x_n) = g(x_1,\cdots, x_n)$ has a solution in natural numbers iff $C'_f \wedge C'_g \wedge |y_f| = |y_g|$ is satisfiable. The proof is complete.
	%
	%%%%%%%%%%%%%%%%%%%%%%%%%%%%%%%%%%%%%%%%%%%%%%%%%%%%%%%%%%%
	%%%%%%%%%%%%%%%%%%%%%%%%%%%%%%%%%%%%%%%%%%%%%%%%%%%%%%%%%%%
	\hide{
		We shall reduce from the aforementioned version of the Hilbert tenth problem. For any polynomial with positive integral  $f(x_1, \cdots, x_n)$ where each coefficient is a positive, we can construct a (division-free) arithmetic circuit (AC) is a directed  acyclic graph with nodes labelled with constants from $\mathbb{Z}$, or with some indeterminates $X_1, \cdots, X_m$, or with the operators $+, -, *$. The nodes labelled with constants are called constant nodes, while those labelled with indeterminates are called input nodes. Both constant and input nodes do not have incoming edges. Internal nodes are those labelled with $+,-,*$. Output node is the one which does not have out-going edges. Without loss of generality we assume that each internal node has in-degree 2, and there is only one output node. Each node in the circuit represents a multivariate polynomial $\mathbb{Z}[X_1, \cdots, X_m]$. Vice verse, each polynomial $f\in \mathbb{Z}[X_1, \cdots, X_m]$ can be represented as an AC, and, if the polynomial has only positive (integral) coefficients, the corresponding AC does not contain nodes labelled by $-$ or negative constants.  
		
		We observe that, given an AC, one can construct an SL[$\concat, \replaceall$] formula over the alphabet $\Sigma=\{a\}$ as follows. Each node $n$ of the AC is associated with a string variable $x_n$. As a result, each input node of the AC labelled by $X_i$ (i.e., the indeterminate) corresponds to a  source variable.   
		\begin{itemize}
			\item For each internal node $n$ labelled by $+$, suppose that $n$ has two children nodes $n_l$ and $n_r$, we introduce a string constraint $x_n= x_{n_l}\concat x_{n_l}$.  
			
			\item For each internal node $n$ labelled by $*$, suppose that $n$ has two children nodes $n_l$ and $n_r$, we introduce a string constraint $x_n= \replaceall(x_{n_l}, a, x_{n_l})$.  		
		\end{itemize}
		Furthermore, we introduce, for each node $n$ labelled by a constant $c$, a regular constraint $x_n=a^c$. 
		
		It is straightforward to verify, according to the semantics of SL[$\concat, \replaceall$], that:
		\begin{itemize}
			\item for relational constraint $x_n= x_{n_l}\concat x_{n_l}$, $|x_n|= |x_{n_l}|+|x_{n_l}|$; 
			\item for relational constraint $x_n= \replaceall(x_{n_l}, a, x_{n_l})$,  $|x_n|= |x_{n_l}|\cdot |x_{n_l}|$; and 
			\item for regular $x_n=a^c$, $|x_n|=c$. 
		\end{itemize}
		
		It follows that for each polynomial $f(x_1, \cdots, x_m)$ with positive integral coefficients, we can construct a straight-line string constraint $\varphi_{f}\wedge\psi_g$ over $\Sigma=\{a\}$ with $y_f$ as the output variant and $y_1, \cdots, y_n$ as source variables such that
		$f(c_1, \cdots, c_m)=|y|$ and, for each $1\leq i\leq m$, $|y_i|= c_i$ (i.e., $y_i=a^{c_i}$).  
		
		Consequently, when given two polynomials $f(x_1, \cdots, x_m)$ and $g(x_1, \cdots, x_m)$, we have straight-line string constraints $\varphi_{f}\wedge \varphi_{g}\wedge \psi_{f}\wedge \psi_g$ with two distinguished two variables  $y_f$ and $y_g$ such that  
		\[\exists x_1, \cdots, x_m. f(x_1, \cdots, x_m)=g(x_1, \cdots, x_m)\mbox{ iff } |y_f|=|y_g|\wedge \varphi_{f}\wedge \varphi_{g}\wedge \psi_{f}\wedge \psi_g\mbox{ is satisfiable} \]
		
		Finally, note that any  SL[$\concat, \replaceall$] constraints can be transformed into SL[$\replaceall$] constraints, we obtain a reduction from the Hilbert's 10th problem to the satisfiability problem of  SL[$\replaceall$] with length constraints, which entail that the latter problem is undecidable. The proof is completed. 
	}
	%%%%%%%%%%%%%%%%%%%%%%%%%%%%%%%%%%%%%%%%%%%%%%%%%%%%%%%%%%%
	%%%%%%%%%%%%%%%%%%%%%%%%%%%%%%%%%%%%%%%%%%%%%%%%%%%%%%%%%%%
\end{proof}

\subsection{Undecidability of Depth-1 Dependency Graph}

We recall the undecidability of a depth-1 dependency graph before providing the proof below.

\medskip

\noindent\textsc{Theorem}\ref{thm-ext-int-strong}
{\em
	For the extension of $\strline[\replaceall]$ with integer constraints, even if $\strline[\replaceall]$ formulae are restricted to those whose dependency graphs are of depth at most one, the satisfiability problem is still undecidable.
}

\medskip

A \emph{linear polynomial} (resp.\ quadratic polynomial) is a polynomial with degree at most one (resp.\ with degree at most two) where each coefficient is an integer. %of the form $a_0 + a_1x_1 + \cdots + a_n x_n$ (resp. a polynomial with degree at most two) where each coefficient $a_i\in \mathbb{Z}$  for $0 \leq i \leq n$. A quadratic polynomial

\begin{theorem}[\cite{ID04}]\label{thm-quad-eq}
	%	There exists some (fixed) $k$ such that no algorithm can solve Diophantine systems in the following form
	%	\[y_1F_1=G_1, t_1H_1=I_1, \cdots, t_kF_k = G_k, t_kH_k = I_k,\] 
	%
	%	where $F_i, G_i, H_i, I_i$ for $1\leq i\leq k$ are nonnegative linear polynomials over natural number variables  $s_1, \cdots, s_m$.
	The following problem is undecidable: Determine whether a system of equations of the following form has a solution in natural numbers, 
	\[
	\begin{array} {l l }
	A_i = B_i, & i =1, \cdots, k,\\
	y_iF_i=G_i \wedge y_i H_i = I_i, & i =1, \cdots, m, 
	\end{array}
	\] 
	%
	where $A_i, B_i, F_i, G_i$ are linear polynomials on the variables $x_1,\cdots, x_n$ (Note that each variable $y_i$ occurs in exactly two quadratic equations).
\end{theorem}

We can get a reduction from the problem in Theorem~\ref{thm-quad-eq} to the satisfiability of the extension of $\strline[\replaceall]$ with integer constraints as follows: For each monomial $y_i x_j$ in the quadratic polynomials, we use an $\strline[\replaceall]$ formula $z_{y_i x_j} = \replaceall(y_i, a, x_j)$ to simulate $y_i x_j$, where $z_{y_i x_j}$ are freshly introduced string variables. Since each equation $y_iF_i=G_i$ or $y_i H_i = I_i$ can be seen as a linear combination of the terms $y_i x_j$ and $x_j$ for $i \in [m]$ and $j \in [n]$, we can replace each variable $x_j$ with $|x_j|$, and each term $y_ix_j$ with $|z_{y_i x_j}|$,  thus transform them into the (linear) integer constraints $F'_i = G'_i$ or $H'_i = I'_i$. Similarly, after replacing each variable $x_j$ with $|x_j|$, we transform each equation $A_i= B_i$ into an integer constraint $A'_i = B'_i$. Therefore, we get a formula 
$$
\begin{array}{l c l }
\bigwedge \limits_{i \in [m], j \in [n]} z_{y_i x_j} = \replaceall(y_i, a, x_j) \wedge \bigwedge \limits_{i \in [m]} y_i \in a^*\ \wedge  \bigwedge \limits_{j \in [n]} x_j \in a^* \  \wedge\\
\hspace{2cm} \bigwedge \limits_{i \in [k]} A'_i = B'_i \wedge \bigwedge \limits_{i \in [m]} (F'_i = G'_i \wedge H'_i = I'_i),
\end{array}
$$
where the dependency graph of the $\strline[\replaceall]$ subformula is of depth at most one.

%%%%%%%%%%%%%%%%%%%%%%%%%%%%%%%%%%%%%%%%%%%%%%%%%
%%%%%%%%%%%%%%%%%%%%%%%%%%%%%%%%%%%%%%%%%%%%%%%%%
\hide{
	From this class of quadratic Diophantine equations, we can introduce string variables $x_1, \cdots, x_k$ and $y_1, \cdots, y_m$, together with relational string constraints 
	\[z_{i,j}=\replaceall(x_i, a, y_j)\]
	for $1\leq i\leq k$ and $1\leq j\leq m$. Note that, for each $i$,  $t_i F_i=G_i$ can be written as
	\begin{equation} \label{eq:dio}
	t_i\cdot \left(a_0+\sum_{j=1}^s a_j s_j\right) =  b_0+\sum_{j=1}^s b_j s_j
	\end{equation}
	where $a$'s and $b$'s are all natural numbers. Moreover, \eqref{eq:dio} holds iff 
	\[a_0\cdot |y_i|+ \sum_{j=1}^s a_j |z_{i,j}| =  b_0+ \sum_{j=1}^s b_j |x_j| \] 
	which is an integer constraint defined in Definition~\ref{def:intconst}. This entails that
}
%%%%%%%%%%%%%%%%%%%%%%%%%%%%%%%%%%%%%%%%%%%%%%%%%
%%%%%%%%%%%%%%%%%%%%%%%%%%%%%%%%%%%%%%%%%%%%%%%%%

\subsection{Undecidability of the Character Constraints}

We provide part of the proof of Proposition~\ref{prop-ext-ch-index}, in particular, we show the undecidability of character constraints.

\begin{proposition}\label{prop-ext-char}
	For the extension of $\strline[\replaceall]$ with character constraints, the satisfiability problem is undecidable. 
\end{proposition}

The arguments for Proposition~\ref{prop-ext-char} proceed as follows. Recall that in the proof of Theorem~\ref{thm-ext-int}, we get a formula $C_f \wedge C_g \wedge |y_f| = |y_g|$ such that $f(x_1,\cdots, x_n) = g(x_1,\cdots, x_n)$ has a solution in natural numbers iff $C_f \wedge C_g \wedge |y_f| = |y_g|$ is satisfiable. Let $\$ \neq a$. Suppose  $z_f = y_f \concat \$$, and $z_g = y_g \concat \$$. Then $|y_f| = |y_g|$ can be captured by $z_f[\mathfrak{n}] = \$[1] \wedge  z_g[\mathfrak{n}] = \$[1]$, where $\mathfrak{n}$ is a variable of type $\intnum$. More precisely, 
%
we have 
\begin{quote}
	\centering
	$C_f \wedge C_g \wedge |y_f|= |y_g|$ is satisfiable \\
	%
	iff \\
	%
	$C_f \wedge C_g \wedge z_f = y_f \concat \$ \wedge z_g = y_g \concat \$ \wedge z_f[\mathfrak{n}] = \$[1] \wedge  z_g[\mathfrak{n}] = \$[1]$ is satisfiable. 
\end{quote}
Therefore, we get a reduction from Hilbert's tenth problem to the satisfiability problem for the extension of $\strline[\replaceall]$ with character constraints. 

%For any two string variables $x,y$ on the unary alphabet $\{a\}$, let $x' = x \concat \$$ and $y' = y \concat \$$, then $|x| = |y|$ iff .
%
% $|x|=|y|$ iff $\exists n. x[n]=y[n]=\$$. 
%
%
%\begin{lemma}
%	For any two strings $x,y\in a^*\$$, $|x|=|y|$ iff $\exists n. x[n]=y[n]=\$$. 
%\end{lemma}
%
%As SL[$\replaceall$] with length constraints is undecidable, we conclude that 
 

\subsection{Undecidability of the $\indexof$ Constraints}

We provide the final part of the proof of Proposition~\ref{prop-ext-ch-index}, in particular, we show the undecidability of $\indexof$ constraints.

\begin{proposition}\label{prop-indexof}
	For the extension of $\strline[\replaceall]$ with the $\indexof$ constraints, the satisfiability problem is undecidable. 
\end{proposition}

Proposition~\ref{prop-ext-char} follows from the following observation and Theorem~\ref{thm-ext-int}: For any two string variables $x,y$ over a unary alphabet, 
$1= \indexof(x,y)$ iff $x$ is a prefix of $y$. Therefore, $|x| = |y|$ iff $1=  \indexof(x,y) \wedge 1= \indexof(y,x)$. This implies that in the proof of Theorem~\ref{thm-ext-int}, we can replace $|y_f| = |y_g|$ with $1=\indexof(y_f, y_g) \wedge 1 = \indexof(y_g, y_f)$ and get a reduction from Hilbert's tenth problem to the satisfiability problem for the extension of $\strline[\replaceall]$ with the $\indexof$ constraints.
Note that $=$ can be simulated as a conjunction of $\leq$ and $\geq$.
 




}



\end{document}
