%!TEX root = main.tex

\section{Motivating Example}\label{sec:mot}

\begin{figure*}[htbp]
\begin{center}
{
%\small
\begin{minted}[linenos]{javascript}
    function DBLPtoACM(authorList)
    {
      var autListReg = /^[A-Z]\w*(\s[A-Z]\w*)*(\sand\s[A-Z]\w*(\s[A-Z]\w*)*)*$/;
      if (autListReg.test(authorList)) {
        var nameReg = /([A-Z]\w*(?:\s[A-Z]\w*)*)\s([A-Z]\w*)((\sand\s)|$)/g;
        return authorList.replace(nameReg, "$2, $1$3");
      }
      else return authorList;
    }
\end{minted}
}
\end{center}
\caption{Change the author list from the DBLP format to the ACM format}
\label{fig-run-exmp}
\end{figure*}

\zhilin{The text description to be modified}

\OMIT{
\begin{figure*}[htbp]
\begin{center}
{
%\small
\begin{minted}[linenos]{javascript}
    function changeNameFormat(authorList)
    {
      var autListReg = /^([\w-\.\s]+,[\w-\.\s]+\sand\s)*[\w-\.\s]+,[\w-\.\s]+$/;
      if (autListReg.test(authorList)) {
        var nameReg = /([\w-\.]+(?:\s+[\w-\.]+)*),\s*([\w-\.]+(?:\s+[\w-\.]+)*)((\s+and\s+)|$)/g;
        return authorList.replace(nameReg, "$1$3");
      }
      else return authorList;
    }
\end{minted}
}
\end{center}
\caption{Change the name format of an author list: A motivating example}
\label{fig-run-exmp}
\end{figure*}
}

We use the JavaScript program in Figure~\ref{fig-run-exmp} as a motivating example to illustrate the main approach of this paper. 
The function ``changeNameFormat''  in Figure~\ref{fig-run-exmp} transforms the name format of an author list from the ACM bibtex style to the DBLP bibtex style. For instance,  if a paper is authored by Alice Brown and John Smith, then the author list in the ACM bibtex style is ``Brown, Alice and Smith, John'', while  in the DBLP bibtex style it is ``Alice Brown and John Smith''. 

%\tl{should the "and" in dblp be removed? Alice Brown, John Smith}\zhilin{I am referring to the bibtex style. Both ACM and DBLP bibtex style contain ``and''}

The input of the function ``changeNameFormat'' is {\sf authorList}, which should follow the pattern specified by the regular expression {\sf autListReg}. Intuitively, {\sf autListReg} stipulates that {\sf authorList} %can be obtained by joining 
%is the concatenation of  
joins the strings of the following structure, with the word ``and'' as the separator: concatenation of two sequences of alphabetic letters (denoted by $\backslash$w), bar (denoted by $-$), dot (denoted by $.$), or the blank symbol (denoted by $\backslash$s), with the comma in between. The symbols \^{} and $\$$ denote the beginning and end of an input.

The name format of each author, except for the last one, is specified by the regular expression {\sf nameReg}  in Figure~\ref{fig-run-exmp}, which describes the pattern of a surname (a sequence of alphabetic letters, $-$ or $.$), followed by the comma, then a given name, finally the word ``and'' or $\$$ denoting the end of the input, where $\backslash s$ represents the blank symbol and $\backslash w$ represents alphabetic letters, digits, or the underline symbol $\_$. There are three capturing groups in {\sf nameReg}, intuitively recording surname, given name, and the word ``and'' or $\$$ respectively. Note that surnames or given names may comprise several words. The subexpression (?:$\backslash$s+$[\backslash$w-.]+) in ${\sf nameReg}$ denotes the non-capturing groups, i.e. matching $\backslash$s+$[\backslash$w-.]+ but not remembering the match.
%Similarly, the regular expression {\sf reLast} describes the name format of the last author, except that it replaces the word ``and'' by the symbol $\$$, denoting the end of the author list. 
%
Notice that the global flag ``g'' is used in {\sf nameReg} so that the name format of each author is transformed. 
\tl{mention  this is replaceAll?}
The name format transformation is done by the {\sf replace} function, i.e. {\sf authorList.replace(nameReg, ``$\$$2 $\$$1 $\$$3'')}, where $\$1$, $\$2$, $\$3$ refer to the match of the first, second and third capturing group respectively. 

For a symbolic execution of the JavaScript program in Figure~\ref{fig-run-exmp}, one needs to  model the semantics of capturing groups as well as references. For this purpose, we introduce prioritized streaming string transducers (PSST, see Section~\ref{sect:psst}). Then {\sf replace(nameReg, ``$\$$2 $\$$1$\$$3'')} can be captured by $\cT_{\sf nameReg}$, where the priorities are used to model the greedy semantics of $+$ or $*$ (see Definition~\ref{def-regex-semantics} in Section~\ref{sec:prel}) and the string variables are used to record the matches of capturing groups.

An intuitive invariant property of the output of {\sf changeNameFormat} is that, if {\sf authorList} matches {\sf autListReg}, then none of its outputs contains the comma ``,''. The invariant property holds iff the following program in single static assignment form is infeasible (that is, there does not exist a value of {\sf authorList} so that the program can execute to the end):
%
\begin{eqnarray}\label{eqn:exmp}
& & \ASSERT{\sf authorList \in autListReg};\nonumber \\
& & \sf ret  := \sf  \cT_{nameReg}(authorList);\nonumber \\
& &  \ASSERT{\sf ret \in .*,.*},
\end{eqnarray}
%
where $\ASSERT{\sf authorList \in autListReg}$ requires that $\sf authorList$ matches $\sf autListReg$, while $\ASSERT{\sf ret \in .*,.*}$ requires that $\sf ret$ contains an occurrence of $,$ ($.$ can match any symbol, except the line breaker).

The path feasibility problem of the program in Equation~(\ref{eqn:exmp}) is solved by a ``backward'' reasoning as follows (see Section~\ref{sec:decision} for the details):  
\begin{itemize}
\item At first, we compute $(\cT_{\sf nameReg})^{-1}(.*,.*)$,  i.e. the pre-image of the language $.*,.*$ under $\cT_{\sf nameReg}$ (see Theorem~\ref{theorem:psst_preimage}),  remove  $\sf  ret  := \sf  \cT_{nameReg}(authorList)$, and add the assertion $\ASSERT{\sf authorList \in (\cT_{\sf nameREg})^{-1}(.*,.*)}$, resulting into the following program,
\begin{eqnarray}\label{eqn:exmp}
& & \ASSERT{\sf authorList \in autListReg};\nonumber \\
%& & \sf ret  := \sf  \cT_{nameReg}(authorList);\nonumber \\
& &  \ASSERT{\sf ret \in .*,.*}; \nonumber \\
& & \ASSERT{\sf authorList \in (\cT_{\sf nameREg})^{-1}(.*,.*)}.
\end{eqnarray}

%
%\item Then, we compute $(\cT_{\sf re})^{-1}\left((\cT_{\sf reLast})^{-1}(\Sigma^*, \Sigma^*)\right)$, i.e. the pre-image of $(\cT_{\sf reLast})^{-1}(\Sigma^*, \Sigma^*)$ under $\cT_{\sf re}$, remove the assignment $ \sf result  := \sf  \cT_{ re}(authorList)$, and add the assertion $\ASSERT{\sf authorList \in (\cT_{ re})^{-1}\left((\cT_{\sf reLast})^{-1}(\Sigma^*, \Sigma^*)\right)}$, resulting into the program 
%\begin{eqnarray}\label{eqn:exmp-final}
%& & \ASSERT{\sf authorList \in (\cT_{ re})^{-1}\left((\cT_{\sf reLast})^{-1}(\Sigma^*, \Sigma^*)\right)}; \nonumber\\
%& & \ASSERT{\sf result \in (\cT_{\sf reLast})^{-1}(\Sigma^*, \Sigma^*)}; \nonumber\\
%& & \sf \ASSERT{ret \in \Sigma^*, \Sigma^*}.
%\end{eqnarray}
%Since the program in Equation~(\ref{eqn:exmp-final}) contains no assignment statements, its path feasibility can be solved by checking the nonemptiness of regular languages.
\item Then, we check the nonemptiness of the intersection of the regular languages $\sf autListReg$ and $(\cT_{\sf nameREg})^{-1}(.*,.*)$.\\
If the intersection is empty, then the invariant property holds, otherwise, it does not.
\end{itemize}

