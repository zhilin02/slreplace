%!TEX root = main.tex

\section{Motivating Example}\label{sec:mot}

\begin{figure*}[htbp]
\begin{center}
{
\small
\begin{minted}{javascript}
    function authorNameDBLPtoACM(authorList)
    {
      var autListReg = 
          /^[A-Z](\w*|.)(\s[A-Z](\w*|.))*(\sand\s[A-Z](\w*|.)(\s[A-Z](\w*|.))*)*$/;
      if (autListReg.test(authorList)) {
        var nameReg = /([A-Z](?:\w*|.)(?:\s[A-Z](?:\w*|.))*)([A-Z](?:\w*|.))/g;
        return authorList.replace(nameReg, "$2, $1");
      }
      else return authorList;
    }
\end{minted}
}
\end{center}
\caption{Change the author list from the DBLP format to the ACM format}
\label{fig-run-exmp}
\end{figure*}


%%%%%%%%%%%%%%%%%%%%%%%%%%%%%%%%%%%%%%%%%%%%%
%%%%%%%%%%%%%%%%%%%%%%%%%%%%%%%%%%%%%%%%%%%%%
\OMIT{
\begin{figure*}[htbp]
\begin{center}
{
%\small
\begin{minted}[linenos]{javascript}
    function changeNameFormat(authorList)
    {
      var autListReg = /^([\w-\.\s]+,[\w-\.\s]+\sand\s)*[\w-\.\s]+,[\w-\.\s]+$/;
      if (autListReg.test(authorList)) {
        var nameReg = /([\w-\.]+(?:\s+[\w-\.]+)*),\s*([\w-\.]+(?:\s+[\w-\.]+)*)((\s+and\s+)|$)/g;
        return authorList.replace(nameReg, "$1$3");
      }
      else return authorList;
    }
\end{minted}
}
\end{center}
\caption{Change the name format of an author list: A motivating example}
\label{fig-run-exmp}
\end{figure*}
}
%%%%%%%%%%%%%%%%%%%%%%%%%%%%%%%%%%%%%%%%%%%%%
%%%%%%%%%%%%%%%%%%%%%%%%%%%%%%%%%%%%%%%%%%%%%

We use the JavaScript program in Figure~\ref{fig-run-exmp} as a motivating example to illustrate the main approach of this paper. 
The function ``authorNameDBLPtoACM''  in Figure~\ref{fig-run-exmp} transforms the name format of an author list from the DBLP bibtex style to the ACM bibtex style. For instance,  if a paper is authored by Alice M. Brown and John Smith, then the author list in DBLP bibtex style is ``Alice M. Brown and John Smith'', while the ACM bibtex style is ``Brown, Alice M. and Smith, John''. 

%\tl{should the "and" in dblp be removed? Alice Brown, John Smith}\zhilin{I am referring to the bibtex style. Both ACM and DBLP bibtex style contain ``and''}

The input of the function ``authorNameDBLPtoACM'' is {\sf authorList}, which should follow the pattern specified by the regular expression {\sf autListReg}. Intuitively, {\sf autListReg} stipulates that {\sf authorList} %can be obtained by joining 
%is the concatenation of  
joins the strings of full names of the following structure, with the word ``and'' as the separator: a full-name string is a concatenation of a given name, middle names, and a family name, separated by the blank symbol (denoted by $\backslash$s), moreover, each of the (given, middle, family) names is a concatenation of a capital alphabetic letter (denoted by [A-Z]) followed by a sequence of letters (denoted by $\backslash$w) or a dot symbol (denoted by $.$).
%of alphabetic letters or  (denoted by $\backslash$w), bar (denoted by $-$), dot (denoted by $.$), or the blank symbol (denoted by $\backslash$s), with the comma in between. 
The symbols \^{} and $\$$ denote the beginning and end of a string input.

The DBLP name format of each author is specified by the regular expression {\sf nameReg}  in Figure~\ref{fig-run-exmp}, which describes the format of a full name.
% of a given name, middle names, and a family name, separated by the blank symbol.
%a family name (a sequence of alphabetic letters, $-$ or $.$), followed by the comma, then a given name, finally the word ``and'' or $\$$ denoting the end of the input, where $\backslash s$ represents the blank symbol and $\backslash w$ represents alphabetic letters, digits, or the underline symbol $\_$. 
\begin{itemize}
\item There are two capturing groups in {\sf nameReg}, one for recording the concatenation of the given name and middle names, and the other for recording the family name. 
Note that the symbols ?: in (?:$\backslash$s[A-Z](?:$\backslash$w*|.)) denote the non-capturing groups, i.e. matching the subexpression, but not remembering the match.
%
\item The \emph{greedy} semantics of the Kleene star * is utilized here to guarantee that the subexpression (?:$\backslash$s[A-Z](?:$\backslash$w*|.))* matches all the middle names (since there may exist multiple middle names) and thus ${\sf nameReg}$ matches the full name. For instance, the first match of ${\sf nameReg}$ in  ``Alice M. Brown and John Smith'' is ``Alice M. Brown'', instead of ``Alice M.''. In comparison, if the semantics of * is assumed to be non-greedy, then (?:$\backslash$s[A-Z](?:$\backslash$w*|.))* can be matched to the empty string, thus ${\sf nameReg}$ is matched to ``Alice M.'', which is what we would like to avoid. Therefore, the greedy semantics of * is essential for the correctness of ``authorNameDBLPtoACM''.
%Similarly, the regular expression {\sf reLast} describes the name format of the last author, except that it replaces the word ``and'' by the symbol $\$$, denoting the end of the author list. 
%
\item Moreover, the global flag ``g'' is used in {\sf nameReg} so that the name format of each author is transformed. 
\end{itemize}
The name format transformation is done by the {\sf replace} function, i.e. {\sf authorList.replace(nameReg, ``$\$$2, $\$$1'')}, where $\$1$ and $\$2$ refer to the match of the first and second capturing group respectively. 

For a symbolic execution of the JavaScript program in Figure~\ref{fig-run-exmp}, one needs to model both the greedy semantics of the Kleene star and store the matches of the two capturing groups in ${\sf nameReg}$ for latter references in "$\$2$, $\$$1". For this purpose, we introduce prioritized streaming string transducers (PSST, see Section~\ref{sect:psst}). Then {\sf replace(nameReg, ``$\$$2, $\$$1'')} is captured by a PSST $\cT_{\sf nameReg}$, where the \emph{priorities} are used to model the greedy semantics of $*$ and the \emph{string variables} are used to record the matches of the capturing groups.

\zhilin{stopped here}

An intuitive invariant property of the output of {\sf authorNameDBLPtoACM} is that, if {\sf authorList} matches {\sf autListReg}, then none of its outputs contains the comma ``,''. The invariant property holds iff the following program in single static assignment form is infeasible (that is, there does not exist a value of {\sf authorList} so that the program can execute to the end):
%
\begin{eqnarray}\label{eqn:exmp}
& & \ASSERT{\sf authorList \in autListReg};\nonumber \\
& & \sf ret  := \sf  \cT_{nameReg}(authorList);\nonumber \\
& &  \ASSERT{\sf ret \in .*,.*},
\end{eqnarray}
%
where $\ASSERT{\sf authorList \in autListReg}$ requires that $\sf authorList$ matches $\sf autListReg$, while $\ASSERT{\sf ret \in .*,.*}$ requires that $\sf ret$ contains an occurrence of $,$ ($.$ can match any symbol, except the line breaker).

The path feasibility problem of the program in Equation~(\ref{eqn:exmp}) is solved by a ``backward'' reasoning as follows (see Section~\ref{sec:decision} for the details):  
\begin{itemize}
\item At first, we compute $(\cT_{\sf nameReg})^{-1}(.*,.*)$,  i.e. the pre-image of the language $.*,.*$ under $\cT_{\sf nameReg}$ (see Theorem~\ref{theorem:psst_preimage}),  remove  $\sf  ret  := \sf  \cT_{nameReg}(authorList)$, and add the assertion $\ASSERT{\sf authorList \in (\cT_{\sf nameREg})^{-1}(.*,.*)}$, resulting into the following program,
\begin{eqnarray}\label{eqn:exmp}
& & \ASSERT{\sf authorList \in autListReg};\nonumber \\
%& & \sf ret  := \sf  \cT_{nameReg}(authorList);\nonumber \\
& &  \ASSERT{\sf ret \in .*,.*}; \nonumber \\
& & \ASSERT{\sf authorList \in (\cT_{\sf nameREg})^{-1}(.*,.*)}.
\end{eqnarray}

%
%\item Then, we compute $(\cT_{\sf re})^{-1}\left((\cT_{\sf reLast})^{-1}(\Sigma^*, \Sigma^*)\right)$, i.e. the pre-image of $(\cT_{\sf reLast})^{-1}(\Sigma^*, \Sigma^*)$ under $\cT_{\sf re}$, remove the assignment $ \sf result  := \sf  \cT_{ re}(authorList)$, and add the assertion $\ASSERT{\sf authorList \in (\cT_{ re})^{-1}\left((\cT_{\sf reLast})^{-1}(\Sigma^*, \Sigma^*)\right)}$, resulting into the program 
%\begin{eqnarray}\label{eqn:exmp-final}
%& & \ASSERT{\sf authorList \in (\cT_{ re})^{-1}\left((\cT_{\sf reLast})^{-1}(\Sigma^*, \Sigma^*)\right)}; \nonumber\\
%& & \ASSERT{\sf result \in (\cT_{\sf reLast})^{-1}(\Sigma^*, \Sigma^*)}; \nonumber\\
%& & \sf \ASSERT{ret \in \Sigma^*, \Sigma^*}.
%\end{eqnarray}
%Since the program in Equation~(\ref{eqn:exmp-final}) contains no assignment statements, its path feasibility can be solved by checking the nonemptiness of regular languages.
\item Then, we check the nonemptiness of the intersection of the regular languages $\sf autListReg$ and $(\cT_{\sf nameREg})^{-1}(.*,.*)$.\\
If the intersection is empty, then the invariant property holds, otherwise, it does not.
\end{itemize}

