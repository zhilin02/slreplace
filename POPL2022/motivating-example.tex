%!TEX root = main.tex

\section{Motivating Example}\label{sec:mot}

\begin{figure*}[htbp]
\begin{center}
{
\small
\begin{minted}{javascript}
    function authorNameDBLPtoACM(authorList)
    {
      var autListReg = 
          /^[A-Z](\w*|.)(\s[A-Z](\w*|.))*(\sand\s[A-Z](\w*|.)(\s[A-Z](\w*|.))*)*$/;
      if (autListReg.test(authorList)) {
        var nameReg = /([A-Z](?:\w*|.)(?:\s[A-Z](?:\w*|.))*)([A-Z](?:\w*|.))/g;
        return authorList.replace(nameReg, "$2, $1");
      }
      else return authorList;
    }
\end{minted}
}
\end{center}
\caption{Change the author list from the DBLP format to the ACM format}
\label{fig-run-exmp}
\end{figure*}


%%%%%%%%%%%%%%%%%%%%%%%%%%%%%%%%%%%%%%%%%%%%%
%%%%%%%%%%%%%%%%%%%%%%%%%%%%%%%%%%%%%%%%%%%%%
\OMIT{
\begin{figure*}[htbp]
\begin{center}
{
%\small
\begin{minted}[linenos]{javascript}
    function changeNameFormat(authorList)
    {
      var autListReg = /^([\w-\.\s]+,[\w-\.\s]+\sand\s)*[\w-\.\s]+,[\w-\.\s]+$/;
      if (autListReg.test(authorList)) {
        var nameReg = /([\w-\.]+(?:\s+[\w-\.]+)*),\s*([\w-\.]+(?:\s+[\w-\.]+)*)((\s+and\s+)|$)/g;
        return authorList.replace(nameReg, "$1$3");
      }
      else return authorList;
    }
\end{minted}
}
\end{center}
\caption{Change the name format of an author list: A motivating example}
\label{fig-run-exmp}
\end{figure*}
}
%%%%%%%%%%%%%%%%%%%%%%%%%%%%%%%%%%%%%%%%%%%%%
%%%%%%%%%%%%%%%%%%%%%%%%%%%%%%%%%%%%%%%%%%%%%

We use the JavaScript program in Figure~\ref{fig-run-exmp} as a motivating example to illustrate the main approach of this paper. The function ``authorNameDBLPtoACM'' in Figure~\ref{fig-run-exmp} transforms %the name format of 
an author list in the DBLP bibtex style to the one in the ACM bibtex style. For instance,  if a paper is authored by Alice M. Brown and John Smith, then the author list in the DBLP bibtex style is ``Alice M. Brown and John Smith'', while it is ``Brown, Alice M. and Smith, John'' in the ACM bibtex style . 

The input of the function ``authorNameDBLPtoACM'' is {\sf authorList}, which is expected to follow the pattern specified by the regular expression {\sf autListReg}. Intuitively, {\sf autListReg} stipulates that {\sf authorList} %can be obtained by joining 
%is the concatenation of  
joins the strings of full names %of the following structure a full-name string is 
as a concatenation of a given name, middle names, and a family name, separated by the blank symbol (denoted by $\backslash$s). Each of the given, middle, family names is a concatenation of a capital alphabetic letter (denoted by [A-Z]) followed by a sequence of letters (denoted by $\backslash$w) or a dot symbol (denoted by $.$). Between names, the word ``and'' is used as the separator.
%of alphabetic letters or  (denoted by $\backslash$w), bar (denoted by $-$), dot (denoted by $.$), or the blank symbol (denoted by $\backslash$s), with the comma in between. 
The symbols \^{} and $\$$ denote the beginning and the end of a string input respectively.

The DBLP name format of each author is specified by the regular expression {\sf nameReg}  in Figure~\ref{fig-run-exmp}, which describes the format of a full name.
% of a given name, middle names, and a family name, separated by the blank symbol.
%a family name (a sequence of alphabetic letters, $-$ or $.$), followed by the comma, then a given name, finally the word ``and'' or $\$$ denoting the end of the input, where $\backslash s$ represents the blank symbol and $\backslash w$ represents alphabetic letters, digits, or the underline symbol $\_$. 
\begin{itemize}
\item There are two capturing groups in {\sf nameReg}, one for recording the concatenation of the given name and middle names, and the other for recording the family name. 
Note that the symbols ?: in (?:$\backslash$s[A-Z](?:$\backslash$w*|.)) denote the non-capturing groups, i.e. matching the subexpression, but not remembering the match.
%
 
\item The \emph{greedy} semantics of the Kleene star * is utilized here to guarantee that the subexpression (?:$\backslash$s[A-Z](?:$\backslash$w*|$\backslash$.))* matches all the middle names (since there may exist multiple middle names) and thus ${\sf nameReg}$ matches the full name. For instance, the first match of ${\sf nameReg}$ in  ``Alice M. Brown and John Smith'' is ``Alice M. Brown'', instead of ``Alice M.''. In comparison, if the semantics of * is assumed to be non-greedy, then (?:$\backslash$s[A-Z](?:$\backslash$w*|$\backslash$.))* can be matched to the empty string, thus ${\sf nameReg}$ is matched to ``Alice M.'', which is \emph{not} what we want. Therefore, the greedy semantics of * is essential for the correctness of ``authorNameDBLPtoACM''.
%=======
%\item The \emph{greedy} semantics of the Kleene star * is utilized here to guarantee that the subexpression (?:$\backslash$s[A-Z](?:$\backslash$w*|.))* matches all the middle names (since there may exist multiple middle names) and thus ${\sf nameReg}$ matches the full name. For instance, the first match of ${\sf nameReg}$ in  ``Alice M. Brown and John Smith'' is ``Alice M. Brown'', instead of ``Alice M.''. In comparison, if the semantics of * is assumed to be non-greedy, then (?:$\backslash$s[A-Z](?:$\backslash$w*|.))* can be matched to the empty string, thus ${\sf nameReg}$ is matched to ``Alice M.'', which is what we would like to avoid. Therefore, the greedy semantics of * is essential for the correctness of ``authorNameDBLPtoACM''.
%>>>>>>> c103b21aed01d0787e4815264e14638bb78265a9
%Similarly, the regular expression {\sf reLast} describes the name format of the last author, except that it replaces the word ``and'' by the symbol $\$$, denoting the end of the author list. 
%
\item The global flag ``g'' is used in {\sf nameReg} so that the name format of each author is transformed. 
\end{itemize}
The name format transformation is done by the {\sf replace} function, i.e. {\sf authorList.replace(nameReg, ``$\$$2, $\$$1'')}, where $\$1$ and $\$2$ refer to the match of the first and second capturing group respectively. 

A natural post-condition of {\sf authorNameDBLPtoACM} is that there exists at least one occurrence of the comma symbol between every two occurrences of ``$\sf and$''. 
%
This post-condition has to be established by the function on \emph{every} execution path.  As
an example, consider the path shown in Fig.~\ref{fig-run-exmp-path},
in which the branches taken in the program are represented as
\texttt{assume} statements. The negated post-condition is captured by
the regular expression in the last \texttt{assume}. For this path, the
post-condition can be proved by showing that the program in
Fig.~\ref{fig-run-exmp-path} is infeasible: there does not exist an
initial value {\sf authorList} so that no assumption fails and the
program executes to the end.
\begin{figure}[tb]
\begin{center}
\small
\begin{minted}[linenos]{javascript}
  var autListReg = 
      /^[A-Z](\w*|.)(\s[A-Z](\w*|.))*(\sand\s[A-Z](\w*|.)(\s[A-Z](\w*|.))*)*$/;
  assume(autListReg.test(authorList));
  var nameReg = /([A-Z](?:\w*|.)(?:\s[A-Z](?:\w*|.))*)([A-Z](?:\w*|.))/g;
  var result = authorList.replace(nameReg, "$2, $1");
  assume(/\sand[^,]*\sand/.test(result));
\end{minted}
\end{center}
\caption{Symbolic execution of a path of the JavaScript program in Fig.~\ref{fig-run-exmp}}
\label{fig-run-exmp-path}
\end{figure}

To enable symbolic execution of the JavaScript programs like in Fig.~\ref{fig-run-exmp-path}, one needs to model both the greedy semantics of the Kleene star and store the matches of capturing groups. For this purpose, we introduce prioritized streaming string transducers (PSST, cf.\ Section~\ref{sec-rwre}) by which {\sf replace(nameReg, ``$\$$2, $\$$1'')} is captured as a PSST $\cT$, where the \emph{priorities} are used to model the greedy semantics of $*$ and the \emph{string variables} are used to record the matches of the capturing groups as well as the return value. Then the  symbolic execution of the program in Fig.~\ref{fig-run-exmp-path} can be equivalently turned into the satisfiability of the following string constraint,
%
\begin{equation}
  \label{eq:motiv}
{\sf authorList} \in {\sf autListReg}\wedge {\sf result} = \cT({\sf authorList}) \wedge {\sf result} \in {\sf postConReg},
\end{equation}
%
where {\sf postConReg} =
\mintinline{javascript}{/^.*\sand[^,]*\sand.*$/}, and {\sf postConReg}
as in Fig.~\ref{fig-run-exmp}.

Our solver is able to show that \eqref{eq:motiv} is unsatisfiable. On
the calculus level (introduced in more detail in
Section~\ref{sect:calculus}), the main inference step applied for this
purpose is the computation of the \emph{pre-image} of {\sf postConReg}
under the function
$\cT$; in other words, we compute the language of all strings that are
mapped to incorrect strings (containing two ``\textsf{and}''s without
a comma in between) by
$\cT$. This inference step relies on the fact that the pre-images of
regular languages under PSSTs are regular (see
Lemma~\ref{lem:psst_preimage}). Denoting the pre-image of {\sf
  postConReg}  by
$\cB$, formula~\eqref{eq:motiv} is therefore equivalent to
\begin{equation}
  \label{eq:motiv2}
  {\sf authorList} \in \cB \wedge
{\sf authorList} \in {\sf autListReg}\wedge {\sf result} = \cT({\sf authorList}) \wedge {\sf result} \in {\sf postConReg}.
\end{equation}

To show that this formula (and thus \eqref{eq:motiv}) is
unsatisfiable, it is now enough to prove that the languages defined by
$\cB$ and {\sf autListReg} are disjoint.
