%!TEX root = main.tex
\paragraph{Validation experiments for the formal semantics} \label{sect:valid}
We have defined %the formal semantics of the regular-expression 
{\regexp}-string matching by constructing {\PSST}s. %out of regular expressions. 
In the sequel, we conduct experiments to validate the formal semantics against the actual JavaScript {\regexp}-string matching.

Let $\opset$ denote the set of {\regexp} operators: alternation $+$, concatenation $\concat$, optional $?$, lazy optional $??$, Kleene star $*$, lazy Kleene star $*?$, Kleene plus $+$, lazy Kleene plus $+?$, repetition $\{m_1,m_2\}$, and lazy repetition $\{m_1,m_2\}?$. Moreover, let $\opset^{2}$ (resp. $\opset^{3}$) denote the set of pairs (resp. triples) of operators from $\opset$. 
Aiming at a good coverage of different syntactical ingredients of {\regexp}, we generate regular expressions for every element of $\opset^{\le 3} = \opset \cup \opset^2 \cup \opset^3$.
As arguments of these operators, we consider the following character sets: $\mathbb{S} = \{$a, $\ldots$, z$\}$, $\mathbb{C}=\{$A, $\ldots$, Z$\}$, $\mathbb{D} = \{0,\ldots,9\}$, and $\mathbb{O}$, the set of ASCII symbols not belonging to $\mathbb{S} \cup \mathbb{C} \cup \mathbb{D}$.
Intuitively, these character sets correspond to JavaScript character classes [a-z], [A-Z], [0-9], and [{\textasciicircum}a-zA-Z0-9] (where {\textasciicircum} denotes complement).
Moreover, for the regular expression generated for each element of $\opset^{\le 3}$, we set the subexpression corresponding to its first component as the capturing group. 
For instance, for the pair $(*?, *)$, we generate the {\regexp} $[([\mathbb{S}^*?])^{*}]$. In the end, we generate $10+10*10+10*10*10 = 1110$ {\regexp}s. 

For each generated {\regexp} $e$, we construct a PSST $\psst_e$, whose output corresponds to the matching of the first capturing group in $e$.  Moreover, we generate from $\psst_e$ an input string $w$ as well as the corresponding output $w'$. We require that the length of $w$ is no less than some threshold (e.g., $10$), in order to avoid the empty string and facilitate a  meaningful comparison with the actual semantics of JavaScript regular-expression matching. 
Let {\sf reg} be the JavaScript regular expression corresponding to $e$. Then we execute the following JavaScript program $\cP_{e,w}$,
\begin{center}
{
\small
\begin{minted}{javascript}
      var x = w; console.log(x.match(reg)[1]);
\end{minted}
}
\end{center}
and confirm that its output is equal to $w'$, thus validating that the formal semantics of  \regexp-string matching defined by PSSTs is consistent with the actual semantics of JavaScript {\sf match} function. For instance, for the {\regexp} expression $[([\mathbb{S}^*?])^{*}]$, we generate from the $\psst_e$ the input string $w= aaaaaaaaaa$, together with the output $a$. Then we generate  the JavaScript program from ${\sf reg}$ and $w$, execute it, and obtain the same output $a$.

In all the generated {\regexp}s, we confirm the consistency of the formal semantics of  \regexp-string matching defined by PSSTs with the actual JavaScript semantics, namely, for each {\regexp}  $e$, the output of the PSST $\psst_e$ on $w$ is equal to the output of the JavaScript program $\cP_{e,w}$.