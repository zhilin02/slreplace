%!TEX root = main.tex

In the rest of this section, we define the string constraint language $\strline$. 

The syntax of $\strline$ is defined by the following rules.
\[
\begin{array}{l c l}
\smallskip
\varphi & \eqdef  & x = y \mid z = x \concat y \mid y  = \extract_{i, e}(x) \mid
y  = \replace_{\pat, \rep}(x) \mid 
\\
& & y = \replaceall_{\pat, \rep}(x)   \mid
 x \in e \mid  \varphi \wedge \varphi \mid \varphi \vee \varphi \mid \neg \varphi \
\label{eq:SL}
\end{array}
\]
where
\begin{itemize}
	\item $\concat$ is the string concatenation operation which concatenates two strings,
%
\item  $e \in${\regexp} and $\pat \in${\regexp},
%
\item for the $\extract$ function, $i \in \Nat$,
%
	\item  for the $\replace$ and $\replaceall$ operation, $\rep \in \refexp$, where $\refexp$ is defined as a concatenation of letters from $\Sigma$, the references $\$i$ ($i \in \Nat$), as well as $\refbefore$ and $\refafter$. (Intuitively, $\$0$ denotes the matching of $\pat$, $\$i$ with $i > 0$ denotes the the matching of the $i$-th capturing group, $\refbefore$ and $\refafter$ denote the prefix before resp. suffix after the matching of $\pat$.)
%
%	\item for assertions, $e \in \regexp$.\philipp{I believe we should remove the word ``assert'', it is easier/more readable to consider constraints like $x \in e \wedge y \in e' \wedge ...$}
\end{itemize}

The $\extract_{i, e}(x)$ function extracts the match of the $i$-th capturing group in the successful match of $e$ to $x$ for $x \in \Lang(e)$ (otherwise, the return value of the function is undefined). Note that $\extract_{i, e}(x)$ returns $x$ if $i=0$. Moreover, if the $i$-the capturing group of $e$ is \emph{not} matched, even if $x \in \Lang(e)$, then $\extract_{i, e}(x)$ returns a special symbol $\nullchar$, denoting the fact that its value is undefined. For instance, when $[[a^+] + ([a^*])]$ is matched to the string $aa$, $[a^+]$, instead of $([a^*])$, will be matched, since $[a^+]$ precedes $([a^*])$. Therefore, $\extract_{1, [[a^+] + ([a^*])]}(aa) = \nullchar$. 

%by the {\PSST} $\cT_e$ constructed from $e$ , with all the string variables, except the string variable corresponding to the $i$-th capturing group, removed.
%
%\zhilin{to be consistent with the definition of semantics of regex-string matching}\philipp{I'm for removing the special case
%  $\nullchar$, just use the empty string instead.}\zhilin{for me, $\nullchar$ is important to guarantee the consistency with Javascript semantics. }


\begin{remark}
The match function in programming languages, e.g. $\sf str.match(reg)$ function in JavaScript, finds the first match of $\sf reg$ in $\sf str$,  assuming that $reg$ does not contain the global flag. We can use $\extract$ to express the first match of $\sf reg$ in $\sf str$ by adding $[\Sigma^{*?}]$ and $[\Sigma^*]$ before and after $\sf reg$ respectively. More generally, the value of the $i$-th capturing group in the first match of a $\regexp$ $\sf reg$ in $\sf str$ can be specified as $\extract_{i+1, {\sf reg'}}({\sf str})$, where ${\sf reg'} = [[[\Sigma^{*?}] \concat ({\sf reg})] \concat [\Sigma^*]]$. The other string functions involving regular expressions, e.g. {\sf exec} and {\sf test}, without global flags, are similar to {\sf match}, thus can be encoded by $\extract$ as well.
\end{remark}

The function $\replaceall_{\pat, \rep}(x)$ is parameterized by the \emph{pattern} $\pat \in \regexp$ and the \emph{replacement string} $\rep \in \refexp$.
For an input string $x$, it identifies all matches of $\pat$ in $x$ and replaces them with strings specified by $\rep$. More specifically, $\replaceall_{\pat, \rep}(x)$ finds the first match of $\pat$ in $x$ and replace the match with $\rep$, let $x'$ be the suffix of $x$ after the first match of $\pat$,  then it finds the first match of $\pat$ in $x'$ and replace the match with $\rep$, and so on.
%The set $\refexp$ of replacement strings is defined using the following syntax, where $i \in \Nat$.
%\[
%    \rep = a \mid \$i \mid \refbefore \mid \refafter
%\]
%That is $\rep$ is a string of characters that may also contain \emph{references}.
%A reference $\$i$ where $i > 0$ is instantiated by the string matched by the $i$th capture group.
%For instance, let $w = 2.5, 3.4$, $\pat = [[([\Gamma^+])\concat .?] \concat ([\Gamma^*])]$ and $\rep = \$1$, then $\replaceall_{\pat, \rep}(w) = 2, 3$.
%
A reference $\$i$ where $i > 0$ is instantiated by the matching of the $i$th capturing group.
There are three special references\footnote{
    The corresponding syntax for $\$0$, $\refbefore$ and $\refafter$ in JavaScript are $\$\&$, $\$`$, and $\$'$.
} $\$0$, $\refbefore$, and $\refafter$.
These are instantiated by the matched text, the text occurring before the match, and the text occurring after the match respectively.
In particular, if the input word is $u v w$ where $v$ has been matched and will be replaced, then $\$0$ takes the value $v$, $\refbefore$ takes the value $u$, and $\refafter$ takes the value $w$.
When there are multiple matches in a $\replaceall$, the values of $\refbefore$ and $\refafter$ are always with respect to the original input string $x$.

The $\replace_{\pat, \rep}(x)$ function is similar to $\replaceall_{\pat, \rep}(x)$, except that it replaces only the first (leftmost) match of $\pat$.

A $\strline$ formula $\varphi$ is said to be \emph{straight-line}, if 1) it contains neither negation nor disjunction, 2) the equations in $\varphi$ can be ordered into a sequence, say $x_1 = t_1, \ldots, x_n = t_n$, such that $x_1,\ldots, x_n$ are mutually distinct, moreover, for each $i \in [n]$, $x_i$ does \emph{not} occur in $t_1, \ldots, t_{i-1}$. Let $\strlinesl$ denote the set of straight-line $\strline$ formulas.

As a crucial step for solving the string constraints in $\strline$, we are going to define the formal semantics of the $\extract_{i, e}(x)$, $\replace_{\pat, \rep}(x)$, and $\replaceall_{\pat, \rep}(x)$ functions in the next section.

%%%%%%%%%%%%%%%%%%%%%%%%%%%%%%%%%%%%%%%%%%%%%%%%%%%%%%%%%%
%%%%%%%%%%%%%%%%%%%%%%%%%%%%%%%%%%%%%%%%%%%%%%%%%%%%%%%%%%
%\OMIT{
%Without loss of generality, we assume that 
%all the $\strline$ programs are in single static assignment (SSA) form, that is 1) each variable $x$ is assigned at most once; and 2) if $x$ is assigned, all its occurrences on the right hand sides of the assignment statements or in assertions are after the assignment statement of $x$.
%%
%For an $\strline$ program $S$, a variable $x$ occurring in $S$ is said to be an \emph{input} variable if $x$ does not occur on the left hand sides of assignment statements.
%
%The \emph{path feasibility} problem of an $\strline$ program is to decide whether there are valuations of the input variables so that the program can execute to the end.
%This problem turns out to be undecidable, this is because of the backreferences in assertion statements or in pattern parameters of the $\replace$/$\replaceall$ function.
%
%\begin{proposition}\label{prop-und}
%The path feasibility problem of $\strline$ is undecidable.
%\end{proposition}
%
%
%We shall show that the path feasibility problem becomes decidable, if the uses of backreferences in assertion statements and pattern parameters of the $\replace$/$\replaceall$ function are forbidden, which turns out to be the situation in practice,  according to the statistics in the literature (see Section~\ref{sec-intro}).
%In the sequel, we will use $\strline_{\sf reg}$ to denote the collection of $\strline$ programs which are free of
%backreferences in assertion statements as well as in pattern parameters of the $\replace$/$\replaceall$ function.
%Note that $\strline_{\sf reg}$  allows backreferences in replacement parameters of $\replace$/$\replaceall$. The main result of this paper is as follows.
%
%
%\begin{theorem}\label{thm-main}
%The path feasibility of $\strline_{\sf reg}$ is decidable.
%\end{theorem}
%The decision procedure for $\strline_{\sf reg}$ utilizes a new automata model called prioritized streaming string transducers, which will be defined in the next section.
%}
%%%%%%%%%%%%%%%%%%%%%%%%%%%%%%%%%%%%%%%%%%%%%%%%%%%%%%%%%%
%%%%%%%%%%%%%%%%%%%%%%%%%%%%%%%%%%%%%%%%%%%%%%%%%%%%%%%%%%
