%!TEX root = main.tex

\section{Formal Semantics of Regular Expression Matching In Javascript}\label{sec-rwre}
Our gaol in this section is to define the formal semantics of regular expression matching in Javascript. 
%
It should be pointed out that if only the set of strings defined by regular expressions are concerned, regular expressions in Javascript (with backreferences ignored) are the same as classical regular expressions in formal language textbooks (e.g. \cite{HU79}). Nevertheless, matching of regular expressions to strings in Javascript, e.g. in the string functions ``exec()'', ``match()'' and ``test()'' , are much more involved: 
\begin{enumerate}
\item in Javascript, the regular expressions are not required to be matched to the whole string, but to a substring, which intuitively corresponds to the first match of the regular expression in the string, moreover, this matching is \emph{deterministic} in the sense that for a given regular expression and a string, the matching returns a \emph{unique} substring (if there is any), 
%
\item regular expressions in Javascript typically contain capturing groups, and the matchings of these capturing groups in strings should also be returned, moreover, these matchings are also deterministic.
\end{enumerate}

In the sequel, we first define the syntax of regular expressions, which is essentially the regular expressions in Javascript, with the backreferences ignored. 
%
Then we formalize the Javascript semantics of matchings of regular expressions to strings. The aforementioned determinism of matchings of regular expressions to strings are attributed to the non-standard semantics of regular-expression operators in Javascript, e.g. the greedy/lazy semantics of Kleene star/plus operators as well as the non-commutative alternation operator.  

The key of the formal semantics is a novel automata model, called prioritized streaming string transducers (abbreviated as PSST), where transition priorities are used to capture the non-standard semantics of regular-expression operators and string variables are used to store the matchings of capturing groups. PSST is a synergy of two automata models introduced before, namely, prioritized finite-state automata \cite{BM17} and streaming string transducers \cite{AC10,AD11}. The formal semantics of Javascript regular-expression matching is defined by constructing a PSST out of regular expressions. 
%
Furthermore, we also do extensive experiments to validate the formal semantics against the actual semantics of regular-expression matching in Javascript.

%%%%%%%%%%%%%%%%%%%%%%%%%%%%%%%%%%%%%%%%
%%%%%%%%%%%%%%%%%%%%%%%%%%%%%%%%%%%%%%%%
\OMIT{
In this section, we introduce the regular expressions which will be used in the string constraints. 
%The semantics of the regular language conforms to the JavaScript language. 
It should be emphasized that the strings accepted by the regular expresses introduced here are still regular, but the parsing of the string is significantly different from classic regular languages. For this purpose, we utilize prioritized finite-state automata \cite{BM17}, which extend classic finite-state automata with priorities, to capture, among others, the greedy/lazy semantics of Kleene star/plus in the regular expression. 
}
%%%%%%%%%%%%%%%%%%%%%%%%%%%%%%%%%%%%%%%%
%%%%%%%%%%%%%%%%%%%%%%%%%%%%%%%%%%%%%%%%

\medskip

Throughout the paper, $\Int^+$ denotes the set of positive integers, and  $\nat$ denotes the set of natural numbers. Furthermore, for $n\in \Int^+$, let $[n]:=\{1, \ldots, n\}$. 
%
We use $\Sigma$ to denote a finite set of letters, called \emph{alphabet}. A \emph{string} over $\Sigma$ is a finite sequence of letters from $\Sigma$. We use $\Sigma^*$ to denote the set of strings over $\Sigma$, $\varepsilon$ to denote the empty string, and $\Sigma^\varepsilon$ to denote $\Sigma \cup \{\varepsilon\}$. A string $w'$ is called a \emph{prefix} of $w$ if $w = w'w''$ for some string $w''$. We use $\pref(w)$ to denote the set of prefixes of $w$. For a prefix $w_1$ of $w$, let $w = w_1 w_2$, then we use $w_1^{-1}w$ to denote $w_2$.


%%%%%%%%%%%%%%%%%%%%%%%%%%%%%%%%%%%%%%%%%
%\hide{
%	\tl{this part can be moved to intro?}
%	Regular expressions are a well-known concept in formal language. %and  have the same expressibility as finite state automata. 
%	Many programming languages provide build-in regular expressions %capabilities either built-in 
%	or otherwise via libraries. Programmers widely use regular expressions in software development, especially in the development of web applications. However, it should be emphasized that regular expressions used in programming languages are considerably different from those in formal language theory, mainly on the following aspects: greedy/non-greedy semantics of the quantifiers ($*$ and its variant $+$), non-commutativity of the alternation operator, capturing groups, and backreferences. In the sequel, we take all these aspects into account and define the class of real-world regular expressions considered in this paper. 
%}
%%%%%%%%%%%%%%%%%%%%%%%%%%%%%%%%%%%%%%%%%  

\subsection{Syntax of regular expressions}

\begin{definition}[Regular expressions, $\regexp$]
	
%\begin{multline*}
\[
\begin{split}
e & \eqdef  \emptyset \mid \varepsilon \mid a \mid [e^?] \mid [e^{??}] \mid  (e) \mid %\$n \mid 
[e + e] \mid [e \concat e] \mid \\
 &          [e^*] \mid [e^+] \mid [e^{*?}] \mid  [e^{+?}] \mid [e^{\{m_1,m_2\}}] \mid [e^{\{m_1,m_2\}?}] 
\end{split}
\]
%\end{multline*}
%	
where $a \in \Sigma$,  $n \in \Int^+$, $m_1,m_2 \in \Nat$ with $m_1 \le m_2$. 
	%	Since $+$ is associative and commutative, we also write $(e_1 + e_2) + e_3$ as $e_1 + e_2 + e_3$ for brevity.  
\end{definition}
%We abbreviate $[e \concat [e^*]]$ as $[e^+]$ and $[e \concat [e^{*?}]]$ as $[e^{+?}]$. 
%
For $\Gamma = \{a_1, \ldots, a_k\}\subseteq \Sigma$, we write $\Gamma$ for  $[[\cdots [a_1 + a_2] + \cdots] + a_k]$ and thus $[\Gamma^\ast] \equiv [[[\cdots [a_1 + a_2] + \cdots] + a_k]^\ast]$. Similarly for $[\Gamma^{\ast?}]$, $[\Gamma^+]$, and $[\Gamma^{+?}]$. We write $|e|$ for the length of $e$, i.e., the number of symbols occurring in $e$.
%
Note that square brackets $[]$ are used for the operator precedence and the parentheses $()$ are used for \emph{capturing groups}. 
 %
%Parenthesis pairs are indexed according to the occurrence sequence of their left parentheses, and it is required that every back reference $\$ n$ occurs after the $n$-th pair of parentheses. For instance, $[[([[a+b]^*]) \concat c] \concat \$1]$ is in $\regexp$, where $\$1$ refers to the matching of the subexpression $[[a+b]^*]$. Intuitively, it denotes the set of strings of the form $u c u$, where $u$ is a string of $a$ and $b$. 
%

The operator $[e^*]$ is the \emph{greedy} Kleene star, meaning that $e$ should be matched as many times as possible. In contrast, the operator $[e^{*?}]$ is the \emph{lazy} Kleene star, meaning $e$ should be matched  as few times as possible. The Kleene plus operators $[e^+]$ and $[e^{+?}]$ are similar to $[e^*]$ and $[e^{*?}]$ but $e$ should be matched at least once. Moreover, as expected,  the counting operators $[e^{\{m_1,m_2\}}]$ require the number of times that $e$ is matched is between $m_1$ and $m_2$ and $[e^{\{m_1,m_2\}?}]$ is the lazy variant. Likewise, the optional operator has greedy and lazy variants $[e^?]$ and $[e^{??}]$, respectively.

For two $\regexp$s $e$ and $e'$, we say that $e'$ is a \emph{subexpression} of $e$,
  	if one of the following conditions holds: 1) $e'=e$, 2) $e = [e_1 \cdot e_2]$ or $[e_1 + e_2]$, and $e'$ is a subexpression of $e_1$ or $e_2$, 3) $e = [e_1^?], [e_1^{??}], [e_1^{\ast}]$, $[e_1^{+}]$, $[e_1^{\ast?}]$, $[e_1^{+?}]$, $e_1^{\{m_1, m_2\}}$, $e_1^{\{m_1, m_2\}?}$ or $( e_1)$, and $e'$ is a subexpression of $e_1$. We use $S(e)$ to denote the set of subexpressions of $e$. %
 
% 
%We use $\cgexp$ to denote the fragment of $\regexp$ excluding backreferences $\$ n$ (where {\sf reg} represents regular languages), and $\refexp$ to denote the set of regular expressions generated by a concatenation of letters and backreferences, formally %regular expressions 
%defined by $e \eqdef \varepsilon \mid a \mid \$n \mid [e \concat e]$.  
%%\tl{define the semantics here?}

%%%%%%%%%%%%%%%%%%%%%%%%%%%%%%%%%%%%%%%%%%
\subsection{Prioritized streaming string transducers (PSST)}

%!TEX root = main.tex

\section{Prioritized Streaming String Transducers}  \label{sect:psst}
%\zhilin{Move from preliminary to here}

In this section, we introduce prioritized streaming string transducers (PSST), which extend prioritized finite-state automata (PFA) proposed in \cite{BM17}. We shall utilize PSST to model  the eager and greedy semantics of $\regexp$ as well as the behavior of $\replaceall$ functions where both capturing groups and back references occur.
%based on which we model the semantics of $\regexp$ defined in Section~\ref{sec:prel} and design the decision procedure in Section~\ref{sec:decision}.

%\paragraph{Prioritized Finite-state automata.}
%
For a finite set $Q$, let $\overline{Q} = \bigcup_{n\in \Nat}\{ (q_1, \ldots, q_n) \mid \forall i \in [n], q_i \in Q \wedge \forall i,j \in[n], i \neq j \rightarrow q_i \neq q_j \}$. Intuitively, $\overline{Q}$ is the set of sequences of non-repetitive elements from $Q$. In particular, the empty sequence $\varepsilon \in \overline{Q}$. Note that the length of each sequence from $\overline{Q}$ is bounded by  $| Q |$. For a sequence $P = (q_1, \ldots, q_n) \in \overline{Q}$ and  $q \in Q$, we write $q \in P$ if  $q = q_i$ for some $i \in [n]$. Moreover, for $P_1 = (q_1, \ldots, q_m) \in \overline{Q}$ and $P_2 = (q'_1, \ldots, q'_n) \in \overline{Q}$, we say $P_1 \cap P_2 = \emptyset$ if $\{q_1, \ldots, q_m\} \cap \{q'_1, \ldots, q'_n\} = \emptyset$.


\begin{definition}[Prioritized Finite-state Automata]\label{def-pfa}
  A \emph{prioritized finite-state automaton} (PFA) over a finite alphabet $\Sigma$ is a tuple $\pnfa=(Q, \Sigma, \delta, \tau, q_0, F)$ where $\delta \in Q
  \times \Sigma \rightarrow \overline{Q}$ and $\tau \in Q \rightarrow \overline{Q} \times \overline{Q}$ such that for every $q \in Q$, if $\tau(q) = (P_1; P_2)$, then $P_1 \cap P_2 = \emptyset$. 
  The definition of $Q$, $q_0$ and $F$ is the same as ordinary FA.
\end{definition}
For $\tau(q) = (P_1; P_2)$, we will use $\pi_1(\tau(q))$ and $\pi_2(\tau(q))$ to denote $P_1$ and $P_2$ respectively.  With slight abuse of notation, we write $q\in (P_1; P_2)$ for $q\in P_1\cup P_2$. Intuitively, $\tau(q)=(P_1; P_2)$ specifies the $\varepsilon$-transitions at $q$, with the intuition that the $\varepsilon$-transitions to the states in $P_1$ resp. $P_2$ have higher resp. lower priorities than the non-$\varepsilon$-transitions out of $q$.
  
A  run of $\pnfa$ on a string $w$ is a sequence $q_0 \sigma'_1 q_1 \ldots \sigma'_m q_m$ such that 
\begin{itemize}
%\item $q_m \in F$,
\item for any $i \in [m]$, either $\sigma'_i \in \Sigma$ and $q_i \in \delta (q_{i - 1}, \sigma'_i)$, or $\sigma'_i = \varepsilon$ and $q_i \in \tau(q_{i-1})$ %\pi_1(\tau(q_{i-1}))\cup \pi_2(\tau(q_{i-1}))$,
\item $w = \sigma'_1 \cdots \sigma'_m$,
%
\item for every subsequence $q_i \sigma'_{i+1} q_{i+1} \ldots \sigma'_{j} q_j$ such that  $i < j$ and $\sigma'_{i+1} = \cdots = \sigma'_j = \varepsilon$, it holds that for every $k, l: i \le k < l < j$, $(q_k, q_{k+1}) \neq (q_l, q_{l+1})$.
%each state $q \in Q$ occurs \emph{at most twice} in the subsequence. 
(Intuitively, each transition occurs at most once in a sequence of $\varepsilon$-transitions.) 
\end{itemize}
Note that it is possible that $\delta(q, \sigma) = ()$, namely, there is no $\sigma$-transition out of $q$. 
It is easy to observe that given a string $w$, the length of a run of $\pnfa$ on $w$ is $O(|w||\cA|)$.
For any two runs $p = q_0 \sigma_1 q_1 \ldots \sigma_m q_m$ and $p' =  q_0 \sigma'_1 q_1' \ldots \sigma'_n q'_n$ such that $\sigma_1 \ldots \sigma_m = \sigma'_1 \ldots \sigma'_n$, we say that $p$ is of a higher priority over $p'$ if 
\begin{itemize}
\item either $p'$ is a prefix of $p$ (in this case, the transitions of $p$ after $p'$ are all $\varepsilon$-transitions), 
%
\item or there is an index $j$ satisfying one of the following constraints:
\begin{itemize}
\item $q_0 \sigma_1 q_1 \ldots q_{j-1} \sigma_j = q_0 \sigma'_1 q'_1 \ldots q'_{j-1} \sigma'_j$, $q_j \neq q'_j$, $\sigma_j \in \Sigma$, and $\delta (q_{j - 1}, \sigma_j) =(\ldots, q_j, \ldots, q_j', \ldots)$,
%
\item $q_0 \sigma_1 q_1 \ldots q_{j-1} \sigma_j = q_0 \sigma'_1 q'_1 \ldots q'_{j-1} \sigma'_j$, $q_j \neq q'_j$, $\sigma_j  = \varepsilon$,  and  either $\pi_1(\tau(q_{j - 1})) = (\ldots, q_j, \ldots, q_j', \ldots)$, or $\pi_2(\tau(q_{j - 1})) = (\ldots, q_j, \ldots, q_j', \ldots)$, or $q_j \in \pi_1(\tau(q_{j - 1}))$ and $q'_j \in \pi_2(\tau(q_{j-1}))$, 
%
\item $q_0 \sigma_1 q_1 \ldots q_{j-1}  = q_0 \sigma'_1 q'_1 \ldots q'_{j-1} $, $\sigma_j  = \varepsilon$, $\sigma'_j  \in \Sigma$, $q_j \in \pi_1(\tau(q_{j - 1}))$, and $q'_j \in \delta(q_{j-1}, \sigma'_j)$, 
%
\item $q_0 \sigma_1 q_1 \ldots q_{j-1}  = q_0 \sigma'_1 q'_1 \ldots q'_{j-1} $, $\sigma_j  \in \Sigma$, $\sigma'_j  = \varepsilon$, $q_j \in \delta(q_{j - 1}, \sigma_j)$, and $q'_j \in \pi_2(\tau(q_{j-1}))$.
\end{itemize}
\end{itemize}

An \emph{accepting} run of $\pnfa$ on $w$ is a run $q_0 \sigma_1 q_1 \ldots \sigma_m q_m$ of $\pnfa$ on $w$ with the \emph{highest} priority such that $q_m \in F$. (Note that a run $q_0 \sigma_1 q_1 \ldots \sigma_m q_m$ of $\pnfa$ with the highest priority may not be accepting, i.e. satisfy $q_m \in F$.) The language of $\pnfa$, denoted as $\Lang(\pnfa)$, is the set of strings on which $\pnfa$ has an accepting run.


Note that PFAs differ from FAs only in the way that a string is accepted; they both define regular languages. 

%\begin{figure}[ht]
%\centering
%\rule{\linewidth}{0cm}
%\includegraphics[scale=0.8]{pfa-epsilon-star.pdf}
%\caption{The PFA for $\varepsilon^\ast$}
%\label{fig-pfa-epsilon-star}
%\end{figure}

\begin{example}\label{exmp-pfa}
The PFAs corresponding to $a^\ast$ and $a^{\ast?}$ respectively are illustrated in Figure~\ref{fig-pfa}: (i) and (ii), where the dashed line represents $\pi_2(\tau(q_0))$ (of lower priority than the $a$-transition), the thicker solid line represents $\pi_1(\tau(q_0))$ (of higher priority than the $a$-transition), and the doubly circled state $q_1$ is a final state.

\begin{figure}[ht]
\centering
%\rule{\linewidth}{0cm}
\includegraphics[scale=0.8]{pfa.pdf}
\caption{The PFAs for $a^\ast$ and $a^{\ast?}$}
\label{fig-pfa}
\end{figure}

\end{example}

%\begin{remark}
%Remark that PFAs in Definition~\ref{def-pfa} are different from pNFAs in \cite{BM17} in the sense that the state set in a pNFA is partitioned into two disjoint subsets and the non-$\varepsilon$-transitions are deterministic, while this is not the case in PFAs. Therefore, PFAs are slightly more flexible than pNFAs in \cite{BM17}. We choose this definition of PFAs as a more natural extension of FAs. 
%\end{remark}

The priorities of PFAs are used to model the eager and greedy semantics of $\regexp$, as we shall see in Section~\ref{construction:pnfa}.


%\paragraph{Prioritized streaming string transducers.}

We then introduce prioritized streaming string transducers, a new class of transducers that combine prioritized transducers \cite{BM17} %which combines the expressive power of 
and streaming string transducers \cite{AC10,AD11}.
  
\begin{definition}[Prioritized Streaming String Transducers]
A \emph{prioritized streaming string transducer} (PSST) is a tuple $\psst = (Q, \Sigma, X, \delta, \tau, E, q_0, F)$, where $Q$ a
finite set of states, $\Sigma$ is the input and output alphabet, $X$ is a finite set of variables, $\delta \in Q \times \Sigma \rightarrow \overline{Q}$, $\tau \in Q \rightarrow \overline{Q} \times \overline{Q}$, $E$ is a partial function from $Q \times \Sigma^\varepsilon \times
  Q$ to $X \rightarrow (X \cup \Sigma)^{\ast}$, i.e. the set of assignments,
   $q_0 \in Q$ is the initial state, and $F$ is a partial function
  from $Q$ to $(X \cup \Sigma)^{\ast}$.

A run of $\psst$ on a string $w$ is a sequence $q_0 \sigma_1 s_1 q_1 \ldots \sigma_m s_m q_m$ such that
\begin{itemize}
%\item $q_m \in F$,
%
\item for each $i \in [m]$, 
\begin{itemize}
\item either $\sigma_i \in \Sigma$, $q_i \in \delta (q_{i-1}, \sigma_i)$, and $s_i = E (q_{i - 1}, \sigma_i, q_i)$, 
\item or $\sigma_i = \varepsilon$, $q_i \in \tau(q_{i-1})$ and $s_i = E (q_{i - 1}, \varepsilon, q_i)$,
\end{itemize}

%\item for every subsequence $q_i \sigma_{i+1} s_{i+1} q_{i+1} \ldots \sigma_{j} s_j q_j$ such that  $i < j$ and $\sigma_{i+1} = \cdots = \sigma_j = \varepsilon$, it holds that $q_i, \ldots, q_j$ are mutually distinct. (Intuitively, loops of $\varepsilon$-transitions are forbidden.) 
\item for every subsequence $q_i \sigma_{i+1} s_{i+1} q_{i+1} \ldots \sigma_{j} s_j q_j$ such that  $i < j$ and $\sigma_{i+1} = \cdots = \sigma_j = \varepsilon$,  it holds that for every $k, l: i \le k < l < j$, $(q_k, q_{k+1}) \neq (q_l, q_{l+1})$.
\end{itemize}

%A run of $\psst$ is the sequence $q_0 \sigma_1 s_1 q_1 \ldots \sigma_m s_m q_m$, where $F (q_m)$ is defined and for each $i \in [m], q_i \in \delta (q_{i-1}, \sigma_i)$ and $s_i = E (q_{i - 1}, \sigma_i, q_i)$. 
For any pair of runs $p = q_0 \sigma_1 s_1 \ldots \sigma_m s_m q_m$ and $p' = q_0 \sigma'_1
  s_1' \ldots \sigma'_n s_n' q_n'$ such that $\sigma_1 \ldots \sigma_m = \sigma'_1 \ldots \sigma'_n$, the definition that $p$ is of a higher priority over
  $p'$ is similar to PFAs.
  % $p \neq p'$ and, for the smallest index $j$ with $q_j \neq q_j'$,
 % $\delta (q_{j - 1}, \sigma_j) = \ldots q_j \ldots q_j' \ldots$
  
An accepting run of $\psst$ on an input $w$ is a run of $\psst$ on $w$ of the highest priority, say $q_0 \sigma_1 s_1 \ldots \sigma_m s_m q_m$, such that $F(q_m)$ is defined. The output of $\psst$ on $w$, denoted by $\psst(w)$, is defined as $\eta_m(F(q_m))$, where $\eta_0(x) = \varepsilon$ for each $x \in X$, and $\eta_{i}(x) = \eta_{i-1}(s_{i}(x))$ for every $1 \le i \le m$ and $x \in X$. Note that here we abuse the notation  $\eta_m(F(q_m))$ and $\eta_{i-1}(s_{i}(x))$ by taking a function $\eta$ from $X$ to $\Sigma^*$ as a function from $(X \cup \Sigma)^*$ to $\Sigma^*$, which maps each $\sigma \in \Sigma$ to $\sigma$ and each $x \in X$ to $\eta(x)$. If there is no accepting run of $\psst$ on $w$, then $\psst(w) = \bot$, namely, the output of $\psst$ on $w$ is undefined. The string relation defined by $\psst$, denoted by $\cR_\psst$,  is $\{(w, \psst(w)) \mid w \in \Sigma^\ast, \psst(w) \neq \bot\}$.
\end{definition}

\begin{example}
The PSST $\cT_{\sf nameReg}=(Q, \Sigma, X, \delta, \tau, E,  q_{0}, F)$ mentioned in Section~\ref{sec:mot} is illustrated in Figure~\ref{fig-psst-exmp}, where $Q = \{q_0, \dots, q_{24}\}$, $X= \{x_1, x_2, x_3, x_4\}$ with $x_1, x_2, x_3$ recording the matches of the 1st, 2nd, 3rd capturing group, and $x_4$ recording the string after the replacements, $F(q_{24}) = x_4$ denotes the final output, and $\delta, \tau, E$ are illustrated by the edges, where the dashed edges denote the $\varepsilon$-transitions of lower priorities than the non-$\varepsilon$-transitions and the symbol $\ell$ is used to denote the currently scanned input letter. For instance, $\delta(q_3, \backslash\mbox{s}) = (q_3)$, $\delta(q_3, \ell) = ()$ for every $\ell \in \Sigma \setminus \{\backslash\mbox{s}\}$, $\tau(q_3) = ((); (q_4))$, and $E(q_3, \backslash\mbox{s}, q_3)(x_1) = x_1 \backslash s$. Since the $\varepsilon$-transition has lower priority than the $\backslash\mbox{s}$-transition at the state $q_3$, whenever the currently scanned letter is $\backslash$s at $q_3$,  $\cT_{\sf nameReg}$ will choose to go to $q_3$ greedily, until there is no more $\backslash$s. (In this case, it has to choose the $\epsilon$-transition and goes to $q_4$.) Note that the identity assignments, e.g. $E(q_3, \backslash\mbox{s}, q_3)(x') = x'$ for $x' \in \{x_2, x_3, x_4\}$, are omitted in Figure~\ref{fig-psst-exmp}, for readability. 
%From $\delta(q_4, \backslash s) = q_5q_{6}$, we know that $q_5$ is prior to $q_6$. 
%Therefore, whenever $\cT_{\sf nameReg}$ reads $\backslash$s at the state $q_3$,  it will choose to go the state $q_5$ greedily, unless this choice would lead to the nonacceptance (in this case, $q_6$ will be chosen). 
\begin{figure*}[ht]
\centering
%\rule{\linewidth}{0cm}
\includegraphics[scale=0.8]{psst-epsilon-exmp.pdf}
\caption{The PSST $\cT_{\sf nameReg}$}
\label{fig-psst-exmp}
\end{figure*}
\end{example}

  
%  $\tmop{Out} (r) =
%  s_{\varepsilon} \circ s_1 \circ s_2 \ldots s_n \circ F (q_n)$ where
%  $s_{\varepsilon}$ is the empty substitution which maps all variables to
%  $\varepsilon$.
  
\begin{definition}[Pre-image]
For a string relation $R \subseteq \Sigma^* \times \Sigma^*$ and $L \subseteq \Sigma^*$, we define the \emph{pre-image} of $L$ under $R$ as $R^{-1}(L):=\{w \in \Sigma^* \mid \exists w'.\ w' \in L \mbox{ and } (w, w') \in R\}$. 
\end{definition}
 
\begin{theorem}[Pre-image of \PSST{}]
  \label{theorem:psst_preimage}
  Given a \PSST{} $\psst = (Q_T, \Sigma$, $X, \delta_T, \tau_T, E_T,  q_{0, T}, F_T)$ and an \FA{} $\Aut
  = (Q_A, \Sigma, \delta_A, q_{0, A}, F_A)$, we can compute an \FA{} $\cB = (Q_B,
  \Sigma, \delta_B, q_{0, B}, F_B)$ in exponential time  such that $\Lang(\cB) = \cR^{-1}_{\cT}(\Lang(\Aut))$.
\end{theorem}

\tl{another way is to first define B as a PFA, could make the construction a bit modular?}\zhilin{Add a counter example for this natural idea.}
 
Let $\psst = (Q_T, \Sigma$, $X, \delta_T, \tau_T, E_T,  q_{0, T}, F_T)$ be a \PSST{}  and $\Aut
  = (Q_A, \Sigma, \delta_A, q_{0, A}, F_A)$ be an \FA{}. Without loss of generality, we assume that $\Aut$ contains no $\varepsilon$-transitions. 

To illustrate the intuition of the FA construction, let us start with the following natural idea of firstly constructing a PFA $\cB$ for the pre-image: $\cB$ simulates the run of $\psst$ on $w$, and, for each $x \in X$, records an $\Aut$-abstraction of the string stored in $x$, that is, the set of state pairs $(p, q) \in Q_A \times Q_A$ such that starting from $p$, $\Aut$ can reach $q$ after reading the string stored in $x$. Specifically, the states of $\cB$ are of the form $(q, \rho)$ with $q \in Q$ and $\rho \in (\cP(Q_A \times Q_A ))^{X}$. Moreover, the priorities of $\cB$ inherit those of $\Aut$. The PFA $\cB$ is then transformed to an equivalent FA by simply dropping all priorities. We refer to this FA as $\cB'$.

Nevertheless, it turns out that this construction method is flawed: A string $w$ is in $\cR^{-1}_{\cT}(\Lang(\Aut))$ iff the (unique) accepting run of $\cT$ on $w$ produces an output $w'$ that is accepted by $\Aut$. However, a string $w$ is accepted by $\cB'$ iff there is a run of $\cT$ on $w$, not necessarily of the highest priority, producing an output $w'$ that is accepted by $\Aut$. The following example illustrates the flaw of the construction above.

\begin{example}
\label{pre-image-count-examp}
Let $\cT$ be the PSST and $\cA$ be the FA in Figure~\ref{fig-pre-image-count-exmp}, that is, 
\begin{itemize}
\item $\cT=(\{q_0, q_1, q_2\}, \{a,b,c\}, \{x_0\}, \delta_T, \tau_T, E_T, q_0, F_T)$, where $\delta_T(q_0, \sigma) = (q_0)$, $\delta_T(q_1, a) = (q_1)$, $\delta_T(q_2, \sigma) = (q_2)$, $\tau_T(q_0) = ((q_1); ())$, $\tau_T(q_1)=((q_0, q_2);())$, and $\tau_T(q_2)= ((); ())$, $E_T(q_0, \sigma, q_0) (x_0) = x_0 \sigma$, $E_T(q_1, \varepsilon, q_0) (x_0) = x_0 c$, $E_T(q_1, \varepsilon, q_2) (x_0) = x_0 c$, $E_T(q_2, \sigma, q_2) (x_0) = x_0 \sigma$, for $\sigma \in\{ a, b\}$. Moreover, $F_T(q_2)= x_0$;
%
\item $\cA = (\{p_0\}, \{a,b,c\}, \delta_A, p_0, \{p_0\})$, where $\delta_A$ = $\{(p_0, \sigma, p_0)$ $\mid \sigma = b, c\}$.
\end{itemize}

Let us consider $w = a$. The accepting run of $\cT$ on $w$ is $q_0 \xrightarrow{\varepsilon} q_1 \xrightarrow[x_0:=x_0c]{\varepsilon} q_0 \xrightarrow[x_0:=x_0a]{a} q_0 \xrightarrow{\varepsilon} q_1 \xrightarrow[x_0:=x_0c]{\varepsilon} q_2$, producing an output $cac \not \in \Lang(\cA)$. Therefore, $a \not \in \cR_\cT^{-1}(\Lang(\cA))$. Nevertheless, if we consider the FA $\cB'$ constructed from $\cT$ and $\cA$,  it turns out that $\cB'$ does accept $w$, witnessed by the run $(q_0, \{(p_0,p_0)\}) \xrightarrow{\varepsilon} (q_1, \{(p_0,p_0)\}) \xrightarrow{a} (q_1, \{(p_0, p_0)\}) \xrightarrow{\varepsilon}  (q_2, \{(p_0, p_0)\})$. On the other hand, the run of $\cB'$ corresponding to the accepting run of $\cT$ on $w$, i.e. $(q_0, \{(p_0,p_0)\}) \xrightarrow{\varepsilon} (q_1, \{(p_0,p_0)\}) \xrightarrow{\varepsilon} (q_0, \{(p_0, p_0)\}) \xrightarrow{a}  (q_0, \emptyset) \xrightarrow{\varepsilon} (q_1, \emptyset) \xrightarrow{\varepsilon} (q_2, \emptyset)$, is not accepting, where $\{(p_0,p_0)\}$ and $\emptyset$ are the $\cA$-abstractions of $x_0$.
\end{example}

\begin{figure}[ht]
\centering
%\rule{\linewidth}{0cm}
\includegraphics[scale=0.8]{pre-image-counter-example.pdf}
\caption{A counterexample to disprove the flawed pre-image construction method}
\label{fig-pre-image-count-exmp}
\end{figure}

\begin{proof}[Proof of Theorem~\ref{theorem:psst_preimage}]
Let $\psst = (Q_T, \Sigma$, $X, \delta_T, \tau_T, E_T,  q_{0, T}, F_T)$ be a \PSST{} and $\Aut
  = (Q_A, \Sigma, \delta_A, q_{0, A}, F_A)$ be an \FA{}. For convenience, we use $\cE(\tau_T)$ to denote $\{(q, q') \mid q' \in \tau_T(q)\}$.

Our goal is to construct a FA $\cB$ that simulates the run of $\psst$ on $w$, and, for each $x \in X$, records an $\Aut$-abstraction of the string stored in $x$, that is, the set of state pairs $(p, q) \in Q_A \times Q_A$ such that starting from $p$, $\Aut$ can reach $q$ after reading the string stored in $x$. To simulate the runs of $\psst$, it is necessary to record all the states accessible from a run of higher priority to ensure the current run is an accepting run of $\psst$ of highest priority. Moreover, $\cB$ also remembers the set of $\varepsilon$-transitions of $\cT$ after the latest non-$\varepsilon$-transition to ensure that no transition occurs twice in a sequence of $\varepsilon$-transitions of $\cT$.

Specifically, each state of $\cB$ is of the form $(q, \rho, \Lambda, S)$, where $q \in Q_T$, $\rho \in (\cP(Q_A \times Q_A ))^{X}$, $\Lambda \subseteq \cE(\tau_T)$, and $S \subseteq Q_T$. Note that when recording in $S$ all the states accessible from a run of higher priority, we do not take the non-repetition of $\varepsilon$-transitions into consideration since if a state is reachable by a sequence of $\varepsilon$-transitions where some $\varepsilon$-transitions are repeated, then there exists also a sequence of non-repeated $\varepsilon$-transitions reaching the state. Moreover, when simulating a $\sigma$-transition of $\cT$ (where $\sigma \in \Sigma$) at a state $(q, \rho, \Lambda, S)$, $\cB$ should saturate the states in $S$ by $\varepsilon$-transitions, that is, compute the set of states that are reachable from the states in $S$ by a sequence of $\varepsilon$-transitions, then apply a $\sigma$-transition to them. For technical reasons, when constructing $\cB$, we assume that this saturation happens when a state is added to $S$ for the first time. Therefore, at a state $(q, \rho, \Lambda, S)$, all the states reachable from the states in $S$ by sequences of $\varepsilon$-transitions in $\cT$ have already been in $S$.
%ready for taking non-$\varepsilon$ transitions, 

Before the construction of $\cB$, we introduce some notations.
\begin{itemize}
\item For $S \subseteq Q_T$, $\delta^{(ip)}_T(S, a) = \{q'_1 \mid \exists q_1 \in S, q'_1 \in \delta_T(q_1, a)\}$.
%
\item For $q \in Q_T$,  if $\tau_T(q) = (P_1, P_2)$, then $\tau^{(ip)}_T(\{q\})=S$ such that $S = P_1 \cup P_2$. 
Moreover, for $S \subseteq Q_T$, we define $\tau^{(ip)}_T(S) = \bigcup \limits_{q \in S} \tau^{(ip)}_T(\{q\})$. We also use $\big(\tau^{(ip)}_T\big)^\ast$ to denote the $\varepsilon$-closure of $\cT$, namely, $\big(\tau^{(ip)}_T\big)^\ast(S) = \bigcup \limits_{n \in \Nat} \big(\tau^{(ip)}_T\big)^{n}(S)$, where $\big(\tau^{(ip)}_T\big)^{0}(S) = S$, and for $n \in \Nat$, $\big(\tau^{(ip)}_T\big)^{n+1}(S) = \tau^{(ip)}_T\big(\big(\tau^{(ip)}_T\big)^{n}(S)\big)$. 
%
\item For $S \subseteq Q_T$ and $\Lambda \subseteq  \cE(\tau_T)$, we use $\big(\tau^{(ip)}_T \backslash \Lambda\big)^\ast(S)$ to denote the set of states reachable from $S$ by sequences of $\varepsilon$-transitions where {\it no} transitions $(q, \varepsilon, q')$ such that $(q, q') \in \Lambda$ are used.
%
%We also use $(\tau^{(ip)}_T)^\ast$ to denote the reflexive transitive closure of $\tau^{(ip)}_T$. \tl{here $(\tau^{(ip)}_T)$ is defined as a function... you mean function composition?}
%\zhilei{I think we can just use the term 'epsilon closure' here?}
%\item For $\sigma \in \Sigma$ and $S \subseteq Q_T$,  we use $\tau^+_T[a, S]$ to denote the set of states that can be obtained from 
% 
\item For $\rho \in (\cP(Q_A \times Q_A ))^{X}$ and $s \in X \rightarrow (X \cup \Sigma)^{\ast}$, we use $s(\rho)$ to denote $\rho'$ that is obtained from $\rho$ as follows: For each $x \in X$, if $s(x) = \varepsilon$, then $\rho'(x) = \{(p, p) \mid p \in Q_A\}$, otherwise, let $s(x) = b_1 \cdots b_\ell$ with $b_i \in \Sigma \cup X$ for each $i \in [\ell]$, then $\rho'(x) = \theta_1 \circ \cdots \circ \theta_\ell$, where $\theta_i = \delta^{(b_i)}_A$ if $b_i \in \Sigma$, and $\theta_i = \rho(b_i)$ otherwise.
\end{itemize}

We are ready to present the formal construction of $\cB =  (Q_B$, $\Sigma$, $\delta_B$, $q_{0, B}, F_B)$. 
\begin{itemize}
\item $Q_B = Q_T \times (\cP(Q_A \times Q_A ))^{X} \times \cP(\cE(\tau_T)) \times \cP(Q_T)$, 
%(Intuitively, the letter $\sigma$ in $(q, \sigma, \rho, S) \in Q_B$ means the next letter to be read at $q$, with $\bot$ represents the end of the input.)

\item $q_{0, B} = (q_{0,T}, \rho_{\varepsilon}, \emptyset, \emptyset)$ where $\rho_{\varepsilon} (x) = \{(q, q) \mid q \in Q\}$ for each $x \in X$,

\item $\delta_{B}$ comprises 
\begin{itemize}
%\item the tuples $(q'_0, \varepsilon, ((q_{0,T},\sigma), \rho_{\varepsilon}, \emptyset))$ where $\sigma \in \Sigma$, $\rho_{\varepsilon} (x) = \{(q, q) \mid q \in Q\}$ for each $x \in X$, 
%
\item the tuples $((q, \rho, \Lambda, S), \sigma, (q_i, \rho', \Lambda', S'))$ such that  
%there exists $s \in \left((X \cup \Sigma\right)^*)^X)$ satisfying
\begin{itemize}
\item $\sigma \in \Sigma$, 
%$\sigma' \in \Sigma \cup \{\bot\}$, 
%
\item $\delta_T (q, \sigma) = (q_1, \ldots, q_i, \ldots, q_m)$, 
%
\item $s = E(q, \sigma, q_i)$, 
%
\item $\rho' = s(\rho)$,
%
\item $\Lambda' = \emptyset$, (Intuitively, $\Lambda$ is reset.)
%
\item let $\tau_T(q) = (P_1, P_2)$, then $S' = \big(\tau^{(ip)}_T\big)^\ast\big(\{ q_1$, $\ldots$, $q_{i - 1} \} \cup \delta^{(ip)}_T\big(S \cup \big(\tau^{(ip)}_T \setminus \Lambda\big)^\ast(P'_1), \sigma\big)\big)$, where $P'_1 = \{q' \in P_1 \mid (q, q') \not \in \Lambda\}$;  (Note that according to the semantics of PSSTs, when computing the set of states reachable from $q$ through an $\varepsilon$-transition to some $q' \in P_1$ first and a sequence of $\varepsilon$-transitions starting from $q'$ next, the transitions $(q'', \varepsilon, q''')$ with $(q'', q''') \in \Lambda$ should be excluded. )
%(Intuitively, we assume that $S$ is the current set of states ready for taking non-$\varepsilon$-transitions and has already included the states belonging to its $\varepsilon$-closure, moreover, $S'$ is the $\varepsilon$-closure of the set of states reached after taking a $\sigma$-transition.)
%the set of states that can be reached from $S \cup (\tau^{(ip)}_T)^\ast(P_1)$ by a sequence of $\varepsilon$-transitions, followed by a $\sigma$-transition, 
%more precisely, suppose $(\tau^{(ip)}_T)^\ast(S \cup P_1) = S''$, then $S' =  \delta^{(ip)}_T(S'', \sigma)$ $\cup$ $\{ q_1$, $\ldots$, $q_{i - 1} \}$; 
%
%$\rho'(x) = \theta_\ell$ such that $\theta_0 = \{(p,p) \mid p \in Q_A\}$, and for each $i \in [\ell]$, if $b_i \in \Sigma$, then $\theta_i = \{(p, p') \mid (p, p'') \in \theta_{i-1}, (p'', b_i, p') \in \delta_A \mbox{ for some } p''\}$, otherwise, $\theta_i = \theta_{i-1} \cdot \rho(x)$. 
\end{itemize}
%
\item the tuples $((q, \rho, \Lambda, S), \varepsilon, (q_i, \rho', \Lambda', S'))$ such that 
%there exists $s \in \left((X \cup \Sigma\right)^*)^X$ satisfying
\begin{itemize}
%, $\sigma' \in \Sigma \cup \{\bot\}$, 
%
\item $\tau_T(q) = ((q_1, \ldots, q_i, \ldots, q_m); \cdots)$, 
%
\item $(q, q_i) \not \in \Lambda$,

\item $s = E(q, \varepsilon, q_i)$, 
%
\item $\rho' = s(\rho)$,
%
\item $\Lambda' = \Lambda \cup \{(q, q_i)\}$, 
%
\item $S' =  S \cup \big(\tau^{(ip)}_T \backslash \Lambda \big)^\ast(\{ q_j \mid j \in [i-1], (q, q_j) \not \in \Lambda \})$;
%
%$\rho'(x) = \theta_\ell$ such that $\theta_0 = \{(p,p) \mid p \in Q_A\}$, and for each $i \in [\ell]$, if $b_i \in \Sigma$, then $\theta_i = \{(p, p') \mid (p, p'') \in \theta_{i-1}, (p'', b_i, p') \in \delta_A \mbox{ for some } p''\}$, otherwise, $\theta_i = \theta_{i-1} \cdot \rho(x)$. 
\end{itemize}
%

\item the tuples $((q, \rho, \Lambda, S), \varepsilon, (q_i, \rho', \Lambda', S'))$ such that 
%there exists $s \in \left((X \cup \Sigma\right)^*)^X$ satisfying
\begin{itemize}
%\item $\sigma \in \Sigma$, 
%$\sigma' \in \Sigma \cup \{\bot\}$, 
%
\item $\tau_T (q) = ((q'_1, \ldots, q'_n); (q_1, \ldots, q_i, \ldots, q_m))$, 
%
\item $(q, q_i) \not \in \Lambda$,
%
\item $s = E(q, \varepsilon, q_i)$,
%
\item $\rho' = s(\rho)$,
%
\item $\Lambda' = \Lambda \cup \{(q, q_i)\}$, 
%
\item $S' = S \cup \{q\} \cup \big(\tau^{(ip)}_T \backslash \Lambda \big)^\ast\big(\big\{q'_j \mid j \in [n], (q, q'_j) \not \in \Lambda \big\} \cup \big\{q_j \mid j \in [i-1], (q, q_j) \not \in \Lambda \big\} \big)$. (Note that here we include $q$ into $S'$, since the non-$\varepsilon$-transitions out of $q$ have higher priorities than the transition $(q, \varepsilon, q_i)$.)
%
%$\rho'(x) = \theta_\ell$ such that $\theta_0 = \{(p,p) \mid p \in Q_A\}$, and for each $i \in [\ell]$, if $b_i \in \Sigma$, then $\theta_i = \{(p, p') \mid (p, p'') \in \theta_{i-1}, (p'', b_i, p') \in \delta_A \mbox{ for some } p''\}$, otherwise, $\theta_i = \theta_{i-1} \cdot \rho(x)$. 
\end{itemize}
\end{itemize}
\item
Moreover, $F_B$ is the set of states $(q, \rho, \Lambda, S) \in Q_B$ such that
\begin{enumerate}
  \item $F_T (q)$ is defined,
%
  \item for any $q' \in S$, $F_T (q')$ is not defined,
  %
  \item if $F_T(q) = \varepsilon$, then $q_{0, A}  \in F_A$, otherwise, 
let $F_T(q) = b_1 \cdots b_\ell$ with $b_i \in \Sigma \cup X$ for each $i \in [\ell]$, then $(\theta_1 \circ \cdots \circ \theta_\ell) \cap (\{q_{0,A}\} \times F_A) \neq \emptyset$, where for each $i \in [\ell]$, if $b_i \in \Sigma$, then $\theta_i = \delta^{(b_i)}_A$, otherwise, $\theta_i = \rho(b_i)$.
\end{enumerate}
\end{itemize}
\end{proof}

Note that the above construction  does not utilize the so-called \tmtextit{copyless} property \cite{AC10,AD11},
  thus it works for general, or \textit{copyful} \PSST{} \cite{FR17}.

\begin{example}
Figure \ref{fig-psst-preimage-exmp} gives the correct FA $\cB$ encoding $\cR^{-1}_{\cT}(\Lang(\Aut))$ in Example \ref{pre-image-count-examp} using the construction method introduced above. Only those states reachable from the initial state $r_0$ are shown. Table \ref{table:psst-preimage} shows the correspondence between the states in the figure and the states of $\cB$, where $\rho_1(x)=\{(p_0,p_0)\}$ and $\rho_2(x)=\emptyset$.

Note that many states are redundant. For efficiency, we minimize the FA after the construction. See Section \ref{sect:impl} for implementation details.
\end{example}


\begin{figure}[ht]
\centering
%\rule{\linewidth}{0cm}
\includegraphics[scale=0.8]{psst-preimage-example.pdf}
\caption{FA $\cB$ that encodes $\cR^{-1}_{\cT}(\Lang(\Aut))$}
\label{fig-psst-preimage-exmp}
\end{figure}

\begin{table}[t]
\centering
\caption{the actual $\cB$ state in Figure 
\label{table:psst-preimage}
\ref{fig-psst-preimage-exmp}}
\begin{tabular}{|c|c|}
    \hline
    Symbol & State of $\cB$\\
    \hline
    $r_0$ & $(q_0, \rho_1, \emptyset, \emptyset)$\\
    \hline
    $r_1$ & $(q_1, \rho_1, \{ (q_0, q_1) \}, \emptyset)$\\
    \hline
    $r_2$ & $(q_2, \rho_1, \{ (q_0, q_1), (q_1, q_2) \}, \{ q_0 \})$\\
    \hline
    $r_3$ & $(q_2, \rho_2, \emptyset, \{ q_0, q_1, q_2 \})$\\
    \hline
    $r_4$ & $(q_2, \rho_1, \emptyset, \{ q_0, q_1, q_2 \})$\\
    \hline
    $r_5$ & $(q_0, \rho_1, \{ (q_0, q_1) (q_1, q_0) \}, \emptyset)$\\
    \hline
    $r_6$ & $(q_0, \rho_2, \emptyset, \emptyset)$\\
    \hline
    $r_7$ & $(q_0, \rho_2, \emptyset, \{ q_0, q_1, q_2 \})$\\
    \hline
    $r_8$ & $(q_1, \rho_2, \{ (q_0, q_1) \}, \{ q_0, q_1, q_2 \})$\\
    \hline
    $r_9$ & $(q_0, \rho_2, \{ (q_0, q_1) (q_1, q_0) \}, \{ q_0, q_1, q_2 \})$\\
    \hline
    $r_{10}$ & $(q_2, \rho_2, \{ (q_0, q_1) (q_1, q_2) \}, \{ q_0, q_1, q_2 \})$\\
    \hline
    $r_{11}$ & $(q_0, \rho_1, \emptyset, \{ q_0, q_1, q_2 \})$\\
    \hline
    $r_{12}$ & $(q_1, \rho_1, \{ (q_0, q_1) \}, \{ q_0, q_1, q_2 \})$\\
    \hline
    $r_{13}$ & $(q_0, \rho_1, \{ (q_0, q_1) (q_1, q_0) \}, \{ q_0, q_1, q_2 \})$\\
    \hline
    $r_{14}$ & $(q_2, \rho_1, \{ (q_0, q_1) (q_1, q_2) \}, \{ q_0, q_1, q_2 \})$\\
    \hline
    $r_{15}$ & $(q_2, \rho_2, \{ (q_0, q_1) (q_1, q_2) \}, \{ q_0 \})$\\
    \hline
    $r_{16}$ & $(q_1, \rho_2, \emptyset, \{ q_0, q_1, q_2 \})$\\
    \hline
    $r_{17}$ & $(q_2, \rho_2, \{ (q_1, q_2) \}, \{ q_0, q_1, q_2 \})$\\
    \hline
    $r_{18}$ & $(q_0, \rho_2, \{ (q_1, q_0) \}, \{ q_0, q_1, q_2 \})$\\
    \hline
    $r_{19}$ & $(q_1, \rho_2, \{ (q_1, q_0) (q_0, q_1) \}, \{ q_0, q_1, q_2 \})$\\
    \hline
    $r_{20}$ & $(q_2, \rho_2, \{ (q_1, q_0) (q_0, q_1) (q_1, q_2) \}, \{ q_0, q_1, q_2 \})$\\
    \hline
    $r_{21}$ & $(q_1, \rho_2, \{ (q_0, q_1) \}, \emptyset)$\\
    \hline
    $r_{22}$ & $(q_0, \rho_2, \{ (q_0, q_1) (q_1, q_0) \}, \emptyset)$\\
    \hline
\end{tabular}

\end{table}


% Note that in the definition of \NSST, there is no \emph{copyless} restriction.






%%%%%%%%%%%%%%%%%%%%%%%%%%%%%%%%%%%%%%%%%%
\subsection{From regular expressions to PSSTs}

%!TEX root = main.tex



%We are ready to define prioritized streaming string transducers. 
%In the definition, 
%
%The special symbol $\nullchar$ is introduced to capture the situation that $\extract_{i,e}(x)$ returns $\nullchar$, i.e. $x \in \Lang(e)$ but the $i$-th capturing group of $e$ is not matched.
%

\paragraph{Semantics}
We now define the formal semantics of $\regexp$. Traditionally, regular expressions are interpreted as a regular language, i.e., a set of strings, which can be defined inductively in a rather straightforward way. In our case when the regular expression is used in string constraints arisen from analysis of programming language such as JavaScript, %owing to the introduction of greedy/lazy semantics,  
what we need is not only the language denoted by the regular expression, but also the intermediate result when parsing a string against the given regular expression. This is especially the case when the capturing group is involved. As a result, we need a more operational (as opposed traditional denotational) account of the semantics for regular expression. To this end, we harness an extension of finite-state automata with priorities, which \emph{defines} how a string is accepted by the given regular expression. We start with the standard finite-state automaton.  


%\subsection{Semantics of \regexp[\sf CG]}
%In this section, we give one of the many semantics of \regexp[\sf CG], which we will utilize for $\replaceall$.
%For two indexed $\regexp$s $e$ and $e'$, we say $e'$ is a \emph{subexpression} of $e$,
%if one of the following conditions holds: 1) $e'=e$, 2) $e = [e_1 \cdot e_2]$ or $[e_1 + e_2]$, and $e'$ is a subexpression of $e_1$ or $e_2$, 3) $e = [e^?], [e^{??}], [e_1^{\ast}]$, $[e_1^{+}]$, $[e_1^{\ast?}]$, $[e_1^{+?}]$, $e_1^{\{m_1, m_2\}}$, $e_1^{\{m_1, m_2\}?}$ or $(_n e_1)_n$, and $e'$ is a subexpression of $e_1$. We use $S(e)$ to denote the set of all subexpressions of $e$. %\tl{is there a difference between $[e_1\cdot e_2]$ and $e_1 e_2$?}
%
% 
%By a mutual induction on $|w|$ and $|e|$, we can show that $|\cM_{w}(e)|$, the size of $\cM_{w}(e)$, is at most $|w||e|$.  

%\begin{example}\label{exmp-regex-match-tree}
%	Let $w= 0250$ and $e = [[([\Gamma^+])\concat .?] \concat ([\Gamma^*])]$ where $\Gamma = \{0,1,\cdots,9\}$. Note that $e$ is essentially {\tt decimalReg} in the motivating example. Then $\cM_{w}(e) = \{T_1,T_2,T_3, T_4\}$ as illustrated in Figure~\ref{fig-regex-semantics-decimal}(i), (ii), (iii), and (iv), where the match trees rooted at $(0, \Gamma)$, $(2, \Gamma)$, and $(5, \Gamma)$ are omitted. % to avoid tediousness.
%	\begin{figure}[htb]
%		\centering
%		%\rule{\linewidth}{0cm}
%		\includegraphics[width=1\textwidth]{regex-semantics-decimal.pdf}
%		\caption{Match trees of $e=[[([\Gamma^+])\concat .?] \concat ([\Gamma^*])]$ to $w= 0250$}
%		\label{fig-regex-semantics-decimal}
%		
%	\end{figure}
%	%\Blindtext
%\end{example}

 
 
%
%\begin{example}\label{exmp-regex-semantics}
%	Let us continue Example~\ref{exmp-regex-match-tree}.  In Fig.~\ref{fig-regex-semantics-decimal}, we have $(T_1)_{(0250, [\Gamma^+])} >_{(w, \idxexp(e))} (T_2)_{(025, [\Gamma^+])}$, since $(0, \Gamma)(2, \Gamma)(5,\Gamma)$ is a proper prefix of $(0, \Gamma)(2, \Gamma)(5,\Gamma)(0, \Gamma)$. Then we deduce that $(T_1)_{(0250, (_1[\Gamma^+])_1)} >_{(w, \idxexp(e))} (T_2)_{(025, (_1[\Gamma^+])_1)}$. Consequently, $(T_1)_{(0250, [(_1[\Gamma^+])_1 \concat .?])} >_{(w, \idxexp(e))} (T_2)_{(025,  [(_1[\Gamma^+])_1 \concat .?])}$ and $T_1 >_{(w, \idxexp(e))} T_2$. Similarly, we have $T_2 >_{(w, \idxexp(e))} T_3$ and $T_3 >_{(w, \idxexp(e))} T_4$. Therefore, $T_1$ is the accepting match of $e$ to $w$, where the first and second capturing group of $e$ are matched to $0250$  and $\varepsilon$ respectively. 
%	%\Blindtext
%\end{example}

%\begin{remark}
%	Our semantics of $\regexp$ follows the 11th Edition of the ECMAScript specification (ES11 for short) \cite{ECMAScript11}, with a focus on the non-commutative union, the greedy/lazy semantics of Kleene star/plus, as well as capturing groups and backreferences.
%	In comparison, POSIX regular expressions require the leftmost and longest match of regular expressions, which we leave as future work.
%\end{remark}


% some examples

%  e = b(a*)a*
%  
%  e' = b(a*?)a*

%\begin{figure}[ht]
%\centering
%\rule{\linewidth}{0cm}
%\includegraphics[scale=0.8]{pfa-epsilon-star.pdf}
%\caption{The PFA for $\varepsilon^\ast$}
%\label{fig-pfa-epsilon-star}
%\end{figure}



%%%%%%%%%%%%%%%%%%%%%%%%%%%%%%%%%%%%%%%%%%%%%%%%%%%%%%%%%%%%%%%%%%%%%%%%%%%%%%%%%%%%%%%%%%%%

%We can adapt the PFA construction in \cite{BDM14}, which in turn is a variant of the standard Thompson construction \cite{Thompson68}, and show the following result. 

%\begin{proposition}\label{prop-rwre-to-pfa}

%We associate with each regular expression $e$ a PFA $\cA_e$ and define the semantics of $e$ as the language accepted by $\cA_e$. As expected, the PFA $\cA_e$ is constructed inductively. 

\paragraph{Case $e =\emptyset$} $\cA_e = (\{q_{\emptyset, 0}\}, \Sigma, \emptyset, \delta, \tau, \emptyset, q_{\emptyset, 0}, (\emptyset, \emptyset))$, where $\delta(q_{\emptyset, 0}, a) = ()$ for every $a \in \Sigma$; $\tau(q_{\emptyset, 0}) = ((); ())$.
		

\paragraph{Case $e = \varepsilon$} $\cA_e = (\{q_{\varepsilon, 0}, f_{\varepsilon,0}\}, \Sigma, \{x \}, \delta, \tau, E, q_{\varepsilon,0}, (\{f_{\varepsilon,0}\}, \emptyset))$, 
%
where $\delta(q_{\varepsilon,0}, a) = \delta(f_{\varepsilon,0}, a) = ()$ for every $a \in \Sigma$; $\tau(q_{\varepsilon,0}) = ((f_{\varepsilon,0}); ())$;  $\tau(f_{\varepsilon,0}) = ((); ())$; for each transition $(q, a, q')$, $E(q,a,q')(x) =xa$.
		
\paragraph{Case $e = a$} $\cA_e = (\{q_{a,0}, q_{a,1}, f_{a,0}\}, \Sigma, \{x\}, \delta, \tau, E, q_{a,0}, (\emptyset, \{f_{a,0}\}))$, where 
$\delta(q_{a,0}, b) = ()$ for every $b \in \Sigma$, $\delta(q_{a,1}, a) = (f_{a,0})$, $\delta(q_{a,1}, b) = ()$ for every $b \in \Sigma \setminus \{a\}$; 
%
$\tau(q_{a,0}) = ((q_{a,1}); ())$, $\tau(q_{a,1}) = ((); ())$, and $\tau(f_{a,0}) = ((); ())$; 
%
for each transition $(q, a, q')$, $E(q,a,q')(x) =xa$.
%		
\begin{figure}[ht]
			\centering
			%\rule{\linewidth}{0cm}
			\includegraphics[width = 0.4\textwidth]{reg2pfa-0.pdf}
			\caption{The PFA $\cA_{\emptyset}$, $\cA_{\varepsilon}$, and $\cA_{a}$ }
			\label{fig-reg2pfa-0}
\end{figure}  
%%%%%%%%%%%%%%%%%%%%%%%%%%%%%%%%%%%%%%%%%%%%%%%%%%%%%%%%%%%%%%%%%%%%%%%%%%%%%%%%%%%%%%%%%%%%%%%%%%%%

		
\paragraph{Case $e = (e_1)$} $\cA_e = \cA_{e_1}$.
		

\paragraph{Case $e = [e_1 + e_2]$} Let 
\[\cA_{e_1} = (Q_{e_1}, \Sigma, X_1, \delta_{e_1}, \tau_{e_1}, E_1,  q_{e_1,0}, (F_{e_1,1}, F_{e_1,2}))\] and 
\[\cA_{e_2} = (Q_{e_2}, \Sigma, X_2, \delta_{e_2}, \tau_{e_2}, E_2, q_{e_2,0}, (F_{e_2, 1}, F_{e_2,2}))\] 
with $X_1\cap X_2=\emptyset$. 
Then 
\[\cA_e = (Q_{e_1} \cup Q_{e_2} \cup \{q_{e,0}\}, \Sigma, X_1\cup X_2, 
		\delta_e, \tau_e, E, q_{e,0}, (F_{e_1,1} \cup F_{e_2,1}, F_{e_1,2} \cup F_{e_2,2}))\] where  
		\begin{itemize}
			\item $q_{e,0}  \not \in Q_{e_1} \cup Q_{e_2}$, 
			\item $\delta_e(q, a) = \delta_{e_i}(q, a)$ for every $i \in \{1,2\}$, $q \in Q_{e_i}$ and $a \in \Sigma$, 
			$\delta_e(q_{e,0}, a)  = ()$ for every $a \in \Sigma$, 
			%
			\item $\tau_e(q) = \tau_{e_i}(q)$ for every $q \in Q_{e_i}$ ($i =1,2$), $\tau_e(q_{e,0}) = ((q_{e_1,0},q_{e_2,0}); ())$.
			\item for each transition $(q, a, q')$ from $\delta_i$ for $i\in\{1,2\}$, $E(q,a,q')(x) =xa$ and $E(q_{e,0},a,q')(x) =\varepsilon$
		\end{itemize}
Fig.~\ref{fig-reg2pfa-1} depicts the construction.  	
		\begin{figure}[ht]
			\centering
			%\rule{\linewidth}{0cm}
			\includegraphics[width = 0.6\textwidth]{reg2pfa-1.pdf}
			\caption{The PFA $\cA_{[e_1+e_2]}$}
			\label{fig-reg2pfa-1}
		\end{figure}  

%%%%%%%%%%%%%%%%%%%%%%%%%%%%%%%%%%%%%%%%%%%%%%%%%%%%%%%%%%%%%%%%%%%%%%%%%%%%%%%%%%%%%%%%%%%%%%%%%%%%%

\paragraph{Case $e = [e_1^?]$} Let $\cA_{e_1} = (Q_{e_1}, \Sigma, X, \delta_{e_1}, \tau_{e_1}, E, q_{e_1,0}, (F_{e_1,1}, F_{e_1,2}))$. Then 
\[\cA_e = (Q_{e_1} \cup \{q_{\varepsilon}, q_{e,0}\}, \Sigma, X, 
		\delta_e, \tau_e, E', q_{e,0}, (\{q_{\varepsilon}\}, F_{e_1,2}))\]
where  
		\begin{itemize}
			\item $q_{e,0}  \not \in Q_{e_1}$
			\item $\delta_e(q, a) = \delta_{e_1}(q, a)$ for every $q \in Q_{e_1}$ and $a \in \Sigma$, $\delta_e(q_{e,0}, a)  = ()$ and $\delta_e(q_{\varepsilon}, a) = ()$ for every $a \in \Sigma$, 
			%
			\item $\tau_e(q) = \tau_{e_1}(q)$ for every $q \in Q_{e_1}$, $\tau_e(q_{e,0}) = ((q_{e_1,0},q_{\varepsilon}); ())$,
			\item for each transition $(q, a, q')$ from $\delta_{e_1}$, $E'(q,a,q')(x) =xa$ and $E(q_{e,0},a,q')(x) =\varepsilon$
		\end{itemize}
%
Fig.~\ref{fig-reg2pfa-6} depicts the construction.
		\begin{figure}[ht]
			\centering
			%\rule{\linewidth}{0cm}
			\includegraphics[width = 0.6\textwidth]{reg2pfa-6.pdf}
			\caption{The PFA $\cA_{[e_1^?]}$}
			\label{fig-reg2pfa-6}
		\end{figure}

%%%%%%%%%%%%%%%%%%%%%%%%%%%%%%%%%%%%%%%%%%%%%%%%%%%%%%%%%%%%%%%%%%%%%%%%%%%%%%%%%%%%%%%%%%%%%%%%%%%%%
\paragraph{Case $e = [e_1^{??}]$} 
Let $\cA_{e_1} = (Q_{e_1},
		\Sigma, \delta_{e_1}, \tau_{e_1}, q_{e_1,0}, (F_{e_1,1}, F_{e_1,2}))$. 
Then 
\[\cA_e = (Q_{e_1} \cup \{q_{\varepsilon}\}, \Sigma,
		\delta_e, \tau_e, q_{e,0}, (\{q_{\varepsilon}\}, F_{e_1,2}))\] 
where 
		\begin{itemize}
			\item $q_{e,0}  \not \in Q_{e_1}$
			\item $\delta_e(q, a) = \delta_{e_1}(q, a)$ for every $q \in Q_{e_1}$ and $a \in \Sigma$, $\delta_e(q_{e,0}, a)  = ()$ and $\delta_e(q_{\varepsilon}, a) = ()$ for every $a \in \Sigma$, 
			%
			\item $\tau_e(q) = \tau_{e_1}(q)$ for every $q \in Q_{e_1}$, $\tau_e(q_{e,0}) = ((q_{\varepsilon}, q_{e_1,0}); ())$.
		\end{itemize}
Fig.~\ref{fig-reg2pfa-7} depicts the construction. 
		\begin{figure}[ht]
			\centering
			%\rule{\linewidth}{0cm}
			\includegraphics[width = 0.5\textwidth]{reg2pfa-7.pdf}
			\caption{The PFA $\cA_{[e_1^{??}]}$}
			\label{fig-reg2pfa-7}
		\end{figure}
	
%%%%%%%%%%%%%%%%%%%%%%%%%%%%%%%%%%%%%%%%%%%%%%%%%%%%%%%%%%%%%%%%%%%%%%%%%%%%%%%%%%%%%%%%%%%%%%%%%%%%%
\paragraph{Case $e = [e_1 \concat e_2]$} 
Let $\cA_{e_1} = (Q_{e_1}, \Sigma, \delta_{e_1}, \tau_{e_1}, q_{e_1,0}, (F_{e_1,1}, F_{e_1,2}))$, $\cA_{e_2} = (Q_{e_2}, \Sigma, \delta_{e_2}, \tau_{e_2}, q_{e_2,0}, (F_{e_2, 1}, F_{e_2,2}))$, and $\cA'_{e_2} = (Q'_{e_2}, \Sigma, \delta'_{e_2}, \tau'_{e_2}, q'_{e_2,0}, (F'_{e_2, 1}, F'_{e_2,2}))$ be a fresh copy of $\cA_{e_2}$. Then 
%
\[\cA_e = ( Q_{e_1} \cup Q_{e_2} \cup Q'_{e_2}, \Sigma, \delta_e, \tau_e, q_{e_1,0}, (F_{e_2,1}, F_{e_2,2} \cup F'_{e_2,1} \cup F'_{e_2,2}))\] where 
	\begin{itemize}
	 \item for every $i \in \{1,2\}$, $q \in Q_{e_i}$ and $a \in \Sigma$, $\delta_e(q, a) = \delta_{e_i}(q, a)$,
			%
	\item for every $q' \in Q'_{e_2}$ and $a \in \Sigma$, $\delta_e(q', a) = \delta'_{e_2}(q',a)$, 
			%    
	\item for every $q \in Q_{e_2}$, $\tau_e(q) = \tau_{e_2}(q)$ and $\tau_e(q') = \tau'_{e_2}(q')$, 
			%
	\item for every $q \in Q_{e_1} \setminus (F_{e_1,1} \cup F_{e_1,2})$, $\tau_e(q) = \tau_{e_1}(q)$, for every $f_{e_1,1} \in F_{e_1,1}$, $\tau_e(f_{e_1,1}) = ((q_{e_2,0}); ())$, and for every $f_{e_1,2} \in F_{e_1,2}$, $\tau_e(f_{e_1,2}) = ((q'_{e_2,0}), ())$.
  \end{itemize}
 Fig.~\ref{fig-reg2pfa-2} depicts the construction. 
		\begin{figure}[ht]
			\centering
			%\rule{\linewidth}{0cm}
			\includegraphics[width = 0.6\textwidth]{reg2pfa-2.pdf}
			\caption{The PFA $\cA_{[e_1\concat e_2]}$}
			\label{fig-reg2pfa-2}
		\end{figure}  
	

%%%%%%%%%%%%%%%%%%%%%%%%%%%%%%%%%%%%%%%%%%%%%%%%%%%%%%%%%%%%%%%%%%%%%%%%%%%%%%%%%%%%%%%%%%%%%%%%%%%%%

\paragraph{Case $e = [e_1^{\ast}]$} 
Let $\cA_{e_1} = (Q_{e_1}, \Sigma, \delta_{e_1}, \tau_{e_1}, q_{e_1,0}, (F_{e_1,1}, F_{e_1,2}))$. Then
\[ \cA_e = (Q_{e_1} \cup \{q_{e,0}, f_{e,0}, f_{e,1}\}, \Sigma, \delta_e, \tau_e, q_{e,0}, (\{f_{e,0}\}, \{f_{e,1}\}))\] where 
		\begin{itemize}
			\item $q_{e,0}, f_{e,0} \not \in Q_{e_1}$,
			
			\item for every $q \in Q_{e_1}$ and $a \in \Sigma$, $\delta_e(q, a) = \delta_{e_1}(q, a)$, 
			%moreover, $\delta(q_0, a) = \delta(f_0, a)  = ()$,
			
			\item for every $q \in Q_{e_1} \setminus (F_{e_1,1} \cup F_{e_1,2})$,  $\tau_e(q) = \tau_{e_1}(q)$, moreover, $\tau_e(q_{e,0}) = ((q_{e_1,0},f_{e,0}); ())$,  $\tau_e(q) = ((q_{e_1,0});())$ for every $q \in F_{e_1,1}$, $\tau_e(q) = ((q_{e_1,0}, f_{e,1});())$ for every $q \in F_{e_1,2}$, $\tau_e(f_{e,0}) =\tau_e(f_{e,1}) = (();())$. (Intuitively, the $\varepsilon$-transitions from $q_{e,0}$ to $q_{e_1,0}$ and $f_{e,0}$, from each $q \in F_{e_1,1}$ to $q_{e_1,0}$, and from $q \in F_{e_1,2}$ to $q_{e_1,0}$ and $f_{e,1}$ respectively are added, moreover, the $\varepsilon$-transition from $q_{e,0}$ to $f_{e,0}$ and from $q \in F_{e_1,2}$ to $f_{e,1}$ are of the lowest priority.)
		\end{itemize}

%%%%%%%%%%%%%%%%%%%%%%%%%%%%%%%%%%%%%%%%%%%%%%%%%%%%%%%%%%%%%%%%%%%%%%%%%%%%%%%%%%%%%%%%%%%%%%%%%%%%%
\paragraph{Case $e = [e_1^{+}]$} Then $\cA_e$ is constructed as $\cA_{[e_1 \concat [e^\ast_1]]}$.
		
%%%%%%%%%%%%%%%%%%%%%%%%%%%%%%%%%%%%%%%%%%%%%%%%%%%%%%%%%%%%%%%%%%%%%%%%%%%%%%%%%%%%%%%%%%%%%%%%%%%%%
\paragraph{Case $e = [e_1^{\ast?}]$} Let $\cA_{e_1} = (Q_{e_1}, \Sigma, \delta_{e_1}, \tau_{e_1}, q_{e_1,0}, (F_{e_1,1}, F_{e_1,2}))$. 
Then 
\[\cA_e = (Q_{e_1} \cup \{q_{e,0}, f_{e,0}, f_{e,1}\}, \Sigma, \delta_e, \tau_e, q_{e,0}, (\{f_{e,0}\}, \{f_{e,1}\}))\]  
where 
		\begin{itemize}
			\item $q_{e,0}, f_{e,0} \not \in Q_{e_1}$,
			
			\item for every $q \in Q_{e_1}$ and $a \in \Sigma$, $\delta_e(q, a) = \delta_{e_1}(q, a)$, 
			%moreover, $\delta(q_0, a) = \delta(f_0, a)  = ()$,
			
			\item for every $q \in Q_{e_1} \setminus (F_{e_1,1} \cup F_{e_1,2})$,  $\tau_e(q) = \tau_{e_1}(q)$, moreover, $\tau_e(q_{e,0}) = ((f_{e,0}, q_{e_1,0}); ())$,  $\tau_e(q) = ((q_{e_1,0});())$ for every $q \in F_{e_1,1}$, $\tau_e(q) = ((f_{e,1}, q_{e_1,0});())$ for every $q \in F_{e_1,2}$, $\tau_e(f_{e,0}) =\tau_e(f_{e,1}) = (();())$. (Intuitively, the $\varepsilon$-transitions from $q_{e,0}$ to $f_{e,0}$ and $q_{e_1,0}$ , from each $q \in F_{e_1,1}$ to  $q_{e_1,0}$, and from each $q \in F_{e_1,2}$ to $f_{e,1}$ and $q_{e_1,0}$ respectively are added, moreover, the $\varepsilon$-transition from $q_{e,0}$ to $q_{e_1,0}$ and from $q \in F_{e_1,2}$ to $q_{e_1,0}$ are of the lowest priority.)
		\end{itemize}

 
%%%%%%%%%%%%%%%%%%%%%%%%%%%%%%%%%%%%%%%%%%%%%%%%%%%%%%%%%%%%%%%%%%%%%%%%%%%%%%%%%%%%%%%%%%%%%%%%%%%%%
\paragraph{Case $e = [e_1^{+?}]$} then $\cA_e$ is constructed as $\cA_{[e_1 \concat [e_1^{*?}]]}$.

Fig.~\ref{fig-reg2pfa-3} depicts the construction. 
\begin{figure}[ht]
	\centering
	%\rule{\linewidth}{0cm}
	\includegraphics[width = 0.8\textwidth]{reg2pfa-3.pdf}
	\caption{The PFA $\cA_{[e_1^\ast]}$ and $\cA_{[e_1^{\ast?}]}$}
	\label{fig-reg2pfa-3}
\end{figure}

%%%%%%%%%%%%%%%%%%%%%%%%%%%%%%%%%%%%%%%%%%%%%%%%%%%%%%%%%%%%%%%%%%%%%%%%%%%%%%%%%%%%%%%%%%%%%%%%%%%%%
\paragraph{Case $e = [e_1^{\{m_1,m_2\}}]$} Then $\cA_e$ is constructed as the concatenation of $\cA_{e_1^{m_1}}$ and $\cA^\prime_{e_1^{\{1,m_2-m_1\}}}$, where $\cA_{e_1^{m_1}}$ is the PFA corresponding to consecutive concatenations of $m_1$ copies of $e_1$, and $\cA^\prime_{e_1^{\{1,m_2-m_1\}}}$ is illustrated in Fig.~\ref{fig-reg2pfa-4}, which consists of $m_2-m_1$ copies of $\cA_{e_1}$, say $(\cA^{(i)}_{e_1})_{i \in [m_2-m_1]}$, as well as the $\varepsilon$-transition from $q^{(1)}_{e_1,0}$ to a fresh state $f^\prime_0$ (of the lowest priority), and the $\varepsilon$-transitions from each $f^{(i)}_{e_1,2} \in F^{(i)}_{e_1,2}$ to $q^{(i+1)}_{e_1,0}$ (of the highest priority), and a fresh state $f^\prime_1$ (of the lowest priority). The accepting states of $\cA^\prime_{e_1^{\{1,m_2-m_1\}}}$ are $(\{f_0'\},\{f_1'\})$. (Intuitively, each $\cA^{(i)}_{e_1}$ accepts only nonempty strings, thus $f^{(i)}_{e_1,1} \in F^{(i)}_{e_1,1}$ contains no outgoing transitions in $\cA^\prime_{e_1^{\{1,m_2-m_1\}}}$. )
		%
		\begin{figure}[ht]
			\centering
			%\rule{\linewidth}{0cm}
			\includegraphics[width = 0.8\textwidth]{reg2pfa-4.pdf}
			\caption{The PFA $\cA^\prime_{e_1^{\{1,m_2-m_1\}}}$}
			\label{fig-reg2pfa-4}
		\end{figure}  


%%%%%%%%%%%%%%%%%%%%%%%%%%%%%%%%%%%%%%%%%%%%%%%%%%%%%%%%%%%%%%%%%%%%%%%%%%%%%%%%%%%%%%%%%%%%%%%%%%%%%
\paragraph{Case $e = [e_1^{\{m_1,m_2\}?}]$} Then $\cA_e$ is constructed as the concatenation of $\cA_{e_1^{m_1}}$ and $\cA^\prime_{e_1^{\{1,m_2-m_1\}?}}$, where $\cA^\prime_{e_1^{\{1,m_2-m_1\}?}}$ is illustrated in Fig.~\ref{fig-reg2pfa-5}, which is the same as $\cA^\prime_{e_1^{\{1,m_2-m_1\}}}$ in Fig.~\ref{fig-reg2pfa-4}, except that the priorities of $\varepsilon$-transition from $q^{(1)}_{e_1,0}$ to $f^\prime_0$ has the highest priority and  the priorities of $\varepsilon$-transitions from each $f^{(i)}_{e_1,2} \in F^{(i)}_{e_1,2}$ to $f^\prime_1$ are reversed.
		\begin{figure}[ht]
			\centering
			%\rule{\linewidth}{0cm}
			\includegraphics[width = 0.8\textwidth]{reg2pfa-5.pdf}
			\caption{The PFA $\cA^\prime_{e_1^{\{1,m_2-m_1\}?}}$}
			\label{fig-reg2pfa-5}
		\end{figure}  
%%%%%%%%%%%%%%%%%%%%%%%%%%%%%%%%%%%%%%%%%%%%%%%%%%%%%%%%%%%%%%%%%%%%%%%%%%%%%%%%%%%%%%%%%%%%%%%%%%%%%%%%

 %can be constructed 
%(in linear time)

% such that 
%	\begin{itemize}
%		\item $\cA_e$ has a unique initial state without incoming transitions and a unique final state without outgoing transitions,
%		%
%		\item for subexpression $e'$ of $e$, $\cA_e$ contains at least one isomorphic copy of $\cA_{e'}$ (i.e. the PFA constructed for $e'$), denoted by ${\sf Sub}_{e'}[\cA_e]$. 
%	\end{itemize}

%Let us use ${\sf Sub}_{e'}[\cA_e]$ to denote the isomorphic copy of $\cA_{e'}$ in $\cA_e$, as mentioned in Proposition~\ref{prop-rwre-to-pfa}.

%\begin{proof}
%	For any $e \in \cgexp$, a PFA $\cA_e$ is constructed recursively in the sequel. The constructed PFA $\cA_e$ satisfies that 
%	\begin{itemize}
%		\item it has a unique initial state without incoming transitions and each of its final states has no outgoing transitions,
%		\item all the transitions out of the initial state are $\varepsilon$-transitions, 
%		\item the set of final states is divided into two disjoint subsets $F_1, F_2$ such that for each $w \in \Sigma^*$ satisfying that $q_0 \xrightarrow[\cA_e]{w} q$ for some $q \in F_1$ (resp. $q \in F_2$), $w = \varepsilon$ (resp. $w \neq \varepsilon$).
%	\end{itemize}



\begin{example}\label{exmp-pfa}
	The PFA corresponding to the RWRE $e = [[([\Gamma^+])\concat .?] \concat ([\Gamma^*])]$ 
	%in Example~\ref{exmp-regex-match-tree}
	%
	is illustrated in Fig.~\ref{fig-pfa}, where the dashed (resp. thicker solid) lines represent the $\varepsilon$-transitions of lower (resp. higher) priorities than non-$\varepsilon$ transitions (if there is any), and the doubly circled states are final states. For instance, $\delta(q_1, \ell) = (q_1)$ for every $\ell \in \{0, \dots, 9\}$, $\delta(q_1, .) = ()$, $\tau(q_1) = ((); (q_2))$. Since the $\varepsilon$-transition has lower priority than the $\ell$-transition at the state $q_1$, whenever the currently scanned letter is $\ell \in \{0,\cdots,9\}$ at $q_1$,  the PFA will choose to go to $q_1$ greedily, until there is no more $\ell  \in \{0,\cdots,9\}$. (In this case, it has to choose the $\epsilon$-transition and goes to $q_2$.)
	%
	\begin{figure}[ht]
		
		\centering
		%\rule{\linewidth}{0cm}
		\includegraphics[width=0.9\textwidth]{pfa-new.pdf}
		\caption{The PFA for $e = [[([\Gamma^+])\concat .?] \concat ([\Gamma^*])]$, where $\Gamma = \{0, \cdots, 9\}$}
		\label{fig-pfa}
		
	\end{figure}
\end{example}



%%%%%%%%%%%%%%%%%%%%%%%%%%%%%%%%%%%%%%%%%%
\subsection{Validation experiments for the formal semantics}

\zhilin{To be written}