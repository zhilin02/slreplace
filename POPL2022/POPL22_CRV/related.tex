\section{Related Work}
\label{sec-related}

%In this section, we will discuss related results. In particular, we will discuss
%(1) results on modelling and reasoning about
%\regexp{} constraints, and (2) results on string constraint solving.


%\subsection
\noindent{\bf Modelling and Reasoning about \regexp{}.}
%This paper is concerned with string constraint solving in general, but the focus is on the interplay of regular expressions in modern programming language and solving constraints involving complex string functions. Both of them are monumental research fields for which we will only discuss the work which are most pertinent to ours. 
%
Variants and extensions of regular expressions to capture their usage in programming languages have received attention %been investigated 
in both theory and practice. In formal language theory, regular expressions with capturing groups and backreferences were considered in \cite{CSY03,CN09} and also more recently in \cite{Freydenberger13,Schmid16,BM17b,FS19}, where the expressibility issues and decision problems were investigated. Nevertheless, some basic features of these regular expression, namely, the non-commutative union and the greedy/lazy semantics of Kleene star/plus, were not addressed therein. In the software engineering community, % have also received attention in the software engineering community. 
some empirical studies were recently reported for these regular expressions, including portability across different programing languages \cite{DMC+19} and DDos attacks \cite{SP18}, as well as how programmers write them in practice \cite{MDD+19}.


Prioritized finite-state automata and %prioritized finite-state 
transducers were proposed in \cite{BM17}. Prioritized finite-state transducers add indexed brackets to the input string in order to identify the matches of capturing groups. It is hard---if not impossible---to use prioritized finite-state transducers to model replace(all) function, e.g., swapping the first and last name as in Example~\ref{exmp-name-swap}. In contrast, {\PSST}s store the matches in string variables, which can then be referred to, allowing us to conveniently model the match and replace(all) function. 
%
Streaming string transducers were used in \cite{ZAM19} to solve the straight-line string constraints with concatenation, finite-state transducers, and regular constraints.

%\subsection
\smallskip
\noindent {\bf String Constraint Solving.}
As we discussed Section \ref{sec-intro}, there has been much research
focussing on string constraint solving algorithms, especially
in the past ten years. Solvers typically use a combination
of techniques to check the satisfiability of string constraints,
including word-based methods, automata-based methods, and unfolding-based methods
like the translation to bit-vector constraints.
We mention among others the following string solvers:
Z3 \cite{Z3}, CVC4 \cite{cvc4}, Z3-str/2/3/4 \cite{Z3-str,Z3-str2,Z3-str3,BerzishMurphy2021},
 ABC \cite{ABC}, Norn
\cite{Abdulla14}, Trau \cite{Z3-trau,AbdullaACDHRR18-trau,Abdulla17}, OSTRICH
\cite{CHL+19}, S2S \cite{DBLP:conf/aplas/LeH18}, Qzy \cite{cox2017model}, Stranger \cite{Stranger}, Sloth
\cite{HJLRV18,AbdullaA+19},
Slog \cite{fang-yu-circuits}, Slent \cite{WC+18}, Gecode+S \cite{DBLP:conf/cpaior/ScottFPS17}, G-Strings \cite{DBLP:conf/cp/AmadiniGST17}, HAMPI
\cite{HAMPI}, and S3 \cite{S3}. 
Most modern string solvers provide support of concatenation and regular 
constraints. The push (e.g. see
\cite{GB16,Vijay-length,HAMPI,Berkeley-JavaScript,LB16,S3})
towards incorporating other functions---e.g. length, 
string-number conversions, replace, replaceAll---in a string theory is an
important theme in the area, owing to the desire to be able to reason 
about complex real-world string-manipulating programs.
These functions, among others, are now part of the SMT-LIB Unicode Strings
standard.\footnote{See
\url{http://smtlib.cs.uiowa.edu/theories-UnicodeStrings.shtml}}

To the best of our knowledge, there is currently no solver that
supports \regexp\ features like greedy/lazing matching or capturing
groups (apart from our own solver \ostrich). This was remarked in
\cite{LMK19}, where the authors try to amend the situation by developing 
\expose{} --- a dynamic symbolic execution engine --- that maps path 
constraints in JavaScript to Z3. The strength of \expose{} is in a thorough
modelling of \regexp{} features, some of which (including backreferences) we do 
not cover in our string constraint language and string solver \ostrich{}. However,
the features that we do not cover are also rare in practice, according to
\cite{LMK19} --- in fact, around 75\% of all the \regexp{} expressions found in
their benchmarks across 415,487 NPM packages can be covered in our fragment.
The strength of \ostrich{} against \expose{} is in a substantial improvement in
performance (by 30--50 fold) and precision. \expose{} does not terminate 
even for simple examples (e.g. for Example \ref{exmp-name-swap} and Example 
\ref{ex:normalize}), which can be solved by our solver within a few seconds.
%we do cover a huge portion of \regexp{} that arise in practice.

%For a recent comparison of the solvers, we refer the
%reader to the survey \cite{Ama20}.

For string constraint solving in general, we refer the readers to the recent survey \cite{Ama20}. In this work, we consider a string constraint language which is undecidable in general, and propose a propagation-based calculus to solve the constraints. However, we also identified a straight-line fragment including concatenation, extract, replace(All) which turns to be decidable. Our decision procedure extends the backward-reasoning approach in \cite{CHL+19}, where only standard one-way and two-way finite-state transducers were considered. 
