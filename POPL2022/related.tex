\section{Related Work}
\label{sec-related}

In this section, we will discuss related results. In particular, we will discuss
(1) results on string solving, and (2) results on modelling and reasoning about 
\regexp{} constraints. 

\subsection{String Constraint Solving}
As we discussed Section \ref{sec-intro}, there has been a wealth of research
activities focussing on the development of string solving algorithms, especially
in the past ten decade or so. We mention among others the following string
solvers:
Z3 \cite{Z3}, CVC4 \cite{cvc4}, Z3-str/2/3/4, ABC \cite{ABC}, Norn
\cite{Abdulla14}, Trau \cite{Z3-trau,AbdullaACDHRR18-trau,Abdulla17}, OSTRICH
\cite{CHL+19}, S2S, Qzy, Stranger \cite{Stranger}, Sloth
\cite{HJLRV18,AbdullaA+19},
Slog \cite{fang-yu-circuits}, Slent \cite{WC+18}, Gecode+S, G-Strings, HAMPI
and 
\cite{HAMPI}. %\cite{??}

\subsection{Modelling and Reasoning about \regexp{}}
%This paper is concerned with string constraint solving in general, but the focus is on the interplay of regular expressions in modern programming language and solving constraints involving complex string functions. Both of them are monumental research fields for which we will only discuss the work which are most pertinent to ours. 

Variants and extensions of regular expressions to capture their usage in programming languages have received attentions %been investigated 
from both theory and practice. In formal language theory, regular expressions with capturing groups and backreferences were considered in \cite{CSY03,CN09} and also more recently in \cite{Freydenberger13,Schmid16,BM17b,FS19}, where the expressibility issues and decision problems were investigated. Nevertheless, some basic features of these regular expression, namely, the non-commutative union and the greedy/lazy semantics of Kleene star/plus, were not addressed therein. In software engineering community, % have also received attention in the software engineering community. 
some empirical studies were reported for these regular expressions recently, including portability across different programing languages \cite{DMC+19} and DDos attacks \cite{SP18}, as well as how programmers write them in practice \cite{MDD+19}.


Prioritized finite-state automata and prioritized finite-state transducers were proposed in \cite{BM17}. Prioritized finite-state transducers add indexed brackets to the input string in order to identify the matches of capturing groups. It is hard---if not impossible---to use prioritized finite-state transducers to model replace(all) function in general, e.g. swapping the first and last name as in Example~\ref{exmp-name-swap}. In contrast, {\PSST}s store the matches of capturing groups into string variables, which can then be referred to, thus allowing us to conveniently model the match and replace(all) function. 
%
Streaming string transducers were also used in \cite{ZAM19} to solve the straight-line string constraints with concatenation, finite-state transducers, and regular constraints.

For string constraints solving in general, we refer the readers to the recent survey \cite{Ama20}. In this work, we consider a string constraint language which is undecidable in general, and propose a propagation-based calculus to solve the constraints. However, we also identified a straight-line fragment including concatenation, extract, replace(All) which turns to be decidable. The decision procedure we use extends the backward-reasoning approach in \cite{CHL+19}, where only standard one-way and two-way finite-state transducers were considered. 
