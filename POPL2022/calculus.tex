%!TEX root = main.tex

\section{A Propagation-Based Calculus for String Constraints}
\label{sect:calculus}

\begin{table}
  \caption{Rules of the one-sided sequent calculus}
  \label{tab:calculus}
  
  \begin{gather*}
    \infer[$\wedge$]
    {\seqq{\varphi \wedge \psi}}
    {\seqq{\varphi, \psi}}
    \quad
    \infer[$\neg\wedge$]
    {\seqq{ \neg(\varphi \vee \psi)}}
    {\seqq{ \neg\varphi, \neg\psi}}
    \quad
    \inferii[$\vee$]
    {\seqq{\varphi \vee \psi}}
    {\seqq{\varphi}}{\seqq{\psi}}
    \quad
    \inferii[$\neg\vee$]
    {\seqq{\neg(\varphi \wedge \psi)}}
    {\seqq{\neg\varphi}}{\seqq{\neg\psi}}
    \quad
    \infer[$\neg\neg$]
    {\seqq{ \neg\neg\varphi}}
    {\seqq{\varphi}}
    \\[1ex]
    \infer[$\not\in$]
    {\seqq{x \not\in e}}
    {\seqq{x \in e^c}}
    \quad
    \inferC{\text{where~} y \text{~is fresh}}
    {\seqq{x \not= f(x_1, \ldots, x_n)}}
    {\seqq{x \not= y, y = f(x_1, \ldots, x_n)}}
    \quad
    \inferii[Cut]
    {\seqq{}}
    {\seqq{x \in e}}{\seqq{x \in e^c}}
    \\[2ex]
    \begin{array}{cl}
      \infer[Close]
      {\seqq{x \in e_1, \ldots, x \in e_n}}
      {}
      &
        \text{if~} \lang{e_1} \cap \cdots \cap\lang{e_n} = \emptyset
      \\[2ex]
      \infer[Subsume]
      {\seqq{x \in e, x \in e_1, \ldots, x \in e_n}}
      {\seqq{x \in e_1, \ldots, x \in e_n}}
      &
        \text{if~} \lang{e_1} \cap \cdots \cap\lang{e_n} \subseteq \lang{e}
      \\[2.5ex]
      \infer[Intersect]
      {\seqq{x \in e_1, \ldots, x \in e_n}}
      {\seqq{x \in e}}
      &
        \text{if~}
        \begin{array}{l}
          n > 1 \text{~and~}
          \\
          \lang{e_1} \cap \cdots \cap\lang{e_n} = \lang{e}
        \end{array}
      \\[4ex]
      \infer[$=$-Prop]
      {\seqq{x \in e, x = y}}
      {\seqq{x \in e, x = y, y \in e}}
      \\[3ex]
      \infer[$\not=$-Prop-Elim]
      {\seqq{x \in e, x \not= y}}
      {\seqq{x \in e, y \in e^c}}
      &
        \text{if~} |\lang{e}| = 1
      \\[3ex]
      \infer[$\not=$-Subsume]
      {\seqq{x \in e_1, x \not= y, y \in e_2}}
      {\seqq{x \in e_1, y \in e_2}}
      &
        \text{if~} \lang{e_1} \cap \lang{e_2} = \emptyset
      \\[4ex]
      \infer[Fwd-Prop]
      {\seqq{x = f(x_1, \ldots, x_n), x_1 \in e_1, \ldots, x_n \in e_n}}
      {\seqq{x \in e, x = f(x_1, \ldots, x_n), x_1 \in e_1, \ldots, x_n \in e_n}}
      &
        \text{if~} \lang{e} = f(\lang{e_1}, \ldots, \lang{e_n})
      \\[3ex]
      \infer[Fwd-Prop-Elim]
      {\seqq{x = f(x_1, \ldots, x_n), x_1 \in e_1, \ldots, x_n \in e_n}}
      {\seqq{x \in e, x_1 \in e_1, \ldots, x_n \in e_n}}
      &
        \text{if~}
        \begin{array}{l}
          \lang{e} = f(\lang{e_1}, \ldots, \lang{e_n})
          \\
          \text{and~} 
          |\lang{e}|= 1
          \end{array}
      \\[3ex]
      \infer[Bwd-Prop]
      {\seqq{x \in e, x = f(x_1, \ldots, x_n)}}
      {\big\{\seqq{x \in e, x = f(x_1, \ldots, x_n),
      x_1 \in e_1^i, \ldots, x_n \in e_n^i}\big\}_{i=1}^k}
      &
        \text{if~}
        \begin{array}{l}
          f^{-1}(\lang{e}) = \\
          \bigcup_{i=1}^k \big( \lang{e_1^i} \times \cdots \times \lang{e_n^i} \big)
          \end{array}
    \end{array}
  \end{gather*}
\end{table}

\philipp{problem: the calculus also uses complemented regular
  expressions; should we add those to the considered regex language?}

We now introduce our calculus for solving string constraints in
$\strline$, state its correctness, and observe that it gives rise to a
decision procedure for the fragment $\strline_{\sf reg}$ of formulas
that are straightline and do not contain back-references. The calculus
is based on the principle of propagating regular language constraints
by computing images and post-images of string functions. To work out
the gist of the approach, the calculus is kept minimalist, but it is
amenable to standard SMT optimizations such as non-chronological
back-tracking and conflict-driven learning, techniques that are
included in our implementation.
%
Our calculus is parameterized in the set of
considered string functions; in this paper, we work with the set
$\{\concat, \extract_{i, e}, \replace_{\pat, \rep}, \replaceall_{\pat,
  \rep}\}$ consisting of concatenation, extraction, and replacement,
but it can be extended by other functions that can be represented as
PSSTs.

\subsection{Sequents and Proof Rules}

The calculus operates on \emph{one-sided sequents,} and can be
interpreted as a sequent calculus in the sense of
Gentzen~\cite{Gentzen35} in which all formulas are located in the
antecedent (left of the turnstile~$\vdash$). A one-sided sequent is a
finite set $\Gamma \subseteq \strline$ of string constraints. For sake
of presentation, we write sequents as lists of formulas separated by
comma, and $\seqq{\varphi_1, \ldots, \varphi_n}$ for the union
$\Gamma \cup \{\varphi_1, \ldots, \varphi_n\}$. We say that a
sequent~$\seqq{}$ is \emph{unsatisfiable} if $\bigwedge \Gamma$ is
unsatisfiable. Our calculus is refutational and has the purpose of
either showing that some initial sequent~$\seqq{}$ is unsatisfiable,
or that it is satisfiable by constructing a solution for it. A
solution is a
sequent~$\seq{x_1 \in w_1, x_2 \in w_2, \ldots, x_n \in w_n}$ that
defines the values of string variables using regexes that only consist
of single words.

\begin{table}
  \caption{Proof of unsatisfiability from Example~\ref{ex:calc1}}
  \label{tab:calcEx1}

  \begin{prooftree}
    \AxiomC{}
    \LeftLabel{\ruleName{Close}}
    \UnaryInfC{$x \in a^+\Sigma^*,
      x = y \concat z, y \in a^+, z \in \Sigma^*,
      x \in (a^c)^+, x = \replaceall_{a,
        b}(x)$}
    \LeftLabel{\ruleName{Fwd-Prop}}
    \UnaryInfC{$x \in a^+\Sigma^*,
      x = y \concat z, y \in a^+, z \in \Sigma^*, x = \replaceall_{a,
        b}(x)$}
    \LeftLabel{\ruleName{Fwd-Prop}}
    \UnaryInfC{$x = y \concat z, y \in a^+, z \in \Sigma^*, x = \replaceall_{a,
        b}(x)$}
    \LeftLabel{\ruleName{$\wedge^*$}}
    \UnaryInfC{$x = y \concat z \wedge y \in a^+ \wedge z \in \Sigma^*
      \wedge x = \replaceall_{a,
    b}(x)$}
  \end{prooftree}
\end{table}

\begin{example}
  \label{ex:calc1}
  We first illustrate the calculus by showing unsatisfiability of the
  constraint:
  \begin{equation}
    \label{eq:calcEx1}
    x = y \concat z \wedge y \in a^+ \wedge z \in \Sigma^*
    \wedge x = \replaceall_{a, b}(x)
  \end{equation}
  To this end, we construct a proof tree that has \eqref{eq:calcEx1}
  as its root by applying proof rules until all proof goals have been
  closed (Table~\ref{tab:calcEx1}). The proof is growing upward, and
  is built by first eliminating the conjunctions~$\wedge$, resulting
  in a list formulas. Next, we apply the rule~\ruleName{Fwd-Prop} for
  \emph{forward-propagation} of a regular expression constraint. Given
  that $y \in a^+, z \in \Sigma^*$, from the
  equation~$x = y \concat z$ we can conclude that $x \in
  a^+\Sigma^*$. From $x \in a^+\Sigma^*$ and
  $x = \replaceall_{a, b}(x)$, we can next conclude that
  $x \in (a^c)^+$, i.e., $x$ cannot contain the letter~$a$. Finally,
  the proof can be closed because the languages~$a^+\Sigma^*$ and
  $(a^c)^+$ are disjoint.

  We next consider the case of a satisfiable formula in $\strline_{\sf reg}$:
  \begin{equation}
    \label{eq:calcEx2}
    x = y \concat z \wedge x \in a^+ \wedge r = \replaceall_{a, b}(x)
  \end{equation}
  [...]
\end{example}

More formally, proof rules are relations between a finite list of
sequents (the premises), and a single sequent (the conclusion). Proofs
are finite trees growing upward, in which each node is labeled with a
sequent, and each non-leaf node is related to the node(s) directly
above it through an instance of a proof rule. A proof branch is a path
from the proof root to a leaf. A branch is closed if a closure rule (a
rule without premises) has been applied to its leaf, and open
otherwise. A proof is closed if all of its branches are closed.

The proof rules of the calculus are shown in Table~\ref{tab:calculus}.

\begin{lemma}[Soundness]
  The sequent calculus defined by Table~\ref{tab:calculus} is sound:
  (i) the root of a closed proof is an unsatisfiable sequent; and (ii)
  if a proof has an open branch that ends with a
  solution~$\seq{x_1 \in w_1, x_2 \in w_2, \ldots, x_n \in w_n}$, then
  the
  assignment~$\{x_1 \mapsto w_1, x_2 \mapsto w_2, \ldots, x_n \mapsto
  w_n\}$ is a satisfying assignment of the root sequent.
\end{lemma}

%%% Local Variables:
%%% mode: latex
%%% TeX-master: "main"
%%% TeX-command-extra-options: "-shell-escape"
%%% End:
