%!TEX root = main.tex

\section{Preliminaries}\label{sec:prel}

Throughout the paper, $\Int^+$ denotes the set of positive integers, and  $\nat$ denotes the set of natural numbers. Furthermore, for $n\in \Int^+$, let $[n]:=\{1, \ldots, n\}$. 

We use $\Sigma$ to denote a finite set of letters, called \emph{alphabet}. Moreover, we use $\nullchar$ to denote a special null symbol and assume that  $\nullchar \not \in \Sigma$. 
A \emph{string} over $\Sigma$ is a finite sequence of letters from $\Sigma$. We use $\Sigma^*$ to denote the set of strings over $\Sigma$ and $\varepsilon$ to denote the empty string. Moreover, for convenience, we use $\Sigma^\varepsilon$ to denote $\Sigma \cup \{\varepsilon\}$. A string $w'$ is called a \emph{prefix} of $w$ if $w = w'w''$ for some string $w''$. We use $\pref(w)$ to denote the set of prefixes of $w$. For a prefix $w_1$ of $w$, let $w = w_1 w_2$, then we use $w_1^{-1}w$ to denote $w_2$.

\begin{definition}[Finite-state automata] \label{def:nfa}
	A \emph{(nondeterministic) finite-state automaton}
	(\FA{}) over a finite alphabet $\ialphabet$ is a tuple $\Aut =
	(\ialphabet, \controls, q_0, \finals, \transrel)$ where 
	$\controls$ is a finite set of 
	states, $q_0\in \controls$ is
	the initial state, $\finals\subseteq \controls$ is a set of final states, and 
	$\transrel\subseteq \controls \times 
	\ialphabet^\varepsilon \times  \controls$ is the
	transition relation. 
\end{definition}

For an input string $w$, a \emph{run} of $\Aut$ on $w$
%(with $a_0 = \EndLeft$ and $a_{n+1} = \EndRight$)
is a sequence $q_0 \sigma_1 q_1 \ldots \sigma_n q_n$ such that $w = \sigma_1 \cdots \sigma_n$ and $(q_{j-1}, \sigma_{j}, q_{j}) \in
\transrel$ for every $j \in [n]$.
The run is said to be \defn{accepting} if $q_n \in \finals$.
A string $w$ is \defn{accepted} by $\Aut$ if there is an accepting run of
$\Aut$ on $w$. In particular, the empty string $\varepsilon$ is accepted by $\Aut$ if $q_0 \in F$. The set of strings accepted by $\Aut$, i.e., the language \defn{recognized} by $\Aut$, is denoted by $\Lang(\Aut)$.
%Since we deal with computational complexity in the sequel, we define
The \defn{size} $|\Aut|$ of $\Aut$ is defined as the cardinality of the set $Q$ of states, which will be 
used when the computational complexity is concerned.

For convenience, for $a \in \Sigma$, we use $\delta^{(a)}$ to denote the  relation $\{(q, q') \mid (q, a, q') \in \delta\}$.