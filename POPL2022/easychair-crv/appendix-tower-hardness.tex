
\newcommand\scon{S}
\newcommand\svar{x}
\newcommand\alphabet{\Sigma}

\newcommand\rebrac[1]{\left[#1\right]}
\newcommand\fbrac[1]{\left\langle#1\right\rangle}

\newcommand\idxi{i}
\newcommand\idxj{j}
\newcommand\idxk{k}
\newcommand\numof{n}

\newcommand\tiles{\Theta}
\newcommand\nontiles{\overline{\Theta}}
\newcommand\hrel{H}
\newcommand\vrel{V}
\newcommand\inittile{\tile^0}
\newcommand\fintile{f}
\newcommand\tile{t}
\newcommand\vartile{d}
\newcommand\linlen{\numof}
\newcommand\tileheight{h}
\newcommand\expheight{m}
\newcommand\spacer{\#}
\newcommand\isnum[2]{R^{#2}_{#1}}
\newcommand\lmark{\langle}
\newcommand\rmark{\rangle}
\newcommand\sftrue[1]{\top_{#1}}
\newcommand\sffalse[1]{\bot_{#1}}
\newcommand\sfvalue{v}

\newcommand\tilesnum[1]{\tiles_{#1}}
\newcommand\hrelnum[1]{\hrel_{#1}}
\newcommand\vrelnum[1]{\vrel_{#1}}
\newcommand\inittilenum[1]{\inittile_{#1}}
\newcommand\fintilenum[1]{\fintile_{#1}}
\newcommand\nmax[1]{N_{#1}}
\newcommand\tenc[2]{[#2]_{#1}}

\newcommand\fullrow[1]{
    \tenc{\expheight}{1} \tile^{#1}_1
        \ldots
        \tenc{\expheight}{\nmax{\expheight}}
            \tile^{#1}_{\nmax{\expheight}}
}

\subsection{Tower-Hardness of String Constraints with Streaming String Transductions}
\label{sec:tower-hard}

We show that the satisfiability problem for $\strlinesl$ is Tower-hard.

\begin{theorem}
The satisfiability problem for $\strlinesl$ is Tower-hard.
\end{theorem}

Our proof will use tiling problems over extremely wide corridors.
We first introduce these tiling problems, then how we will encode potential solutions as words.
Finally, we will show how $\strlinesl$ can verify solutions.

\subsubsection{Tiling Problems}

A *tiling problem* is a tuple
$\tup{\tiles, \hrel, \vrel, \inittile, \fintile}$
where
    $\tiles$ is a finite set of tiles,
    $\hrel \subseteq \tiles \times \tiles$ is a horizontal matching relation,
    $\vrel \subseteq \tiles \times \tiles$ is a vertical matching relation, and
    $\inittile, \fintile \in \tiles$ are initial and final tiles respectively.

A solution to a tiling problem over a $\linlen$-width corridor is a sequence
\[
    \begin{array}{c}
        \tile^1_1 \ldots \tile^1_\linlen \\
        \tile^2_1 \ldots \tile^2_\linlen \\
        \ldots \\
        \tile^\tileheight_1 \ldots \tile^\tileheight_\linlen
    \end{array}
\]
where
$\tile^1_1 = \inittile$,
$\tile^\tileheight_\linlen = \fintile$,
and for all
$1 \leq \idxi < \linlen$
and
$1 \leq \idxj \leq \tileheight$
we have
$\tup{\tile^\idxj_\idxi, \tile^\idxj_{\idxi+1}} \in \hrel$
and for all
$1 \leq \idxi \leq \linlen$
and
$1 \leq \idxj < \tileheight$
we have
$\tup{\tile^\idxj_\idxi, \tile^{\idxj+1}_\idxi} \in \vrel$.
Note, we will assume that $\inittile$ and $\fintile$ can only appear at
the beginning and end of the tiling respectively.

Tiling problems characterise many complexity classes~\cite{BGG97}. In
particular, we will use the following facts.

\begin{itemize}
\item
    For any $\linlen$-space Turing machine, there exists a tiling problem
    of size polynomial in the size of the Turing machine, over a corridor
    of width $\linlen$, that has a solution iff the $\linlen$-space Turing
    machine has a terminating computation.

\item
    There is a fixed
    $\tup{\tiles, \hrel, \vrel, \inittile, \fintile}$
    such that for any width $\linlen$ there is a unique solution
    \[
      \begin{array}{c}
          \tile^1_1 \ldots \tile^1_\linlen \\
          \tile^2_1 \ldots \tile^2_\linlen \\
          \ldots \\
          \tile^\tileheight_1 \ldots \tile^\tileheight_\linlen
      \end{array}
    \]
    and moreover $\tileheight$ is exponential in $\linlen$. One such
    example is a Turing machine where the tape contents represent a binary
    number. The Turing machine starts from a tape containing only $0$s
    and finishes with a tape containing only $1$s by repeatedly
    incrementing the binary encoding on the tape. This Turing machine can
    be encoded as the required tiling problem.
\end{itemize}

\subsubsection{Large Numbers}

The crux of the proof is encoding large numbers that can take values
between $1$ and $\expheight$-fold exponential.

A linear-length binary number could be encoded simply as a sequence of bits
\[
    b_0 \ldots b_\linlen \in \set{0,1}^\linlen \ .
\]
To aid with later constructions we will take a more oblique approach.
Let
$\tup{\tilesnum{1}, \hrelnum{1}, \vrelnum{1}, \inittilenum{1}, \fintilenum{1}}$
be a copy of the fixed tiling problem from the previous section for
which there is a unique solution, whose length must be exponential in
the width. In the future, we will need several copies of this problem,
hence the indexing here. Note, we assume each copy has disjoint tile
sets. Fix a width $\linlen$ and let $\nmax{1}$ be the corresponding
corridor length. A \emph{level-1} number can encode values from $1$ to
$\nmax{1}$. In particular, for $1 \leq \idxi \leq \nmax{1}$ we define
\[
    \tenc{1}{\idxi} = \tile^\idxi_1 \ldots \tile^\idxi_\linlen
\]
where
$\tile^\idxi_1 \ldots \tile^\idxi_\linlen$
is the tiling of the $\idxi$th row of the unique solution to the tiling problem.

A \emph{level-2} number will be derived from tiling a corridor of width
$\nmax{1}$, and thus the number of rows will be doubly-exponential. For
this, we require another copy $\tup{\tilesnum{2}, \hrelnum{2},
\vrelnum{2}, \inittilenum{2}, \fintilenum{2}}$ of the above tiling
problem. Moreover, let $\nmax{2}$ be the length of the solution for a
corridor of width $\nmax{1}$. Then for any $1 \leq \idxi \leq \nmax{2}$ we
define
\[
    \tenc{2}{\idxi} =
        \tenc{1}{1} \tile^\idxi_1
        \tenc{1}{2} \tile^\idxi_2
        \ldots
        \tenc{1}{\nmax{1}} \tile^\idxi_{\nmax{1}}
\]
where
$\tile^\idxi_1 \ldots \tile^\idxi_{\nmax{1}}$
is the tiling of the $\idxi$th row of the unique solution to the tiling
problem. That is, the encoding indexes each tile with it's column
number, where the column number is represented as a level-1 number.

In general, a *level-$\expheight$* number is of length
$(\expheight-1)$-fold exponential and can encode numbers
$\expheight$-fold exponential in size. We use a copy
$\tup{\tilesnum{\expheight},
      \hrelnum{\expheight},
      \vrelnum{\expheight},
      \inittilenum{\expheight},
      \fintilenum{\expheight}}$
of the above tiling problem and use a corridor of width
$\nmax{\expheight-1}$. We define $\nmax{\expheight}$ as the length of
the unique solution to this problem. Then, for any $1 \leq \idxi \leq
\nmax{\expheight}$ we have
\[
    \tenc{\expheight}{\idxi} =
        \tenc{\expheight-1}{1} \tile^\idxi_1
        \tenc{\expheight-1}{2} \tile^\idxi_2
        \ldots
        \tenc{\expheight-1}{\nmax{\expheight-1}}
            \tile^\idxi_{\nmax{\expheight-1}}
\]
where
$\tile^\idxi_1 \ldots \tile^\idxi_{\nmax{\expheight-1}}$
is the tiling of the $\idxi$th row of the unique solution to the tiling problem.

Note that we can define regular languages to check that a string is a
large number. In particular
\[
    \isnum{\expheight}{\linlen} =
    \begin{cases}
        \rebrac{\tilesnum{1}}^\linlen & \expheight = 1 \\
        \rebrac{
            \isnum{\expheight-1}{\linlen}
            \tilesnum{\expheight}
        }^\ast  & \expheight > 1 \ .
    \end{cases}
\]

\subsubsection{Hardness Proof}

We show that the satisfiability problem for $\strlinesl$ is
Tower-hard. We first introduce the basic framework of solving a hard
tiling problem. Then we discuss the two phases of transductions required
by the reduction. These are constructing a large boolean formula, and
then evaluating the formula. This two phases are described in separate
sections.

\paragraph{The Framework}

The proof is by reduction from a tiling problem over an
$\expheight$-fold exponential width corridor. In general, solving such
problems is hard for $\expheight$-ExpSpace.

Let $\nmax{\expheight}$ be the width of the corridor. Fix a tiling
problem
\[
    \tup{\tiles, \hrel, \vrel, \inittile, \fintile} \ .
\]

We will compose an $\strlinesl$ formula $\scon$ with a free
variable $\svar$. If $\scon$ is satisfiable, $\svar$ will contain a
string encoding a solution to the tiling problem. In particular, the
value of $\svar$ will be of the form
\[
    \begin{array}{c}
        \fullrow{1} \spacer \\
        \fullrow{2} \spacer \\
        \ldots \\
        \fullrow{\tileheight} \spacer \ .
    \end{array}
\]
That is, each row of the solution is separated by the $\spacer$ symbol.
Between each tile of a row is it's index, encoded using the large number
encoding described in the previous section.

The formula $\scon$ will use a series of replacements and assertions to verify
that the tiling encoded by $\svar$ is a valid solution to the tiling problem.
We will give the formula in three steps.

We will define the alphabet to be
\[
    \alphabet = \tiles \cup \nontiles
\]
where $\tiles$ is the set of tiles, and $\nontiles$ is the set of characters required to encode large numbers, plus $\spacer$.

The first part is
\[
    \begin{array}{l}
        \ASSERT{x \in
            \rebrac{
                \rebrac{
                    \isnum{\expheight}{\linlen} \tiles
                }^\ast
                \spacer
            }^\ast
        }; \\
        \ASSERT{x \in \isnum{\expheight}{\linlen} \inittile}; \\
        \ASSERT{x \in \alphabet^\ast \fintile \spacer}; \\
        \ASSERT{x \in
            \rebrac{
                \rebrac{
                    \sum\limits_{\tup{\tile_1, \tile_2} \in \hrel}
                        \isnum{\expheight}{\linlen} \tile_1
                        \isnum{\expheight}{\linlen} \tile_2
                }^\ast
                \rebrac{\isnum{\expheight}{\linlen} \tiles}^?
                \spacer
            }^\ast
        }; \\
        \ASSERT{x \in
            \rebrac{
                \rebrac{\isnum{\expheight}{\linlen} \tiles}
                \rebrac{
                    \sum\limits_{\tup{\tile_1, \tile_2} \in \hrel}
                        \isnum{\expheight}{\linlen} \tile_1
                        \isnum{\expheight}{\linlen} \tile_2
                }^\ast
                \rebrac{\isnum{\expheight}{\linlen} \tiles}^?
                \spacer
            }^\ast
        }; \\
    \end{array}
\]

The first asserts simply verify the format of the value of $\svar$ is as
expected and moreover, the first appearing element of $\tiles$ in the string is
$\inittile$, and the last element is $\fintile$.

The final two assertions check the horizontal tiling relation.  In particular,
the first checks that even pairs of tiles are in $\hrel$, while the second
checks odd pairs are in $\hrel$.


The main challenge is checking the vertical tiling relation. This is
done by a series of transductions operating in two main phases. The
first phase rewrites the encoding into a kind of large Boolean formula,
which is then evaluated in the second phase.

\paragraph{Constructing the Large Boolean Formula}

The next phase of the formula is shown below and explained afterwards.
For convenience, we will describe the construction using transductions.
After the explanation, we will describe how to achieve these transductions
using $\replaceall$.
\[
    \begin{array}{l}
        \svar^1_\expheight = \ap{\psst^1_\expheight}{\svar}; \\
        \svar^2_\expheight = \ap{\psst^2_\expheight}{\svar^1_\expheight}; \\
        \svar^3_\expheight = \ap{\psst^3_\expheight}{\svar^2_\expheight}; \\
        \svar^1_{\expheight-1}
            = \ap{\psst^1_{\expheight-1}}{\svar^3_\expheight}; \\
        \svar^2_{\expheight-1}
            = \ap{\psst^2_{\expheight-1}}{\svar^1_{\expheight-1}}; \\
        \svar^3_{\expheight-1}
            = \ap{\psst^3_{\expheight-1}}{\svar^2_{\expheight-1}}; \\
        \ldots \\
        \svar^1_1 = \ap{\psst^1_1}{\svar^3_2}; \\
        \svar^2_1 = \ap{\psst^2_1}{\svar^1_1}; \\
        \svar^3_1 = \ap{\psst^3_1}{\svar^2_1}; \\
        \svar_0 = \ap{\psst_0}{\svar^3_1} \ .
    \end{array}
\]

The Boolean formula is constructed by rewriting the encoding stored in
$\svar$. We need to check the vertical tiling relation by comparing
$\tile^\idxi_\idxj$ with $\tile^{\idxi+1}_\idxj$. However, these are
separated by a huge number of other tiles, which also need to be checked
against their counterpart in the next row.

The goal of the transductions is to "rotate" the encoding so that
instead of each tile being directly next to its horizontal counterpart,
it is directly next to its vertical counterpart. Our transductions do
not quite achieve this goal, but instead place the tiles in each row next to
potential vertical counterparts. The Boolean formula contains large
disjunctions over these possibilities and use the indexing by large
numbers to pick out the correct pairs.

The idea is best illustrated by showing the first three transductions,
$\psst^1_\expheight$, $\psst^2_\expheight$, and $\psst^3_\expheight$.
We start with
\[
    \begin{array}{c}
        \fullrow{1} \spacer \\
        \fullrow{2} \spacer \\
        \ldots \\
        \fullrow{\tileheight} \spacer \ .
    \end{array}
\]
The transducer $\psst^1_\expheight$ saves the row it is currently reading.
Then, when reading the next row, it outputs each index and tile of the current
row followed by a copy of the last row. The output is shown below. We use a
disjunction symbol to indicate that, after the transduction, the tile should
match one of the tiles copied after it. Between each pair of a tile and a
copied row, we use the conjunction symbol to indicate that every disjunction
should have one match. The result is shown below. To aid readability, we
underline the copied rows. The parentheses $\fbrac{}$ are also inserted to aid future parsing.
\[
    \begin{array}{c}
        \fbrac{
            \tenc{\expheight}{1} \tile^2_1
                \lor
                \underline{\fullrow{1}}
        }
        \land
        \ldots
        \land
        \fbrac{
            \tenc{\expheight}{\nmax{\expheight}} \tile^2_{\nmax{\expheight}}
                \lor
                \underline{\fullrow{1}}
        } \\
        \land \ldots \land \\
        \fbrac{
            \tenc{\expheight}{1} \tile^\tileheight_1
                \lor
                \underline{\fullrow{\tileheight-1}}
        }
        \land
        \ldots
        \land
        \fbrac{
            \tenc{\expheight}{\nmax{\expheight}}
                \tile^\tileheight_{\nmax{\expheight}}
                \lor
                \underline{\fullrow{\tileheight-1}}
        } \ .\\
    \end{array}
\]

After this transduction, we apply $\psst^2_\expheight$. This
transduction forms pairs of a tile, with all tiles following it from the
previous row (up to the next $\land$ symbol). This leaves us with a
conjunction of disjunctions of pairs. Inside each disjunct, we need to
verify that one pair has matching indices and tiles that satisfy the
vertical tiling relation $\vrel$. The result of the second transduction
is shown below.
\[
    \begin{array}{c}
        \fbrac{
            \tenc{\expheight}{1} \tile^2_1
            \tenc{\expheight}{1} \tile^1_1
            \lor
            \ldots
            \lor
            \tenc{\expheight}{1} \tile^2_1
            \tenc{\expheight}{\nmax{\expheight}} \tile^1_{\nmax{\expheight}}
        }
            \land
            \ldots
            \land
            \fbrac{
                \tenc{\expheight}{\nmax{\expheight}} \tile^2_{\nmax{\expheight}}
                \tenc{\expheight}{1} \tile^1_1
                \lor
                \ldots
                \lor
                \tenc{\expheight}{\nmax{\expheight}} \tile^2_{\nmax{\expheight}}
                \tenc{\expheight}{\nmax{\expheight}} \tile^1_{\nmax{\expheight}}
            } \\
        \land \ldots \land \\
        \fbrac{
            \tenc{\expheight}{1} \tile^\tileheight_1
            \tenc{\expheight}{1} \tile^{\tileheight-1}_1
            \lor
            \ldots
            \lor
            \tenc{\expheight}{1} \tile^\tileheight_1
            \tenc{\expheight}{\nmax{\expheight}}
                \tile^{\tileheight-1}_{\nmax{\expheight}}
        }
            \land
            \ldots
            \land
            \fbrac{
                \tenc{\expheight}{\nmax{\expheight}}
                    \tile^\tileheight_{\nmax{\expheight}}
                \tenc{\expheight}{1} \tile^{\tileheight-1}_1
                \lor
                \ldots
                \lor
                \tenc{\expheight}{\nmax{\expheight}}
                    \tile^\tileheight_{\nmax{\expheight}}
                \tenc{\expheight}{\nmax{\expheight}}
                    \tile^{\tileheight-1}_{\nmax{\expheight}}
            } \ .\\
    \end{array}
\]
Notice that we now have each tile in a pair with its vertical neighbour,
but also in a pair with every other tile in the row beneath. The indices
can be used to pick out the right pairs, but we will need further
transductions to analyse the encoding of large numbers.

To simplify matters, we apply $\psst^3_\expheight$. This transduction removes
the tiles from the string, retaining each pair of indices where the tiles
satisfy the vertical tiling relation. When the tiling relation is not
satisfied, we insert $\sffalse{\expheight}$. We use $\lmark$, $\spacer$, and
$\rmark$ to delimit the indices. We are left with a string of the form
\[
    \bigwedge \bigvee \lmark
        \tenc{\expheight}{\idxi} \spacer \tenc{\expheight}{\idxj}
    \rmark
    \lor \sffalse{\expheight} \lor \cdots \lor \sffalse{\expheight} \ .
\]
We will often elide the $\sffalse{\expheight}$ disjuncts for clarity. They
will remain untouched until the formula is evaluated in the next section.

We consider a pair $\lmark \tenc{\expheight}{\idxi} \spacer
\tenc{\expheight}{\idxj} \rmark$ to evaluate to true whenever $\idxi = \idxj$.
The truth of the formula can be computed accordingly. However, it's not
straightforward to check whether $\idxi = \idxj$ as they are large numbers. The
key observation is that they are encoded as solutions to indexed tiling
problems, which means we can go through a similar process to the transductions
above.

First, recall that $\tenc{\expheight}{\idxi}$ is of the form
\[
    \tenc{\expheight-1}{1} \vartile^\idxi_1
    \tenc{\expheight-1}{2} \vartile^\idxi_2
    \ldots
    \tenc{\expheight-1}{\nmax{\expheight-1}}
        \vartile^\idxi_{\nmax{\expheight-1}}
\]
where we use $\vartile$ to indicate tiles instead of $\tile$.

We apply three transductions $\psst^1_{\expheight-1}$,
$\psst^2_{\expheight-1}$, and $\psst^3_{\expheight-1}$. The first copies
the first index of each pair directly after the tiles of the second
index. That is, each pair
\[
    \lmark \tenc{\expheight}{\idxi} \spacer \tenc{\expheight}{\idxj} \rmark
\]
is rewritten to
\[
    \fbrac{
        \tenc{\expheight-1}{1} \vartile^\idxj_1
        \lor
        \tenc{\expheight}{\idxi}
    }
    \land
    \ldots
    \land
    \fbrac{
        \tenc{\expheight-1}{\nmax{\expheight-1}}
            \vartile^\idxj_{\nmax{\expheight-1}}
        \lor
        \tenc{\expheight}{\idxi}
    } \ .
\]
Then we apply a similar second transduction: each disjunction is
expanded into pairs of indices and tiles. The result is
\[
    \begin{array}{c}
        \fbrac{
            \tenc{\expheight-1}{1} \vartile^\idxj_1
            \tenc{\expheight-1}{1} \vartile^\idxi_1
            \lor
            \ldots
            \lor
            \tenc{\expheight-1}{1} \vartile^\idxj_1
            \tenc{\expheight-1}{\nmax{\expheight-1}}
                \vartile^\idxi_{\nmax{\expheight-1}}
        } \\
        \land
        \ldots
        \land \\
        \fbrac{
            \tenc{\expheight-1}{\nmax{\expheight-1}}
                \vartile^\idxj_{\nmax{\expheight-1}}
            \tenc{\expheight-1}{1} \vartile^\idxi_1
            \lor
            \ldots
            \lor
            \tenc{\expheight-1}{\nmax{\expheight-1}}
                \vartile^\idxj_{\nmax{\expheight-1}}
            \tenc{\expheight-1}{\nmax{\expheight-1}}
                \vartile^\idxi_{\nmax{\expheight-1}}
        } \ .
    \end{array}
\]
The third transduction replaces with $\sffalse{\expheight-1}$ all pairs where
we don't have $\vartile^\idxj_\idxk = \vartile^\idxi_{\idxk'}$ (recall, we need
to check that $\idxi = \idxj$ so the tiles at each position should be the
same). As before, for a single pair, this leaves us with a string formula of
the form
\[
    \bigwedge \bigvee \lmark
        \tenc{\expheight-1}{\idxi'} \spacer \tenc{\expheight-1}{\idxj'}
    \rmark
    \lor \sffalse{\expheight-1} \lor \cdots \lor \sffalse{\expheight-1} \ .
\]
Again, we will elide the $\sffalse{\expheight-1}$ disjuncts for clarity as they
will be untouched until the formula is evaluated.  Recalling that there are
many pairs in the input string, the output of this series of transductions is a
string formula of the form
\[
    \bigwedge \bigvee \bigwedge \bigvee \lmark
        \tenc{\expheight-1}{\idxi'} \spacer \tenc{\expheight-1}{\idxj'}
    \rmark \ .
\]

We repeat these steps using $\psst^1_{\expheight-2}$,
$\psst^2_{\expheight-2}$, $\psst^3_{\expheight-2}$ all the way down to
$\psst^1_1$, $\psst^2_1$, $\psst^3_1$. We are left with a string formula of the
form
\[
    \bigwedge \bigvee \cdots \bigwedge \bigvee \lmark
        \tenc{1}{\idxi'} \spacer \tenc{1}{\idxj'}
    \rmark \ .
\]
Recall each $\tenc{1}{\idxi'}$ is of the form
\[
    \vartile^{\idxi'}_1 \ldots \vartile^{\idxi'}_\linlen \ .
\]
The final step interleaves the tiles of the two numbers. The result is a string
formula of the form
\[
    \bigwedge \bigvee \cdots \bigwedge \bigvee \bigwedge \vartile \vartile' \ .
\]
This is the formula that is evaluated in the next phase.

To complete this section we need to implement the above transductions using
$\replaceall$.

First, consider $\psst^1_\expheight$.  We start with
\[
    \begin{array}{c}
        \fullrow{1} \spacer \\
        \fullrow{2} \spacer \\
        \ldots \\
        \fullrow{\tileheight} \spacer \ .
    \end{array}
\]
We are aiming for
\[
    \begin{array}{c}
        \fbrac{
            \tenc{\expheight}{1} \tile^2_1
                \lor
                \fullrow{1}
        }
        \land
        \ldots
        \land
        \fbrac{
            \tenc{\expheight}{\nmax{\expheight}} \tile^2_{\nmax{\expheight}}
                \lor
                \fullrow{1}
        } \\
        \land \ldots \land \\
        \fbrac{
            \tenc{\expheight}{1} \tile^\tileheight_1
                \lor
                \fullrow{\tileheight-1}
        }
        \land
        \ldots
        \land
        \fbrac{
            \tenc{\expheight}{\nmax{\expheight}}
                \tile^\tileheight_{\nmax{\expheight}}
                \lor
                \fullrow{\tileheight-1}
        } \ .\\
    \end{array}
\]
We use two $\replaceall$s. The first uses $\refbefore$ to do the main work of
copying the previous row into the current row a huge number of times. In fact,
$\refbefore$ will copy too much, as it will copy everything that came before,
not just the last row. The second $\replaceall$ will cut down the contents of
$\refbefore$ to only the last row. That is, we first apply
$\replaceall_{\pat_1, \rep_1}$ and then $\replaceall_{\pat_2, \rep_2}$ where
\[
    \begin{array}{rcl}
        \pat_1 &=& (\tile) \\
        \rep_1 &=& \$1 \triangleleft \refbefore \triangleright
    \end{array}
\]
and $\triangleleft$ and $\triangleright$ are two characters not in $\alphabet$,
and, letting $\alphabet_\spacer = \alphabet \setminus \set{\spacer}$,
\[
    \begin{array}{rcl}
        \pat_2 &=& \triangleleft
                \alphabet_\spacer^\ast \spacer (
                    \alphabet_\spacer^\ast
                ) \spacer \alphabet_\spacer^\ast
            \triangleright \\
        \rep_2 &=& \lor \$1 \ .
    \end{array}
\]
That is, the first replace adds after each tile the entire preceding string,
delimited by $\triangleleft$ and $\triangleright$. The second replace picks out
the final row of each string between $\triangleleft$ and $\triangleright$ and
adds the $\lor$. Notice that the second replace does not match anything between
$\triangleleft$ and $\triangleright$ on the first row. In fact, we need another
$\replaceall$ to delete the first row. That is $\replaceall_{\pat_3, \rep_3}$
where
\[
    \begin{array}{rcl}
        \pat_3 &=& \rebrac{
            \alphabet \cup \set{\triangleleft, \triangleright}
            }^\ast \triangleright \alphabet_\spacer^\ast \spacer \\
        \rep_3 &=& \varepsilon \ .
    \end{array}
\]
Notice, the pattern above matches any row containing at least one
$\triangleright$. This means only the first row will be deleted as delimiters
have already been removed from the other rows.  To complete the step, we
replace all $\spacer$ with $\land$ and insert the parenthesis $\fbrac{}$ using another $\replaceall$ (and a concatenation at the beginning and the end of the string).

The transduction $\psst^2_\expheight$ uses similar techniques to the above and
we leave the details to the reader. The same is true of the other similar
transductions $\psst^1_{\idxi}$ and $\psst^2_{\idxi}$.

Transduction $\psst^3_\expheight$ (and similarly the other $\psst^2_\idxi$)
replaces all pairs
\[
    \tenc{\expheight}{\idxi} \tile^2_\idxi
    \tenc{\expheight}{\idxi + 1} \tile^1_{\idxi+1}
\]
that do not satisfy the vertical tiling relation with $\sffalse{\expheight}$,
and rewrites them to
\[
    \lmark
        \tenc{1}{\idxi'} \spacer \tenc{1}{\idxj'}
    \rmark
\]
if the vertical tiling relation is matched. This can be done in two steps:
first replace the non-matches, then replace the matches. To replace the
non-matches we use $\replaceall_{\pat_1, \rep_1}$ where
\[
    \begin{array}{rcl}
        \pat_1 &=& \sum\limits_{\tup{\tile_1, \tile_2} \notin \vrel}
            \isnum{\expheight}{\linlen} \tile_1
            \isnum{\expheight}{\linlen} \tile_2 \\
        \rep_1 &=& \sffalse{\expheight} \ .
    \end{array}
\]
For the matches we use $\replaceall_{\pat_2, \rep_2}$ where
\[
    \begin{array}{rcl}
        \pat_2 &=& \sum\limits_{\tup{\tile_1, \tile_2} \in \vrel}
            (\isnum{\expheight}{\linlen}) \tile_1
            (\isnum{\expheight}{\linlen}) \tile_2 \\
        \rep_1 &=& \langle \$1 \spacer \$2 \rangle \ .
    \end{array}
\]

The final transduction takes a string of the form
\[
    \bigwedge \bigvee \cdots \bigwedge \bigvee \lmark
        \tenc{1}{\idxi'} \spacer \tenc{1}{\idxj'}
    \rmark
\]
where each $\tenc{1}{\idxi'}$ is of the form
\[
    \vartile^{\idxi'}_1 \ldots \vartile^{\idxi'}_\linlen \ .
\]
We need to interleave the tiles of the two numbers, giving a string of the form
\[
    \bigwedge \bigvee \cdots \bigwedge \bigvee \bigwedge \vartile \vartile' \ .
\]
This can be done with a single $\replaceall_{\pat, \rep}$ where
\[
    \begin{array}{rcl}
        \pat &=&
            \langle
                (\tilesnum{1}) \ldots (\tilesnum{1})
            \spacer
                (\tilesnum{1}) \ldots (\tilesnum{1})
            \rangle \\
        \rep &=&
            \langle
                \$1 \$(\linlen + 1)
                \land \cdots \land
                \$\linlen \$(2\linlen)
            \rangle \ . \\
    \end{array}
\]

\paragraph{Evaluating the Large Boolean Formula}

The final phase of $\scon$ evaluates the Boolean formula and is shown below.
Again we write the formula using transductions and explain how they can be done
with $\replaceall$.
\[
    \begin{array}{c}
        \svar^\land_0 = \ap{\psst_0}{\svar_0}; \\
        \svar^\lor_1 = \ap{\psst^\land_0}{\svar^\land_0}; \\
        \svar^\land_1 = \ap{\psst^\lor_1}{\svar^\lor_1}; \\
        \svar^\lor_2 = \ap{\psst^\land_1}{\svar^\land_1}; \\
        \svar^\land_2 = \ap{\psst^\lor_2}{\svar^\lor_2}; \\
        \svar^\land_3 = \ap{\psst^\land_2}{\svar^\land_2}; \\
        \ldots \\
        \svar^\lor_\expheight
            = \ap{\psst^\land_{\expheight-1}}{\svar^\land_{\expheight-1}}; \\
        \svar^\land_\expheight
            = \ap{\psst^\lor_\expheight}{\svar^\lor_{\expheight}}; \\
        \svar_f
            = \ap{\psst^\land_\expheight}{\svar^\lor_{\expheight}}; \\
        \ASSERT{\svar_f \in \pat_f}
    \end{array}
\]

The first transducer $\psst_1$ reads the string formula
\[
    \bigwedge \bigvee \cdots \bigwedge \bigvee \bigwedge \vartile \vartile' \ .
\]
copies it to its output, except replacing each pair
$\vartile \vartile'$
with $\sftrue{1}$ if $\vartile = \vartile'$ and with $\sffalse{1}$ otherwise.
This is requires two simple $\replaceall$ calls.

The remaining transductions evaluate the innermost disjunction or conjunction as
appropriate (the parenthesis $\fbrac{}$ are helpful here). For example
$\psst^\lor_1$
replaces the innermost
$\bigvee \sfvalue$
with $\sftrue{1}$ if $\sftrue{1}$ appears somewhere in the disjunction and
$\sffalse{1}$ otherwise.
This can be done by greedily matching any sequence of characters from
$\set{\sftrue{1}, \sffalse{1}, \lor}$
that contains at least one $\sftrue{1}$ and replacing the sequence with $\sftrue{1}$,
then greedily matching any remaining sequence of
$\set{\sffalse{1}, \lor}$
and replacing it with $\sffalse{1}$.
The evaluation of conjunctions works similarly, but inserts $\sftrue{2}$ and $\sffalse{2}$ in the move to the next level of evaluation.

The final assert checks that $\svar_f$ contains only the character $\sftrue{\expheight+1}$
and fails otherwise.

This completes the reduction.
