%!TEX root = popl2018.tex

\section{Undecidable extensions}

In this section, we extend the language SL with integer and character constraints. The language will use two types of variables, str and int. 

We start by defining integer constraints, which expresses length or number of occurrences of symbols in words. 

\begin{definition}
	Each term is either 
	\begin{enumerate}
		\item an integer variable $n$;
		\item $|x|$ for a string variable;
		\item $|x|_a$ for $a\in \Sigma$
	\end{enumerate}
\end{definition}

Recall the Hilbert 10th problem, which is, for any given Diophantine equation (a polynomial equation with integer coefficients and a finite number of unknowns), to decide whether the equation has a solution with all unknowns taking integer values. It is easy to observe that given two polynomials with positive integral coefficients over the same set of variables $x_1, \cdots, x_n$, it is undecidable to check whether $f(x_1, \cdots, x_n)=g(x_1, \cdots, x_n)$ has a solution in natural numbers. 

\begin{theorem}
	The satisfiability problem for SL with length constraints is undecidable. 
\end{theorem}

\begin{proof}
	We shall reduce from the aforementioned version of the Hilbert tenth problem. For any polynomial with positive integral  $f(x_1, \cdots, x_n)$ where each coefficient is a positive 
\end{proof}

\subsection{Undecidability of character and length constraints}

\subsection{Extensions with disequalities and IndexOf}