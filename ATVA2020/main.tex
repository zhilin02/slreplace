 \documentclass{llncs}
\usepackage{minted}
%\documentclass[envcountsame, fleqn]{llncs}
  \usepackage{amsmath, amssymb, latexsym}
  \usepackage{graphicx}
\usepackage{epic,eepic}

%\pagestyle{plain}
\usepackage{listings}
\usepackage{psfrag}
\usepackage{rotating}

\usepackage{url}
\usepackage{amssymb,amsthm, epsfig,amstext}
\usepackage{txfonts}
\usepackage{algorithmic}
\usepackage[linesnumbered,ruled,noend]{algorithm2e}
\usepackage{graphicx}

\usepackage{mathrsfs}

\usepackage{color, colortbl}

\renewcommand{\floatpagefraction}{.8}%


\setminted{numbersep=5pt, xleftmargin=12pt}
\usemintedstyle{friendly} 

\usepackage{todonotes}
\usepackage{multirow}
\usepackage{float,color}
\usepackage{picinpar,color,xcolor,wrapfig}
\setcounter{secnumdepth}{4}
\sloppy

\pagestyle{plain}

%%%%%%%%%%%%%%%   MACROS  %%%%%%%%%%%%%%%%%%


%!TEX root = main.tex

\newcommand{\set}[1]{\{ #1 \}}
\newcommand{\sequence}[2]{(#1, \ldots, #2)}
\newcommand{\couple}[2]{(#1,#2)}
\newcommand{\pair}[2]{(#1,#2)}
\newcommand{\triple}[3]{(#1,#2,#3)}
\newcommand{\quadruple}[4]{(#1,#2,#3,#4)}
\newcommand{\tuple}[2]{(#1,\ldots,#2)}
\newcommand{\Nat}{\ensuremath{\mathbb{N}}}
\newcommand{\Rat}{\ensuremath{\mathbb{Q}}}
\newcommand{\Rea}{\ensuremath{\mathbb{R}}}
\newcommand{\Int}{\ensuremath{\mathbb{Z}}}
%\newcommand{\true}{\top}
%\newcommand{\false}{\perp}
\newcommand{\bottom}{\perp}
%% \newcommand{\powerset}[1]{{\cal P}(#1)}
\newcommand{\npowerset}[2]{{\cal P}^{#1}(#2)}
\newcommand{\finitepowerset}[1]{{\cal P}_f(#1)}
\newcommand{\level}[2]{L_{#1}(#2)}
\newcommand{\card}[1]{\mbox{card}(#1)}
\newcommand{\range}[1]{\mathtt{ran}(#1)}
\newcommand{\astring}{s}

\newcommand{\Cc}{\mathcal{C}}

\newcommand{\intnum}{\mathbb{Z}}


\newcommand {\notof}{\ensuremath{\neg}}
\newcommand {\myand}{\ensuremath{\wedge}}
\newcommand {\myor}{\ensuremath{\vee}}
\newcommand {\mynext}{\mbox{{\sf X}}}
\newcommand {\until}{\mbox{{\sf U}}}
\newcommand {\sometimes}{\mbox{{\sf F}}}
\newcommand {\previous}{\mynext^{-1}}
\newcommand {\since}{\mbox{{\sf S}}}
\newcommand {\fminusone}{\mbox{{\sf F}}^{-1}}
\newcommand {\everywhere}[1]{\mbox{{\sf Everywhere}}(#1)}



\newcommand{\aatomic}{{\rm A}}
\newcommand{\aset}{X}
\newcommand{\asetbis}{Y}
\newcommand{\asetter}{Z}

\newcommand{\avarprop}{p}
\newcommand{\avarpropbis}{q}
\newcommand{\avarpropter}{r}
\newcommand{\varprop}{{\rm PROP}} % Set of atomic propositions (for a given logic)

% formulae

\newcommand{\aformula}{\astateformula} % a formula
\newcommand{\aformulabis}{\astateformulabis} % another formula (when at least 2 are present)
\newcommand{\aformulater}{\astateformulater} % another formula (when at least 3 are present)
\newcommand{\asetformulae}{X}
\newcommand{\subf}[1]{sub(#1)}

\newcommand{\aautomaton}{{\mathbb A}}
\newcommand{\aautomatonbis}{{\mathbb B}}

%\newcommand {\length}[1] {\ensuremath{|#1|}}



% Equivalences
\newcommand{\egdef}{\stackrel{\mbox{\begin{tiny}def\end{tiny}}}{=}} % =def=
\newcommand{\eqdef}{\stackrel{\mbox{\begin{tiny}def\end{tiny}}}{=}} % =def=
\newcommand{\equivdef}{\stackrel{\mbox{\begin{tiny}def\end{tiny}}}{\equivaut}} % <=def=>
\newcommand{\equivaut}{\;\Leftrightarrow\;}

\newcommand{\ainfword}{\sigma}

\newcommand{\amap}{\mathfrak{f}}
\newcommand{\amapbis}{\mathfrak{g}}

\newcommand{\step}[1]{\xrightarrow{\!\!#1\!\!}}
\newcommand{\backstep}[1]{\xleftarrow{\!\!#1\!\!}}

\newcommand {\aedge}[1] {\ensuremath{\stackrel{#1}{\longrightarrow}}}
\newcommand {\aedgeprime}[1] {\ensuremath{\stackrel{#1}{\longrightarrow'}}}
\newcommand {\afrac}[1] {\ensuremath{\mathit{frac}(#1)}}
\newcommand {\cl}[1] {\ensuremath{\mathit{cl}(#1)}}
\newcommand {\sfc}[1] {\ensuremath{\mathit{sfc}(#1)}}
\newcommand {\dunion} {\ensuremath{\uplus}}
\newcommand {\edge} {\ensuremath{\longrightarrow}}
\newcommand {\emptyword}{\ensuremath{\epsilon}}
\newcommand {\floor}[1] {\ensuremath{\lfloor #1 \rfloor}}
\newcommand {\intersection} {\ensuremath{\cap}}
\newcommand {\union} {\ensuremath{\cup}}
\newcommand {\vals}[2] {\ensuremath{\mathit{val}_{#2}(#1)}}



\newcommand {\pspace} {\textsc{pspace}}
\newcommand {\nlogspace} {\textsc{nlogspace}}
\newcommand {\logspace} {\textsc{logspace}}
\newcommand {\expspace} {\textsc{expspace}}
\newcommand {\nexpspace} {\textsc{nexpspace}}
\newcommand {\exptime} {\textsc{exptime}}
\newcommand {\np} {\textsc{np}}
\newcommand {\threeexptime} {\textsc{3exptime}}
\newcommand {\polytime} {\textsc{p}}
\newcommand{\twoexpspace}{\textsc{2expspace}}
\newcommand{\threeexpspace}{\textsc{3expspace}}
\newcommand {\nexptime} {\textsc{nexptime}}

\newcommand {\nonelementary} {\textsc{non-elementary}}

\newcommand {\elementary} {\textsc{elementary}}


\newcommand{\aalphabet}{\Sigma}     % an alphabet, A is already used for atoms
\newcommand{\aword}{\mathfrak{u}}
\newcommand{\awordbis}{\mathfrak{v}}



\newcommand{\aassertion}{P}
\newcommand{\aassertionbis}{Q}
\newcommand{\aexpression}{e}
\newcommand{\aexpressionbis}{f}
\newcommand{\avariable}{\mathtt{x}}
\newcommand{\uniquevar}{\mathtt{u}}
\newcommand{\uniquevarbis}{\mathtt{v}}
\newcommand{\avariablebis}{\mathtt{y}}
\newcommand{\avariableter}{\mathtt{z}}
\newcommand{\nullconstant}{\mathtt{null}}
\newcommand{\nilvalue}{nil}
\newcommand{\emptyconstant}{\mathtt{emp}}
\newcommand{\infheap}{\mathtt{inf}}
\newcommand{\saturated}{\mathtt{Saturated}}

\newcommand{\astateformula}{\phi}
\newcommand{\astateformulabis}{\psi}
\newcommand{\astateformulater}{\varphi}
%%
\newcommand{\separate}{\ast}
\newcommand{\sep}{\separate}
\newcommand{\size}{\mathtt{size}}
\newcommand{\sizeeq}[1]{\mathtt{size} \ = \ #1}
\newcommand{\alloc}[1]{\mathtt{alloc}(#1)}
\newcommand{\allocb}[2]{\mathtt{alloc}^{-1}[#2](#1)}
\newcommand{\isol}[1]{\mathtt{isoloc}(#1)}
\newcommand{\icell}{\mathtt{isocell}}
\newcommand{\malloc}{\mathtt{malloc}}
\newcommand{\cons}{\mathtt{cons}}
\newcommand{\new}{\mathtt{new}}
\newcommand{\free}[1]{\mathtt{free} \ #1}
\newcommand{\maxform}[1]{\mathtt{maxForms}(#1)}
\newcommand{\locations}[1]{\mathtt{loc}(#1)}
\newcommand{\values}{\mathtt{Val}}
\newcommand{\aheap}{\mathfrak{h}}
\newcommand{\avaluation}{\mathfrak{V}}
\newcommand{\heaps}{\mathcal{H}}
\newcommand{\astore}{\mathfrak{s}}
\newcommand{\stores}{\mathcal{S}}
\newcommand{\amodel}{\mathfrak{M}}
\newcommand{\alabel}{\ell}

\newcommand{\aprogram}{\mathtt{PROG}}
\newcommand{\programs}{\mathtt{P}}
\newcommand{\ctprograms}{\programs^{ct}}
\newcommand{\aninstruction}{\mathtt{instr}}
\newcommand{\ainstruction}{\mathtt{instr}}
\newcommand{\instructions}{\mathtt{I}}
\newcommand{\aguard}{\ensuremath{g}}
\newcommand{\guards}{\ensuremath{G}}
\newcommand{\domain}[1]{\mathtt{dom}(#1)}
\newcommand{\memory}{\stores\times\heaps}
\newcommand{\skipinstruction}{\mathtt{skip}}

\newcommand{\execution}{\mathtt{comp}}
\newcommand{\aux}{\mathtt{embd}}
\newcommand{\runof}{run}
\newcommand{\anexecution}{e}


\newcommand{\aletter}{\ensuremath{a}}
\newcommand{\aletterbis}{\ensuremath{b}}
\newcommand{\alocation}{\mathfrak{l}}

\newcommand{\pointsl}[1]{\stackrel{#1}{\hookrightarrow}}
\newcommand{\ppointsl}[1]{\stackrel{#1}{\mapsto}}
\newcommand{\ourhook}[1]{\stackrel{#1}{\hookrightarrow}}
\newcommand{\ltrue}{{\sf true}}
\newcommand{\lfalse}{{\sf false}}


\newcommand{\variables}{\mathtt{FVAR}}
\newcommand{\pvariables}{\mathtt{PVAR}}
\newcommand{\secvariables}{\mathtt{SVAR}}
\newcommand{\logique}[1]{\mathtt{FO}(#1)}



\newcommand{\atranslation}{\mathfrak{t}}
\newcommand{\nbpred}[1]{\widetilde{\sharp #1}}
\newcommand{\nbpredstar}[1]{\widetilde{\sharp #1}^{\star}}
\newcommand{\isolated}{\mathtt{isol}}
\newcommand{\stdmarks}{\mathtt{envir}}
\newcommand{\relation}[1]{\mathtt{relation}_{#1}}
\newcommand{\freevar}{\mathtt{FV}}
\newcommand{\notonmark}{\mathtt{notonenv}}
\newcommand{\InVal}[1]{\mathtt{InVal}\!\left(#1\right)}
\newcommand{\NotOnEnv}[1]{\mathtt{NotOnEnv}\!\left(#1\right)}
\newcommand{\PartOfVal}[1]{\mathtt{PartOfVal}\!\left(#1\right)}
%\newcommand{\nbpreds}[3]{\sharp #1 \geq #2}
\newcommand{\defstyle}[1]{{\emph{#1}}}

\newcommand{\cut}[1]{}
\newcommand{\interval}[2]{[#1,#2]}
\newcommand{\buniquevar}{\overline{\uniquevar}}
\newcommand{\bbuniquevar}{\overline{\overline{\uniquevar}}}
\newcommand{\magicwand}{\mathop{\mbox{$\mbox{$-~$}\!\!\!\!\ast$}}}
\newcommand{\wand}{\magicwand}
\newcommand{\septraction}{\stackrel{\hsize0pt \vbox to0pt{\vss\hbox to0pt{\hss\raisebox{-6pt}{\footnotesize$\lnot$}\hss}\vss}}{\magicwand}}
%% \newcommand{\reach}{\mathtt{reach}}
\mathchardef\mhyphen="2D % hyphen while in math mode

\newcommand{\adataword}{\mathfrak{dw}}
\newcommand{\adatum}{\mathfrak{d}}

\newcommand{\collectionknives}{\mathtt{ks}}
\newcommand{\collectionknivesfork}[1]{\mathtt{ksfs}_{=#1}}
\newcommand{\collectionknivesforks}{\mathtt{ksfs}}
\newcommand{\collectionkniveslargeforks}{\mathtt{kslfs}}


\newcommand{\acounter}{\mathtt{C}}

\newcommand{\fotwo}[3]{{\mbox{FO2}_{#1,#2}(#3)}}
\newcommand{\mtrans}[1]{t\!\left(#1\right)^{\Box}}
\newcommand{\mbtrans}[2]{\mtrans{#2}_{#1}}


\newcommand{\alogic}{\mathfrak{L}}


\newcommand{\semantics}[1]{\ensuremath{[ #1 ]}}


\newcommand{\adomino}{\mathfrak{d}}
\newcommand{\atile}{\mathfrak{d}}
\newcommand{\atiling}{\mathfrak{t}}

\newcommand{\hori}{\mathtt{h}}
\newcommand{\verti}{\mathtt{v}}
\newcommand{\domi}{\mathtt{d}}

\newcommand{\cpyrel}{\mathfrak{cp}}

\newcommand{\cntcmp}{\mathfrak{C}}

\newcommand{\heapdag}{\mathfrak{G}}

\newcommand{\onmainpath}{\mathtt{mp}}

\newcommand{\tree}{\mathtt{tree}}

%\newcommand{\tile}{\mathtt{tile}}

\newcommand{\type}{\mathtt{type}}

\newcommand{\ptype}{\mathtt{ptype}}

\newcommand{\exttype}{\mathtt{exttype}}

\newcommand{\anctypes}{\mathtt{AncTypes}}

\newcommand{\destypes}{\mathtt{DesTypes}}

\newcommand{\inctypes}{\mathtt{IncTypes}}

\newcommand{\treeic}{\mathtt{treeIC}}

\newcommand{\trs}{\mathfrak{trs}}


\newcommand{\nin}{\not \in}
\newcommand{\cupplus}{\uplus}
\newcommand{\aunarypred}{\mathtt{P}}


\newcommand{\hide}[1]{}

\newcommand{\eval}[2]{\llbracket#1\rrbracket_{#2}}
\newcommand\cur{\mathsf{cur}}
\newcommand\dom{\mathsf{dom}}
\newcommand\rng{\mathsf{rng}}

\newcommand\dd{\mathbb{D}}
\newcommand\nat{\mathbb{N}}


\newcommand\cA{\mathcal{A}}
\newcommand\cB{\mathcal{B}}
\newcommand\cC{\mathcal{C}}
\newcommand\cE{\mathcal{E}}
\newcommand\cG{\mathcal{G}}
\newcommand\Ll{\mathcal{L}}
\newcommand\cM{\mathscr{M}}
\newcommand\cP{\mathcal{P}}
\newcommand\cR{\mathcal{R}}
\newcommand\cS{\mathcal{S}}
\newcommand\cT{\mathcal{T}}

\newcommand\vard{\mathfrak{d}}

\newcommand\replace{\mathsf{replace}}
\newcommand\replaceall{\mathsf{replaceAll}}
\newcommand\sreplaceall{\mathsf{sreplaceAll}}
\newcommand\reverse{\mathsf{reverse}}
\newcommand\indexof{\mathsf{indexOf}}
\newcommand\length{\mathsf{length}}
\newcommand\substring{\mathsf{substring}}
\newcommand\charat{\mathsf{charAt}}
\newcommand\extract{\mathsf{extract}}

\newcommand\revsym{\pi}

\newcommand\strline{\mathsf{SL}}

\newcommand\pstrline{\mathsf{SL_{pure}}}

\newcommand\search{\mathsf{search}}

\newcommand\verify{\mathsf{vfy}}

\newcommand\searchleft{\mathsf{left}}

\newcommand\searchlong{\mathsf{long}}


\newcommand\pref{\mathsf{Pref}}

\newcommand\wprof{\mathsf{WP}}

\newcommand\vars{\mathsf{Vars}}

\newcommand\dep{\mathsf{Dep}}
\newcommand\ptn{\mathsf{Ptn}}

\newcommand\src{\mathsf{src}}
\newcommand\strtorep{\mathsf{strToRep}}

\newcommand\rpleft{\mathsf{l}}
\newcommand\rpright{\mathsf{r}}


\newcommand\srcnd{\mathsf{srcND}}

\newcommand\ctxt{\mathsf{ctxt}}


\newcommand\ctxts{\mathsf{Ctxts}}

\newcommand\sprt{\mathsf{sprt}}

\newcommand\val{\mathsf{val}}

\newcommand\srclen{\mathsf{srcLen}}

\newcommand\rpleftlen{\mathsf{lLen}}


\newcommand\dfs{\mathsf{DFS}}

\newcommand\repr{\mathsf{rep}}

\newcommand\red{\mathsf{red}}

\newcommand\gfun{\mathcal{F}}


\newcommand{\leftmost}{{\sf leftmost}}
\newcommand{\longest}{{\sf longest}}

\newcommand{\arbidx}{{\sf Idx_{arb}}}
\newcommand{\dmdidx}{{\sf Idx_{dmd}}}
\newcommand{\lftlen}{{\sf Len_{lft}}}


\newcommand{\ASSERT}[1]{\textsf{assert}\left(#1\right)}

\newcommand{\concat}{\cdot}

\newcommand{\parabs}{\Theta} % parametr abstraction
\newcommand{\arity}{r}

\newcommand{\FA}{FA}
\newcommand{\FFA}{2FA}
\newcommand{\SFFA}{S2FA}
\newcommand{\FT}{FT}
\newcommand{\FFT}{2FT}
\newcommand{\FunFT}{FFT}
\newcommand{\SFFT}{S2FT}
\newcommand{\PT}{PT}
\newcommand{\PPT}{2PT}
\newcommand{\SPPT}{S2PT}
\newcommand{\RBPPT}{RB2PT}
\newcommand{\RBSPPT}{RBS2PT}
\newcommand{\SA}{SA}
\newcommand{\SSA}{2SA}
\newcommand{\ST}{ST}
\newcommand{\SST}{2ST}
\newcommand{\SPT}{SPT}
\newcommand{\SSPT}{2SPT}
\newcommand{\RBSSPT}{RB2SPT}

\newcommand{\ialphabet}{\Sigma}
\newcommand{\oalphabet}{\Gamma}


\newcommand{\EndLeft}{\ensuremath{\vartriangleright}}
\newcommand{\EndRight}{\ensuremath{\vartriangleleft}}

\newcommand{\Lang}{\mathscr{L}}
\newcommand{\Tran}{\mathscr{T}}

\newcommand{\NFA}{\mathcal{A}}
\newcommand{\NFAB}{\mathcal{B}}
\newcommand{\NFT}{\mathcal{T}}

\newcommand{\CEFA}{\mathcal{A}}

\newcommand{\Transducer}{\ensuremath{T}}
\newcommand{\controls}{\ensuremath{Q}}
\newcommand{\finals}{\ensuremath{F}}
\newcommand{\transrel}{\ensuremath{\delta}}


\newcommand{\Left}{\ensuremath{-1}}
\newcommand{\Right}{\ensuremath{1}}
\newcommand{\Stay}{\ensuremath{0}}


\newcommand{\defn}[1]{\emph{#1}}

\newcommand{\conacc}{\Omega}

\newcommand{\reginvrel}{\textbf{RegInvRel}}

\newcommand{\prerec}{\reginvrel}

\newcommand{\regmondec}{\textbf{RegMonDec}}

\newcommand\rcdim{\mathsf{art}}

\newcommand\rcdep{\mathsf{asgn}}

\newcommand\rcasrt{\mathsf{asrt}}

\newcommand\rcreg{\mathsf{reg}}

\newcommand\rcsreg{\mathsf{sreg}}


\newcommand\tower{\mathrm{Tower}}

\newcommand\rcphi{\mathsf{fnsize}}
\newcommand\rcpsi{\mathsf{fasize}}

%%%%%%%%%%%%
% Auto escape eg

\newcommand\linkvar{\mintinline{html}{$li}}
\newcommand\linktextvar{\mintinline{html}{$txt}}
\newcommand\urlstarttag{\texttt{[URL]}}
\newcommand\urlendtag{\texttt{[LRU]}}
\newcommand\htmlstarttag{\texttt{[HTML]}}
\newcommand\htmlendtag{\texttt{[LMTH]}}

\newcommand{\Pre}{\textsf{Pre}}

\newcommand\bigO{\mathcal{O}}

\newcommand\Aut{\mathcal{A}}

\newcommand{\tup}[1]{\left( #1 \right)}

\newcommand\ap[2]{{#1}\mathord{\brac{#2}}}
%\newcommand\ap[2]{{#1}\mathord{\brac{#2}}}

\newcommand{\opset}{\mathscr{O}}

\newcommand{\strlineall}{$\strline$[\FT, $\replaceall$, $\reverse$]}

\newcommand{\strlineconcat}{$\strline$[$\concat$, $\sreplaceall$, $\reverse$, \FT]}

\newcommand{\strlinefft}{$\strline$[$\concat$, $\replaceall$, $\reverse$, \FunFT]}


%%% Macros for expspace lower bound with replaceall

% General
\newcommand\brac[1]{\left(#1\right)}

% \setcomp{ele}{comp} = { ele | comp }
\newcommand\setcomp[2]{\left\{{#1}\ \middle|\ {#2}\right\}}

% \lang{A} = L(A)
\newcommand\lang[1]{\mathcal{L}\mathord{\brac{#1}}}

%% Optimisations

\newcommand\caleybox[1]{\llparenthesis #1 \rrparenthesis}
\newcommand\internalchar{\flat}

% Tools

\newcommand\stranger{Stranger}

\newcommand\transducerbench{\textsc{Transducer}}
\newcommand\slogbench{\textsc{SLOG+}}
\newcommand\slogbenchr{\textsc{SLOG+(replace)}}
\newcommand\slogbenchra{\textsc{SLOG+(replaceall)}}
\newcommand\kaluzabench{\textsc{Kaluza}}
\newcommand\pyexbench{\textsc{PyEx}}
\newcommand\pyextdbench{\textsc{PyEx-}td}
\newcommand\pyexztbench{\textsc{PyEx-}z3}
\newcommand\pyexzzbench{\textsc{PyEx-}zz}

\newtheorem{fact}{Fact}
\newtheorem{remark}{Remark}


\newcommand\slint{${\rm SL}_{\rm int}$}

\newcommand\cslint{SL$^{\dag}_{int}$}

\newcommand{\regexp} {{\sf Regex}}
%\newcommand{\cgexp} {{\sf RWRE_{reg}}}
\newcommand{\pcre} {{\sf PCRE}}

\newcommand{\lasat}{${\rm SAT}_{\rm CEFA}[{\rm LIA}]$}

\newcommand{\prjnum}{{\rm Prj}_{\rm num}}

\newcommand{\uwp}{{\rm uwp}}

\newcommand\urlxsssanitise{{\sf urlXssSanitise}}

%%%%%%%%%% Start TeXmacs macros
\newcommand{\colons}{\,:\,}
\newcommand{\tmop}[1]{\ensuremath{\operatorname{#1}}}
\newcommand{\tmtextit}[1]{{\itshape{#1}}}
\newcommand{\tmtextbf}[1]{{\bfseries{#1}}}
%\newtheorem{definition}{Definition}
%{\theorembodyfont{\rmfamily}\newtheorem{note}{Note}}
%{\theorembodyfont{\rmfamily}\newtheorem{remark}{Remark}}
%\newtheorem{theorem}{Theorem}
%\newtheorem{note}{Note}
%\newtheorem{remark}{Remark}
\newcommand\NSST{{\sf NSST}}
\newcommand\PSST{{\sf PSST}}
\newcommand\refexp{{\sf REP}}
\newcommand\ssym{{\sf Start}}
\newcommand\esym{{\sf End}}

\newcommand\pnfa{\mathcal{A}}
\newcommand\psst{\mathcal{T}}

\newcommand\pat{\mathsf{pat}}
\newcommand\rep{\mathsf{rep}}

\newcommand\refbefore{\$^{\leftarrow}}
\newcommand\refafter{\$^{\rightarrow}}

\newcommand\nullchar{\mathsf{null}}

\newcommand\idxexp{{\sf idx}}


\newcommand\cvc{CVC4}
\newcommand\zthree{Z3-str3}
\newcommand\trau{Trau}
\newcommand\trauplus{Trau+}
\newcommand\zthreetrau{Z3-Trau}
\newcommand\ostrich{EMU}
\newcommand\expose{ExpoSE}
\newcommand\sloth{Sloth}
\newcommand\slent{Slent}
\newcommand\Tool{\text{OSTRICH+}}
\newcommand{\OMIT}[1]{}

\newcommand{\seq}[1]{\ensuremath{#1}}
\newcommand{\seqq}[1]{\seq{\Gamma\ifx#1\relax\else,#1\fi}}

\newcommand{\infer}[3][]{%
  \AxiomC{$#3$}
  \ifx#1\relax\else\LeftLabel{\textsc{#1}}\fi
  \UnaryInfC{$#2$}
  \DisplayProof
}
\newcommand{\inferC}[3]{%
  \AxiomC{$#3$}
  \RightLabel{$~~#1$}
  \UnaryInfC{$#2$}
  \DisplayProof
}
\newcommand{\inferii}[4][]{%
  \AxiomC{$#3$}
  \AxiomC{$#4$}
  \ifx#1\relax\else\LeftLabel{\textsc{#1}}\fi
  \BinaryInfC{$#2$}
  \DisplayProof
}


%%%%%%%%%%%%%%%%%%%%%%%%%%%%%%%%%%%%%%%%%%%%

\newcommand{\zhilin}[1]{\color{cyan} {ZL: #1 :LZ} \color{black}}
\newcommand{\tl}[1]{\color{purple} {TL: #1 :LT} \color{black}}
\newcommand{\anthony}[1]{\color{blue} {AL: #1 :LA} \color{black}}
\newcommand{\matt}[1]{\color{blue} {MH: #1 :HM} \color{black}}
\newcommand{\philipp}[1]{\color{blue} {PR: #1 :RP} \color{black}}

%%%%%%%%%%%%%%%%%%%%%%%%%%%%%%%%%%%%%%%%%%%%


\title{A Decision Procedure for Path Feasibility of String \\
 Manipulating Programs  with Integer Data Type}


\author{Taolue Chen\inst{1} \and Matthew Hague\inst{2} \and Jinlong He\inst{3} \and Denghang Hu\inst{3} \and Anthony Widjaja Lin\inst{4} \and Philipp Ruemmer\inst{5} \and Albin Stjerna\inst{5} \and Zhilin Wu\inst{4}}

\institute{Birkbeck, University of London, UK
	\and Royal Holloway, University of London, UK
	\and Institute of Software, Chinese Academy of Sciences, China
    \and Uppsala University, Sweden
	\and  Technical University of Kaiserslautern, Germany
}

\begin{document}


%Regular Papers should not exceed 20 pages in LNCS format, not counting references and appendices.
%
%Regular papers at CAV 2020 will follow a full double blind review process, which means that author names and affiliations must be omitted from the submission. Additionally, if a submission refers to prior work done by the authors, the reference should be made in the third person. These are firm submission requirements, and any regular paper that does not conform to these requirements will be rejected without review.
%
%Authors can include a clearly marked appendix at the end of their submissions that is exempt from the page limit restrictions. However, the reviewers are not obliged to read the contents of these appendices. These papers should contain original research and sufficient detail to assess the merits and relevance of the contribution. Papers will be evaluated on basis of a combination of correctness, technical depth, significance, novelty, clarity, and elegance.

\maketitle

%\vspace{-1cm}

\begin{abstract}
\begin{abstract}
%% some background on regular expressions
Regular expressions (RE) are a classical concept in formal language theory.
%, which are expressions built from characters by the operators of concatenation, union, and Kleene star. 
Real-world regular expressions (RWRE) in programming languages differ from REs 
in the non-standard semantics of operators (e.g. non-commutative union and 
greedy/lazy Kleene star), as well as  the additional features like capturing
groups and backreferences. 
%% symbolic execution requires faithful encoding of regex semantics
%String constraint solvers are one of the cornerstones of the analysis and verification of string manipulating programs. Faithful encoding of the semantics of real-world regular expressions in string constraint solvers facilitates more precise program analysis and verification.
%% state of the art string constraint solvers
%The semantics of real-word regular expressions are tricky and vary in different programming languages. It is challenging for string constraint solvers to support real-world regular expressions.
While REs are supported by every state-of-the-art string constraint solver, 
RWREs are thus far unsupported. Recent works have suggested that the mismatch 
between REs %in string constraint 
%solvers and RWREs in programming languages 
%hinders the precision and efficiency 
makes it difficult for a symbolic execution engine to deal with RWREs,
which are hitherto approximated by a CEGAR-based approach, which performs
many satisfiability checks of string constraints with only REs.
%; hitherto,
%RWREs are either approximated away
%
%by resorting to approximate 
%encoding of RWREs and counter-example guided abstraction refinements (CEGAR).
%
%The semantics of real-world regular expressions are better to be encoded as faithfully as possible. For instance, in dynamic symbolic execution of string manipulating programs, in order to generate the inputs for execution paths and improve coverage,  the semantics of real-world regular expressions are better to be encoded as faithfully as possible in string constraint solvers. 
%% contribution
In this paper we propose an approach of natively supporting RWREs in string
constraint solving. 
The key of our approach is to introduce a new automata model called prioritized streaming string transducers (PSST), 
which combines priorities in prioritized finite transducers and string variables in streaming string transducers, to model the string functions involving RWREs.
%non-standard semantics of regular expression operators can be modeled by priorities and new features of capturing groups and back references can be modeled by string variables. 
%the non-standard semantics of regular expression operators can be modeled by priorities and new features of capturing groups and back references can be modeled by string variables. 
Based on PSSTs, we design a decision procedure for string constraints with 
RWREs and provide its implementation.
%% implementation and experiments
%We implement the decision procedure 
%and do extensive experiments to evaluate its performance. 
We evaluate its performance on on over 160,000 string constraints generated 
from RWREs in open-source programs.
Our approach dramatically improves the CEGAR-based approach for RWREs in both 
precision and efficiency. 
%To the best of our knowledge, this work represents the first string constraint solver supporting RWREs.
\end{abstract}

\end{abstract}

%\section{Introduction}
%\label{intro}
%
%\section{Preliminaries}
%\label{prel}
%
%Definition for NFA, NFT.

%%%%%%%%%%%%%%%%%%%%%%%%%%%%%%%%%%%%%%%%%%%%%%%%%%%%%%%
\section{Introduction} \label{sec:intro}

%!TEX root = main.tex

\section{Introduction}


% general intro on string constraint solving

%
%Strings are among the most important data types. 
Modern high-level programming languages like JavaScript, Python, Java,
and PHP natively support a variety of string operations, and use
strings to store and process virtually any kind of data or code.
Applied string operations range from concatenation, splitting, and
replacement, to complex functions like regular expression matching and
character
encoding/decoding.  As a result, string-manipulating programs are
notoriously subtle, error-prone, and their potential bugs may bring
severe security consequences. A typical example is cross-site
scripting (XSS), which is among the OWASP Top 10 Application Security
Risks.
%Regular expressions are widely used in string-manipulating programs. 
An effective and increasingly popular method for identifying such bugs
in the program is symbolic execution, possibly in combination with dynamic
analysis. In a nutshell, this technique analyses a static path in a
a program under consideration, by viewing it as a constraint $\phi$, whose 
feasibility can be checked by constraint solvers.

Regular expression matching is one of the most important string operations
in programming languages \cite{Berkeley-JavaScript,BM17,LMK19,HAMPI}.
Most state-of-the-art string constraint solvers (e.g.
Z3, CVC4, Z3-str/2/3/4, ABC, Norn, Trau, OSTRICH, S2S, Qzy, Stranger, Sloth, Slog, Slent, Gecode+S, G-Strings, HAMPI... \anthony{make sure we add all}) therefore support
the \emph{regular expression constraints}, e.g., matching a string with a 
regular expression, as we know it from formal language theory. Unfortunately, 
\emph{Real-world Regular Expressions} (RWRE) in programming languages are dramatically different from 
\emph{classical Regular Expressions} (RE) in formal language theory. 
%In the sequel, we call the former as \emph{real-word} regular expressions and the latter as \emph{classical} regular expressions. 
Classical regular expressions are built from letters by the operators of
concatenation, union, and Kleene star, and have nice compositional semantics. On
the other hand, RWREs differ from classical ones mainly in the following two 
aspects: 1) non-standard semantics of 
operators, e.g., the non-commutative union, the greedy/lazy Kleene star, and 2) new 
features, e.g., capturing groups and backreferences.
RWREs are in general more expressive than classical REs, e.g., it is known that
with backreferences one can easily generate languages that are not even 
context-free (e.g. see \cite{FS19,Aho90,BM17b}). %It is an open question whether 
\begin{example}
    Consider the RWRE \mintinline{javascript}{(\d+)(\d*)}. It has two capturing
    groups (i.e. each within a pair of opening/closing brackets), each matches
    a string of digits (signified by \mintinline{javascript}{\d}). The second 
    capturing group
    could be matched with an empty sequence of digits. Given a string of digits
    (e.g. \texttt{"2050"}), the entire string will always be matched by the
    first subexpression \mintinline{javascript}{(\d+)}, owing to the greedy semantics of
    RWREs. 

    Consider another RWRE \mintinline{javascript}{(\d+)\1\1}. This contains two
    backreferences (i.e. \mintinline{javascript}{\1}), each is to be 
    matched precisely with
    the matched word of the first capturing group. So, it
    accepts precisely the set $L$ of all the words $www$, where $w$ is a 
    nonempty sequence of digits, which is not a context-free language.
    \qed
\end{example}

%The semantics of RWREs are tricky and can be different in different programming languages. 
%Real-world regular expressions are challenging for string constraint solvers. The state-of-the-art string constraint solvers e.g. CVC4 and Z3-str only support classical regular expressions. 

RWREs in real-world programs are also commonly used in combination with
other string operations (e.g. match and replace(all) functions \cite{LMK19}),
which pose additional challenges to symbolic execution engines.
On a given string $s$ and a RWRE $e$, the match function allows one to extract 
the last match of a capturing group in $e$ with respect to $s$. 
For the replace function, on a given string $s$, a matching pattern RWRE $e$, and a replacement string $t$, it replaces the first match (or all 
matches, if the global flag is enabled) of $e$ in $s$ by $t$. Here $t$
could contain references to the matches of various capturing groups
in $e$. 
\begin{example}
    We give a more extensive example in Section \ref{sec:mot}, which 
    simultaneously involves both match and replace. Consider the snippet
    \begin{minted}{javascript}
        var namesReg = /([A-Za-z]+) ([A-Za-z]+)/g;
        var newAuthorList = authorList.replace(nameReg,$2, $1);
    \end{minted}
    Assuming \texttt{authorList} is given as a \texttt{;}-separated 
    list of author names --- first name, followed by a last name ---
    the above program would convert this to last name, followed by first name
    format. For example, \texttt{"Don Knuth; Alan Turing; Bob Floyd"} would
    be converted to \texttt{"Knuth, Don; Turing, Alan; Floyd, Bob"}.
    \qed
\end{example}
\OMIT{
The semantics of RWREs drastically affect the behaviors of these functions. In particular, one must take a special care of the
greedy/lazy semantics of Kleene star, which cannot 
be captured in a complete way as constraints over word equations and classical 
REs. 
\anthony{More to come}
}


Since the state-of-the-art string solvers support only classical REs instead of
RWREs, % are not primitively supported by state-of-the-art string solvers
%(in fact, they are in general ,
existing symbolic execution approaches that handle
string-manipulating programs with RWREs apply a workaround.
We mention here Aratha \cite{aratha} and Expose \cite{LMK19}, both of which are
symbolic execution engines for JavaScript programs.
%symbolic executors of string manipulating programs, e.g. 
Aratha performs a very rough approximation to the 
non-standard semantics of regular expressions, e.g., a backreference
is replaced by the regular expression $\ialphabet^*$ that accepts all words.
%referred to by the backreference 
%operator. 
On the other hand, Expose attempts to exploit the power of string 
equations and classical REs (as implemented in Z3 \cite{Z3}) supported by string
solvers to capture the 
semantics of RWREs. Unfortunately, the semantics of RWREs cannot 
in general be fully captured by string constraints with REs. 
%This is caused by
%the aforementioned features of RWREs: greedy semantics
%, especially in the
%presence of the greedy semantics of backreferences. 
%It is even an open
%question if even the greedy semantics of RWREs have to be 
For this reason, 
Expose attempts to instead approximate the semantics of RWREs in the style of 
CEGAR (counter-example guided abstraction and refinement). This, however,
results in a rather severe price in both precision and performance: The refinement process may not terminate and the symbolic execution of a simple program with RWREs may need to be refined for tens of times. 
%For one, satisfiability of string equations with regular constraints is
%well-known to be PSPACE-complete \cite{J16,Kozen77,P04}. For another, to the 
%best of our knowledge, no existing string solver is complete for string 
%equations with regular constraints.

%Typical string operations involving RWREs in programming languages include match, exec, test, search/find, and replace.



%In particular, 
 %to approach the genuine semantics of real-world regular expressions. 
%Although the CEGAR approach of Expose made a first step towards tackling the semantics of real-world regular expressions in the analysis and verification of string-manipulating programs, it is still unsatisfactory in both the precision and performance: 1) although CEGAR approximates the semantics of real-world regular expressions to a greater precision, it is still imprecise, 2) tens of refinement steps or even more are needed for simple string-manipulating programs containing regular expressions. Direct support of real-world regular expressions in string constraint solvers would facilitate the improvement of both the precision and scalability of symbolic executions of string manipulating programs.

\paragraph*{Contributions}
This paper proposes a novel approach to support real-world regular expressions in string constraint solving. The main idea of our approach is to propose a new automata model, called prioritized streaming string transducers (PSST). The model of PSST extends and combines prioritized finite-state automata \cite{BM17} and streaming string transducers \cite{AC10,AD11}. With PSST, we encode the non-standard semantics of regular expression operators by priorities and deal with capturing groups and backreferences by string variables. 
The widely used string functions involving regular expressions, e.g. match and replace(all), can be easily transformed into PSSTs. 

We then design a decision procedure for a class of string constraints with real-world regular expressions. The decision procedure extends the backward reasoning approach proposed in \cite{CHL+19} to PSSTs. Specifically, we show that the pre-images of regular languages under PSSTs are regular and can be computed effectively. 

We implement the decision procedure in our new solver \ostrich,
on top of the existing open-source solver~OSTRICH~\cite{CHL+19},
 and carry out extensive experiments to evaluate the performance. For the benchmarks, we generate \zhilin{xxx} Javascript programs from a library of real-world regular expressions \cite{DMC+19}, by using some simple Javascript program template containing match and replace functions.  Then we generate all the path constraints for each Javascript program and put them into one SMT file. We run {\ostrich} on these SMT files. The average running time on each file is \zhilin{xxx} seconds. For comparison, we also run Expose on the Javascript programs. The average running time for each program is \zhilin{xxx} seconds. The huge difference of the running time shows that our approach can reason about RWREs in a much more precise and faster way than the CEGAR-based approach.


\paragraph*{Related work.}

RWREs have been investigated in formal language theory. Regular expressions with capturing groups and backreferences have been considered in \cite{CSY03,CN09,Freydenberger13,Schmid16,FS19}, where the expressibility issues and decision problems were investigated. Nevertheless, some basic features of RWREs, namely, the non-commutative union and the greedy/lazy semantics of Kleene star, were not addressed therein.

Prioritized finite-state automata and prioritized finite-state transducers were proposed in \cite{BM17}. Prioritized finite-state transducers add indexed brackets to the input string in order to identify the matches of capturing groups. In contrast, PSSTs store the matches of capturing groups into string variables, which can then be referred to later on, thus allowing to conveniently model the match and replace(all) function. 
%
Streaming string transducers were used in \cite{ZAM19} to solve the straight-line string constraints with concatenation, finite-state transducers, and regular constraints.

RWREs have also received the attentions in software engineering community. Some empirical studies were reported for RWREs recently, including portability across different programing languages \cite{DMC+19} and DDos attacks \cite{SP18}, as well as how programmers write RWREs in practice \cite{MDD+19}.

%%%%%%%%%%%%%%%%%%%%%%%%%%%%%%%%%%%%%%%%%%%%%%%%%%
%%%%%%%%%%%%%%%%%%%%%%%%%%%%%%%%%%%%%%%%%%%%%%%%%%
\hide{
Strings are a fundamental data type in virtually all programming languages.
%Their generic nature can, however, lead to many subtle programming
%bugs, some with security consequences, e.g., cross-site scripting
%(XSS), which is among the OWASP Top 10 Application Security Risks
%\cite{owasp17}. 

One effective automatic testing method for identifying subtle programming errors  is based on \emph{symbolic execution}
\cite{king76} and combinations with dynamic analysis
called \emph{dynamic symbolic execution} \cite{jalangi,DART,EXE,CUTE,KLEE}.
See \cite{symbex-survey} for an excellent survey. 

Unlike purely random testing,
which runs only \emph{concrete} program executions on different
inputs, the techniques of symbolic execution analyse \emph{static} paths
(also called symbolic executions) through the software system under test.
Such a path can be viewed as a constraint $\varphi$ (over
appropriate data domains) and the hope is that a fast
solver is available for checking the satisfiability of $\varphi$ (i.e. to check
the \emph{feasibility} of the static path), which can be used for generating
inputs that lead to certain parts of the program or an erroneous behaviour.
%a undesirable program behaviour.
%or an exploration of a new part of the
%system.


%
In this paper, we focus on two string operations with emphasis on practical usage of  regular expressions. Rather than textbook style regular expressions, regular expressions used in programming languages are considerably more involved. On particular feature we consider is the capturing group. This is particularly useful for string pattern matching 
%Many regular expression matching libraries perform matching as a form of parsing by using capturing groups,and thus 
where it can be returned what subexpression matched which substring. 

%This form of regular expression matching requires theoretical un-derpinnings different from classical regular expressions as defined in formal language theory. 


%which effective serves as a register when matching the regular expression to a string. Accompanying to the capturing groups 


%Many regular expression matching libraries perform matching as a form of parsing by using capturing groups,and thus output what subexpression matched which substring[9]. This form of regular expression matching requires theoretical un-derpinnings different from classical regular expressions as defined in formal language theory. 
%
%A popular implementation strategy used for performing regular expression matching (or parsing) with capturing groups, used for example in Java, .NET and the PCRE library[14], is a worst-case exponential time depth-first search strategy. A formal approach to matching with capturing groups can be obtained by using finite state transducers that output annotations on the input string to signify what subexpression matched which substring[16]. 
%
%A complicating factor in this approach is introduced by the fact that the matching semantics dictates a single output string for each input string, obtained by using rules to determine a “highest priority” match among the potentially exponentially many possible ones (in contrast,[6]discusses non-deterministic capturing groups).

The \emph{string-replace function}, 
which may be used to replace all occurrences of a string matching a pattern by 
another string. 

The replace function (especially 
the replace-all functionality) is omnipresent in HTML5 applications
\cite{LB16,TCJ16,YABI14}. 
%\mat{What does it mean for the replace function to be convincingly argued?}

A regular expression (shortened as regex) is a sequence of characters that define a search pattern. Usually such patterns are used by string-searching algorithms for "find" or "find and replace" operations on strings, or for input validation.  

The semantics of regular expressions with capturing groups and backreferences is rather involved. One of the reasons is that different languages may choose different semantics for a regex to match the string when the regex is served as a pattern. 

To capture the semantics, priority is introduced, giving rise to an extension of the standard finite-state automata. However, this is not sufficient for capturing string operations. For that purpose, we introduce  a new transducer model, prioritized streaming string transducer (PSST) which is a combination of priority which is essential for modelling capturing groups and streaming transducers which are a highly expressive formalism for modelling string operations. 
}
%%%%%%%%%%%%%%%%%%%%%%%%%%%%%%%%%%%%%%%%%%%%%%%%%%
%%%%%%%%%%%%%%%%%%%%%%%%%%%%%%%%%%%%%%%%%%%%%%%%%%


%%%%%%%%%%%%%%%%%%%%%%%%%%%%%%%%%%%%%%%%%%%%%%%%%%%%%%%
\section{Preliminaries}\label{sec:prel}

%!TEX root = main.tex

We write $\Nat$ and $\Int$ for the sets of natural and integer numbers, respectively. For $n \in \Nat$ with $n \ge 1$, $[n]$ denotes $\{1, \ldots, n\}$; for $m,n \in \Nat$ with $m \le n$,  $[m, n]$ denotes $\{ i \in \Nat \mid m \le i \le n \}$. Throughout the paper, $\Sigma$ is a finite alphabet, and $a,b,\ldots$ range over letters.

Let $\eta_1: X \rightarrow Z$ and $\eta_2: Y \rightarrow Z$ be two functions such that $X \cap Y = \emptyset$. We use $\eta_1 \cup \eta_2$ to denote the function $\eta: X \cup Y \rightarrow Z$ such that for each $\alpha \in X \cup Y$, $\eta(\alpha) = \eta_1(\alpha)$  if $\alpha \in X$, and $\eta(\alpha) = \eta_2(\alpha)$ otherwise. 

\paragraph*{Strings, languages, and transductions.}
A string over $\Sigma$ is a (possibly empty) sequence of elements from $\Sigma$,
denoted by $u, v, w, \ldots$. An empty string is denoted by $\varepsilon$.  We write $\Sigma^*$ (resp., $\Sigma^+$) for the set of all (resp. nonempty) strings over $\Sigma$.

Let $u$ be a string over $\Sigma$. We use $|u|$ to denote the number of letters in $u$. In particular, $|\varepsilon|=0$. Moreover, for $a \in \Sigma$, let $|u|_a$ denote the number of occurrences of $a$ in $u$. 
Assume $u=a_0\cdots a_{n-1}$ is nonempty and $i<j \in [0,n-1]$. %Then a \emph{position} of $u$ is a number $i \in [|u|]$ (Note that the first position is $1$, instead of  0). In addition, 
We let $u[i]$ denote $a_i$ and $u[i,j]$ for the substring %of $u$ starting from $i$ and ending with $j$ (i.e., 
$a_i\cdots a_j$. 
%\tl{check later for consistency}\zhilin{the indices start from 0, to be consistent with the semantics of substring and indexof}\zhilin{i do use it in section 4}

Let $u, v$ be two strings. We use $u \cdot v$ to denote the \emph{concatenation} of $u$ and $v$, that is, the string $w$ such that $|w|= |u| + |v|$ and for each $i \in [0, |u|-1]$, $w[i]= u[i]$, and for each $i \in [0,|v|-1]$, $w[|u|+i]=v[i]$. The string $u$ is said to be a \emph{prefix} of $v$ if $v = u \cdot v'$ for some string $v'$.
In addition, if $u \neq v$, then $u$ is said to be a \emph{strict} prefix of $v$. If $u$ is a prefix of $v$, that is, $v = u \cdot v'$ for some string $v'$, then 
we use $u^{-1} v$ to denote $v'$. In particular, $\varepsilon^{-1} v = v$.
If $u=a_0 \cdots a_{n-1}$ is nonempty, then we use $u^{(r)}$ to denote the \emph{reverse} of $u$, that is, $u^{(r)}= a_{n-1 }\cdots a_0$. %\tl{$u^r$ check later for consistency}


A \emph{language} over $\Sigma$ is a subset of $\Sigma^*$, denoted by  $L_1, L_2, \dots$. For two languages $L_1$ and $L_2$, let $L_1 \cdot L_2$ denote the concatenation of $L_1$ and $L_2$, that is, the language $\{u_1 \cdot u_2 \mid u_1 \in L_1, u_2 \in L_2\}$. 
For a language $L$ and $n \in \Nat$, we define the \emph{iteration} $L^n$ of $L$  inductively by $L^0=\{\varepsilon\}$ and $L^{n} =L \cdot L^{n-1}$ for $n > 0$. We also use $L^*$ to denote an arbitrary number of iterations of $L$, that is, $L^* = \bigcup _{n \in \Nat} L^n$. Moreover, let $L^+ = \bigcup _{n \in \Nat \setminus \{0\}} L^n$. 

A \emph{transduction} over $\Sigma$ is a binary relation over $\Sigma^*$, namely, a subset of $\Sigma^* \times \Sigma^*$. We will use $T_1, T_2,\ldots$ to denote transductions. For two transductions $T_1$ and $T_2$, we will use $T_1 \cdot T_2$ to denote the \emph{composition} of $T_1$ and $T_2$, namely, $T_1 \cdot T_2 = \{(u, w) \in \Sigma^* \times \Sigma^* \mid \emph{there exists } v \in \Sigma^* \mbox{ s.t. } (u,v) \in T_1 \mbox{ and } (v,w) \in T_2\}$.

\paragraph*{Regular languages.}
A language $L$ is \emph{regular} if it can be defined by a regular expression, or equivalently by a finite automaton.  
Regular expressions $\regexp$ are defined by:
%
	\[e \eqdef \emptyset \mid \varepsilon \mid a \mid e + e \mid e \concat e \mid e^*, \mbox{ where } a \in \Sigma. \]
%
Since $+$ is associative and commutative, we also write $(e_1 + e_2) + e_3$ as $e_1 + e_2 + e_3$ for brevity. We use the abbreviation $e^+ \equiv e \concat e^*$. Moreover, for $\Gamma = \{a_1, \ldots, a_n\}\subseteq \Sigma$, we use the abbreviations $\Gamma \equiv a_1 + \cdots + a_n$ and $\Gamma^\ast \equiv (a_1 + \cdots + a_n)^\ast$. 

We define $\Ll(e)$ to be the language defined by $e$, that is, the set of strings that match $e$, inductively as follows: $\Ll(\emptyset) =\emptyset$,
%\begin{itemize}
%\item 
$\Ll(\varepsilon) =\{\varepsilon\}$,
%
%\item 
$\Ll(a)= \{a\}$,
%
%\item 
$\Ll(e_1 + e_2) = \Ll(e_1) \cup \Ll(e_2)$,
%
%\item 
$\Ll(e_1 \concat e_2) = \Ll(e_1) \cdot \Ll(e_2)$,
%
%\item 
$\Ll(e_1^*)=(\Ll(e_1))^*$.
%\end{itemize}
In addition, we use $|e|$ to denote the number of symbols occurring in $e$.

A \emph{nondeterministic finite automaton} (NFA) $\NFA$ is a tuple $(Q, \Sigma, \delta, I, F)$, where $Q$ is a finite set of states, $\Sigma$ is a finite alphabet, $\delta \subseteq Q \times \Sigma \times Q$ is the transition relation, $I,F \subseteq Q$ are the set of initial and final states respectively. For readability, we write a transition $(q, a, q') \in \delta$ as $q \xrightarrow[\delta]{a} q'$. Moreover, when $\delta$ is clear from context, we omit $\delta$ in $q \xrightarrow[\delta]{a} q'$ and write $q \xrightarrow{a} q'$. The \emph{size} of an NFA $\NFA$, denoted by $|\NFA|$, is defined as the number of transitions of $\NFA$.
%
A \emph{run} of $\NFA$ on a string $w = a_1 \cdots a_n$ is a sequence of transitions $q_0 \xrightarrow{a_1} q_1 \cdots q_{n-1} \xrightarrow{a_n} q_n$ with $q_0 \in I$. The run is \emph{accepting} if $q_n \in F$.
A string $w$ is accepted by an NFA $\NFA$ if there is an accepting run of $\NFA$ on $w$. In particular, the empty string $\varepsilon$ is accepted by $\NFA$ if $I \cap F \neq \emptyset$. The language defined by $\NFA$, denoted by $\Lang(\NFA)$, is the set of strings accepted by $\NFA$. An NFA $\NFA$ is said to be \emph{deterministic} if $I$ is a singleton, moreover, for every $q \in Q$ and $a \in \Sigma$, there is at most one state $q' \in Q$ such that $(q, a, q') \in \delta$.

\paragraph*{Recognizable relations.} Intuitively, a \emph{recognisable relation} is simply a finite union of Cartesian products of regular languages. Formally, an $\arity$-ary relation $R\subseteq \Sigma^*\times \cdots\times \Sigma^*$ is \emph{recognisable} if $R=\bigcup_{i=1}^n L^{(i)}_1\times \cdots\times L^{(i)}_\arity$ where $L^{(i)}_j$ is regular for each $j\in [\arity]$. A \emph{representation} of a recognisable relation $R=\bigcup_{i=1}^n L^{(i)}_1\times \cdots\times L^{(i)}_\arity$ is $(\NFA^{(i)}_1, \ldots, \NFA^{(i)}_\arity)_{1 \le i \le n}$ such that each $\NFA^{(i)}_j$ is an NFA with $\Lang(\NFA^{(i)}_j)=L^{(i)}_j$. The tuples $(\NFA^{(i)}_1, \ldots, \NFA^{(i)}_\arity)$ are called the \emph{disjuncts} of the representation and the NFAs $\NFA^{(i)}_j$ are called the \emph{atoms} of the representation.



%\paragraph*{Automata models.} 

 
%Fix a finite \emph{alphabet} $\Sigma$. Elements in $\Sigma^*$ are called \emph{strings}. Let $\varepsilon$ denote the empty string and  $\Sigma^+ = \Sigma^* \setminus \{\varepsilon\}$. We will use $a,b,\ldots$ to denote letters from $\Sigma$ and $u, v, w, \ldots$ to denote strings from $\Sigma^*$. For a string $u \in \Sigma^*$, let $|u|$ denote the \emph{length} of $u$ (in particular, $|\varepsilon|=0$), moreover, 

%Let $w=a_1\cdots a_n$ be a string. The reverse of $w$, denoted by $w^{(r)}$, is $a_n \cdots a_1$. 


\paragraph*{Finite transducers.} A \emph{nondeterministic finite transducer} (NFT) $\NFT$ is an extension of NFA with outputs. Formally, an NFT $\NFT$ is a tuple $(Q, \Sigma, \delta, I, F)$, where $Q, \Sigma, I, F$ are as in NFA and the transition relation $\delta$ is a finite subset of $Q \times \Sigma \times Q \times \Sigma^*$. Similarly to NFA, for readability, we write a transition $(q, a, q', u) \in \delta$ as $q \xrightarrow[\delta]{a, u} q'$ or $q \xrightarrow{a, u} q'$. The \emph{size} of an NFT $\NFT$, denoted by $|\NFT|$, is defined as the sum of the sizes of the transitions of $\NFT$, where the size of a transition $q \xrightarrow{a, u} q'$ is defined as $|u|+3$.
%
A run of $\NFT$ over a string $w=a_1 \cdots a_n$ is a sequence of transitions $q_0 \xrightarrow{a_1, u_1} q_1 \cdots q_{n-1} \xrightarrow{a_n, u_n} q_n$ with $q_0 \in I$. The run is accepting if $q_n \in F$. The string $u_1 \cdots u_n$ is called the output of the run. The transduction defined by $\NFT$, denoted by $\Tran(\NFT)$, is the set of string pairs $(w, u)$ such that there is an accepting run of $T$ on $w$, with the output $u$. An NFT $\NFT$ is said to be \emph{deterministic} if $I$ is a singleton, and, for every $q \in Q$ and $a \in \Sigma$  there is at most one pair $(q', u) \in Q \times \Sigma^*$ such that $(q, a, q', u) \in \delta$.
%
In this paper, we are mainly interested in \emph{functional} transducers (FFT), i.e., transducers that define functions instead of relations.
(For instance, deterministic transducers are always functional.)

%\tl{to put in a different place later} Note that our decision procedure is applicable to general transducers as well, however, the EXPSPACE complexity bound is not, because the distributive property $f^{-1}(L_1\cap L_2)= f^{-1}(L_1)\cap f^{-1}(L_2)$ for regular languages $L_1, L_2$ only holds for functional transducers $f$.  

%We remark that an FT usually defines a relation.

\medskip

In this paper, we consider logics involving two data types, i.e., the string data type and the integer data type. We will use $u, v, \dots$ to denote string constants,  $c, d,\dots$ to denote integer constants, $x, y, \dots$ to denote string variables, and $i, j, \dots$ to denote  integer variables.


\paragraph*{Linear integer arithmetic.}  A linear integer arithmetic (abbreviated as LIA, essentially Presburger) formula $\phi$ is defined by the following rules
\[
\begin{array}{l c l}
t & ::=  & i \mid c \mid ct \mid t + t, \mbox{ where } c \in \Int, \\
\phi &::= & t \ o \ t \mid \neg \phi \mid \phi \vee \phi \mid \exists i.\ \phi, \mbox{ where } o \in \{=, \neq, \le, \ge, <, >\}.
\end{array}
\]
The \emph{size} of an LIA formula $\phi$, denoted by $|\phi|$, is defined as the number of symbols in $\phi$.
Let $\phi$ be an LIA formula and $i$ be a variable occurring in $\phi$. Then an occurrence of $i$ in $\phi$ is said to be \emph{free}  if the occurrence is not under the scope of $\exists i$. A formula $\phi$ is \emph{quantifier-free} if it does not contain quantifiers. The semantics of LIA formulas is standard and its definition is omitted here.
For a quantifier-free LIA formula $\phi$ that contains the free variables $i_1, \ldots, i_k$, we use $\cM(\phi)$ to denote the set of models of $\phi$, namely, $\cM(\phi) = \left\{(n_1, \ldots , n_k) \in \Int^k \mid \phi[n_1/i_1, \ldots, n_k/i_k] \mbox{ is evaluated to } true\right\}$, where $\phi[n_1/i_1, \ldots, n_k/i_k]$ is the formula obtained from $\phi$ by simultaneously replacing $i_1,\ldots, i_k$ with $n_1,\ldots, n_k$. An \emph{existential} LIA formula is a LIA formula where all the existential quantifiers are under the scope of an even number of negation symbols.


%%%%%%%%%%%%%%%%%%%%%%%%%%%%%%%%%%%%%%%%%%%%%%%%%%%%%%%

\section{String-Manipulating Programs with Integer Data Type}\label{sec:logic}

%!TEX root = main.tex

\section{The string logic}\label{sec:logic}



%%%%%%%%%%%%%%%%%%%%%%%%%%%%%%%%%%%%%%%%


%\subsection{Prioritized Streaming String Transducer}
%Below, we introduce  a new class of prioritized transducer \cite{BM17} which combines the expressive power of streaming string transducer \cite{AC10,AD11}.
%
%\begin{definition}[Prioritized Streaming String Transducer]
%	A \emph{prioritized streaming string transducer} (PSST) is a tuple $\psst = (Q, \Sigma, X, E, \delta, q_0, F)$, where $Q$ a
%	finite set of states, $\Sigma$ is the input and output alphabet, and $X$ a finite set of variables. $E$ is a partial function from $Q \times \Sigma \times
%	Q$ to $X \rightarrow (X \cup \Sigma)^{\ast}$, i.e. the set of assignment,
%	$\delta \in Q \times \Sigma \rightarrow \overline{Q}$ and $F$ is a partial function
%	from $Q$ to $(X \cup \Sigma)^{\ast}$.
%\end{definition}
%
%A run of $\psst$ is the sequence $q_0 \sigma_1 s_1 q_1 \ldots \sigma_m s_m q_m$, where $F (q_m)$ is defined and for each $i \in [m], q_i \in \delta (q_{i-1}, \sigma_i)$ and $s_i = E (q_{i - 1}, \sigma_i, q_i)$. For any two runs on $w = \sigma_1 \ldots \sigma_m$, denoted by $p = q_0 \sigma_1 s_1 \ldots \sigma_m s_m q_m$ and $p' = q_0 \sigma_1
%s_1' \ldots \sigma_m s_m' q_m'$, we say that $p$ is of a higher priority over
%$p'$ if $p \neq p'$ and, for the smallest index $j$ with $q_j \neq q_j'$,
%$\delta (q_{j - 1}, \sigma_j) = \ldots q_j \ldots q_j' \ldots$
%
%The accepting run of $\psst$ on input $w$ is the run of the highest priority. The output of $T$ on w, denoted by $T(w)$, is defined as $\pi_m(F(q_m))$, where $\pi_0(x) = \varepsilon$ for each $x \in X$, and $\pi_{i}(x) = \pi_{i-1}(s_{i}(x))$ for $1 \le i \le m$ and $x \in X$. Note that here we abuse the notation  $\pi_m(F(q_m))$ and $\pi_{i-1}(s_{i}(x))$ by taking a function $\pi$ from $X$ to $\Sigma^*$ as a function from $(X \cup \Sigma)^*$ to $\Sigma^*$, which maps each $\sigma \in \Sigma$ to $\sigma$ and each $x \in X$ to $\pi(x)$.  
%
%%  $\tmop{Out} (r) =
%%  s_{\varepsilon} \circ s_1 \circ s_2 \ldots s_n \circ F (q_n)$ where
%%  $s_{\varepsilon}$ is the empty substitution which maps all variables to
%%  $\varepsilon$.
%
%\begin{definition}[pre-image]
%	For a string relation $R \subseteq \Sigma^* \times \Sigma^*$ and $L \subseteq \Sigma^*$, we define the \emph{pre-image} of $L$ under $R$ as $R^{-1}(L):=\{w \in \Sigma^* \mid \exists w'.\ w' \in L \mbox{ and } (w, w') \in R\}$. 
%\end{definition}
%
%\begin{theorem}[pre-image of \PSST{}]
%	\label{theorem:psst_preimage}
%	Given a \PSST{} $\psst = (Q_T, \Sigma$, $X, E, \delta_T, q_{0, T}, F_T)$ and \FA{} $A
%	= (Q_A, \Sigma, \delta_A, q_{0, A}, F_A)$, we can compute an \FA{} $B = (Q_B,
%	\Sigma, \delta_B, q_{0, B}, F_B)$ in exponential time  such that $\Lang(B) = \cR^{-1}_T(\Lang(\Aut))$.
%\end{theorem}
%
%\begin{proof}
%	Intuitively, $B$ simulates the run of $\psst$ on $w$, and, for each $x \in X$, records the set of state pairs $(p, q) \in Q_A \times Q_A$ such that starting from $p$, $A$ can reach $q$ after reading the string stored in $x$. Moreover, $B$ also records all the states accessible from a run with higher priority to ensure the current run is the accepting one of $\psst$.
%	
%	Formally, $Q_B = Q_T \times (\cP(Q_A \times Q_A ))^{X} \times \cP(Q_T)  $, $q_{0, B} = (q_{0, T}, \rho_{\varepsilon}, \emptyset)$ where $\rho_{\varepsilon} (x) = \{(q, q) \mid q \in Q\}$ for each $x \in X$, and $\delta_{B}$ comprises the tuples $((q, \rho, S), a, (q_i, \rho', S'))$ such that there exists $s \in \left((X \cup \Sigma\right)^*)^X$ satisfying
%	\begin{itemize}
%		\item $\delta_T (q, a) = (q_1 \ldots q_i \ldots q_m)$, 
%		\item $s = E(q,a,q_i)$.
%		\item $S' = \delta_T^{\ast} (S, a) \cup \{ q_1, \ldots, q_{i - 1} \}$, where $\delta_T^{\ast}(S,a) = \{q' \mid \exists q \in S, q' \in \delta_T(q,a)\}$.
%		\item and $\rho'$ is obtained from $\rho$ and $s$ as follows: for each $x \in X$, if $s(x) = \varepsilon$, then $\rho'(x) = \{(p, p) \mid p \in Q_A\}$, otherwise, let $s(x) = b_1 \cdots b_\ell$ with $b_i \in \Sigma \cup X$ for each $i \in [\ell]$, then $\rho'(x) = \theta_1 \circ \cdots \circ \theta_\ell$, where $\theta_i = \delta^{(b_i)}_A$ if $b_i \in \Sigma$, and $\theta_i = \rho(b_i)$ otherwise.
%		%
%		%$\rho'(x) = \theta_\ell$ such that $\theta_0 = \{(p,p) \mid p \in Q_A\}$, and for each $i \in [\ell]$, if $b_i \in \Sigma$, then $\theta_i = \{(p, p') \mid (p, p'') \in \theta_{i-1}, (p'', b_i, p') \in \delta_A \mbox{ for some } p''\}$, otherwise, $\theta_i = \theta_{i-1} \cdot \rho(x)$. 
%	\end{itemize}
%	
%	Moreover, $F_B$ is the set of states $(q, \rho, S) \in Q_B$ such that
%	\begin{enumerate}
%		\item $F_T (q)$ is defined,
%		\item For any $q' \in S$, $F_T (q')$ is not defined
%		
%		\item if $F_T(q) = \varepsilon$, then $q_{0, A}  \in F_A$, otherwise, 
%		let $F_T(q) = b_1 \cdots b_\ell$ with $b_i \in \Sigma \cup X$ for each $i \in [\ell]$, then $(\theta_1 \circ \cdots \circ \theta_\ell) \cap (\{q_{0,A}\} \times F_A) \neq \emptyset$, where for each $i \in [\ell]$, if $b_i \in \Sigma$, then $\theta_i = \delta^{(b_i)}_A$, otherwise, $\theta_i = \rho(b_i)$.
%	\end{enumerate}
%\end{proof}
%
%Note that the above construction  does not utilize the so-called \tmtextit{copyless} property \cite{AC10,AD11},
%thus it works for general, or \tmtextit{copyful} \PSST{} \cite{FR17}.

% Note that in the definition of \NSST, there is no \emph{copyless} restriction.



%%%%%%%%%%%%%%%%%%%%%%%%%%%%%%%%%%%%%%%%%%%%%%%%%%%%%


%We will use $\$ 1, \$2, \cdots$ to denote the references to capturing groups in regular expressions.

%We define the set of reference expressions as follows: 

%We consider the symbolic execution of the string-manipulating programs.

%\begin{definition}[The constraint language $\strline$] 
We define the string-manipulating language $\strline$ %considered in this paper 
as follows. 
%The $\strline$ language is defined by
\[
\begin{array}{l}
S \eqdef  z:= x \concat y \mid y := \extract_{i, e}(x) \mid  
%& &  
%y := \reverse(x) 
y := \replaceall_{\pat, \rep}(x)   \mid 
%y := \Transducer(x)\  \mid\\
 \ASSERT{x \in e} \mid S; S\
\label{eq:SL}
%a ::= f(x_1,\ldots,x_n), \qquad b ::= g(x_1,\ldots,x_n)
\end{array}
\]
%\tl{to avoid confusion, write  $\ASSERT{x \in A}$?} 
where 
\begin{itemize}
	\item $\concat$ is the string concatenation operation which concatenates two strings,
%
\item for the $\extract$ function, $i \in \Nat$, $e \in \cgexp$,
%
	\item  for the $\replaceall$ operation, $\pat\in \cgexp$, $\rep \in \refexp$, %$\replaceall$ is the replace-all function to be defined shortly,
%	\item $\reverse$ is the string function which reverses a string; 
%	\item $\Transducer$ is a \PSST,
%
	\item for assertions, $e \in \regexp$.
\end{itemize} 
%and $R$ is a recognisable relation represented by a collection of tuples of \FA{}s.
%\end{definition}
%
%\zhilei{Should we add NSST constraints? Basically NSST can express more than PSST.}
%\tl{maybe just use NSST to replace PSST?}
%
%\zhilei{PSST is needed for decision procedure. NSST can be decided too, but the algorithm is very similar, so maybe too tedious to add both }

%It is evident that the $\reverse$ function is subsumed by \PSST{}s.

%
%\begin{remark}
%	Zhilin mentioned that we might introduce a function which takes a string and a pattern with capturing groups, and does sort of pattern matching to extra substrings. This function can be captured by the transducer $T$. We will formalise this later.
%\end{remark}

The $\extract$ function is used to model the regular-expression match function in programming languages.
%, e.g. $\sf str.match(regexp)$ function in Javascript. 
Specifically, the $\extract_{i, e}(x)$ function extracts the match of the $i$-th capturing group in the accepting match of $e$ to $x$ for $x \in \Lang(e)$ (otherwise, the return value of the function is undefined). Note that $\extract_{i, e}(x)$ returns $x$ if $i=0$. For instance, assuming $e = [[([\Gamma^+])\concat .?] \concat ([\Gamma^*])]$,   $\extract_{1, e}(0250)=0250$ and $\extract_{2, e}(0250)=\varepsilon$, as shown in Example~\ref{exmp-regex-semantics}. 

\begin{remark}
The match function in programming languages, e.g. $\sf str.match(reg)$ function in JavaScript, finds the first match of $\sf reg$ in $\sf str$. We can use $\extract$ to express the first match of $\sf reg$ in $\sf str$ by adding $[\Sigma^{*?}]$ and $[\Sigma^*]$ before and after $\sf reg$ respectively. More generally, the value of the $i$-th capturing group in the first match of a $\regexp$ $\sf reg$ in $\sf str$ can be specified as $\extract_{i+1, {\sf reg'}}({\sf str})$, where ${\sf reg'} = [[[\Sigma^{*?}] \concat ({\sf reg})] \concat [\Sigma^*]]$.
\end{remark}

The $\replaceall_{\pat, \rep}(x)$ function is parameterized by the  %\emph{subject} string, the second parameter is a 
\emph{pattern} $\pat \in \cgexp$ and the \emph{replacement} string $\rep \in \refexp$. For a given input string $x$, the function identifies all the %the first, second, $\dots$, 
matches of $\pat$ in $x$ and replace them with the corresponding strings specified by the replacement string. (In the replacement string,  references may be used which refer to %are replaced by 
the corresponding matches of the capturing groups.)  For instance, let $\pat = [[([\Gamma^+])\concat .?] \concat ([\Gamma^*])]$ and $\rep = \$1$. We have $\replaceall_{\pat, \rep}(2.5,3.4) = 2,3$. 

Without loss of generality, we assume that all the $\strline$ programs are in single static assignment (SSA) form, that is, each variable $x$ is assigned at most once. Moreover, if it is assigned, all its occurrences on the right hand sides of the assignment statements or in assertions are after the assignment statement of $x$.
%
For an $\strline$ program $S$, a variable $x$ occurring in $S$ is said to be an \emph{input} variable if $x$ does not occur on the left hand sides of assignment statements. The \emph{path feasibility} problem of an $\strline$ program is to decide whether there are valuations of the input variables such that the program can execute to the end.


%
%For the semantics of $\replaceall$ function, in particular when the pattern is a regular expression, we adopt the \emph{leftmost and longest} matching. 




%For instance, $\replaceall(aababaab, (ab)^+, c) =ac\cdot \replaceall(aab, (ab)^+, c)= acac$, since the leftmost and longest matching of $(ab)^+$ in $aababaab$ is $abab$. Here we require that the language defined by the pattern parameter does \emph{not} contain the empty string, in order to avoid the troublesome definition of the semantics of the matching of the empty string. We refer the reader to \cite{CCHLW18} for the formal semantics of the $\replaceall$ function. To be consistent with the notation in this paper, for each regular expression $e$, we define
%the string function $\replaceall_e: \ialphabet^* \times \ialphabet^* \rightarrow \ialphabet^*$ such that for $u, v \in \ialphabet^*$, $\replaceall_e(u, v) = \replaceall(u, e, v)$, and we write $\replaceall(x, e, y)$ as $\replaceall_e(x,y)$.

It turns out that the path feasibility problem is undecidable, attributed to the the back references in assertion statements. 

\begin{proposition}\label{prop-und}
The path feasibility problem of $\strline$ is undecidable.
\end{proposition}

We shall show that the path feasibility problem is decidable, if the uses of back references in assertion statements are forbidden, which turns out to be the situation in practice.\footnote{A partial evidence is that the occurrences of regular expressions with back references occupy only less than $1\%$ in the NPM package, according to the statistics collected in \cite{LMK19}.} In the sequel, we will use $\strline_{\sf reg}$ to denote the collection of $\strline$ programs which are free of %where no 
back references in assertion statements. %We state 
The main result of this paper is as follows.

\begin{theorem}\label{thm-main}
The path feasibility of $\strline_{\sf reg}$ is decidable in XXX. \zhilin{complexity should be added}
\end{theorem}
The decision procedure for $\strline_{\sf reg}$ utilizes a new model called prioritized streaming string transducers, which will be defined in the next section.


%%%%%%%%%%%%%%%%%%%%%%%%%%%%%%%%%%%%%%%%%%%%%%%%%%%%%%%%

%\section{Automata-Theoretic Foundations}\label{sec:cefa}

%%!TEX root = main.tex


%In this section, we introduce cost-enriched regular languages and recognisable relations, as the extensions of regular languages and recognisable relations, moreover, we investigate the decidability and complexity of a related decision problem, thus laying down the theoretical foundations of the decision procedure in the next section. 

\subsection{Cost-Enriched Regular Languages and Recognisable Relations} \label{sect:ce}

Let $k \in \Nat$ with $k > 0$. A \emph{$k$-cost-enriched string} is $(w, (n_1, \cdots, n_k))$ where $w$ is a string and $n_i \in \intnum$ for all $i \in [k]$. 
A \emph{$k$-cost-enriched language} $L$ is a subset of $\Sigma^* \times \intnum^k$. For our purpose, we identify a ``regular" fragment of cost-enriched languages as follows. 
%Note that all the cost-enriched strings in $L$ are associated with the same number of costs (i.e., $k$).

\begin{definition}[Cost-enriched regular languages]
Let $k \in \Nat$ with $k > 0$. A $k$-cost-enriched language is \emph{regular} (abbreviated as CERL) if it can be accepted by a \emph{cost-enriched finite automaton}. 

A cost-enriched finite automaton (CEFA) $\CEFA$ is a tuple $(Q, \Sigma, R, \delta, I, F)$ where 
\begin{itemize}
\item $Q, \Sigma, I, F$ are defined as in NFAs, 
%
\item $R=(r_1, \cdots, r_k)$ is a vector of (mutually distinct) \emph{cost registers}, 
%
\item $\delta$ is the transition relation which is a finite set of tuples $(q, a, q', \eta)$ where $q, q' \in Q$, $a \in \Sigma$, and %$\eta: R \rightarrow \intnum$ 
$\eta: R \rightarrow \Int$
is a cost register update function. \\
For convenience, we usually write $(q, a, q', \eta) \in \Delta$ as $q \xrightarrow{a, \eta} q'$.
\end{itemize}
%
A \emph{run} of $\CEFA$ on a $k$-cost-enriched string $(a_1 \cdots a_m, (n_1, \cdots,n_k))$ is a  transition sequence $q_0 \xrightarrow{a_1, \eta_1} q_1 \cdots q_{m-1} \xrightarrow{a_m, \eta_m} q_m$ such that $q_0 \in I$ and $n_i = \sum \limits_{1\leq j\leq m}\eta_j(r_i)$ for each $i \in [k]$ (Note that the initial values of cost registers are zero). The run is \emph{accepting} if $q_m \in F$. A $k$-cost-enriched string $(w, (n_1, \cdots,n_k))$ is accepted by $\CEFA$ if there is an accepting run of $\CEFA$ on $(w, (n_1, \cdots,n_k))$. In particular, $(\varepsilon, n)$ is accepted by $\CEFA$ if $n=0$ and $I\cap F \neq \emptyset$.
The $k$-cost-enriched language defined by $\CEFA$, denoted by $\Lang(\CEFA)$, is the set of $k$-cost-enriched strings accepted by $\CEFA$. 
%A cost-enriched language $L \subseteq \Sigma^* \times \intnum^k$ is called a cost-enriched regular language (CERL) if there is a CEFA $\NFA$ such that $L = \Lang(\NFA)$.
\end{definition}
The \emph{size} of a CEFA $\CEFA=(Q, \Sigma, R, \delta, I, F)$, denoted by $|\CEFA|$, is defined as the sum of the sizes of its transitions, where the size of each transition $(q, a, q', \eta)$ is $\sum \limits_{r \in R} \lceil \log_2 (|\eta(r)|) \rceil +3$. Note here  the integer constants in $\CEFA$ are encoded in binary.

\begin{remark}
CEFAs can be seen as a variant of Cost Register Automata \cite{RLJ+13}, by admitting nondeterminism and discarding partial final cost functions. CEFAs are also closely related to monotonic counter machines \cite{LB16}. The main difference is that CEFAs discard guards in transitions and allow binary-encoded integers in cost updates, while monotonic counter machines allow guards in transitions but restrict the cost updates to being monotonic and unary, i.e. $0,1$ only. Moreover, we explicitly define CEFAs as language acceptors for  cost-enriched languages.
\end{remark}

\begin{example}[CEFA for $\length$]\label{exm:len}
The string function $\length$ can be captured by CEFAs. For any NFA $\NFA=(Q, \Sigma,  \delta, I, F)$, it is not difficult to see that the cost-enriched language $\{(w, \length(w)) \mid w\in \Lang(\NFA)\}$ is accepted by a CEFA, i.e., 
$(Q, \Sigma, (r_1), \delta', I, F)$  %$Q =I=F= \{q_0\}$, $R=(r_1)$, and 
such that for each $(q, a, q')\in \delta$, we let $(q, a, q', \eta)\in \delta'$, where $\eta(r_1) = 1$. 

For later use, we identify a special $\CEFA_{\rm len}= (\{q_0\}, \Sigma, (r_1), \{(q_0, a, q_0, \eta) \mid \eta(r_1) = 1\}, \{q_0\}, \{q_0\})$. In other words, $\CEFA_{\rm len}$ accepts $\{(w, \length(w)) \mid w\in \Sigma^*\}$.
%
%
%to denote the CEFA $(Q, \Sigma, (r_1), \delta', I, F)$ for the special NFA $\NFA$ such that $\Lang(\NFA)=\Sigma^*$, specifically, $\NFA_{\rm len} = (\{q_0\}, \Sigma, (r_1), \{(q_0, a, q_0, \eta) \mid \eta(r_1) = 1\}, \{q_0\}, \{q_0\})$.
%\tl{it is a bit vague; you mean for any RE $e$,  is a cerl?}
\end{example}

We can show that the function $\indexof_v(\cdot, \cdot)$ can be captured by CEFAs as well, in the sense that, for any NFA $\NFA$ and constant string $v$, we can construct a CEFA %$\CEFA'$ such that 
accepting $\left\{(w, (n, \indexof_v(w, n)))\mid w\in \Lang(\NFA), n \le \indexof_v(w, n) < |w| \right\}$. %$R(\CEFA')=(r_1,r_2)$ and $\Lang(\CEFA')=\{(w, (n, \indexof_v(w, n)))\mid w\in \Lang(\NFA) \}$. 
The construction is slightly technical and can be found in Appendix, Section~\ref{appendix:cefa_indexof}.

For convenience, we use $\CEFA_{\indexof_v}$ which accepts $\{(w, (n, \indexof_v(w, n)))\mid w\in \Sigma^*, n \le \indexof_v(w, n) < |w| \}$. %to denote the constructed CEFA $\CEFA'$ for the special NFA $\NFA$ such that $\Lang(\NFA)=\Sigma^*$. 
Note that $\CEFA_{\indexof_v}$ does not model the corner cases in the semantics of $\indexof_v$, for instance, $\indexof_v(w, n) = -1$ if $v$ does not occur after the position $n$ in $w$.
In the sequel, we will give an example of the construction of $\CEFA_{\indexof_v}$ for the special case that $v$ is a single letter.  

\begin{example}[CEFA for $\indexof_a$]\label{exm:indexof}
Let $a \in \Sigma$. Then  $\CEFA_{\indexof_a} = (\{(q_0, q_1, q_2)\}, \Sigma, (r_1,r_2), \delta_{\indexof_a}, \{q_0\}, \{q_2\})$, where $\delta_{\indexof_a}$ comprises the tuples
\begin{itemize}
\item $(q_0, b, q_0, \eta)$ such that $b \in \Sigma$, $\eta(r_1)=1$, $\eta(r_2)=1$,
%
\item $(q_0, b, q_1, \eta)$ such that $b \in \Sigma$, $\eta(r_1)=0$, $\eta(r_2) = 1$,
%
\item $(q_0, a, q_2, \eta)$ such that $\eta(r_1)=0$, $\eta(r_2) = 0$,
%
\item $(q_1, b, q_1, \eta)$ such that $b \in \Sigma \setminus \{a\}$, $\eta(r_1)=0$, $\eta(r_2)=1$,
%
\item $(q_1, a, q_2, \eta)$ such that $\eta(r_1)=0$, $\eta(r_2)=0$,
%
\item $(q_2, b, q_2, \eta)$ such that $b \in \Sigma$, $\eta(r_1)=0$, $\eta(r_2)=0$.
\end{itemize}
Intuitively, $r_1$ corresponds to the starting position $i$ of $\indexof_a(x, i)$, $r_2$ corresponds to the output of $\indexof_a(x, i)$, $q_0$ specifies that the current position is before $i$, $q_1$ specifies that the current position is after $i$, while $a$ has not occurred yet, and $q_2$ specifies that $a$ has occurred after $i$. 
\end{example}


Given two CEFAs $\CEFA_1 = ( Q_1, \Sigma, R_1, \delta_1, I_1, F_1)$ and $\CEFA_2 = (Q_2, \Sigma, \delta_2, R_2, I_2, F_2)$ with $R_1 \cap R_2 = \emptyset$ (where %the notation is abused a bit, viewing 
$R_1$ and $R_2$ are treated as sets), the product of $\CEFA_1$ and $\CEFA_2$, denoted by $\CEFA_1 \times \CEFA_2$, is defined as $(Q_1 \times Q_2, \Sigma, R_1 \cup R_2, \delta, I_1 \times I_2, F_1 \times F_2)$, where $\delta$ comprises the tuples $((q_1, q_2), \sigma, (q'_1, q'_2), \eta)$ such that $(q_1, \sigma, q'_1, \eta_1) \in \delta_1$, $(q_2, \sigma, q'_2, \eta_2) \in \delta_2$, and $\eta = \eta_1\cup \eta_2$.  %for some $\eta_1, \eta_2$.


For a CEFA $\CEFA$, we use $R(\CEFA)$ to denote the vector of cost registers occurring in $\CEFA$. %Note that cost registers of $\CEFA$ are simply integer variables to store costs in $\CEFA$.
%Moreover, for a CEFA $\NFA$ and 
Suppose $\CEFA$ is  CEFA with $R(\CEFA)=(r_1,\cdots, r_k)$ and $\vec{i} = (i_1,\cdots, i_k)$ is a vector of mutually distinct integer variables such that $R(\NFA) \cap \vec{i} = \emptyset$. We use $\CEFA[\vec{i}/R(\CEFA)]$ to denote the CEFA obtained from $\CEFA$ by simultaneously replacing $r_j$ with $i_j$ for $j \in [k]$. 

\smallskip

%Let $(k_1,\cdots, k_l) \in \Nat^l$ with $k_j > 0$ for every $j \in [l]$. A  $(k_1,\cdots, k_l)$-cost-enriched relation $\cR$ is a subset of $(\Sigma^* \times \intnum^{k_1}) \times \cdots (\Sigma^* \times \intnum^{k_l})$.

%%%%%%%%%%%%%%%%%%%%%%%%%%%%%%%%%%%%%%%%%%%%%%%%%%%%%%%%%%%%%%%%%%%%%%%%%%%%%%%%%%%%%%%%%%%%%%%
\begin{definition}[Cost-enriched recognisable relations]
Let $(k_1,\cdots, k_l) \in \Nat^l$ with $k_j > 0$ for every $j \in [l]$. A cost-enriched recognisable relation (CERR)  $\cR \subseteq (\Sigma^* \times \intnum^{k_1}) \times \cdots  \times (\Sigma^* \times \intnum^{k_l})$ is a finite union of products of CERLs. Formally,
$\cR = \bigcup \limits_{i=1}^n L_{i,1 } \times \cdots \times L_{i, l}$, 
	where for every $j \in [l]$, $L_{i,j} \subseteq \Sigma^* \times \intnum^{k_j}$ is a CERL. 
	A CEFA representation of $\cR$ is a collection of CEFA tuples $(\CEFA_{i,1}, \cdots, \CEFA_{i,l})_{i \in [n]}$ such that $\Lang(\CEFA_{i,j}) = L_{i,j}$ for every $i \in [n]$ and $j \in [l]$.
\end{definition}

\begin{example}\label{exm:CERR}
The relation 
\[\cR=\left\{((w_1, |w_1|), (w_2, |w_2|)) \mid  w_1 \in \Lang((aa)^*), w_2 \in \Lang(b(bb)^*), |w_1|+|w_2| \ge 2\right\}\] 
is a CERR since 
$\cR = L_{1,1} \times L_{1,2} \cup L_{2,1} \times L_{2,2}$, where $L_{1,1}=\{(w_1, |w_1|) \mid w_1 \in \Lang((aa)^*)\}$, $L_{1,2}=\{(w_2, |w_2|) \mid  w_2 \in \Lang(bbb(bb)^*)\}$, $L_{2,1}=\{(w_1, |w_1|) \mid w_1 \in \Lang(aa(aa)^*)\}$, and $L_{2,2}=\{(w_2, |w_2|) \mid  w_2 \in \Lang(b(bb)^*)\}$.
%\begin{itemize}
%\item $L_{1,1}=\{(w_1, |w_1|) \mid w_1 \in \Lang((aa)^*)\}$, 
%\item $L_{1,2}=\{(w_2, |w_2|) \mid  w_2 \in \Lang(bbb(bb)^*)\}$ , 
%\item $L_{2,1}=\{(w_1, |w_1|) \mid w_1 \in \Lang(aa(aa)^*)\}$,  and
%\item $L_{2,2}=\{(w_2, |w_2|) \mid  w_2 \in \Lang(b(bb)^*)\}$. 
%\end{itemize}
Note that $L_{i,j}$ for $i,j\in[2]$ are CERLs, with corresponding CEFAs $\CEFA_{i,j}$ by  Example~\ref{exm:len}. It follows that %Moreover, 
$(\CEFA_{i,1}, \CEFA_{i,2})_{i \in [2]}$ gives a CEFA representation of $\cR$. %, where 
%\tl{to be simplified.}
%\begin{itemize}
%\item  $\CEFA_{1,1} = (\{p_0, p_1\}, \{a,b\}, (r_{1}), \delta_{1,1}, \{p_0\}, \{p_0\})$ defines $L_{1,1}$, such that $\delta_{1,1}= \{(p_0, a, p_1,\eta_1)$, $(p_1, a, p_0, \eta_1)\}$, with $\eta_1(r_{1})=1$,
%%
%\item $\CEFA_{1,2}=(\{q_0,q_1,q_2\}, \{a,b\}, (r_{2}), \delta_{1,2}, \{q_0\}, \{q_1\})$ defines $L_{1,2}$, such that $\delta_{1,2}=\{(q_0, b, q_1,\eta_2), (q_1, b, q_2, \eta_2), (q_2, b, q_1,\eta_2)\}$, with $\eta_2(r_{2})=1$,
%%
%\item $\CEFA_{2,1} = (\{p_0, p_1,p_2, p_3\}, \{a,b\}, (r_{1}), \delta_{2,1}, \{p_0\}, \{p_2\})$ defines $L_{2,1}$, such that $\delta_{2,1} = \{(p_0, a, p_1, \eta_1), (p_1, a, p_2, \eta_1), (p_2, a, p_3, \eta_1), (p_3, a, p_2, \eta_1)\}$, with $\eta_1(r_{1})=1$, and
%%
%\item $\CEFA_{2,2}=(\{q_0,q_1,q_2\}, \{a,b\}, (r_{2}), \delta_{2,2}, \{q_1\})$ defines $L_{2,2}$, such that $\delta_{2,2}= (q_0, b, q_1,\eta_2), (q_1, b, q_2, \eta_2), (q_2, b, q_1,\eta_2)\}$, with $\eta_2(r_{2})=1$.
%\end{itemize}
\end{example}

%\subsection{Satisfiability of Linear Integer Arithmetic Formula with respect to CEFAs}
%
%In the sequel, we consider a decision problem for CEFAs which will be used for the decision procedure of the path feasibility problem for {\slint} in the next section.
%
%\begin{definition}[{\lasat} Problem]\label{def-la-sat-cefa}
%	The satisfiability problem of LIA formulas w.r.t. CEFAs (abbreviated as {\lasat} problem) is defined as follows.
%	
%	\textbf{Input}: a given quantifier-free LIA formula $\phi$ and CEFAs $\CEFA_1,\cdots,\CEFA_m$, such that $\CEFA_i=(Q_i, \Sigma, R_i, \delta_i, I_i, F_i)$ for every $i\in [m]$, 
%  $R_i \cap R_j = \emptyset$ for every $1 \le i < j \le m$, and
% the free variables of $\phi$ are from $\bigcup_{i\in [m]} R_i$, 
% 
%Decide whether %$\phi$ is satisfiable w.r.t. $\CEFA_1, \cdots, \CEFA_m$, namely, whether 
%there are an assignment function $\theta: \bigcup \limits_{i \in [m]} R_i \rightarrow \Int$ and strings $w_1, \cdots, w_m$  
%	such that  $\phi[(\theta(R_i)/R_i)_{i \in [m]}]$ hold and $(w_i, \theta(R_i)) \in \Lang(\NFA_i)$ for every $i \in [m]$.
%\end{definition}
%%Note that in Definition~\ref{def-la-sat-cefa}, registers in $\NFA_i$'s may intersect. 
%
%\begin{example}
%Let $\phi \equiv r_1 = r_2$ and $\CEFA_{1,1}, \CEFA_{1,2}$ be two CEFAs in Example~\ref{exm:CERR} defining $L_{1,1}, L_{1,2}$ respectively. (Recall that $R(\CEFA_{1,1})=(r_1)$ and $R(\CEFA_{1,2})=(r_2)$.) Then it is easy to see that $\phi$ is unsatisfiable w.r.t. $\CEFA_{1,1}$ and $\CEFA_{1,2}$, since for each $(w_1, n_1) \in L_{1,1}$, $n_1$ must be even, while for each $(w_2, n_2) \in L_{1,2}$, $n_2$ must be odd. Hence $n_1$ and $n_2$ cannot be equal.
%\end{example}
%
%\begin{theorem}\label{thm-la-sat-cefa}
%	The {\lasat} problem is NP-complete.
%\end{theorem}
%The proof is given in Appendix, Section~\ref{appendix:thm-la-sat-cefa}.


%%%%%%%%%%%%%%%%%%%%%%%%%%%%%%%%%%%%%%%%%%%%%%%%%%%%%%%%

\section{Decision Procedures for Path Feasibility}\label{sec:dec}

%!TEX root = main.tex

In this section, we present a decision procedure for the path feasibility problem of {\slint}. A distinguished feature of the decision procedure is that it conducts backward computation which is local and can be done in a modular way. To support this, we extend  a regular language with quantitative information of the strings in the language, giving rise to cost-enriched regular languages and corresponding finite automata (Section \ref{sect:ce}). The crux of the decision procedure is thus to show that the  pre-images of cost-enriched regular languages under the string operations in {\slint} (i.e., concatenation $\concat$, $\replaceall_{e,u}$, $\reverse$, FFTs $\NFT$, and $\substring$) are representable by so called cost-enriched recognisable relations (Section \ref{sect:pre}). The overall decision procedure is presented in Section~\ref{sec:dc}, supplied by additional complexity analysis and examples. 

\vspace{-3mm}
\subsection{Cost-Enriched Regular Languages and Recognisable Relations} \label{sect:ce}

Let $k \in \Nat$ with $k > 0$. A \emph{$k$-cost-enriched string} is $(w, (n_1, \cdots, n_k))$ where $w$ is a string and $n_i \in \intnum$ for all $i \in [k]$. A \emph{$k$-cost-enriched language} $L$ is a subset of $\Sigma^* \times \intnum^k$. For our purpose, we identify a ``regular" fragment of cost-enriched languages as follows.  

\begin{definition}[Cost-enriched regular languages]
	Let $k \in \Nat$ with $k > 0$. A $k$-cost-enriched language is \emph{regular} (abbreviated as CERL) if it can be accepted by a \emph{cost-enriched finite automaton}. 
	
	A cost-enriched finite automaton (CEFA) $\CEFA$ is a tuple $(Q, \Sigma, R, \delta, I, F)$ where 
	\begin{itemize}
		\item $Q, \Sigma, I, F$ are defined as in NFAs, 
		%
		\item $R=(r_1, \cdots, r_k)$ is a vector of (mutually distinct) \emph{cost registers}, 
		%
		\item $\delta$ is the transition relation which is a finite set of tuples $(q, a, q', \eta)$ where $q, q' \in Q$, $a \in \Sigma$, and %$\eta: R \rightarrow \intnum$ 
		$\eta: R \rightarrow \Int$
		is a cost register update function. \\
		For convenience, we usually write $(q, a, q', \eta) \in \Delta$ as $q \xrightarrow{a, \eta} q'$.
	\end{itemize}
	%
	A \emph{run} of $\CEFA$ on a $k$-cost-enriched string $(a_1 \cdots a_m, (n_1, \cdots,n_k))$ is a  transition sequence $q_0 \xrightarrow{a_1, \eta_1} q_1 \cdots q_{m-1} \xrightarrow{a_m, \eta_m} q_m$ such that $q_0 \in I$ and $n_i = \sum \limits_{1\leq j\leq m}\eta_j(r_i)$ for each $i \in [k]$ (Note that the initial values of cost registers are zero). The run is \emph{accepting} if $q_m \in F$. A $k$-cost-enriched string $(w, (n_1, \cdots,n_k))$ is accepted by $\CEFA$ if there is an accepting run of $\CEFA$ on $(w, (n_1, \cdots,n_k))$. In particular, $(\varepsilon, n)$ is accepted by $\CEFA$ if $n=0$ and $I\cap F \neq \emptyset$.
	The $k$-cost-enriched language defined by $\CEFA$, denoted by $\Lang(\CEFA)$, is the set of $k$-cost-enriched strings accepted by $\CEFA$. 
	%A cost-enriched language $L \subseteq \Sigma^* \times \intnum^k$ is called a cost-enriched regular language (CERL) if there is a CEFA $\NFA$ such that $L = \Lang(\NFA)$.
\end{definition}
The \emph{size} of a CEFA $\CEFA=(Q, \Sigma, R, \delta, I, F)$, denoted by $|\CEFA|$, is defined as the sum of the sizes of its transitions, where the size of each transition $(q, a, q', \eta)$ is $\sum \limits_{r \in R} \lceil \log_2 (|\eta(r)|) \rceil +3$. Note here  the integer constants in $\CEFA$ are encoded in binary.

\begin{remark}
	CEFAs can be seen as a variant of Cost Register Automata \cite{RLJ+13}, by admitting nondeterminism and discarding partial final cost functions. CEFAs are also closely related to monotonic counter machines \cite{LB16}. The main difference is that CEFAs discard guards in transitions and allow binary-encoded integers in cost updates, while monotonic counter machines allow guards in transitions but restrict the cost updates to being monotonic and unary, i.e. $0,1$ only. Moreover, we explicitly define CEFAs as language acceptors for  cost-enriched languages.
\end{remark}

\begin{example}[CEFA for $\length$]\label{exm:len}
	The string function $\length$ can be captured by CEFAs. For any NFA $\NFA=(Q, \Sigma,  \delta, I, F)$, it is not difficult to see that the cost-enriched language $\{(w, \length(w)) \mid w\in \Lang(\NFA)\}$ is accepted by a CEFA, i.e., 
	$(Q, \Sigma, (r_1), \delta', I, F)$  %$Q =I=F= \{q_0\}$, $R=(r_1)$, and 
	such that for each $(q, a, q')\in \delta$, we let $(q, a, q', \eta)\in \delta'$, where $\eta(r_1) = 1$. 
	
	For later use, we identify a special $\CEFA_{\rm len}= (\{q_0\}, \Sigma, (r_1), \{(q_0, a, q_0, \eta) \mid \eta(r_1) = 1\}, \{q_0\}, \{q_0\})$. In other words, $\CEFA_{\rm len}$ accepts $\{(w, \length(w)) \mid w\in \Sigma^*\}$.
	%
	%
	%to denote the CEFA $(Q, \Sigma, (r_1), \delta', I, F)$ for the special NFA $\NFA$ such that $\Lang(\NFA)=\Sigma^*$, specifically, $\NFA_{\rm len} = (\{q_0\}, \Sigma, (r_1), \{(q_0, a, q_0, \eta) \mid \eta(r_1) = 1\}, \{q_0\}, \{q_0\})$.
	%\tl{it is a bit vague; you mean for any RE $e$,  is a cerl?}
\end{example}

We can show that the function $\indexof_v(\cdot, \cdot)$ can be captured by CEFAs as well, in the sense that, for any NFA $\NFA$ and constant string $v$, we can construct a CEFA %$\CEFA'$ such that 
accepting $\left\{(w, (n, \indexof_v(w, n)))\mid w\in \Lang(\NFA), n \le \indexof_v(w, n) < |w| \right\}$. %$R(\CEFA')=(r_1,r_2)$ and $\Lang(\CEFA')=\{(w, (n, \indexof_v(w, n)))\mid w\in \Lang(\NFA) \}$. 
The construction is slightly technical and can be found in Appendix, Section~\ref{appendix:cefa_indexof}.

For convenience, we use $\CEFA_{\indexof_v}$ which accepts $\{(w, (n, \indexof_v(w, n)))\mid w\in \Sigma^*, n \le \indexof_v(w, n) < |w| \}$. %to denote the constructed CEFA $\CEFA'$ for the special NFA $\NFA$ such that $\Lang(\NFA)=\Sigma^*$. 
Note that $\CEFA_{\indexof_v}$ does not model the corner cases in the semantics of $\indexof_v$, for instance, $\indexof_v(w, n) = -1$ if $v$ does not occur after the position $n$ in $w$.

%coimmented for space
%In the sequel, we will give an example of the construction of $\CEFA_{\indexof_v}$ for the special case that $v$ is a single letter.  
%
%\begin{example}[CEFA for $\indexof_a$]\label{exm:indexof}
%	Let $a \in \Sigma$. Then  $\CEFA_{\indexof_a} = (\{(q_0, q_1, q_2)\}, \Sigma, (r_1,r_2), \delta_{\indexof_a}, \{q_0\}, \{q_2\})$, where $\delta_{\indexof_a}$ comprises the tuples
%	\begin{itemize}
%		\item $(q_0, b, q_0, \eta)$ such that $b \in \Sigma$, $\eta(r_1)=1$, $\eta(r_2)=1$,
%		%
%		\item $(q_0, b, q_1, \eta)$ such that $b \in \Sigma$, $\eta(r_1)=0$, $\eta(r_2) = 1$,
%		%
%		\item $(q_0, a, q_2, \eta)$ such that $\eta(r_1)=0$, $\eta(r_2) = 0$,
%		%
%		\item $(q_1, b, q_1, \eta)$ such that $b \in \Sigma \setminus \{a\}$, $\eta(r_1)=0$, $\eta(r_2)=1$,
%		%
%		\item $(q_1, a, q_2, \eta)$ such that $\eta(r_1)=0$, $\eta(r_2)=0$,
%		%
%		\item $(q_2, b, q_2, \eta)$ such that $b \in \Sigma$, $\eta(r_1)=0$, $\eta(r_2)=0$.
%	\end{itemize}
%	Intuitively, $r_1$ corresponds to the starting position $i$ of $\indexof_a(x, i)$, $r_2$ corresponds to the output of $\indexof_a(x, i)$, $q_0$ specifies that the current position is before $i$, $q_1$ specifies that the current position is after $i$, while $a$ has not occurred yet, and $q_2$ specifies that $a$ has occurred after $i$. 
%\end{example}


Given two CEFAs $\CEFA_1 = ( Q_1, \Sigma, R_1, \delta_1, I_1, F_1)$ and $\CEFA_2 = (Q_2, \Sigma, \delta_2, R_2, I_2, F_2)$ with $R_1 \cap R_2 = \emptyset$, %(where %the notation is abused a bit, viewing  $R_1$ and $R_2$ are treated as sets), 
the product of $\CEFA_1$ and $\CEFA_2$, denoted by $\CEFA_1 \times \CEFA_2$, is defined as $(Q_1 \times Q_2, \Sigma, R_1 \cup R_2, \delta, I_1 \times I_2, F_1 \times F_2)$, where $\delta$ comprises the tuples $((q_1, q_2), \sigma, (q'_1, q'_2), \eta)$ such that $(q_1, \sigma, q'_1, \eta_1) \in \delta_1$, $(q_2, \sigma, q'_2, \eta_2) \in \delta_2$, and $\eta = \eta_1\cup \eta_2$.  %for some $\eta_1, \eta_2$.


For a CEFA $\CEFA$, we use $R(\CEFA)$ to denote the vector of cost registers occurring in $\CEFA$. %Note that cost registers of $\CEFA$ are simply integer variables to store costs in $\CEFA$.
%Moreover, for a CEFA $\NFA$ and 
Suppose $\CEFA$ is  CEFA with $R(\CEFA)=(r_1,\cdots, r_k)$ and $\vec{i} = (i_1,\cdots, i_k)$ is a vector of mutually distinct integer variables such that $R(\NFA) \cap \vec{i} = \emptyset$. We use $\CEFA[\vec{i}/R(\CEFA)]$ to denote the CEFA obtained from $\CEFA$ by simultaneously replacing $r_j$ with $i_j$ for $j \in [k]$. 

%\smallskip

%Let $(k_1,\cdots, k_l) \in \Nat^l$ with $k_j > 0$ for every $j \in [l]$. A  $(k_1,\cdots, k_l)$-cost-enriched relation $\cR$ is a subset of $(\Sigma^* \times \intnum^{k_1}) \times \cdots (\Sigma^* \times \intnum^{k_l})$.

%%%%%%%%%%%%%%%%%%%%%%%%%%%%%%%%%%%%%%%%%%%%%%%%%%%%%%%%%%%%%%%%%%%%%%%%%%%%%%%%%%%%%%%%%%%%%%%
\begin{definition}[Cost-enriched recognisable relations]
	Let $(k_1,\cdots, k_l) \in \Nat^l$ with $k_j > 0$ for every $j \in [l]$. A cost-enriched recognisable relation (CERR)  $\cR \subseteq (\Sigma^* \times \intnum^{k_1}) \times \cdots  \times (\Sigma^* \times \intnum^{k_l})$ is a finite union of products of CERLs. Formally,
	$\cR = \bigcup \limits_{i=1}^n L_{i,1 } \times \cdots \times L_{i, l}$, 
	where for every $j \in [l]$, $L_{i,j} \subseteq \Sigma^* \times \intnum^{k_j}$ is a CERL. 
	A CEFA representation of $\cR$ is a collection of CEFA tuples $(\CEFA_{i,1}, \cdots, \CEFA_{i,l})_{i \in [n]}$ such that $\Lang(\CEFA_{i,j}) = L_{i,j}$ for every $i \in [n]$ and $j \in [l]$.
\end{definition}

%\begin{example}\label{exm:CERR}
%	The relation 
%	\[\cR=\left\{((w_1, |w_1|), (w_2, |w_2|)) \mid  w_1 \in \Lang((aa)^*), w_2 \in \Lang(b(bb)^*), |w_1|+|w_2| \ge 2\right\}\] 
%	is a CERR since 
%	$\cR = L_{1,1} \times L_{1,2} \cup L_{2,1} \times L_{2,2}$, where $L_{1,1}=\{(w_1, |w_1|) \mid w_1 \in \Lang((aa)^*)\}$, $L_{1,2}=\{(w_2, |w_2|) \mid  w_2 \in \Lang(bbb(bb)^*)\}$, $L_{2,1}=\{(w_1, |w_1|) \mid w_1 \in \Lang(aa(aa)^*)\}$, and $L_{2,2}=\{(w_2, |w_2|) \mid  w_2 \in \Lang(b(bb)^*)\}$.
%	%\begin{itemize}
%	%\item $L_{1,1}=\{(w_1, |w_1|) \mid w_1 \in \Lang((aa)^*)\}$, 
%	%\item $L_{1,2}=\{(w_2, |w_2|) \mid  w_2 \in \Lang(bbb(bb)^*)\}$ , 
%	%\item $L_{2,1}=\{(w_1, |w_1|) \mid w_1 \in \Lang(aa(aa)^*)\}$,  and
%	%\item $L_{2,2}=\{(w_2, |w_2|) \mid  w_2 \in \Lang(b(bb)^*)\}$. 
%	%\end{itemize}
%	Note that $L_{i,j}$ for $i,j\in[2]$ are CERLs, with corresponding CEFAs $\CEFA_{i,j}$ by  Example~\ref{exm:len}. It follows that %Moreover, 
%	$(\CEFA_{i,1}, \CEFA_{i,2})_{i \in [2]}$ gives a CEFA representation of $\cR$. %, where 
%	%\tl{to be simplified.}
%	%\begin{itemize}
%	%\item  $\CEFA_{1,1} = (\{p_0, p_1\}, \{a,b\}, (r_{1}), \delta_{1,1}, \{p_0\}, \{p_0\})$ defines $L_{1,1}$, such that $\delta_{1,1}= \{(p_0, a, p_1,\eta_1)$, $(p_1, a, p_0, \eta_1)\}$, with $\eta_1(r_{1})=1$,
%	%%
%	%\item $\CEFA_{1,2}=(\{q_0,q_1,q_2\}, \{a,b\}, (r_{2}), \delta_{1,2}, \{q_0\}, \{q_1\})$ defines $L_{1,2}$, such that $\delta_{1,2}=\{(q_0, b, q_1,\eta_2), (q_1, b, q_2, \eta_2), (q_2, b, q_1,\eta_2)\}$, with $\eta_2(r_{2})=1$,
%	%%
%	%\item $\CEFA_{2,1} = (\{p_0, p_1,p_2, p_3\}, \{a,b\}, (r_{1}), \delta_{2,1}, \{p_0\}, \{p_2\})$ defines $L_{2,1}$, such that $\delta_{2,1} = \{(p_0, a, p_1, \eta_1), (p_1, a, p_2, \eta_1), (p_2, a, p_3, \eta_1), (p_3, a, p_2, \eta_1)\}$, with $\eta_1(r_{1})=1$, and
%	%%
%	%\item $\CEFA_{2,2}=(\{q_0,q_1,q_2\}, \{a,b\}, (r_{2}), \delta_{2,2}, \{q_1\})$ defines $L_{2,2}$, such that $\delta_{2,2}= (q_0, b, q_1,\eta_2), (q_1, b, q_2, \eta_2), (q_2, b, q_1,\eta_2)\}$, with $\eta_2(r_{2})=1$.
%	%\end{itemize}
%\end{example}

 
\vspace{-4mm}
\subsection{Pre-images of CERLs under string operations} \label{sect:pre}

To unify the presentation, %of the decision procedure, %in this section, we usually keep the string operations abstract by only mentioning the input and output data types, namely, 
we consider string functions $f: (\Sigma^* \times \Int^{k_1}) \times \cdots \times (\Sigma^* \times \Int^{k_l}) \rightarrow \Sigma^*$. (If there is no integer input parameter, then $k_1,\cdots,k_l$ are zero.)  
%where each integer input parameter (if there is any) is assumed to be affiliated to a unique string input parameter. 
%Note that  in general $f$ can be nondeterministic, namely, on one input, $f$ may output several  strings.

\begin{definition}[Cost-enriched pre-images of CERLs] \label{def:preimage}
Suppose that $f: (\Sigma^* \times \Int^{k_1}) \times \cdots \times (\Sigma^* \times \Int^{k_l}) \rightarrow \Sigma^*$ is a string function, $L \subseteq \Sigma^* \times \Int^{k_0}$ is a CERL defined by a CEFA $\CEFA=(Q, \Sigma, R, \delta, I, F)$ with $R= (r_1, \cdots, r_{k_0})$. Then the $R$-cost-enriched pre-image of $L$ under $f$, denoted by $f^{-1}_R(L)$, is a pair $(\cR, \vec{t})$ such that 
\begin{itemize}
\item $\cR \subseteq (\Sigma^* \times \Int^{k_1 + k_0}) \times \cdots \times (\Sigma^* \times \Int^{k_l + k_0})$;
\item $\vec{t} = (t_1, \cdots ,t_{k_0})$ is a vector of linear integer terms where for each $i \in [k_0]$, $t_i$ is a term whose variables are from $\left\{r^{(1)}_i, \cdots, r^{(l)}_i\right\}$ which are fresh cost registers and are disjoint from $R$ in $\CEFA$;

%[intuitively, each cost register $r_i$ is split into $l$ cost registers $r^{(1)}_i, \cdots,r^{(l)}_i$, one for each string input parameter, and $t_i$ tells how to compute $r_i$ from $r^{(1)}_i, \cdots,r^{(l)}_i$]
\item $L$ is equal to the language comprising the $k_0$-cost-enriched strings
%
\[\left(w_0, t_1\left[d^{(1)}_{1}/r^{(1)}_1, \cdots, d^{(l)}_{1}/r^{(l)}_1\right], \cdots, t_{k_0}\left[d^{(1)}_{k_0}/r^{(1)}_{k_0}, \cdots, d^{(l)}_{k_0}/r^{(l)}_{k_0}\right]
\right), \]
%
such that 
\[w_0 = f\left((w_1, \vec{c_1}), \cdots, (w_l, \vec{c_l}\right)) \mbox{ for some } ((w_1, (\vec{c_1}, \vec{d_1})), \cdots, (w_l, (\vec{c_l}, \vec{d_l}))) \in \cR,\]
where $\vec{c_j} \in \Int^{k_j}$, $\vec{d_j} = (d^{(j)}_{1}, \cdots, d^{(j)}_{k_0}) \in \Int^{k_0}$ for $j\in [l]$.
%
%$\vec{c_1} \in \Int^{k_1}$, $\cdots$, $\vec{c_l} \in \Int^{k_l}$, $\vec{d_1} = (d^{(1)}_{1}, \cdots, d^{(1)}_{k_0}) \in \Int^{k_0}$, $\cdots$, and $\vec{d_l} = (d^{(l)}_{1},\cdots, d^{(l)}_{k_0}) \in \Int^{k_0}$.
\end{itemize}
The $R$-cost-enriched pre-image of $L$ under $f$, say $f^{-1}_R(L)=(\cR, \vec{t})$, is said to be CERR-definable if $\cR$ is a CERR. 
\end{definition}

Definition~\ref{def:preimage} is essentially a semantic definition of the pre-images. For the decision procedure, one desires an effective representation of a CERR-definable $f^{-1}_R(L)=(\cR, \vec{t})$ in terms of CEFAs. Namely,
a CEFA representation of %a CERR-definable $f^{-1}_R(L)=
$(\cR, \vec{t})$ (where $t_j$ is over $\left\{r^{(1)}_j, \cdots, r^{(l)}_j\right\}$ for $j\in [k_0]$)
is a tuple $((\CEFA_{i,1}, \cdots, \CEFA_{i, l})_{i \in [n]}, \vec{t})$ such that $(\CEFA_{i,1}, \cdots, \CEFA_{i, l})_{i \in [n]}$ is a CEFA representation of $\cR$, where $R(\CEFA_{i,j})=\left(r'_{j,1}, \cdots, r'_{j,k_j}, r^{(j)}_1, \cdots,r^{(j)}_{k_0}\right)$ for each $i \in [n]$ and $j \in [l]$. (The cost registers $r'_{1,1}, \cdots, r'_{1,k_1},\cdots, r'_{l,1}, \cdots, r'_{l,k_l}$ %, r^{(1)}_1, \cdots,r^{(1)}_{k_0}, \cdots, r^{(l)}_1, \cdots,r^{(l)}_{k_0}$ 
are mutually distinct and freshly introduced.) %\tl{$r^{(1)}_1, \cdots,r^{(1)}_{k_0}, \cdots, r^{(l)}_1, \cdots,r^{(l)}_{k_0}$ are actually introduced above?}


\begin{example}[$\substring^{-1}_R(L)$]\label{exm:pre-image}
Let $\Sigma = \{a\}$ and $L = \{(w, |w|) \mid w \in \Lang((aa)^*) \}$. Evidently $L$  is a CERL defined by a CEFA $\CEFA = (Q, \Sigma, R, \delta, \{q_0\}, \{q_0\})$ with $Q=\{q_0,q_1\}$, $R=(r_1)$ and $\delta = \{(q_0, a, q_1), (q_1, a, q_0)\}$. Since $\substring$  is  from $\Sigma^* \times \Int^2$ to $\Sigma^*$, $\substring^{-1}_R(L)$, the $R$-cost-enriched pre-image of $L$ under $\substring$, is the pair $(\cR, t)$, where $t=r^{(1)}_1$ (note that in this case $l=1$, $k_0=1$, and $k_1=2$) and 
%
$$\cR = \{(w, n_1, n_2, n_2) \mid w \in \Lang(a^*), n_1 \ge 0, n_2 \ge 0, n_1+n_2 \le |w|, n_2 \mbox{ is even}\},$$ 
%
which is represented by $(\CEFA', t)$ such that $\CEFA'= (Q', \Sigma, R', \delta', I', F')$, where 
\begin{itemize}
\item $Q' = Q \times \{p_0, p_1, p_2\}$, (Intuitively, $p_0$, $p_1$, and $p_2$ denote that the current position is before the starting position, between the starting position and ending position, and after the ending position of the substring respectively.) 
\item $R'= \left(r'_{1,1}, r'_{1,2}, r^{(1)}_1 \right)$, 
\item $I' =\{(q_0,p_0)\}$, $F'=\{(q_0, p_2), (q_0, p_0)\}$ (where $(q_0, p_0)$ is used to accept the $3$-cost-enriched strings $(w, n_1, 0, 0)$ with $0 \le n_1 \le |w|$), and 
\item $\delta'$ is  
\[
\left\{
\begin{array}{l}
(q_0, p_0) \xrightarrow{a, \eta_1} (q_0, p_0), (q_0, p_0) \xrightarrow{a, \eta_2} (q_1, p_1), (q_1, p_1) \xrightarrow{a, \eta_2} (q_0, p_1), \\
(q_0, p_1) \xrightarrow{a, \eta_2} (q_1, p_1), (q_1, p_1) \xrightarrow{a, \eta_2} (q_0, p_2), (q_0, p_2) \xrightarrow{a, \eta_3} (q_0, p_2)
\end{array}
\right\},
\] 
where $\eta_1(r'_{1,1})=1$, $\eta_1(r'_{1,2})=0$, $\eta_1(r^{(1)}_1)=0$, $\eta_2(r'_{1,1})=0$, $\eta_2(r'_{1,2})=1$, and $\eta_2(r^{(1)}_1)=1$, $\eta_3(r'_{1,1})=0$, $\eta_3(r'_{1,2})=0$, and $\eta_3(r^{(1)}_1)=0$.
\end{itemize}
%
Therefore, $\substring^{-1}_R(L)$ is CERR-definable.
\end{example}


It turns out that for each string function $f$ in the assignment statements of {\slint}, the cost-enriched pre-images of CERLs under $f$ are CERR-definable.

\begin{proposition}\label{prop:pre-image}
Let $L$ be a CERL defined by a CEFA $\CEFA = (Q, \Sigma, R, \delta, I, F)$. Then for each string function $f$ ranging over $\concat$, $\replaceall_{e,u}$, $\reverse$, FFTs $\NFT$, and $\substring$, $f^{-1}_R(L)$ is CERR-definable. In addition,
\begin{itemize}
\item a CEFA representation of $\concat^{-1}_R(L)$ can be computed in time $\bigO(|\CEFA|^2)$, 
%
\item a CEFA representation of $\reverse^{-1}_R(L)$ (resp. $\substring^{-1}_R(L)$) can be computed in time $\bigO(|\CEFA|)$,
%
\item a CEFA representation of  $(\Tran(\NFT))^{-1}_R(L)$ can be computed in time polynomial in $|\CEFA|$ and exponential in $|\NFT|$,
%
\item a CEFA representation of  $(\replaceall_{e,u})^{-1}_R(L)$ can be computed in time polynomial in $|\CEFA|$ and exponential in $|e|$ and $|u|$.
\end{itemize}
\end{proposition}

The proof of Proposition~\ref{prop:pre-image} is given in Appendix, Section~\ref{app:pre-image}.

\vspace{-2mm}
%%%%%%%%%%%%%%%%%%%%%%%%%%%%%%%
\subsection{The Decision Procedure}\label{sec:dc}
%%%%%%%%%%%%%%%%%%%%%%%%%%%%%%%%
%
Let $S$  be an {\slint} program. %We show how to decide the path feasibility of $S$. 
Without loss of generality, we assume that for every occurrence of assignments of the form $y:= \substring(x, t_1, t_2)$, it holds that $t_1$ and $t_2$ are integer variables. This is not really a restriction, since, for instance, if in $y:= \substring(x, t_1, t_2)$, neither $t_1$ nor $t_2$ is an integer variable, then we introduce fresh integer variables $i$ and $j$, replace $t_1, t_2$ by $i,j$ respectively, and add $\ASSERT{i=t_1};\ASSERT{j = t_2}$ in $S$.
We present a decision procedure for the path feasibility problem of $S$ which is divided into five steps. %is nondeterministic and divided into three steps. 
%\begin{description}
%\item[Step I: Preprocessing.] 
%
%\item 

\medskip
\noindent {\bf Step I: Reducing to atomic assertions.}%Removing $\vee$ and $\wedge$}.

\smallskip
Note first that in our language, each assertion is a positive Boolean combination of atomic formulas of the form $x\in \CEFA$ or $t_1\ o\ t_2$ (cf. Section~\ref{sec:logic}).  
Nondeterministically choose, for each assertion $\ASSERT{\varphi}$ of $S$, a set of atomic formulas $\Phi_\varphi = \{\alpha_1,\cdots,\alpha_n\}$ such that $\varphi$ holds when atomic formulas in $\Phi_\varphi$ are true.  %therein, say $\Phi_\varphi = \{\alpha_1,\cdots,\alpha_n\}$, so that $\eta_{\Phi}$, the Boolean valuation associated with $\Phi$, satisfies $\varphi$, where $\eta_{\Phi}$ assigns $\ltrue$ to $\alpha_1,\cdots, \alpha_n$ and $\lfalse$ to the other atomic formulas in $\varphi$. 

Then each assertion $\ASSERT{\varphi}$ in $S$ with $\Phi_\varphi = \{\alpha_1,\cdots,\alpha_n\}$ is replaced by $\ASSERT{\alpha_1}; \cdots; \ASSERT{\alpha_n}$, and thus $S$ constrains atomic assertions only. 

\medskip 
\noindent {\bf Step II: Dealing with the case splits in the semantics of $\indexof_v$ and $\substring$.}%Removing $\vee$ and $\wedge$}.

\smallskip

For each integer term of the form $\indexof_v(x,i)$ in $S$, nondeterministically choose one of the following five options (which correspond to the semantics of $\indexof_v$ in Section~\ref{sec:logic}).
\begin{itemize}
\item[(1)] Add $\ASSERT{i < 0}$ to $S$, and replace $\indexof_v(x,i)$ with $\indexof_v(x,0)$ in $S$. 
%
\item[(2)] Add $\ASSERT{i < 0};\ASSERT{x \in \NFA_{\overline{\Sigma^*v\Sigma^*}}}$ to $S$; replace $\indexof_v(x,i)$ with $-1$ in $S$.
%
\item[(3)] Add $\ASSERT{i \ge \length(x)}$ to $S$, and replace $\indexof_v(x,i)$ with $-1$ in $S$.
%
\item[(4)] Add $\ASSERT{i \ge 0}; \ASSERT{i < \length(x)}$ to $S$.
%
\item[(5)] Add 
%
$$
\begin{array}{l}
\ASSERT{i \ge 0}; \ASSERT{i < \length(x)}; \ASSERT{j=\length(x)-i}; \\
\ \ \ \ y:=\substring(x, i, j); \ASSERT{y \in \NFA_{\overline{\Sigma^*v\Sigma^*}}}
\end{array}
$$ 
to $S$, where $y$ is a fresh string variable, $j$ is a fresh integer variable, and $\NFA_{\overline{\Sigma^*v\Sigma^*}}$ is an NFA defining the language $\{w \in \Sigma^*\mid v \mbox{ does not occur as a substring in } w\}$. Replace $\indexof_v(x, i)$ with $-1$ in $S$.
\end{itemize}

%Nondeterministically choose a subset of $\Theta$ of the assignment statements of the form $y:=\substring(x, i, j)$ in $S$, and 
%
For each assignment $y:=\substring(x, i, j)$ in $S$, nondeterministically choose one of the following three options (which correspond to the semantics of $\substring$ in Section~\ref{sec:logic}).
\begin{itemize}
\item[(1)] Add the statements $\ASSERT{i \ge 0}; \ASSERT{i + j \le \length(x)}$ to $S$. 
%
\item[(2)] Add the statements 
$\ASSERT{i \ge 0}; \ASSERT{i \le \length(x)};\ASSERT{i+j  > \length(x)}$; $\ASSERT{i'  = \length(x)-i}$
to $S$, and replace $y:=\substring(x, i, j)$ with $y:=\substring(x, i, i')$, where $i'$ is a fresh integer variable.
%
\item[(3)] Add the statement $\ASSERT{i < 0}; \ASSERT{y \in \NFA_\varepsilon}$ to $S$, and  remove $y:=\substring(x, i, j)$ from $S$, where $\NFA_\varepsilon$ is the NFA defining the language $\{\varepsilon\}$.
\end{itemize}

\medskip
\noindent {\bf Step III: Removing $\length$ and $\indexof$}.

\smallskip

%Repeat the following procedure until there are no occurrences of $\indexof$.

For each term $\length(x)$ in $S$, we introduce a \emph{fresh} integer variable $i$, replace every occurrence of $\length(x)$ by $i$, and add the statement $\ASSERT{x \in \CEFA_{\rm len}[i/r_1]}$ to $S$. (See Example~\ref{exm:len} for the definition of $\CEFA_{\rm len}$.)  

For each term $\indexof_v(x, i)$ occurring in $S$, introduce two fresh integer variables $i_1$ and $i_2$, replace every occurrence of $\indexof_v(x, i)$ by $i_2$, and add the statements $\ASSERT{i=i_1}; \ASSERT{x \in \CEFA_{\indexof_v}[i_1/r_1, i_2/r_2]}$ to $S$.  (See Appendix, Section~\ref{appendix:cefa_indexof} for the construction of $\CEFA_{\rm \indexof_v}$.) 
%(See Example~\ref{exm:indexof} for an illustration of $\CEFA_{\rm \indexof_v}$.)

%
%\end{enumerate}
%
%\item 

\medskip
\noindent {\bf Step IV: Removing the assignment statements backwards}.

\smallskip

Repeat the following procedure until $S$ contains no assignment statements.
%
\begin{quote}
Suppose $y := f(x_1, \vec{i_1}, \cdots, x_l, \vec{i_l})$ is the \emph{last} assignment of $S$, where $f: (\Sigma^* \times \Int^{k_1}) \times \cdots \times (\Sigma^* \times \Int^{k_l}) \rightarrow \Sigma^*$ is a string function and $\vec{i_j}= (i_{j,1}, \cdots, i_{j, k_j})$ for each $j \in [l]$.
%\tl{as we are considering functional transducers, shall we change to $... \rightarrow \Sigma^*$; if decided, step ii will be updated as the product would be unnecessary}%from $\concat$, $\replaceall_{e,u}$, $\reverse$, $\NFT$, and $\substring$.
\\
Let $\{\CEFA_1, \cdots, \CEFA_s\}$ be the set of all CEFAs such that $\ASSERT{y \in \CEFA_j}$ occurs in $S$ for every $j \in [s]$. 
%Construct $\NFA = \NFA_1 \times \cdots \times \NFA_s$
%\footnote{Since the cost registers of CEFAs are always freshly introduced, it holds that $R(\CEFA_1)$, $\cdots$,  $R(\CEFA_s)$ are mutually disjoint.} 
%with $R(\CEFA) = (r_1, \cdots, r_{k_0})$. 
Let $j \in [s]$ and $R(\CEFA_j)=(r_{j,1}, \cdots, r_{j, \ell_j})$. Then from Proposition~\ref{prop:pre-image}, 
a CEFA representation of $f^{-1}_{R(\CEFA_j)}(\Lang(\CEFA_j))$, say $\left(\left(\cB^{(1)}_{j, j'}, \cdots, \cB^{(l)}_{j, j'}\right)_{j' \in [m_j]}, \vec{t}\right)$, can be effectively computed from $\NFA$ and $f$, where we write
\[
R\left(\cB^{(j'')}_{j, j'}\right)=\left((r')^{(j'',1)}_{j}, \cdots, (r')_{j}^{(j'',k_{j''})}, r^{(j'')}_{j, 1}, \cdots,r^{(j'')}_{j, \ell_j} \right)
\]
for each $j' \in [m_j]$ and $j'' \in [l]$, and $\vec{t}=(t_1,\cdots, t_{\ell_j})$. Note that the cost registers $(r')^{(1,1)}_{j}, \cdots, (r')_{j}^{(1,k_1)}, \cdots, (r')^{(l,1)}_{j}, \cdots, (r')_{j}^{(l,k_l)}, r^{(1)}_{j, 1}, \cdots,r^{(1)}_{j, \ell_j}, \cdots, r^{(l)}_{j, 1}, \cdots,r^{(l)}_{j, \ell_j}$ are mutually distinct and freshly introduced, moreover, $R\left(\cB^{(j'')}_{j, j'_1}\right)=R\left(\cB^{(j'')}_{j, j'_2}\right)$ for distinct $j'_1,j'_2 \in [m_j]$.
%

Remove $y := f(x_1, \vec{i_1}, \cdots, x_l, \vec{i_l})$, as well as all the statements $\ASSERT{y \in \CEFA_1}$, $\cdots$, $\ASSERT{y \in \CEFA_s}$ from $S$. For every $j \in [s]$, nondeterministically choose $j' \in [m_j]$, and add the following statements to $S$, 
%
\[
\begin{array}{l}
\ASSERT{x_1 \in \cB^{(1)}_{j, j'}};\ \cdots;\ \ASSERT{x_l \in \cB^{(l)}_{j, j'}}; S_{j, j', \vec{i_1}, \cdots, \vec{i_l}}; S_{j, \vec{t}}\\
\end{array}
\]
where 
\[
\begin{array}{l c c}
S_{j, j', \vec{i_1}, \cdots, \vec{i_l}} & \equiv & \ASSERT{i_{1,1} = (r')^{(1,1)}_{j, j'}}; \cdots; \ASSERT{i_{1,k_1} = (r')^{(1,k_1)}_{j, j'}};\\
& & \cdots\\
 & & \ASSERT{i_{l,1} = (r')^{(l,1)}_{j, j'}}; \cdots; \ASSERT{i_{l,k_l} = (r')^{(l,k_l)}_{j, j'}}
\end{array}
\]
and
\[
\begin{array}{l}
S_{j, \vec{t}} \equiv \ASSERT{r_{j, 1} = t_1}; \cdots, \ASSERT{r_{j, \ell_j} = t_{\ell_j}}.
\end{array}
\]
%
\end{quote}

\medskip
\noindent{\bf Step V: Final satisfiability checking.}%Solving the {\lasat} problem}.

\smallskip

In this step, $S$ %is a {\slint} program containing 
contains no assignment statements and only assertions of the form $\ASSERT{x \in \CEFA}$ and  $\ASSERT{t_1\ o\ t_2}$  where $\CEFA$ are CEFAs and $t_1, t_2$ are linear integer terms. %no assignment statements and all the integer terms are linear integer arithmetic terms. 
%
Let $X$ denote the set of string variables occurring in $S$.
For each $x \in X$, let $\Lambda_x=\{\CEFA_{x}^1, \cdots, \CEFA_{x}^{s_x}\}$ denote the set of CEFAs $\CEFA$ such that $\ASSERT{x \in \CEFA}$ appears in $S$. 
%$R_x$ be the vector of cost registers obtained by concatenating $R(\CEFA_{x,1})$, $\cdots$, and $R(\CEFA_{x,s_x})$. 
Moreover, let $\phi$ denote the conjunction of all the LIA formulas $t_1\ o\ t_2$ occurring in $S$. It is straightforward to observe that $\phi$ is over %the set of integer variables 
$R'=\bigcup_{x\in X, j \in [s_x]}R(\CEFA_{x}^{j})$. Then the path feasibility of $S$ is reduced to \emph{the satisfiability problem of LIA formulas w.r.t. CEFAs (abbreviated as {\lasat} problem)} which is defined as 
\begin{quote}
deciding whether $\phi$ is satisfiable w.r.t. $(\Lambda_x)_{x \in X}$, namely,  %\begin{definition}[{\lasat} Problem]\label{def-la-sat-cefa}
	%The satisfiability problem of LIA formulas w.r.t. CEFAs (abbreviated as {\lasat} problem) is defined as follows.
%	
%	\textbf{Input}: a given quantifier-free LIA formula $\phi$ and CEFAs $\CEFA_1,\cdots,\CEFA_m$, such that $\CEFA_i=(Q_i, \Sigma, R_i, \delta_i, I_i, F_i)$ for every $i\in [m]$, 
%	$R_i \cap R_j = \emptyset$ for every $1 \le i < j \le m$, and
%	the free variables of $\phi$ are from $\bigcup_{i\in [m]} R_i$, 
whether %$\phi$ is satisfiable w.r.t. $\CEFA_1, \cdots, \CEFA_m$, namely, whether 
there are an assignment function $\theta: R' \rightarrow \Int$ and strings $(w_x)_{x \in X}$ such that  $\phi[\theta(R')/R']$ holds and $(w_x, \theta(R(\CEFA_{x}^{j}))) \in \Lang(\CEFA_{x}^{j})$ for every $x \in X$ and $j \in [s_x]$.
\end{quote}
This {\lasat} problem is decidable and $\pspace$-complete;
%A proof of Proposition~\ref{prop:la-sat-cefa-inter} 
The proof can be found in Appendix, Section~\ref{app:sat-cefa}.  
%\end{definition}

%\begin{definition}[{\lasat} Problem]\label{def-la-sat-cefa}
%	The satisfiability problem of LIA formulas w.r.t. CEFAs (abbreviated as {\lasat} problem) is defined as follows.
%	
%	\textbf{Input}: a given quantifier-free LIA formula $\phi$ and CEFAs $\CEFA_1,\cdots,\CEFA_m$, such that $\CEFA_i=(Q_i, \Sigma, R_i, \delta_i, I_i, F_i)$ for every $i\in [m]$, 
%  $R_i \cap R_j = \emptyset$ for every $1 \le i < j \le m$, and
% the free variables of $\phi$ are from $\bigcup_{i\in [m]} R_i$, 
% 
%Decide whether %$\phi$ is satisfiable w.r.t. $\CEFA_1, \cdots, \CEFA_m$, namely, whether 
%there are an assignment function $\theta: \bigcup \limits_{i \in [m]} R_i \rightarrow \Int$ and strings $w_1, \cdots, w_m$  
%	such that  $\phi[(\theta(R_i)/R_i)_{i \in [m]}]$ hold and $(w_i, \theta(R_i)) \in \Lang(\NFA_i)$ for every $i \in [m]$.
%\end{definition}

\begin{proposition}\label{prop:la-sat-cefa-inter}
{\lasat} is $\pspace$-complete.
%	Given a family of CEFAs $\{ \CEFA_i^{j} \}_{i\in I,j\in J_i}$ each of which carries a vector of registers $R_i^j$ and a quantifier-free LIA formula $\phi$ such that  $ R_i^{j} $ are pairwise disjoint and the variables of $\phi$ are from $R'=\bigcup_{i,j} R_i^j$. Deciding whether  %such that the free variables of $\phi$ are from $\bigcup_{i\in [m]} R_i$,  and here are 
%	there are an assignment function $\theta: R' \rightarrow \Int$ and strings $(w_i)_{i \in I}$ such that  $\phi[\theta(R' )/R']$ holds and $(w_i, \theta(R_i^j)) \in \Lang(\CEFA_{i}^j)$ for every $i \in I$ and $j \in J_i$ is $\pspace$-complete. 
\end{proposition}

%%%%%% commented for space %%%%%%%%%%%%%%%
%A natural idea for the proof for Proposition~\ref{prop:la-sat-cefa-inter} is to compute, for each string variable $x \in X$, an existential LIA formula $\phi_x$ defining the Parikh image of the product of the CEFAs in $\Lambda_x$, and then to solve the satisfiability of $\phi \wedge \bigwedge_{x \in X} \phi_x$. Nevertheless, computing the product of CEFAs followed by the computation of its Parikh image would require exponential space. To circumvent this exponential space blowup, %instead of computing the product of CEFAs in $\Lambda_x$ explicitly, 
%we utilise Proposition 16 in \cite{LB16} to show a small model property for the {\lasat} problem: If a model of  {\lasat}, specifically, an assignment function $\theta: R' \rightarrow \Int$ and strings $(w_x)_{x \in X}$, exists, then $\theta$ can be assumed to satisfy that for each $x \in X$ and $r \in R_x=\bigcup_{j \in [s_x]} R(\CEFA_{x}^{j})$, $\theta(r)$ is at most exponential in the size of $\CEFA_{x}^{j}$ for $j \in [s_x]$ and $|\phi|$. Since the binary encodings of $\theta(r)$ can be stored in polynomial space, one can \emph{nondeterministically} guess the strings $(w_x)_{x \in X}$, simulate the runs of $\CEFA_{x}^{j}$ on $w_x$, and finally evaluate $\phi$ over the register values after all CEFAs $\CEFA_{x}^{j}$ accept, in polynomial space. 
%%%%%% commented for space %%%%%%%%%%%%%%%

\smallskip
\noindent\emph{Complexity analysis of the decision procedure.} Step I and Step II can be done in nondeterministic linear time. Step III can be done in linear time. In Step IV, for each input string variable $x$ in $S$, at most exponentially many CEFAs can be generated for $x$, each of which is of at most exponential size. Therefore, Step IV can be done in nondeterministic exponential space. By Proposition~\ref{prop:la-sat-cefa-inter}, Step V can be done in exponential space. Therefore, we conclude that the path feasibility problem of {\slint} programs is in $\nexpspace$, thus in $\expspace$ by Savitch's theorem \cite{complexity-book}.  
 
\begin{remark}
	In this paper, we focus on functional finite transducers (cf.\ Section~\ref{sec:prel}). Our decision procedure is applicable to general finite transducers as well with minor adaptation. However, the $\expspace$ complexity upper-bound does not hold any more, because the distributive property $f^{-1}(L_1\cap L_2)= f^{-1}(L_1)\cap f^{-1}(L_2)$ for regular languages $L_1, L_2$ only holds for functional finite transducers $f$.  
\end{remark}

%An example to demonstrate the decision procedure applied to the program in Example~\ref{exmp:logic} is provided in the Appendix, Section~\ref{app:urlexample}.

%\begin{example}
%Consider the program $S$ associated with {\tt urlXssSanitise(url)} in Section~\ref{sec:intro}. % Example~\ref{exmp:running}.
%%\[
%%\begin{array}{l}
%%y := \replaceall_{\Sigma \setminus ), \varepsilon}(x); z:= \replaceall_{\Sigma \setminus (, \varepsilon}(x);\\
%%\ASSERT{\length(y) = \length(z)}; \ASSERT{\indexof_{(}(x,0) < \indexof_{)}(x,0)}.
%%\end{array}
%%\] 
%We show how to decide its path feasibility. % of $S$. 
%\begin{description}
%\item[Step I.]   Vacuous since $S$ contains only atomic assertions already. %neither disjunction nor conjunction.
%%
%\item[Step II.] Nondeterministically choose to replace $\indexof_\#(\mathtt{url1}, 0)$ with $-1$ and add $\ASSERT{\mathtt{url1} \in \NFA_{\overline{\Sigma^*\#\Sigma^*}}}$ to $S$.  
%%
%\item[Step III.] Replace $\length(\mathtt{url1})$ with $i'_1$ and add $\ASSERT{\mathtt{url1} \in \CEFA_{\rm len}[i'_1/r_1]}$ to $S$, moreover, replace $\indexof_?(\mathtt{url1}, 0)$ with $i'_3$ and add $\ASSERT{0= i'_2}; \ASSERT{\mathtt{url1} \in \CEFA_{\indexof}[i'_2/r_1, i'_3/r_2]}$ to $S$, where $i'_1, i'_2, i'_3$ are fresh integer variables. Then we get the following program (still denoted by $S$), 
%\[ 
%\begin{array}{l}
%    \ASSERT{\mathtt{prothostpath} \in \NFA_\varepsilon}; \ASSERT{\mathtt{querfrag} \in \NFA_\varepsilon}; \mathtt{url1} := \NFT_{\rm trim}(\mathtt{url}); \\
%    \ASSERT{\mathtt{qmarkpos} = i'_3}; \ASSERT{\mathtt{sharppos} =-1 }; \ASSERT{\mathtt{qmarkpos} \ge 0}; \\ 
%    \mathtt{prothostpath1} := \substring(\mathtt{url1}, 0, \mathtt{qmarkpos});\\
%   \mathtt{querfrag1} := \substring(\mathtt{url1, qmarkpos}, i'_1 - \mathtt{qmarkpos});\\
%    \mathtt{querfrag2} := \replaceall_{\mathtt{script},\ \varepsilon}(\mathtt{querfrag1});\\
%    \mathtt{url2} := \mathtt{prothostpath1} \concat \mathtt{querfrag2}; \ASSERT{\mathtt{querfrag2} \in  \NFA_{\Sigma^*\mathtt{script}\Sigma^*}};  \\
%    \ASSERT{\mathtt{url1} \in  \NFA_{\overline{\Sigma^*\#\Sigma^*}}}; \ASSERT{\mathtt{url1} \in \CEFA_{\rm len}[i'_1/r_1]}; \\
%    \ASSERT{0= i'_2}; \ASSERT{\mathtt{url1} \in \CEFA_{\indexof}[i'_2/r_1, i'_3/r_2]}.
%\end{array}
%\]
%%
%\item[Step IV.] Since there is no CEFA constraints for $\mathtt{url2}$, removing the last assignment statement of $S$, i.e. $\mathtt{url2} := \mathtt{prothostpath1} \concat \mathtt{querfrag2}$, is easy and in this case we add no statements to $S$. After this, $\mathtt{querfrag2} := \replaceall_{\mathtt{script},\ \varepsilon}(\mathtt{querfrag1})$ becomes the last assignment statement. Suppose $\NFA'=(Q', \Sigma, \delta', I', F')$ is an NFA representing $(\replaceall_{\mathtt{script},\ \varepsilon})^{-1}_\emptyset(\Lang(\NFA_{\Sigma^*\mathtt{script}\Sigma^*}))$, namely, the pre-image of $\Lang(\NFA_{\Sigma^*\mathtt{script}\Sigma^*})$ under $\replaceall_{\mathtt{script},\ \varepsilon}$. Then we remove this assignment statement and add $\ASSERT{\mathtt{querfrag1 \in \NFA'}}$, resulting into the following program
%\[ 
%\begin{array}{l}
%    \ASSERT{\mathtt{prothostpath} \in \NFA_\varepsilon}; \ASSERT{\mathtt{querfrag} \in \NFA_\varepsilon}; \mathtt{url1} := \NFT_{\rm trim}(\mathtt{url}); \\
%    \ASSERT{\mathtt{qmarkpos} = i'_3}; \ASSERT{\mathtt{sharppos} =-1 }; \ASSERT{\mathtt{qmarkpos} \ge 0}; \\ 
%    \mathtt{prothostpath1} := \substring(\mathtt{url1}, 0, \mathtt{qmarkpos});\\
%   \mathtt{querfrag1} := \substring(\mathtt{url1, qmarkpos}, i'_1 - \mathtt{qmarkpos});\\
%%    \mathtt{querfrag2} := \replaceall_{\mathtt{script},\ \varepsilon}(\mathtt{querfrag1});\\
%%    \mathtt{url2} := \mathtt{prothostpath1} \concat \mathtt{querfrag2}; 
%    \ASSERT{\mathtt{querfrag2} \in  \NFA_{\Sigma^*\mathtt{script}\Sigma^*}};  
%    \ASSERT{\mathtt{url1} \in  \NFA_{\overline{\Sigma^*\#\Sigma^*}}}; \\
%    \ASSERT{\mathtt{url1} \in \CEFA_{\rm len}[i'_1/r_1]};  \ASSERT{0= i'_2}; \\
%    \ASSERT{\mathtt{url1} \in \CEFA_{\indexof}[i'_2/r_1, i'_3/r_2]};  \ASSERT{\mathtt{querfrag1} \in \NFA'}.
%\end{array}
%\]
%
%From Example~\ref{exm:pre-image}, we know that $\substring^{-1}_\emptyset(\Lang(\NFA'))$ can be represented by some CEFA $\cB=(Q'', R'', \delta'', I'', F'')$ with $Q''= Q' \times \{p_0,p_1,p_2\}$ and $R''=(r'_{1,1}, r'_{1,2})$ (where $r'_{1,1}$ and $r'_{1,2}$ are fresh integer variables). Then we remove $\mathtt{querfrag1} := \substring(\mathtt{url1, qmarkpos}, i'_1 - \mathtt{qmarkpos})$, add $\ASSERT{\mathtt{url1} \in \cB};\ASSERT{\mathtt{r'_{1,1}= qmarkpos}}; \ASSERT{r'_{1,2}=i'_1 - \mathtt{qmarkpos}}$, and get the following program
%\[ 
%\begin{array}{l}
%    \ASSERT{\mathtt{prothostpath} \in \NFA_\varepsilon}; \ASSERT{\mathtt{querfrag} \in \NFA_\varepsilon}; \mathtt{url1} := \NFT_{\rm trim}(\mathtt{url}); \\
%    \ASSERT{\mathtt{qmarkpos} = i'_3}; \ASSERT{\mathtt{sharppos} =-1 }; \ASSERT{\mathtt{qmarkpos} \ge 0}; \\ 
%    \mathtt{prothostpath1} := \substring(\mathtt{url1}, 0, \mathtt{qmarkpos});\\
%%   \mathtt{querfrag1} := \substring(\mathtt{url1, qmarkpos}, i'_1 - \mathtt{qmarkpos});\\
%%    \mathtt{querfrag2} := \replaceall_{\mathtt{script},\ \varepsilon}(\mathtt{querfrag1});\\
%%    \mathtt{url2} := \mathtt{prothostpath1} \concat \mathtt{querfrag2}; 
%    \ASSERT{\mathtt{querfrag2} \in  \NFA_{\Sigma^*\mathtt{script}\Sigma^*}};  
%    \ASSERT{\mathtt{url1} \in  \NFA_{\overline{\Sigma^*\#\Sigma^*}}}; \\
%    \ASSERT{\mathtt{url1} \in \CEFA_{\rm len}[i'_1/r_1]};  \ASSERT{0= i'_2}; \\
%    \ASSERT{\mathtt{url1} \in \CEFA_{\indexof}[i'_2/r_1, i'_3/r_2]};  \ASSERT{\mathtt{querfrag1} \in \NFA'};\\
%    \ASSERT{\mathtt{url1} \in \cB};\ASSERT{\mathtt{r'_{1,1} = qmarkpos} }; \ASSERT{r'_{1,2}=i'_1 - \mathtt{qmarkpos}}.
%\end{array}
%\]
%We can continue the process until the problem contains no assignment statement.
%%
%\item[Step V.]  Straightforward by utilising Proposition~\ref{prop:la-sat-cefa-inter}. 
%\end{description}
%\end{example}

%and the {\lasat} problem is $\np$-complete according to  Theorem~\ref{thm-la-sat-cefa}, we deduce that Step III can be fulfilled in  nondeterministic doubly exponential time. Therefore, we conclude that the path feasibility problem is in 2$\nexptime$. \zhilin{The complexity analysis is for deterministic string operations.}



%\paragraph*{Functional string operations.}
%We would like to remark that if all the string operations $f$ in $S$ are \emph{deterministic}, then the product of  the CEFAs in $\rho$ can be avoided and the pre-image can be computed \emph{distributively} for every CEFA in $\rho$. \zhilin{I think we should only present the decision procedure for deterministic string operations, since this is the case for {\slint}.}

%\subsection{Satisfiability of Linear Integer Arithmetic Formula with respect to CEFAs}

%In the sequel, we consider a decision problem for CEFAs which will be used for the decision procedure of the path feasibility problem for {\slint} in the next section.


%Note that in Definition~\ref{def-la-sat-cefa}, registers in $\NFA_i$'s may intersect. 

%\begin{example}
%	Let $\phi \equiv r_1 = r_2$ and $\CEFA_{1,1}, \CEFA_{1,2}$ be two CEFAs in Example~\ref{exm:CERR} defining $L_{1,1}, L_{1,2}$ respectively. (Recall that $R(\CEFA_{1,1})=(r_1)$ and $R(\CEFA_{1,2})=(r_2)$.) Then it is easy to see that $\phi$ is unsatisfiable w.r.t. $\CEFA_{1,1}$ and $\CEFA_{1,2}$, since for each $(w_1, n_1) \in L_{1,1}$, $n_1$ must be even, while for each $(w_2, n_2) \in L_{1,2}$, $n_2$ must be odd. Hence $n_1$ and $n_2$ cannot be equal.
%\end{example}

%\begin{theorem}\label{thm-la-sat-cefa}
%	The {\lasat} problem is NP-complete.
%\end{theorem}
%The proof is given in Appendix, Section~\ref{appendix:thm-la-sat-cefa}.




%%%%%%%%pre-image of concatenation: removed%%%%%%%%
%%%%%%%%pre-image of concatenation: removed%%%%%%%%
%\hide{
%\begin{example}\label{exm:pre-image}
%Let $L = \{(w, |w|) \mid w \in \Lang((aa)^*b(bb)^*) \}$. Evidently $L$  is a CERL defined by a CEFA $\CEFA = (Q, \{a,b\}, R=(r_1), \delta, I, F)$. Since the concatenation function $\concat$  is  from $\Sigma^* \times \Sigma^*$ to $\Sigma^*$, $\concat^{-1}_R(L)$, the $R$-cost-enriched pre-image of $L$ under concatenation $\concat$, is the pair $(\cR, t)$, where $t=r^{(1)}_1+r^{(2)}_1$ (note that in this case $l=2$, $k_0=1$, and $k_j=0$ for $j\in [l]$) and 
%\[\cR = L_{1,1} \times L_{1,2} \cup L_{2,1} \times L_{2,2} \cup L_{3,1} \times L_{3,2} \cup L_{4,1} \times L_{4,2} \cup L_{5,1} \times L_{5,2},\]
%such that
%\begin{itemize}
%\item $L_{1,1} = \{(w_1, |w_1|) \mid w_1 \in \Lang((aa)^*)\}$ and $L_{1,2} = \{(w_2, |w_2|) \mid w_2 \in \Lang(b(bb)^*)\}$,
%%
%\item $L_{2,1} = \{(w_1, |w_1|) \mid w_1 \in \Lang((aa)^*)\}$ and $L_{2,2} = \{(w_2, |w_2|) \mid w_2 \in \Lang((aa)^*b(bb)^*)\}$,
%%
%\item $L_{3,1} = \{(w_1, |w_1|) \mid w_1 \in \Lang(a(aa)^*)\}$ and $L_{3,2} = \{(w_2, |w_2|) \mid w_2 \in \Lang(a(aa)^*b(bb)^*)\}$,
%%
%\item $L_{4,1} = \{(w_1, |w_1|) \mid w_1 \in \Lang((aa)^*b(bb)^*)\}$ and $L_{4,2} = \{(w_2, |w_2|) \mid w_2 \in \Lang((bb)^*)\}$,
%%
%\item $L_{5,1} = \{(w_1, |w_1|) \mid w_1 \in \Lang((aa)^*(bb)^*)\}$ and $L_{5,2} = \{(w_2, |w_2|) \mid w_2 \in \Lang(b(bb)^*)\}$.
%\end{itemize}
%%
%It is easy to see that $\cR$ is a CERR. Thus $\concat^{-1}_R(L)$ is CERR-definable.
%\end{example}
%}
%%%%%%%%pre-image of concatenation: removed%%%%%%%%
%%%%%%%%pre-image of concatenation: removed%%%%%%%%




%%%%%%%%%%%%%%%%The original example for the decision procedure%%%%%%%%%%%%%
%%%%%%%%%%%%%%%%The original example for the decision procedure%%%%%%%%%%%%%
%%%%%%%%%%%%%%%%The original example for the decision procedure%%%%%%%%%%%%%
%%%%%%%%%%%%%%%%The original example for the decision procedure%%%%%%%%%%%%%
\hide{
\begin{example}
Consider the program $S$ in Example~\ref{exmp:running}.
\[
\begin{array}{l}
y := \replaceall_{\Sigma \setminus ), \varepsilon}(x); z:= \replaceall_{\Sigma \setminus (, \varepsilon}(x);\\
\ASSERT{\length(y) = \length(z)}; \ASSERT{\indexof_{(}(x,0) < \indexof_{)}(x,0)}.
\end{array}
\] 
We show how to decide the path feasibility of $S$. 
\begin{description}
\item[Step I.]   Vacous since $S$ contains only atomic asserations already. %neither disjunction nor conjunction.
%
\item[Step II.] Nondeterministically choose $\mathcal{I} = \emptyset$.  
%
\item[Step III.] From the fact that $R(\CEFA_{\rm len})=(r_1)$ and $R(\CEFA_{\indexof_v})=(r_1,r_2)$, we introduce fresh variables $i_1$ for $\length(y)$, $i_2$ for $\length(z)$, $j_1, j_2$ for $\indexof_{(}(x,0)$, and $j'_1,j'_2$ for $\indexof_{)}(x,0)$. Then replace $\length(y)$ with $i_1$, $\length(z)$ with $i_2$, $\indexof_{(}(x,0)$ with $j_2$, $\indexof_{)}(x,0)$ with $j'_2$ respectively. Moreover, add the statements $\ASSERT{y \in \CEFA_{\rm len}[i_1/r_1]}$, $\ASSERT{z \in \CEFA_{\rm len}[i_2/r_1]}$, $\ASSERT{j_1 = 0}; \ASSERT{x \in \CEFA_{\indexof_{(}}[j_1/r_1, j_2/r_2]}$, and $\ASSERT{j'_1 = 0}; \ASSERT{x \in \CEFA_{\indexof_{)}}[j'_1/r_1, j'_2/r_2]}$, resulting in  the program 
\[
\begin{array}{l}
y := \replaceall_{\Sigma \setminus ), \varepsilon}(x); z:= \replaceall_{\Sigma \setminus (, \varepsilon}(x);\\
\ASSERT{i_1 = i_2}; \ASSERT{ j_2 < j'_2};\\
\ASSERT{y \in \CEFA_{\rm len}[i_1/r_1]}; \ASSERT{z \in \CEFA_{\rm len}[i_2/r_1]};\\
\ASSERT{j_1 = 0}; \ASSERT{x \in \CEFA_{\indexof_{(}}[j_1/r_1, j_2/r_2]};  \\
\ASSERT{j'_1 = 0}; \ASSERT{x \in \CEFA_{\indexof_{)}}[j'_1/r_1, j'_2/r_2]}.
\end{array}
\] 
%
\item[Step IV.] Recall that $\CEFA_{\rm len}=(\{q_0\}, \Sigma, (r_1), \delta_{\rm len}, \{q_0\}, \{q_0\})$, where $\delta_{\rm len} = \{(q_0, a, q_0, \eta) \mid a \in \Sigma, \eta(r_1)=1\}$. Moreover, let $\NFT_{\replaceall_{\Sigma \setminus (, \varepsilon}} = (\{p_0\}, \Sigma, \delta_{\replaceall_{\Sigma \setminus (, \varepsilon}}, \{p_0\}, \{p_0\})$, where $\delta_{\replaceall_{\Sigma \setminus (, \varepsilon}} = \{(p_0, a, p_0, a), (p_0, (, p_0, \varepsilon) \mid a \in \Sigma \setminus \{(\}\}$. Then a CEFA representation of $(\replaceall_{\Sigma \setminus (, \varepsilon})^{-1}_{(i_2)}(\Lang(\CEFA_{\rm len}[i_2/r_1]))$ is 
%
\[\CEFA' = \left((\{(p_0, q_0)\}, \Sigma, (i^{(1)}_2), \delta', \{(p_0, q_0)\}, \{(p_0, q_0)\}), (i^{(1)}_2)\right),\]
%
where $\delta'$ comprises the tuples $((p_0,q_0), (, (p_0,q_0), \eta_1)$ and $((p_0,q_0), a, (p_0,q_0), \eta_0)$ such that $a \in \Sigma \setminus \{(\}$, $\eta_1(i^{(1)}_2)=1$, and $\eta_0(i^{(1)}_2)=0$. 

Remove  $z:= \replaceall_{\Sigma \setminus (, \varepsilon}(x)$ and $\ASSERT{z \in \CEFA_{\rm len}[i_2/r_1]}$ from the program and add the statement $\ASSERT{i_2=i^{(1)}_2};\ASSERT{x \in \CEFA'}$. 

Similarly, a CEFA representation of $(\replaceall_{\Sigma \setminus ), \varepsilon})^{-1}_{(i_2)}(\Lang(\CEFA_{\rm len}[i_1/r_1]))$ is  
\[\CEFA'' = \left((\{(p_0, q_0)\}, \Sigma, (i^{(1)}_1), \delta'', \{(p_0, q_0)\}, \{(p_0, q_0)\}), (i^{(1)}_1)\right),\]
%
where $\delta''$ comprises the tuples $((p_0,q_0), ), (p_0,q_0), \eta_1)$ and $((p_0,q_0), a, (p_0,q_0), \eta_0)$ such that $a \in \Sigma \setminus \{)\}$, $\eta_1(i^{(1)}_1)=1$, and $\eta_0(i^{(1)}_1)=0$. 

Remove  $y:= \replaceall_{\Sigma \setminus ), \varepsilon}(x)$ and $\ASSERT{y \in \CEFA_{\rm len}[i_1/r_1]}$ from the program and add the statement $\ASSERT{i_1=i^{(1)}_1}; \ASSERT{x \in \CEFA''}$, resulting in the following program,
\[
\begin{array}{l}
%y := \replaceall_{\Sigma \setminus ), \varepsilon}(x); z:= \replaceall_{\Sigma \setminus (, \varepsilon}(x);\\
\ASSERT{i_1 = i_2}; \ASSERT{ j_2 < j'_2};\\
%\ASSERT{y \in \CEFA_{\rm len}[i_1/r_1]}; \ASSERT{z \in \CEFA_{\rm len}[i_2/r_1]};\\
\ASSERT{j_1 = 0}; \ASSERT{x \in \CEFA_{\indexof_{(}}[j_1/r_1, j_2/r_2]};  \\
\ASSERT{j'_1 = 0}; \ASSERT{x \in \CEFA_{\indexof_{)}}[j'_1/r_1, j'_2/r_2]};\\
\ASSERT{i_2=i^{(1)}_2}; \ASSERT{x \in \CEFA'}; \ASSERT{i_1=i^{(1)}_1}; \ASSERT{x \in \CEFA''}.
\end{array}
\] 
%
\item[Step V.]  Straightforward by utilising Proposition~\ref{prop:la-sat-cefa-inter}. 
\end{description}

\end{example}
}


%%%%%%%%%%%%%%%%%%%%%%%%%%%%%%%%%%%%%%%%%%%%%%%%
%%%%%%%%%%%%%%%%CERR linear integer functions removed%%%%%%%%%%%%%
%%%%%%%%%%%%%%%%CERR linear integer functions removed%%%%%%%%%%%%%
%%%%%%%%%%%%%%%%%%%%%%%%%%%%%%%%%%%%%%%%%%%%%%%%

\hide{
\begin{definition}[CERR linear integer functions]
An integer function $g: \Sigma^* \times \Int^{k_1} \times \Sigma^* \times \Int^{k_l} \rightarrow 2^\Int$ is  \emph{linear} if there is a pair $(\cR, t)$ such that $\cR \subseteq \Sigma^* \times \Int^{k_1+1} \times \Sigma^* \times \Int^{k_l+1}$ is a CERR and $t$ a linear integer term over $r^{(1)}, \cdots, r^{(l)}$ such that for all $\vec{c_1} \in \Int^{k_1}, \cdots, \vec{c_l} \in \Int^{k_l}$, and $d_1, \cdots, d_l \in \Int$, it holds that $(w_1, (\vec{c_1}, d_1), \cdots, w_l, (\vec{c_l}, d_l)) \in \cR$ iff $t[d_1/r^{(1)}, \cdots, d_l/r^{(l)}] \in g(w_1, \vec{c_1}, \cdots, w_l, \vec{c_l})$.  

For a CERR linear integer function $g$ witnessed by the pair $(\cR, t)$, a CEFA representation of $g$ is a tuple $((\NFA_{i,1}, \cdots, \NFA_{i, l})_{i \in [n]}, t)$, where $(\NFA_{i,1}, \cdots, \NFA_{i, l})_{i \in [n]}$ is a CEFA representation of $\cR$.

\end{definition}

\begin{example}
The string functions $\length$ and $\indexof_u$ are CERR linear integer functions, whose CEFA representations can be found in Section~\ref{sec-cslint}.
\end{example}

Suppose $A = A_r \wedge A_i$, where $A_r$ is a conjunction of atomic formulae of the form $x \in \NFA$, and $A_i$ is linear arithmetic formula (containing no integer functions). By computing the product construction of CEFAs, $A_r$ can be rewritten as $x_1 \in \NFA_1 \wedge \cdots \wedge x_n \in \NFA_n$, where $x_1,\cdots, x_n$ are mutually distinct. Therefore, the path feasibility of $S \wedge A$ is exactly the satisfiability of $A_i$ w.r.t. the CEFAs $\NFA_1, \cdots, \NFA_n$. From Theorem~\ref{thm-incra-la-sat}, we conclude that the path feasibility of  {\slint} is decidable.




Without loss of generality, we assume that string functions only apply to string variables.


Let $S':=S$ and $A':=A$. Moreover, let $A'':= \ltrue$. Then execute the following procedure to (partially) flatten the integer terms.
\begin{description}
\item[Step 1.] Recursively apply the following transformation until $S' \wedge A'$ contains no more occurrences of integer functions: Select an occurrence of integer functions, say $g(x_1, \vec{t_1}, \cdots, x_k, \vec{t_k})$, such that 
%it is a \emph{proper} subterm of the other integer term and 
{\it none} of $\vec{t_1}, \cdots, \vec{t_k}$ contains occurrences of integer functions, introduce a fresh integer variable $i$, let $S' \wedge A'$ be the formula obtained by replacing $g(x_1, \vec{t_1}, \cdots, x_k, \vec{t_k})$ with $i$, moreover, let $A'':= A'' \wedge i = g(x_1, \vec{t_1}, \cdots, x_k, \vec{t_k})$.
%
\item[Step 2.] It comprises the following two substeps. 
\begin{enumerate}
\item For each occurrence of string functions in $S'$, say $f(x_1, \vec{t_1}, \cdots, x_k, \vec{t_k})$, suppose $\vec{t_j} = (t_{j,1}, \cdots, t_{j, l_j})$ for each $j \in [k]$, introduce fresh integer variables $i_{j, j'}$ for $j \in [k]$ and $j' \in [l_j]$, replace $f(x_1, \vec{t_1}, \cdots, x_k, \vec{t_k})$ with $f(x_1, \vec{i_1}, \cdots, x_k, \vec{i_k})$ in $S'$, where $\vec{i_j} = (i_{j,1}, \cdots, i_{j, l_j})$ for each $j \in [k]$, and let $A':=A' \wedge \bigwedge \limits_{j \in [k], j' \in [l_j]} i_{j, j'} = t_{j, j'}$. 
\item For each occurrence of integer functions in $A''$, say $g(x_1, \vec{t_1}, \cdots, x_k, \vec{t_k})$, suppose $\vec{t_j} = (t_{j,1}, \cdots, t_{j, l_j})$ for each $j \in [k]$, introduce fresh integer variables $i_{j, j'}$ for $j \in [k]$ and $j' \in [l_j]$, replace $g(x_1, \vec{t_1}, \cdots, x_k, \vec{t_k})$ with $g(x_1, \vec{i_1}, \cdots, x_k, \vec{i_k})$ in $A''$, where $\vec{i_j} = (i_{j,1}, \cdots, i_{j, l_j})$ for each $j \in [k]$, and let $A':=A' \wedge \bigwedge \limits_{j \in [k], j' \in [l_j]} i_{j, j'} = t_{j, j'}$. 
\end{enumerate}
%
\item[Step 3.] Let $S:=S'$ and $A:=A'' \wedge A' $.
\end{description}
The aforementioned flattening procedure is a bit technical, for simplicity, we may assume that the integer terms are fully flattened, including the arithmetic operations.

Note that after the aforementioned flattening procedure, the resulting formula $S \wedge A$ satisfies the following property: 
\begin{quote}
The integer terms in all the occurrences of string and integer functions  are integer variables, moreover, each integer variable occurs at most once in these string and integer functions.  \hfill ($*$)
\end{quote}
Therefore, in the sequel, we assume that $S \wedge A$ satisfies the property ($*$).

\begin{theorem}\label{thm-sl-int-dec}
Path feasibility of {\slint} satisfying the semantic conditions is decidable.
\end{theorem}

\begin{proof}
In the following, we extend the generic decision procedure in \cite{CHL+19}, where NFA is replaced by CEFA.

Let $S \wedge A$ be an {\slint} formula (satisfying the property ($*$)).

For each occurrence of $i = g(x_1, \vec{i'_1}, \cdots, x_k, \vec{i'_k})$ in $A$ with $g$ an integer function, apply the following nondeterministic transformation to $A$: 
\begin{quote}
According to the 1st semantic condition, $g$ is a CERR linear integer function and a CEFA representation of $g$, say $((\NFA_{j,1}, \cdots, \NFA_{j, k})_{j \in [m]}, t)$, can be computed effectively from $g$. Consider $((\NFA'_{j,1}, \cdots, \NFA'_{j, k})_{j \in [m]}, t')$, where $\NFA'_{j,1}=\NFA_{j,1}[\vec{i'_1}/R(\NFA_{j,1})]$, $\cdots$, $\NFA'_{j,k}=\NFA_{j,k}[\vec{i'_k}/R(\NFA_{j,k})]$, and $t' = t[i^{(1)}/r^{(1)}, \cdots, i^{(k)}/r^{(k)}]$.
Nondeterministically choose $j \in [m]$, and replace $i = g(x_1, \vec{i'_1}, \cdots, x_k, \vec{i'_k})$ by $x_1 \in \NFA'_{j,1} \wedge \cdots \wedge x_k \in \NFA'_{j,k} \wedge i = t'$ in $A$.
\end{quote}
Note that after this transformation, $S \wedge A$ contains no occurrences of integer functions, moreover, as a result of the property ($*$), for every variable $x$, all the CEFAs to which $x$ belongs satisfy that their sets of registers are  mutually disjoint.

Then repeat the following procedure until $S$ becomes empty.
%
\begin{quote}
Suppose $y := f(x_1, \vec{i_1}, \cdots, x_k, \vec{i_k})$ is the last assignment of $S$. 
\\
Let $\rho := \{\NFA_1, \cdots, \NFA_s\}$ be the set of all CEFAs such that $y \in \NFA_j$ occurs in $A$ for each $j \in [s]$. Construct $\NFA = \NFA_1 \times \cdots \times \NFA_s$ (Recall that the sets of registers of $\NFA_1$, $\cdots$, $\NFA_s$ are mutually disjoint). Let  the vector of registers in $\NFA$ be $R = (r'_1, \cdots, r'_n)$. Then according to the 2nd semantic condition, 
a CEFA representation of the $R$-cost enriched pre-image of $\Lang(\NFA)$ under $f$, say $((\cB_{j, 1}, \cdots, \cB_{j, k})_{j \in [\ell]}, \vec{t})$, can be effectively computed from $\NFA$ and $f$. Consider $((\cB'_{j, 1}, \cdots, \cB'_{j, k})_{j \in [\ell]}, \vec{t'})$, where $\cB'_{j, 1} = \cB_{j, 1}[\vec{i_1}/R(\cB_{j,1}), \vec{(r')^{(1)}}/\vec{r^{(1)}}]$, $\cdots$, $\cB'_{j,k}=\cB_{j,k}[\vec{i_k}/R(\cB_{j,k}), \vec{(r')^{(k)}}/\vec{r^{(k)}}]$ (with $\vec{r^{(1)}}= (r^{1}_1, \cdots, r^{(1)}_n)$, similarly for $\vec{r^{(2)}}$ and so on), and $\vec{t'} = \vec{t}[\vec{r'_1}/\vec{r_1}, \cdots, \vec{r'_n}/\vec{r_n}]$ (with $\vec{r_1} = (r^{(1)}_1, \cdots, r^{(k)}_1)$, similarly for $\vec{r^{(2)}}$ and so on). 
\\
Nondeterministically choose $j \in [\ell]$ and let 
$$A:= A \wedge x_1 \in \cB'_{j, 1} \wedge \cdots \wedge x_k \in \cB'_{j, k}  \wedge \bigwedge \limits_{j' \in [n]} r'_{j'} = t'_{j'}.$$
%
Remove $y := f(x_1, \vec{i_1}, \cdots, x_k, \vec{i_k})$ from $S$.
\end{quote}

We would like to remark that if all the string functions $f$ in $S \wedge A$ are \emph{deterministic}, then the product of CEFAs before the pre-image computation can be avoided and the pre-image can be computed \emph{distributively} for CEFAs in $\rho$.

In the end, we get a formula $S \wedge A$ where $S$ is empty. Suppose $A = A_r \wedge A_i$, where $A_r$ is a conjunction of atomic formulae of the form $x \in \NFA$, and $A_i$ is linear arithmetic formula (containing no integer functions). By computing the product construction of CEFAs, $A_r$ can be rewritten as $x_1 \in \NFA_1 \wedge \cdots \wedge x_n \in \NFA_n$, where $x_1,\cdots, x_n$ are mutually distinct. Therefore, the path feasibility of $S \wedge A$ is exactly the satisfiability of $A_i$ w.r.t. the CEFAs $\NFA_1, \cdots, \NFA_n$. From Theorem~\ref{thm-incra-la-sat}, we conclude that the path feasibility of  {\slint} is decidable.
\qed
\end{proof}

\begin{corollary}
Path feasibility of {\cslint} is decidable.
\end{corollary}
}

%%%%%%%%%%%%%%%%%%%%%%%%%%%%%%%%%%%%%%%%%%%%%%%%%%%%%%%%

%\section{Implementation}
%
%\section{Implementation}

%%%%%%%%%%%%%%%%%%%%%%%%%%%%%%%%%%%%%%%%%%%%%%%%%%%%%%%%

\section{Evaluations}

%!TEX root = main.tex

We have compared {\ostrich}+ with the state-of-the-art solvers on a wide range of benchmarks.  
We will discuss the benchmarks in Section~\ref{sec:bench}, then describe the details of the experiments as well as the experimental results in Section~\ref{sec:exp-res}.
%and SLENT \cite{WC+18}. 

 %
%OSTRICH implements the optimised
%decision procedure for string functions as described in Section 5.1 (i.e. using distributivity of regular
%constraints across pre-images of functions) and has built-in support for concatenation, reverse, FFT,
%and replaceAll. Moreover, since the optimisation only requires that string operations are functional,
%we can also support additional functions that satisfy RegInvRel, such as replacee which replaces
%only the first (leftmost and longest) match of a regular expression. OSTRICH is extensible and new
%string functions can be added quite simply (Section 6.3).
%Our implementation adds a new theory solver for conjunctive formulas representing path
%feasibility problems to Princess (Section 6.1). This means that we support disjunction as well as
%conjunction in formulas, as long as every conjunction of literals fed to the theory solver corresponds
%to a path feasibility problem. OSTRICH also implements a number of optimisations to efficiently
%compute pre-images of relevant functions (Section 6.2). OSTRICH is entirely written in Scala and is
%open-source. We report on our experiments with OSTRICH in Section 6.4. The tool is available on
%GitHub6.  


\subsection{Benchmarks}\label{sec:bench}
 
%In
%particular, we compared  with CVC4 1.6 \cite{}, SLOTH \cite{},
%and Z3 configured to use the Z3-str3 string solver \cite{}. 

%We considered several sets
%of benchmarks which are described in the next sub-section. The results are given in Section 6.4.2.
%
%
%In [Holík et al. 2018] SLOTH was compared with S3P [Trinh et al. 2016] where inconsistent
%behaviour was reported. We contacted the S3P authors who report that the current code is unsupported;
%moreover, S3P is being integrated with Z3. Hence, we do not compare with this tool.
%
%
%The first set of benchmarks we call \textbf{Transducer}. It combines the benchmark
%sets of Stranger [Yu et al. 2010] and the mutation XSS benchmarks of [Lin and Barceló 2016]. The
%first (sub-)set appeared in [Holík et al. 2018] and contains instances manually derived from PHP
%programs taken from the website of Stranger [Yu et al. 2010]. It contains 10 formulae (5 sat, 5
%unsat) each testing for the absence of the vulnerability pattern .*<script.* in the program output.
%These formulae contain between 7 and 42 variables, with an average of 21. The number of atomic
%constraints ranges between 7 and 38 and averages 18. These examples use disjunction, conjunction,
%regular constraints, and concatenation, replaceAll. They also contain several one-way functional
%transducers (defined in SMTLIB in [Holík et al. 2018]) encoding functions such as addslashes and
%trim used by the programs. Note that transducers have been known for some time to be a good
%framework for specifying sanitisers and browser transductions (e.g., see the works by Minamide,
%Veanes, Saxena, and others [D’Antoni and Veanes 2013; Hooimeijer et al. 2011; Minamide 2005;
%Weinberger et al. 2011]), and a library of transducer specifications for such functions is available
%(e.g. see the language BEK [Hooimeijer et al. 2011]).
%
%The second (sub-)set was used by [Holík et al. 2018; Lin and Barceló 2016] and consists of 8
%formulae taken from [Kern 2014; Lin and Barceló 2016]. These examples explore mutation XSS
%vulnerabilities in JavaScript programs. They contain between 9 and 12 variables, averaging 9.75, and
%9-13 atomic constraints, with an average of 10.5. They use conjunctions, regular constraints, concatenation,
%replaceAll, and transducers providing functions such as htmlEscape and escapeString.
%Our next set of benchmarks, SLOG, came from the SLOG tool [Wang et al. 2016] and consist of
%3,392 instances. They are derived from the security analysis of real web applications and contain 1-
%211 string variables (average 6.5) and 1-182 atomic formula (average 5.8).We split these benchmarks
%into two sets SLOG (replace) and SLOG (replaceall). Each use conjunction, disjunction, regular
%constraints, and concatenation. The set SLOG (replace) contains 3,391 instances and uses replace.
%SLOG (replaceall) contains 120 instances using the replaceAll operation.
%
%
%Our next set of benchmarks Kaluza is the well-known set of Kaluza benchmarks [Saxena et al.
%2010] restricted to those instances which satisfy our semantic conditions (roughly 80\% of the
%benchmarks). Kaluza contains concatenation, regular constraints, and length constraints, most of
%which admit regular monadic decomposition. There are 37,090 such benchmarks (28 032 sat).
%We also considered the benchmark set of [Chen 2018a,b]. This contains 42 hand-crafted benchmarks
%using regular constraints, concatenation, and replaceAll with variables in both argument
%positions. The benchmarks contain 3-7 string variables and 3-9 atomic constraints.
% 

Our evaluation focuses on problems that combine string with integer
constraints.  For this, we consider the following four sets of
benchmarks, all in SMT-LIB~2 format.

%\paragraph*{
\smallskip
\noindent \transducerbench+
is derived from the {\transducerbench} benchmark suite of {\ostrich}
\cite{CHL+19}.  The {\transducerbench} suite involves seven
transducers: toUpper (replacing all lowercase letters with their
uppercase ones) and its dual toLower, htmlEscape \cite{htmlEscape} and
its dual htmlUnescape, escapeString \cite{escapeString}, addslashes
\cite{addslashes}, and trim \cite{trim}. These transducers are
collected from Stranger \cite{YABI14} and SLOTH
\cite{HJLRV18}. Initially none of the benchmarks involved integers. In
{\transducerbench+}, we encode four security-relevant properties of
transducers~\cite{BEK}, with the help of the functions~$\charat$ and
$\length$:
\begin{itemize}
\item idempotence: given $\NFT$, whether
  $\forall x.\ \NFT(\NFT(x)) = \NFT(x)$;
\item duality: given $\NFT_1$ and
  $\NFT_2$, whether $\forall x.\ \NFT_2(\NFT_1(x)) = x$;
\item commutativity: given $\NFT_1$ and $\NFT_2$, whether
  $\forall x.\ \NFT_2(\NFT_1(x)) = \NFT_1(\NFT_2(x))$;
\item equivalence: given $\NFT_1$ and $\NFT_2$, whether
  $\forall x.\ \NFT_1(x) = \NFT_2(x)$.
\end{itemize}

%
%\footnote{These problems have been investigated for transducers in \cite{BEK}.} 
%into the path feasibility of {\slint} programs. 
For instance, we encode non-idempotence of $\NFT$ into the path feasibility of the {\slint} program $y:=\NFT(x); z:=\NFT(y); S_{y \neq z}$, where $y$ and $z$ are two fresh string variables, and $S_{y \neq z}$ is the {\slint} program encoding $y \neq z$, which uses $\length$ and $\charat$ (see Section~\ref{sec:logic} for the details of $S_{y \neq z}$). We also include in {\transducerbench+} three {\slint} programs generated from the {\tt urlXssSanitise} function in Section~\ref{sec:intro}, %Example~\ref{exmp:running}, 
where the transducer $\NFT_{\rm trim}$ is used. 
%In total, we obtain 94 instances for the {\transducerbench+} benchmark suite. 
%\begin{description}
%\item[Idempotence.] For each FFT $\NFT$,  we encode the non-idempotence of $\NFT$, namely deciding whether $\exists x.\ \NFT(\NFT(x)) \neq \NFT(x)$, into the path feasibility of the {\slint} program $y:=\NFT(x); z:=\NFT(y); S_{y \neq z}$, where $y$ and $z$ are two fresh string variables, and $S_{y \neq z}$ is the {\slint} program encoding $y \neq z$, which uses $\length$ and $\charat$ (see Section~\ref{sec:logic} for the details of $S_{y \neq z}$).
%
%\item[Duality.] For each pair of distinct FFTs $\NFT_1$ and $\NFT_2$, we encode the non-duality of $\NFT_1$ and $\NFT_2$, namely deciding whether $\exists x.\ \NFT_2(\NFT_1(x)) \neq x$, into the path feasibility of the {\slint} program $y:=\NFT_1(x); z: = \NFT_2(y); S_{x \neq z}$.
%
%\item[Commutativity.] For each pair of distinct FFTs $\NFT_1$ and $\NFT_2$, we encode the non-commutativity of $\NFT_1$ and $\NFT_2$, namely deciding whether $\exists x.\ \NFT_2(\NFT_1(x)) \neq \NFT_1(\NFT_2(x))$, into the path feasibility of the {\slint} program $y:=\NFT_1(x); z: = \NFT_2(y); y':= \NFT_2(x); z': = \NFT_1(y'); S_{z \neq z'}$.
%
%\item[Equivalence.] For each pair of distinct FFTs $\NFT_1$ and $\NFT_2$, we encode the non-equivalence of $\NFT_1$ and $\NFT_2$, namely deciding whether $\exists x.\ \NFT_1(x) \neq \NFT_2(x)$, into the path feasibility of the {\slint} program $y:=\NFT_1(x); z: = \NFT_2(x); S_{y \neq z}$.
%
%\end{description}

\smallskip
\noindent{\slogbench} is adapted from the SLOG benchmark suite%used by the SLOG tool
~\cite{fang-yu-circuits}, containing 3,511~instances about strings only.
%The SLOG are derived from the security analysis of real web applications and contain 1-211 string variables (average 6.5) and 1-182 atomic formula (average 5.8).
%Note that the SLOG benchmark suite addresses strings only. 
We  obtain  {\slogbench}  by choosing a string variable $x$ for each instance,
%SLOG benchmark instance ,
%\footnote{A string variable is called the output variable if it occurs in the left-hand side of an assignment statement, but does not appear in the right-hand sides of the assignment statements.} 
%say $x$, 
and adding the statement $\ASSERT{\length(x) < 2 \wedge \indexof_{a}(x, 0)}$ for some $a \in \Sigma$.
% and $c \in \Nat$ with $c > 1$.
As in \cite{CHL+19}, we  split  {\slogbench}  into  \slogbenchr\ and \slogbenchra,  comprising 3,391 and 120 instances respectively. In addition to %the aforementioned 
the $\indexof$ and $\length$ functions, the benchmarks use regular constraints and concatenation;  {\slogbenchr} also contains the $\replace$ function (replacing the first occurrence), while {\slogbenchra}  uses the $\replaceall$ function (replacing all occurrences).

%Each use conjunction, disjunction, regular constraints, and concatenation.
%The \slogbenchr\ sub-suite contains 3,391 instances \zhilin{the number to be checked}, while \slogbenchra\ sub-suite contains 120 instances \zhilin{the number to be checked}.

\smallskip
\noindent \pyexbench~\cite{ReynoldsWBBLT17} 
contains 25,421 instances  derived by the PyEx tool, a symbolic execution engine for Python programs. The {\pyexbench} suite was generated by the CVC4 group from four popular Python packages: httplib2, pip, pymongo, and requests. These instances use regular constraints, concatenation, $\length$, $\substring$, and $\indexof$ functions. Following \cite{ReynoldsWBBLT17}, the {\pyexbench} suite is further divided into three parts: {\pyextdbench},  {\pyexztbench} and {\pyexzzbench}, comprising 5,569, 8,414 and 11438  instances, respectively. 

\smallskip
\noindent\kaluzabench~\cite{Berkeley-JavaScript}
%
is the most well-known  benchmark suite in  literature, 
containing 47,284 instances with regular constraints, concatenation, and the $\length$ function. The 47,284 benchmarks include 28,032 satisfiable and 9,058 unsatisfiable problems in SSA form.


\definecolor{Gray}{gray}{0.9}
%\newcolumntype{g}{>{\columncolor{Gray}}c}
\begin{table}[tbp]
\begin{center}
\begin{tabular}{|c|c|c|c|c|c|}
\hline
Benchmark & Output &  \cvc & \zthree &  \zthreetrau & \ostrich+ \\
\hline
\hline
\multirow{3}{*}{\transducerbench+(94)} & \cellcolor{Gray} sat &  \cellcolor{Gray}$-$ & \cellcolor{Gray}$-$ & \cellcolor{Gray}$-$ & \cellcolor{Gray}\bf{84}\\
\cline{2-6}
 & unsat &$-$  &$-$ &$-$ &\bf{4}\\
\cline{2-6}
 & \cellcolor{Gray}  inconclusive  &\cellcolor{Gray}$-$    &\cellcolor{Gray}$-$  &\cellcolor{Gray}$-$  &\cellcolor{Gray}6\\
%\cline{2-6} 
% & wrong &$-$  & $-$  &$-$ &0 \\
\hline
\hline
\multirow{3}{*}{\slogbenchra(120)} & \cellcolor{Gray} sat &  \cellcolor{Gray}\bf{104}  & \cellcolor{Gray}$-$ & \cellcolor{Gray}$-$  &98 \cellcolor{Gray}\\
\cline{2-6}
 & unsat &11  &$-$  &$-$ &\bf{12}\\
\cline{2-6}
 &\cellcolor{Gray} inconclusive & \cellcolor{Gray}5  &\cellcolor{Gray}$-$ &\cellcolor{Gray}$-$ &\cellcolor{Gray}10\\
%\cline{2-6}
% & wrong &0 &$-$ &$-$  &0 \\
\hline
\hline
\multirow{3}{*}{\slogbenchr(3,391)} & \cellcolor{Gray} sat &  \cellcolor{Gray}\bf{1,309} & \cellcolor{Gray}878 & \cellcolor{Gray}$-$ & \cellcolor{Gray}584 \\
\cline{2-6}
 & unsat & \bf{2,082} & 2,066  &$-$ &\bf{2,082}\\
\cline{2-6}
 &\cellcolor{Gray}  inconclusive & \cellcolor{Gray}0  &  \cellcolor{Gray}447   &  \cellcolor{Gray}$-$ &\cellcolor{Gray}725\\
%\cline{2-6}
% & wrong &0 & 0  &$-$ &0\\
\hline
\hline
\multirow{3}{*}{\pyextdbench(5,569)} & \cellcolor{Gray} sat & \cellcolor{Gray}4,224 & \cellcolor{Gray}4,068 &  \cellcolor{Gray} \bf{4,266} & \cellcolor{Gray}4,141\\
\cline{2-6}
 & unsat & 1,284 & 1,289 & \bf{1,295} &1,203\\
\cline{2-6}
 &\cellcolor{Gray} inconclusive &\cellcolor{Gray}61 &\cellcolor{Gray}212   &\cellcolor{Gray}8 &\cellcolor{Gray}225\\
%\cline{2-6}
% &wrong &0 & 0 &  0 &0 \\
\hline
\hline
\multirow{3}{*}{\pyexztbench(8,414)} & \cellcolor{Gray} sat & \cellcolor{Gray}6,346 & \cellcolor{Gray}6,040 & \cellcolor{Gray}\bf{7,003} & \cellcolor{Gray}5,489\\
\cline{2-6}
 & unsat & 1,358  & 1,370  &\bf{1,394} &1,239\\
\cline{2-6}
 & \cellcolor{Gray}inconclusive &\cellcolor{Gray}710 &\cellcolor{Gray}1,004 &\cellcolor{Gray} 17 &\cellcolor{Gray}1,686\\
%\cline{2-6}
% & wrong & 0 &0 & 0 &0 \\
\hline
\hline
\multirow{3}{*}{\pyexzzbench(11,438)} & \cellcolor{Gray} sat & \cellcolor{Gray} 10,078 & \cellcolor{Gray} 8,804 & \cellcolor{Gray} \bf{10,129} & \cellcolor{Gray}9,033\\
\cline{2-6}
 & unsat & 1,204 & 1,207  &   \bf{1,222} &868\\
\cline{2-6}
 &\cellcolor{Gray}  inconclusive &\cellcolor{Gray}156 & \cellcolor{Gray}1,427  &  \cellcolor{Gray} 87 &\cellcolor{Gray}1,537 \\
%\cline{2-6}
% & wrong &  0 & 0 & 0&0 \\
\hline
\hline
\multirow{3}{*}{\kaluzabench (47,284)} & \cellcolor{Gray} sat &  \cellcolor{Gray} \bf{35,264} & \cellcolor{Gray} 33,438 & \cellcolor{Gray} 34,769 & \cellcolor{Gray}27,962\\
\cline{2-6}
 & unsat & \bf{12,014} &  11,799  &\bf{12,014}  &9,058\\
\cline{2-6}
 &\cellcolor{Gray} inconclusive &\cellcolor{Gray}6 & \cellcolor{Gray}2,047  &\cellcolor{Gray}501 &\cellcolor{Gray}10,264 \\
%\cline{2-6}
% & wrong &  0 & 0 &0 &0 \\
\hline 
\hline
\multirow{2}{*}{Total(76,310)} & \cellcolor{Gray} solved & \cellcolor{Gray}\bf{75,278}  & \cellcolor{Gray}70,959 & \cellcolor{Gray}72,092 & \cellcolor{Gray}61,857\\
\cline{2-6}
 &  unsolved &1,032  & 5,351  & 4,218 &14,453  \\
\hline
\end{tabular}
\end{center}
\caption{Experimental results on different benchmark suites.  '--' means that the tool is not applicable to the benchmark suite, and 'inconclusive' means that a tool gave up, timed out, or crashed.}
\label{tab-experiment}
\end{table}%


\subsection{Experiments}\label{sec:exp-res}

We compare {\ostrich}+ to {\cvc}~\cite{cvc4}, {\zthree}~\cite{Z3-str}, and {\zthreetrau} \cite{Z3-trau}. 
The experiments are executed on an Intel Xeon Silver 4210 2.20GHz and 2.19GHz CPU (2-core) and 8GB main memory, running 64bit Ubuntu 18.04 LTS OS and Java 1.8. We use a timeout of 30~seconds (wall-clock time), and report the number of satisfiable and unsatisfiable problems solved by each of the systems.
Table~\ref{tab-experiment} summarises the experimental results. We did not observe incorrect answers by any tool.
% (note that the ground-truth are known and have been verified \tl{check this}  

There are two additional state-of-the-art solvers  {\slent} and {\trauplus} which were not included in
the evaluation. We exclude
{\slent}~\cite{DBLP:conf/kbse/WangCYJ18} because it uses its own input
format laut, which is different from the SMT-LIB~2 format used for our
benchmarks; also, {\transducerbench+} is beyond the scope of {\slent}.
%
{\trauplus}~\cite{AbdullaA+19}  integrates {\trau} with {\sloth} to deal with both finite transducers and integer constraints. We were unfortunately unable
to obtain a working version of {\trauplus} by the deadline, possibly because {\trau} requires two separate versions of Z3 to run. In addition, the algorithm~\cite{AbdullaA+19} focuses on length-preserving transducers, which means that {\transducerbench}+ is beyond the scope of \trauplus.

%  We also believe that {\transducerbench}+ is actually beyond the scope of {\trauplus} since   {\trau} and {\sloth} are rather separated in in {\trauplus}, namely, {\sloth} is used to deal with finite transducers and {\trau} is used to handle integer constraints. For instances from {\transducerbench}+, the two types of constraints are tightly coupled. The approach adopted by {\trauplus} cannot handle these constraints. %, while {\sloth} is unable to solve integer constraints and {\trau}  is incapable of dealing with finite transducers.


%Therefore, at the first sight, we had better compare {\ostrich}+ with {\trau+} on the {\transducerbench}+ benchmark suite. Nevertheless, 
%We failed to run {\trau+}  possibly due to the fact that {\trau} requires two versions of Z3 to run. Moreover, although {\trauplus} integrates {\trau} with {\sloth},  %in order to deal with both finite transducers (including the $\replaceall$ function) and integer constraints. 
%%,  though we were able to compile the source code of \trauplus.  
%After some deeper analysis, we realised that {\transducerbench}+ benchmark suite is actually beyond the scope of {\trauplus} since {\trauplus} integrates {\trau} with {\sloth} in a shallow and largely separated way in the sense that it uses {\sloth} to deal with finite transducers and {\trau} to handle integer constraints, while {\sloth} is unable to solve integer constraints and {\trau}  is incapable of dealing with finite transducers. % since it is incapable of dealing with finite transducers.
%We do not compare with {\ostrich} since {\ostrich} basically focuses on pure string constraints not involving the integer data type, while the majority of the benchmarks considered in this paper contain integer arithmetic constraints. 
%In the beginning, we also planned to compare {\ostrich}+ with two other solvers, namely {\trauplus} \cite{AbdullaA+19} and {\slent} \cite{WC+18}, but finally gave up, with some justifications of this decision explained below. 
%\begin{itemize}
%\item {\trauplus} integrates {\trau} with {\sloth} in order to deal with both finite transducers (including the $\replaceall$ function) and integer constraints. Therefore, at the first sight, we had better compare {\ostrich}+ with {\trau+} on the {\transducerbench}+ benchmark suite. Nevertheless, we failed to run {\trau+} (even after some hard work), possibly due to the fact that {\trau} requires two versions of Z3 to run.
%%,  though we were able to compile the source code of \trauplus.  
%After some deeper analysis, we realised that {\transducerbench}+ benchmark suite is actually beyond the scope of {\trauplus} since {\trauplus} integrates {\trau} with {\sloth} in a shallow and largely separated way in the sense that it uses {\sloth} to deal with finite transducers and {\trau} to handle integer constraints, while {\sloth} is unable to solve integer constraints and {\trau}  is incapable of dealing with finite transducers.
%%
%\item {\slent} uses an input format called laut which is different from the widely used smtlib2 format. Therefore, it is hard to compare {\ostrich}+ with {\slent} on {\kaluzabench} and {\pyexbench} benchmark suites. Moreover, {\transducerbench}+ is beyond the scope of {\slent} since it is incapable of dealing with finite transducers.
%\end{itemize}


{\ostrich}+ is the only tool applicable to the problems in
{\transducerbench}+. With a timeout of 30s, \ostrich+ can solve 88 of
the benchmarks, but this number rises to 94 when using a longer
timeout of 600s. Given the complexity of those benchmarks, this is an
encouraging result.

On {\slogbenchra}, {\ostrich}+ and {\cvc} are very close: {\ostrich}+ solves 98 satisfiable instances, slightly less than the 104 instances solved by {\cvc}, while {\ostrich}+ solves one more unsatisfiable instance than {\cvc} (12 versus 11). The suite is beyond the scope of {\zthree} and {\zthreetrau}, which do not support $\replaceall$.

On {\slogbenchr}, {\ostrich}+, {\cvc}, and {\zthree} solve a similar
number of unsatisfiable problems, while {\cvc} solves the largest
number of satisfiable instances (1,309). The  suite %{\slogbenchr} benchmark
is beyond the scope of {\zthreetrau} which does not support
$\replace$.

On the three \pyexbench\ suites, {\zthreetrau} consistently solves the
largest number of problems, by some margin. \ostrich+ solves a similar
number of problems as \zthree. Interpreting the results, however, it
has to be taken into account that \pyexbench\ also includes problems
that are not in SSA form, and therefore beyond the scope of
\ostrich+. \philipp{how many?}

\iffalse
For {\pyextdbench}, each of the four solvers is able to solve more than 95\% of the benchmark instances and their performances are close, with {\zthreetrau} the best, followed by {\cvc}, then {\zthree}, and finally {\ostrich}+.  

For {\pyexztbench}, {\zthreetrau} is the best, solving almost all instances. %, except 17 of them. 
{\cvc} and {\zthree} are ranked in the second and third place. The performance of {\ostrich}+ on this benchmark suite is not very impressive, only solving around 80\% instances.

For {\pyexzzbench}, %\zhilin{to be done}
{\zthreetrau} again is the best, which is followed closely by {\cvc}. %solving almost all instances. %, except 17 of them. 
%{\cvc} and {\zthree} are ranked in the second and third place. 
The performance of {\ostrich}+ and {\zthree} are level, but are 10\% worse than  {\zthreetrau} and {\cvc}. %on this benchmark suite is not very impressive, only solving around 80\% instances.
\fi

The {\kaluzabench} problems can be solved most effectively by {\cvc}.
\ostrich+ can solve almost all of the 28,032 satisfiable
problems in SSA form, and all 9,058 unsatisfiable problems in SSA
form. This is consistent with the results in  \cite{CHL+19} for
{\ostrich}.

\iffalse
{\zthreetrau} solves slightly less instances, but solves the same number of unsatisfiable instances as {\cvc}. {\zthree} is the third on the {\kaluzabench} suite. 
%The performance of 
{\ostrich}+ %is not that satisfactory on this suite and it  
solves  around 80\%  instances. The main reason is that about 20\% {\kaluzabench} instances cannot be written as programs in the SSA form, thus cannot be handled by%out of the scope of 
{\ostrich} or {\ostrich}+. This  is consistent with the observation for {\ostrich} \cite{CHL+19}.
\fi

In summary, we can observe that \ostrich+ is competitive with other
solvers, while it is also able to handle benchmarks that are beyond the
scope of the other tools due to the combination of string operations (in particular transducer applications) and integer
constraints. Interestingly, the experiments show that \ostrich+, at
least in its current state, is better at solving unsatisfiable
problems than satisfiable problems; this might be an artefact of the
use of nuXmv for analysing products of CEFAs. We expect that further
optimisation of our algorithm will lead to additional performance
improvements. For instance, a natural optimisation that is not yet
included in our implementation is to use standard finite automata like in \ostrich, as opposed to CEFAs, for simpler problems such as the \kaluzabench\ benchmarks.

\iffalse
%Last but not least, we would like to summarise and highlight some facts and observations about {\ostrich}+  according to the experimental results in Table~\ref{tab-experiment}.
In summary, 
%\begin{itemize}
%\item 
{\ostrich}+ is the only string solver that is capable of solving string constraints involving both finite transducers and integer constraints. In particular, it is the only string solver that is able to verify idempotence, duality, commutativity, and equivalence of HTML sanitisation operations.
%
{\ostrich}+ is generally better at solving unsatisfiable benchmark instances. It solves the most instances on both {\slogbenchra} and {\slogbenchr} benchmark suites. Moreover, the number of the unsatisfiable instances of {\pyexbench} resp. {\kaluzabench} solved by {\ostrich}+ is close to that of the best solver. %, compared to the number of solved satisfiable instances.   \zhilin{to be double checked later}

%\end{itemize}
Admittedly, {\ostrich}+ has not exceeded other mature solvers when dealing with relatively simple string operations. However, considering that it implemented a ()complete) decision procedure (while others rely on heuristics) and its strength in dealing with complex string operations and integer constraints, we may conclude that {\ostrich}+ is competitive, and has much potential to improve when heuristics are introduced to deal with  satisfiable  instances more efficiently. 
\fi

%From the experimental results, we can see that \ostrich+ is the only solver that 





%%%%%%%%%%%%%%%%%%%%%%%%%%%%%%%%%%%%%%%%%%%%%%%%%%%%%%%%%
\section{Conclusion}

In this paper, we have proposed  an expressive string constraint language which can specify constraints on both strings and integers.  We provided an automata-theoretic decision procedure for the path feasibility problem of this language. The decision procedure is simple, generic, and amenable to implementation, giving rise to a new solver OSTRICH+.  We have evaluated OSTRICH+ on  a wide range of existing and newly created benchmarks, and have obtained very encouraging results. 
OSTRICH+ is shown to be the first solver  which is capable of tackling finite transducers and integer constraints with completeness guarantees. Meanwhile, it demonstrates competitive performance against some of the best state-of-the-art string constraint solvers.


%%%%%%%%%%%%%%%%%%%%%%%%%%%%%%%%%%%%%%%%%%%%%%%%%%%%%%%%%%%%%%%%%%%%%%%%%%%%%%%

\newpage 
\bibliographystyle{abbrv}
\bibliography{string}

%\iffalse
\newpage
%\setcounter{page}{1}

\begin{appendix}
%!TEX root = main.tex


\section{Construction of $\CEFA_{\indexof_v}$} \label{appendix:cefa_indexof}

In this section, we show that the function $\indexof_v(\cdot, \cdot)$ can be captured by CEFA. Technically, for any NFA $\NFA$ and constant string $v$, we can construct a CEFA %$\CEFA'$ such that 
accepting $\{(w, (n, \indexof_v(w, n)))\mid w\in \Lang(\NFA) \}$. %$R(\CEFA')=(r_1,r_2)$ and $\Lang(\CEFA')=\{(w, (n, \indexof_v(w, n)))\mid w\in \Lang(\NFA) \}$. 

%The construction is slightly technical and can be found in Appendix, Section~\ref{appendix:cefa_indexof}.
For this purpose, we need a concept of window profiles of  string positions w.r.t. $v$, which are elements of $\{\bot, \top\}^{n-1}$. The window profiles facilitate recognising the first occurrence of $v$ in the input string. 
Intuitively, given a string $u$, the window profile of a position $i$ in $u$ w.r.t. $v$ encodes the matchings of prefixes of $v$ to the suffixes of $u[0,i]$ (see \cite{CCH+18} for the details). For $\pi = \pi_1 \cdots \pi_{n-1} \in \{\bot, \top\}^{n-1}$ and $b \in \Sigma$, we use $\uwp(\vec{\pi}, b)$ to represent the window profile updated from $\pi$ after reading the letter $b$, specifically, $\uwp(\vec{\pi}, b) = \vec{\pi'}$ such that  
\begin{itemize}
\item $\pi'_1 = \top$ iff $b = a_1$, 
%
\item for each $i \in [n-2]$, $\pi'_{i+1} = \top$ iff $\pi_{i} = \top$ and $b = a_{i+1}$. 
\end{itemize}
Let $WP_v$ denote the set of window profiles of string positions w.r.t. $v$. From the result in \cite{CCH+18}, we know that $|WP_v| \le |v|$. 
%Then the set of window profiles of $v$, denoted by $WP_v$, is computed by setting $WP_0 := \{\bot^{n-1}\}$ and iterating the following procedure, until $WP_i = WP_{i+1}$:
%\[WP_{i+1}:=WP_i \cup \{\uwp(\vec{\pi}, b) \mid \vec{\pi} \in WP_i, b \in \Sigma\}.\] 
%Therefore, the aforementioned iteration terminates in at most $|v|$ steps.\\
%
%

Suppose $v = a_1 \cdots a_n$ with $n \ge 2$. 
Then $\indexof_v$ is captured by the CEFA $\CEFA_{\indexof_v}=(Q, \Sigma, R, \delta, I, F)$, such that 
\begin{itemize}
\item $Q = \{q_0, q_1\} \cup WP_v \cup WP_v \times [n]$, 
\item $R=(r_1, r_2)$ (where $r_1,r_2$ represent the input and output positions of $\indexof_v$ respectively), 
\item $I=\{q_0\}$, 
\item $F=\{q_1\}$, and 
\item $\delta$ comprises 
\begin{itemize}
\item the tuples $(q_0, a, q_0, \eta)$ such that $a \in \Sigma$, $\eta(r_1)=1$, and $\eta(r_2) = 1$,
%
\item the tuples $(q_0, a, \vec{\pi}, \eta)$ such that $a \in \Sigma$, $\vec{\pi} = \theta \bot^{n-2}$ where $\theta  = \top$ iff $a = a_1$, $\eta(r_1) = 0$, and $\eta(r_2)= 0$ (recall that the first position of a string is $0$),
% 
\item the tuples  $(\vec{\pi}, a, \uwp(\vec{\pi}, a), \eta)$ such that $\vec{\pi} \in WP_u$, $a \in \Sigma$, $\pi_{n-1} = \bot$ or $a \neq a_{n}$, $\eta(r_1) = 0$, and $\eta(r_2)= 1$,
%
\item the tuples $(\vec{\pi}, a, (\uwp(\vec{\pi}, a), 1), \eta)$ such that $\vec{\pi} \in WP_u$, $a = a_1$, $\pi_{n-1} = \bot$ or $a \neq a_{n}$, $\eta(r_1) = 0$, and $\eta(r_2)= 1$,
%
\item the tuples $((\vec{\pi}, i),  a, (\uwp(\vec{\pi}, a), i+1), \eta)$ such that $\vec{\pi} \in WP_u$, $i \in [n-2]$, $a = a_{i+1}$, $\pi_{n-1} = \bot$ or $a \neq a_{n}$, $\eta(r_1) = 0$, and $\eta(r_2)= 0$,
%
\item the tuples $((\vec{\pi}, n-1),  a, q_1, \eta)$ such that $\vec{\pi} \in WP_u$, $a = a_{n}$, $\eta(r_1) =0$, and $\eta(r_2)= 0$,
%
\item the tuples  $(q_1, a, q_1, \eta)$ such that $a \in \Sigma$, $\eta(r_1) = 0$, and $\eta(r_2)= 0$.
\end{itemize}
\end{itemize}

%%%%%%%%%%%%%%%%%%%%%%%%%%%%%%%%%%%%%%%%%%%%%%%%%%%%%%%%%%%%%%%%%%%%%%%%%%%%%%


\section{Proof of Proposition~\ref{prop:pre-image}}\label{app:pre-image}


\noindent {\bf Proposition~\ref{prop:pre-image}}.
\emph{Let $L$ be a CERL defined by a CEFA $\CEFA = (Q, \Sigma, R, \delta, I, F)$. Then for each string function $f$ ranging over $\concat$, $\replaceall_{e,u}$, $\reverse$, FFTs $\NFT$, and $\substring$, $f^{-1}_R(L)$ is CERR-definable. In addition,
\begin{itemize}
\item a CEFA representation of $\concat^{-1}_R(L)$ can be computed in time $\bigO(|\CEFA|^2)$, 
%
\item a CEFA representation of $\reverse^{-1}_R(L)$ (resp. $\substring^{-1}_R(L)$) can be computed in time $\bigO(|\CEFA|)$,
%
\item a CEFA representation of  $(\Tran(\NFT))^{-1}_R(L)$ can be computed in time polynomial in $|\CEFA|$ and exponential in $|\NFT|$,
%
\item a CEFA representation of  $(\replaceall_{e,u})^{-1}_R(L)$ can be computed in time polynomial in $|\CEFA|$ and exponential in $|e|$ and $|u|$.
\end{itemize}
}

\begin{proof}
	Let $\CEFA=(Q, \Sigma, R, \delta, I, F)$ be a CEFA with $R= (r_1, \cdots, r_k)$. We show how to construct a CEFA representation of $f^{-1}_R(L)$ for each function $f$ in {\slint}.
	
	%%%%%%%%%%%%%%%%%%%%%%%%%%%%%%%%%%%%%%%%%%%%%%%%%%%%%%%%%%%%%%%%%%%%%%%%%%%%%%
	\paragraph*{$\concat^{-1}_R(L)$.}
	%
	A CEFA representation of $\concat^{-1}_R(L)$ is given by $((\CEFA_{I, q}, \NFA_{q, F})_{q \in Q}, \vec{t})$, where 
	\begin{itemize}
		\item $\CEFA_{I, q}=(Q, \Sigma, R^{(1)}, \delta^{(1)}, I, \{q\})$ and  $\CEFA_{q, F}=(Q, \Sigma, R^{(2)}, \delta^{(2)}, \{q\}, F)$ such that 
		\begin{itemize}
			\item $R^{(1)} = (r^{(1)}_1, \cdots, r^{(1)}_k)$, $R^{(2)} = (r^{(2)}_1, \cdots, r^{(2)}_k)$, 
			\item $\delta^{(1)}$ comprises the tuples $(q, a, q', \eta')$ satisfying that there exists $\eta$ such that $(q, a, q', \eta) \in \delta$ and for each $j \in [k]$, and $\eta'(r^{(1)}_j)=\eta(r_j)$,  similarly for $\delta^{(2)}$,
		\end{itemize}
		\item and $\vec{t} = (r^{(1)}_1 + r^{(2)}_1, \cdots, r^{(1)}_k + r^{(2)}_k)$.
	\end{itemize}
	Note that the size of $((\CEFA_{I, q}, \NFA_{q, F})_{q \in Q}, \vec{t})$ is $\bigO(|\CEFA|^2)$.
	%
	%%%%%%%%%%%%%%%%%%%%%%%%%%%%%%%%%%%%%%%%%%%%%%%%%%%%%%%%%%%%%%%%%%%%%%%%%%%%%%
	%
	\paragraph*{$\reverse^{-1}_R(L)$.} 
	%
	A CEFA representation of $\reverse^{-1}_R(L)$ is given by $(\CEFA^{(r)}, \vec{t})$, where 
	\begin{itemize}
		\item $\CEFA^{(r)}=(Q, \Sigma, R^{(1)}, \delta', F, I)$ such that 
		\begin{itemize}
			\item $R^{(1)}=(r^{(1)}_1,\cdots,r^{(1)}_k)$, and 
			\item $\delta'$ comprises the tuples $(q', a, q, \eta')$ satisfying that there exists $\eta$ such that $(q, a, q', \eta) \in \delta$, and $\eta'(r^{(1)}_i) = \eta(r_i)$ for each $i \in [k]$,
		\end{itemize}
		%
		\item and $\vec{t}=(r^{(1)}_1, \cdots, r^{(1)}_k)$. 
	\end{itemize}
	Note that $\Lang(\CEFA^{(r)}) = \{(w^{(r)}, \vec{n}) \mid (w, \vec{n}) \in \Lang(\CEFA)\}$, and the size of $(\CEFA^{(r)}, \vec{t})$ is $\bigO(|\CEFA|)$.
	
	%%%%%%%%%%%%%%%%%%%%%%%%%%%%%%%%%%%%%%%%%%%%%%%%%%%%%%%%%%%%%%%%%%%%%%%%%%%%%%
	\paragraph*{$\substring^{-1}_R(L)$.}
	A CEFA representation of $\substring^{-1}_R(L)$ is given by $(\cB, \vec{t})$, where 
	\begin{itemize}
		\item $\cB = (Q', \Sigma, R', \delta', I', F')$ such that 
		\begin{itemize}
			\item $Q' = Q \times \{p_0, p_1, p_2\}$, (intuitively, $p_0$, $p_1$, and $p_2$ denote that the current position is before the starting position, between the starting position and ending position, and after the ending position respectively)
			%
			\item $R' = \left(r'_{1,1}, r'_{1,2}, r^{(1)}_1,\cdots, r^{(1)}_k \right)$, (intuitively, $r'_{1,1}$ denotes the starting position, and $r'_{1,2}$ denotes the length of the substring)
			%
			\item $I'=I \times \{p_0\}$, $F'=F' \times \{p_2\} \cup (I \cap F) \times \{p_0\}$,
			%
			\item and $\delta'$ comprises 
			\begin{itemize}
				\item the tuples $((q, p_0), a, (q, p_0), \eta')$ such that $q \in I$, $a \in \Sigma$, and $\eta'$ satisfies that $\eta'(r'_{1,1})= 1$, and $\eta'(r'_{1,2}) = 0$, and $\eta'(r^{(1)}_j)=0$ for each $j \in [k]$, 
				%
				\item the tuples $((q, p_0), a, (q', p_1), \eta')$ such that $q \in I$ and there exists $\eta$ satisfying that $(q, a, q', \eta) \in \delta$, moreover, $\eta'(r'_{1,1})=0$ (recall that the positions of strings start at $0$), $\eta'(r'_{1,2}) = 1$, and $\eta'(r^{(1)}_j)=\eta(r_j)$ for each $j \in [k]$,
				%
				\item the tuples $((q, p_0), a, (q', p_2), \eta')$ such that $q \in I$ and there exists $\eta$ satisfying that $(q, a, q', \eta) \in \delta$, moreover, $q' \in F$, and $\eta'(r'_{1,1})=0$ (recall that the positions of strings start at $0$), $\eta'(r'_{1,2}) = 1$, and $\eta'(r^{(1)}_j)=\eta(r_j)$ for each $j \in [k]$,  
				%
				\item the tuples $((q, p_1), a, (q', p_1), \eta')$ such that there exists $\eta$ satisfying that $(q, a, q', \eta) \in \delta$, $\eta'(r'_{1,1}) = 0$, and $\eta'(r'_{1,2}) = 1$, and $\eta'(r^{(1)}_j)=\eta(r_j)$ for each $j \in [k]$,
				%
				\item the tuples $((q, p_1), a, (q', p_2), \eta')$ such that $q' \in F$, and there exists $\eta$ satisfying that $(q, a, q', \eta) \in \delta$, moreover, $\eta'(r'_{1,1}) = 0$, $\eta'(r'_{1,2}) = 1$, and $\eta'(r^{(1)}_j)=\eta(r_j)$ for each $j \in [k]$,
				%
				%\item the tuples $((q, p_1), a, (q, p_2), \eta')$ such that $q \in F$ and $\eta'(r^{(1)}_j)=0$ for each $j \in [k]$, $\eta'(r'_1) = 0$, and $\eta'(r'_2) = 1$,
				%
				\item the tuples $((q, p_2), a, (q, p_2), \eta')$ such that $q \in F$, $\eta'(r'_{1,1}) = 0$, and $\eta'(r'_{1,2}) = 0$, and $\eta'(r^{(1)}_j)=0$ for each $j \in [k]$,
				%
			\end{itemize}
		\end{itemize}
		\item $\vec{t}=(r^{(1)}_1, \cdots, r^{(1)}_k)$.
	\end{itemize}
	Note that the size of $(\cB, \vec{t})$ is $\bigO(|\CEFA|)$.
\zhilin{substring cleaned, the other operations to be cleaned}	
	%%%%%%%%%%%%%%%%%%%%%%%%%%%%%%%%%%%%%%%%%%%%%%%%%%%%%%%%%%%%%%%%%%%%%%%%%%%%%%
	%
	%
	\paragraph*{$(\Tran(\NFT))^{-1}_R(L)$.}
	%
	Suppose $\NFT = (Q', \Sigma, \delta', I', F')$. Then a CEFA representation of $(\Tran(\NFT))^{-1}_R(L)$ is given by 
	$(\cB, \vec{t})$, where 
	\begin{itemize}
		\item $\cB$ simulates the run of $\NFT$ on the input string, meanwhile, it simulates the run of $\CEFA$ on the output string of $\NFT$, formally, $\cB= (Q' \times Q, \Sigma, R^{(1)}, \delta'', I' \times I, F' \times F)$ such that 
		\begin{itemize}
			\item $R^{(1)}  = (r^{(1)}_1, \cdots, r^{(1)}_k)$, and
			\item $\delta''$ comprises the tuples $((q'_1, q_1), a, (q'_2, q_2), \eta')$ satisfying one of the following conditions,
			\begin{itemize}
				\item there exist $u = a_1 \cdots a_n \in \Sigma^+$ and a transition sequence $p_0 \xrightarrow[\delta]{a_1, \eta_1} p_2 \cdots p_{n-1} \xrightarrow[\delta]{a_n, \eta_n} p_{n}$ in $\CEFA$ such that $(q'_1, a, q'_2, u) \in \delta'$, $p_0 = q_1$, $p_{n}= q_2$, and for each $j \in [k]$,  $\eta'(r^{(1)}_j) = \eta_1(r_j) + \cdots + \eta_n(r_j)$,
				%
				\item $(q'_1, a, q'_2, \varepsilon) \in \delta'$, $q_1 = q_2$, and $\eta'(r^{(1)}_j) =0$ for each $j \in [k]$,
			\end{itemize}
		\end{itemize}
		%
		\item $\vec{t}=(r^{(1)}_1, \cdots, r^{(1)}_k)$.
	\end{itemize}
	Note that the number of transitions of $\cB$ can be exponential in the worst case, since it summarises the updates of cost registers of $\CEFA$ on the output strings of the transitions of $\NFT$. More precisely,  let
	\begin{itemize}
		\item $\ell$ be the maximum length of the output strings of transitions of $\NFT$, 
		\item $N$ be the maximum number of transitions between a given pair of states of $\CEFA$, and
		\item  $C$ be the maximum absolute value of the integer constants occurring in $\CEFA$,
	\end{itemize}
	then $|\delta''|$, the cardinality of $\delta''$, is bounded by $|\delta'| \times |Q|^2 \times N^\ell $, and the integer constants occurring in each transition of $\delta''$ are bounded by $\ell C$. Therefore, 
	the size of $(\cB, \vec{t})$ is 
	\[
	\bigO(|\delta'| \times |Q|^2 \times N^\ell \times k \log_2 (\ell C)).
	\] 
	Since $|\delta'|, \ell \le |\NFT|$, $|Q|, N, k \le |\CEFA|$, and $C \le 2^{|\CEFA|}$, we deduce that the size of $(\cB, \vec{t})$ is 
	$
	\bigO( |\NFT| \times  |\CEFA|^2 \times |\CEFA|^{|\NFT|} \times |\CEFA|^2 \log_2(|\NFT|))= |\CEFA|^{\bigO(|\NFT|)} |\NFT| \log_2(|\NFT|).$
	%
	
	%%%%%%%%%%%%%%%%%%%%%%%%%%%%%%%%%%%%%%%%%%%%%%%%%%%%%%%%%%%%%%%%%%%%%%%%%%%%%%
	\paragraph*{$(\replaceall_{e,u})^{-1}_R(L)$.}
	%
	From the result in \cite{CCH+18}, we know that  a NFT $\NFT_{e,u}=(Q', \Sigma, \delta', I', F')$ can be constructed to capture $\replaceall_{e,u}$.  Moreover, 
	\begin{itemize}
		\item $|Q'|$, as well as $|\delta'|$, is $2^{\bigO(|e|)}$,
		\item $\ell$, the maximum length of the output strings of transitions of $\NFT_{e,u}$, is $|u|$.
	\end{itemize}
	Then a CEFA representation of $(\replaceall_{e,u})^{-1}_R(L)$ can be constructed as that of $(\Tran(\NFT_{e,u}))^{-1}_R(L)$.
	Let $N$ denote the maximum number of transitions between a given pair of states of $\CEFA$, and $C$ be the maximum absolute value of the integer constants occurring in $\CEFA$, which is bounded by $2^{|\CEFA|}$. Then the CEFA representation of $(\replaceall_{e,u})^{-1}_R(L)$ is of size 
	\[
	\bigO(|\delta'| \times |Q|^2 \times N^\ell \times k \log_2 (\ell C)) = 2^{\bigO(|e|)} |\CEFA|^2 |\CEFA|^{|u|} |\CEFA|^2 \log_2 |u|=2^{\bigO(|e|)} |\CEFA|^{\bigO(|u|)}.
	\]
	%
	according to the aforementioned discussion for NFTs.
	% 
	%
\end{proof}

%%%%%%%%%%%%%%%%%%%%%%%%%%%%%%%%%%%%%%%%%%%%%%%%%%%%%%%%%%%%%%%%%%%%%%%%%%%%%%%%%%%%%%%%%%%%%%%%%%%%%%%
\section{Proof of Proposition~\ref{prop:la-sat-cefa-inter}}\label{app:sat-cefa}

\noindent{\bf Proposition~\ref{prop:la-sat-cefa-inter}}.
\emph{The {\lasat} problem is $\pspace$-complete.}

\begin{proof}
	The lower bound follows from the {\pspace}-hardness of the intersection problem of NFAs. 
	
	For the upper bound, let $\{ \CEFA_i^{j} \}_{i\in I,j\in J_i}$ be a family of CEFAs  each of which carries a vector of registers $R_i^j$ and  $\phi$ be a quantifier-free LIA formula such that  $ R_i^{j} $ are pairwise disjoint and the variables of $\phi$ are from $R'=\bigcup_{i,j} R_i^j$. 
	%Deciding whether  there are an assignment function $\theta: R' \rightarrow \Int$ and strings $(w_i)_{i \in I}$ such that  $\phi[\theta(R' )/R']$ holds and $(w_i, \theta(R_i^j)) \in \Lang(\CEFA_{i}^j)$ for every $i \in I$ and $j \in J_i$ is $\pspace$-complete. 
	
	At first, we observe that we can focus on \emph{monotonic CEFAs} where the cost registers are monotone in the sense that their values are non-decreasing, in other words, they can only be updated with natural number constants. This observation is justified by the following reduction.
	
	For each register $r \in R^i_j$, introduce two registers $r^+, r^-$. Let $(R^i_j)^{\pm}$ denote the vector of registers by replacing each $r \in R^i_j$ with $(r^+, r^-)$. Intuitively,  for each $r \in R^i_j$, the updates of $r$ in $\CEFA_i^{j} $ are split into the non-negative ones and negative ones, with the former stored in $r^+$ and the latter in $r^-$. Suppose $(R')^{\pm} = \bigcup_{i,j} (R_i^j)^{\pm}$. Then we construct monotonic CEFAs $(\cB_i^{j})_{i \in I, j \in J_i}$ and an LIA formula $\phi^\pm$ such that
	\begin{quote}
		there are an assignment function $\theta: R' \rightarrow \Int$ and strings $(w_i)_{i \in I}$ such that  $\phi[\theta(R' )/R']$ holds and $(w_i, \theta(R_i^j)) \in \Lang(\CEFA_{i}^j)$ for every $i \in I$ and $j \in J_i$ 
		\begin{center} if and only if \end{center}
		there are an assignment function $\theta^\pm: (R')^\pm \rightarrow \Nat$ and strings $(w_i)_{i \in I}$ such that  $\phi^\pm[\theta^\pm((R')^\pm)/(R')^\pm]$ holds and $(w_i, \theta^\pm((R_i^j)^\pm)) \in \Lang(\cB_{i}^j)$ for every $i \in I$ and $j \in J_i$.
	\end{quote}
	For $i \in I$ and $j \in J_i$, the CEFA $\cB_{i}^j$ is obtained from $\CEFA_{i}^j$ by replacing each transition $(q, a, q', \eta)$ in $\CEFA_i^j$ by the transition $(q, a, q', \eta')$ such that for each $r \in R_j^j$, 
	\[
	\eta'(r^+) = \left\{ \begin{array}{l  l}\eta(r), & \mbox{ if } \eta(r) \ge 0 \\ 0 & \mbox{ otherwise} \end{array}\right.,  \eta'(r^-) = \left\{ \begin{array}{l  l} 0, & \mbox{ if } \eta(r) \ge 0 \\ -\eta(r) & \mbox{ otherwise} \end{array}\right..
	\]
	In addition, $\phi^\pm$ is obtained from $\phi$ by replacing each $r \in R'$ with $r^+-r^-$.
	
	\smallskip
	
	It remains to prove the {\lasat} problem for monotonic CEFAs is in {\pspace}, namely,
	\begin{quote}
		given a family of monotonic CEFAs $\{ \CEFA_i^{j} \}_{i\in I,j\in J_i}$ each of which carries a vector of registers $R_i^j$ and a quantifier-free LIA formula $\phi$ such that  $ R_i^{j} $ are pairwise disjoint,  and the variables of $\phi$ are from $R'=\bigcup_{i,j} R_i^j$, deciding whether  there are an assignment function $\theta: R' \rightarrow \Nat$ and strings $(w_i)_{i \in I}$ such that  $\phi[\theta(R' )/R']$ holds and $(w_i, \theta(R_i^j)) \in \Lang(\CEFA_{i}^j)$ for every $i \in I$ and $j \in J_i$ is in {\pspace}.
	\end{quote}
	
	
	We use Proposition 16 in \cite{LB16} to show the result. Proposition 16 in \cite{LB16} is about monotonic counter machines, which can be seen as monotonic CEFAs where each transition contain no alphabet symbol, moreover, $\eta(r) \in \{0,1\}$ for the update function $\eta$ therein.
	
	For each $i\in I$ and $j\in J_i$, let $(\CEFA')_i^j$ be the monotonic counter machine obtained from $\CEFA_i^{j}$ by the following two-step procedure:
	\begin{enumerate}
		\item {[Remove the alphabet symbols]}: Remove alphabet symbols $a$ in each transition $(q, a, q', \eta)$ of $\CEFA_i^{j}$.
		%
		\item {[From binary encoding to unary encoding]}: Replace each transition $(q, q', \eta)$ such that $\ell = \max_{r \in R_i^j} \eta(r) > 1$ with a sequence of transitions $(q, p_1,\eta'_1), \cdots, (p_{\ell-1}, p_\ell, \eta'_\ell)$, where $p_1, \cdots,p_r$ are the freshly introduced states, moreover, $\eta'_j(r) = 1$ if $\eta(r) \ge j$, and $\eta'_j(r)=0$ otherwise. 
	\end{enumerate}
	
	According to Proposition 16 in \cite{LB16}, we have the following, 
	\begin{quote}
		Given a family of monotonic counter machines $\{ \cC_i \}_{i\in I}$ each of which carries a vector of counters $R_i$ and a quantifier-free LIA formula $\phi$ such that $ R_i$ are pairwise disjoint,  and the variables of $\phi$ are from $R'=\bigcup_{i} R_i$. If there are an assignment function $\theta: R' \rightarrow \Nat$ and strings $(w_i)_{i \in I}$ such that  $\phi[\theta(R' )/R']$ holds and $(w_i, \theta(R_i)) \in \Lang(\cC_{i})$ for every $i \in I$, then there are desired $\theta$ and $(w_i)_{i \in I}$ such that for each $i \in I$ and $r \in R_i$, $\theta(r)$ is at most polynomial in the number of states in $\cC_i $, exponential in $|R_i|$, and exponential in $|\phi|$.
	\end{quote}
	For each $i \in I$, let $\cC_i$ be the product of monotonic counter machines $(\CEFA')_i^j$ for $j \in J_i$. 
	From the fact that the number of states of $(\CEFA')_i^j$ is at most the product of the number of transitions of $\CEFA_i^j$ and $B_{\CEFA_i^j}$ (where $B_{\CEFA_i^j}$ denotes the maximum natural number constants $\eta(r)$ in $\CEFA_i^j$), we deduce the following,
	\begin{quote}
		if there are an assignment function $\theta: R' \rightarrow \Nat$ and strings $(w_i)_{i \in I}$ such that  $\phi[\theta(R' )/R']$ holds and $(w_i, \theta(R_i^j)) \in \Lang(\CEFA_i^j)$ for every $i \in I$ and $j \in J_i$, then there are desired $\theta$ and $(w_i)_{i \in I}$ such that for each $r \in \bigcup_{j \in J_i} R^j_i$, $\theta(r)$ is at most polynomial in the product of the number of transitions in $\CEFA_i^j$ and $B_{\CEFA_i^j}$ for $j \in J_i$, exponential in $\left|\bigcup_{j \in J_i} R^j_i \right|$, and exponential in $|\phi|$.
	\end{quote}
	
	Therefore, one can nondeterministically guess the strings $(w_i)_{i \in I}$, and for each $i \in I$ and $j \in J_i$, simulate the runs of CEFAs $\CEFA_i^j$ on $w_i$, in polynomial space, since  the values of all the registers can be assumed to be at most exponential, thus their binary encodings can be stored in polynomial space. From Savitch's theorem \cite{complexity-book}, we conclude that the {\lasat} problem for monotonic CEFAs is in {\pspace}.
\end{proof}

%%%%%%%%%%%%%%%%%%%%%%%%%%%%%%%%%%%%%%%%%%%%%%%%

\section{Example of {\tt urlXssSanitise(url)} for the decision procedure} \label{app:urlexample}

%\begin{example}
	Consider the program $S$ associated with {\tt urlXssSanitise(url)} in Section~\ref{sec:intro}. % Example~\ref{exmp:running}.
	%\[
	%\begin{array}{l}
	%y := \replaceall_{\Sigma \setminus ), \varepsilon}(x); z:= \replaceall_{\Sigma \setminus (, \varepsilon}(x);\\
	%\ASSERT{\length(y) = \length(z)}; \ASSERT{\indexof_{(}(x,0) < \indexof_{)}(x,0)}.
	%\end{array}
	%\] 
	We show how to decide its path feasibility. % of $S$. 
	\begin{description}
		\item[Step I.]   Vacuous since $S$ contains only atomic assertions already. %neither disjunction nor conjunction.
		%
		\item[Step II.] Nondeterministically choose to replace $\indexof_\#(\mathtt{url1}, 0)$ with $-1$ and add $\ASSERT{\mathtt{url1} \in \NFA_{\overline{\Sigma^*\#\Sigma^*}}}$ to $S$.  
		%
		\item[Step III.] Replace $\length(\mathtt{url1})$ with $i'_1$ and add $\ASSERT{\mathtt{url1} \in \CEFA_{\rm len}[i'_1/r_1]}$ to $S$, moreover, replace $\indexof_?(\mathtt{url1}, 0)$ with $i'_3$ and add $\ASSERT{0= i'_2}; \ASSERT{\mathtt{url1} \in \CEFA_{\indexof}[i'_2/r_1, i'_3/r_2]}$ to $S$, where $i'_1, i'_2, i'_3$ are fresh integer variables. Then we get the following program (still denoted by $S$), 
		\[ 
		\begin{array}{l}
		\ASSERT{\mathtt{prothostpath} \in \NFA_\varepsilon}; \ASSERT{\mathtt{querfrag} \in \NFA_\varepsilon}; \mathtt{url1} := \NFT_{\rm trim}(\mathtt{url}); \\
		\ASSERT{\mathtt{qmarkpos} = i'_3}; \ASSERT{\mathtt{sharppos} =-1 }; \ASSERT{\mathtt{qmarkpos} \ge 0}; \\ 
		\mathtt{prothostpath1} := \substring(\mathtt{url1}, 0, \mathtt{qmarkpos});\\
		\mathtt{querfrag1} := \substring(\mathtt{url1, qmarkpos}, i'_1 - \mathtt{qmarkpos});\\
		\mathtt{querfrag2} := \replaceall_{\mathtt{script},\ \varepsilon}(\mathtt{querfrag1});\\
		\mathtt{url2} := \mathtt{prothostpath1} \concat \mathtt{querfrag2}; \ASSERT{\mathtt{querfrag2} \in  \NFA_{\Sigma^*\mathtt{script}\Sigma^*}};  \\
		\ASSERT{\mathtt{url1} \in  \NFA_{\overline{\Sigma^*\#\Sigma^*}}}; \ASSERT{\mathtt{url1} \in \CEFA_{\rm len}[i'_1/r_1]}; \\
		\ASSERT{0= i'_2}; \ASSERT{\mathtt{url1} \in \CEFA_{\indexof}[i'_2/r_1, i'_3/r_2]}.
		\end{array}
		\]
		%
		\item[Step IV.] Since there is no CEFA constraints for $\mathtt{url2}$, removing the last assignment statement of $S$, i.e. $\mathtt{url2} := \mathtt{prothostpath1} \concat \mathtt{querfrag2}$, is easy and in this case we add no statements to $S$. After this, $\mathtt{querfrag2} := \replaceall_{\mathtt{script},\ \varepsilon}(\mathtt{querfrag1})$ becomes the last assignment statement. Suppose $\NFA'=(Q', \Sigma, \delta', I', F')$ is an NFA representing $(\replaceall_{\mathtt{script},\ \varepsilon})^{-1}_\emptyset(\Lang(\NFA_{\Sigma^*\mathtt{script}\Sigma^*}))$, namely, the pre-image of $\Lang(\NFA_{\Sigma^*\mathtt{script}\Sigma^*})$ under $\replaceall_{\mathtt{script},\ \varepsilon}$. Then we remove this assignment statement and add $\ASSERT{\mathtt{querfrag1 \in \NFA'}}$, resulting into the following program
		\[ 
		\begin{array}{l}
		\ASSERT{\mathtt{prothostpath} \in \NFA_\varepsilon}; \ASSERT{\mathtt{querfrag} \in \NFA_\varepsilon}; \mathtt{url1} := \NFT_{\rm trim}(\mathtt{url}); \\
		\ASSERT{\mathtt{qmarkpos} = i'_3}; \ASSERT{\mathtt{sharppos} =-1 }; \ASSERT{\mathtt{qmarkpos} \ge 0}; \\ 
		\mathtt{prothostpath1} := \substring(\mathtt{url1}, 0, \mathtt{qmarkpos});\\
		\mathtt{querfrag1} := \substring(\mathtt{url1, qmarkpos}, i'_1 - \mathtt{qmarkpos});\\
		%    \mathtt{querfrag2} := \replaceall_{\mathtt{script},\ \varepsilon}(\mathtt{querfrag1});\\
		%    \mathtt{url2} := \mathtt{prothostpath1} \concat \mathtt{querfrag2}; 
		\ASSERT{\mathtt{querfrag2} \in  \NFA_{\Sigma^*\mathtt{script}\Sigma^*}};  
		\ASSERT{\mathtt{url1} \in  \NFA_{\overline{\Sigma^*\#\Sigma^*}}}; \\
		\ASSERT{\mathtt{url1} \in \CEFA_{\rm len}[i'_1/r_1]};  \ASSERT{0= i'_2}; \\
		\ASSERT{\mathtt{url1} \in \CEFA_{\indexof}[i'_2/r_1, i'_3/r_2]};  \ASSERT{\mathtt{querfrag1} \in \NFA'}.
		\end{array}
		\]
		
		From Example~\ref{exm:pre-image}, we know that $\substring^{-1}_\emptyset(\Lang(\NFA'))$ can be represented by some CEFA $\cB=(Q'', R'', \delta'', I'', F'')$ with $Q''= Q' \times \{p_0,p_1,p_2\}$ and $R''=(r'_{1,1}, r'_{1,2})$ (where $r'_{1,1}$ and $r'_{1,2}$ are fresh integer variables). Then we remove $\mathtt{querfrag1} := \substring(\mathtt{url1, qmarkpos}, i'_1 - \mathtt{qmarkpos})$, add $\ASSERT{\mathtt{url1} \in \cB};\ASSERT{\mathtt{r'_{1,1}= qmarkpos}}; \ASSERT{r'_{1,2}=i'_1 - \mathtt{qmarkpos}}$, and get the following program
		\[ 
		\begin{array}{l}
		\ASSERT{\mathtt{prothostpath} \in \NFA_\varepsilon}; \ASSERT{\mathtt{querfrag} \in \NFA_\varepsilon}; \mathtt{url1} := \NFT_{\rm trim}(\mathtt{url}); \\
		\ASSERT{\mathtt{qmarkpos} = i'_3}; \ASSERT{\mathtt{sharppos} =-1 }; \ASSERT{\mathtt{qmarkpos} \ge 0}; \\ 
		\mathtt{prothostpath1} := \substring(\mathtt{url1}, 0, \mathtt{qmarkpos});\\
		%   \mathtt{querfrag1} := \substring(\mathtt{url1, qmarkpos}, i'_1 - \mathtt{qmarkpos});\\
		%    \mathtt{querfrag2} := \replaceall_{\mathtt{script},\ \varepsilon}(\mathtt{querfrag1});\\
		%    \mathtt{url2} := \mathtt{prothostpath1} \concat \mathtt{querfrag2}; 
		\ASSERT{\mathtt{querfrag2} \in  \NFA_{\Sigma^*\mathtt{script}\Sigma^*}};  
		\ASSERT{\mathtt{url1} \in  \NFA_{\overline{\Sigma^*\#\Sigma^*}}}; \\
		\ASSERT{\mathtt{url1} \in \CEFA_{\rm len}[i'_1/r_1]};  \ASSERT{0= i'_2}; \\
		\ASSERT{\mathtt{url1} \in \CEFA_{\indexof}[i'_2/r_1, i'_3/r_2]};  \ASSERT{\mathtt{querfrag1} \in \NFA'};\\
		\ASSERT{\mathtt{url1} \in \cB};\ASSERT{\mathtt{r'_{1,1} = qmarkpos} }; \ASSERT{r'_{1,2}=i'_1 - \mathtt{qmarkpos}}.
		\end{array}
		\]
		We can continue the process until the problem contains no assignment statement.
		%
		\item[Step V.]  Straightforward by utilising Proposition~\ref{prop:la-sat-cefa-inter}. 
	\end{description}
%\end{example}








%%%%%%%%%%%%%%%%%%%%%%%%%%%%%%
\section{Algorithms for case splits in the semantics of $\indexof_v$ and $\substring$}\label{app:case-split-semantics}

\begin{algorithm}[htbp]
  \small
  \KwIn{$active$: set of CEFA constraints,  $arith$: arithmetic constraint,
    $\mathit{funApps}$: acyclic set of assignment statements, and $(\mathcal{I}_l)_{l \in [5]}$: subsets of $\indexof_v(x,i)$ string terms}
  \KwResult{$(active, arith, \mathit{funApps})$\newline}
  
\For{each $\indexof_v(x, i) \in \mathcal{I}_1$}
		{
			$arith \leftarrow arith[\indexof_v(x, 0)/\indexof_v(x,i)] \wedge i < 0$\;
		}
		\For{each $\indexof_v(x, i) \in \mathcal{I}_2$}
		{
			$active \leftarrow active \cup \{x \in \NFA_{\overline{\Sigma^* v \Sigma^*}}\}$\;
			$arith \leftarrow arith[-1/\indexof_v(x,i)] \wedge i < 0$\;
		}
		\For{each $\indexof_v(x, i) \in \mathcal{I}_3$}
		{
			$arith \leftarrow arith[-1/\indexof_v(x,i)] \wedge i \ge \length(x)$\;
		}
		\For{each $\indexof_v(x, i) \in \mathcal{I}_4$}
		{
			$arith \leftarrow arith[-1/\indexof_v(x,i)] \wedge i \ge 0 \wedge i < \length(x)$\;
		}
		\For{each $\indexof_v(x, i) \in \mathcal{I}_5$}
		{
			choose fresh variables $y$ and $j$\;
			$active \leftarrow active \cup \{y \in \NFA_{\overline{\Sigma^* v \Sigma^*}}\}$\;
			$arith \leftarrow arith[-1/\indexof_v(x,i)] \wedge i \ge 0 \wedge i < \length(x) \wedge j = \length(x)-i$\;
			 $\mathit{funApps} \leftarrow \mathit{funApps} \cup \{y:=\substring(x, i, j)\}$\;
		}		
\caption{$\mathit{indexofCaseSplit}$ for case splits in the semantics of $\indexof_v$}\label{alg:indexof}
\end{algorithm}

\begin{algorithm}[htbp]
  \small
  \KwIn{$active$: set of CEFA constraints,  $arith$: arithmetic constraint,
    $\mathit{funApps}$: acyclic set of assignment statements, and $(\mathcal{I}_l)_{l \in [5]}$: subsets of $\indexof_v(x,i)$ string terms}
  \KwResult{$(active, arith, \mathit{funApps})$\newline}

  		\For{each $y:=\substring(x, i, j) \in \mathcal{J}_1$}
		{
			 $arith \leftarrow arith \wedge i \ge 0 \wedge i+j \le \length(x)$;
		}
		\For{each $y:=\substring(x, i, j) \in \mathcal{J}_2$}
		{
			 choose a fresh integer variable $i'$\;
			 $arith \leftarrow arith \wedge i \ge 0 \wedge i \le \length(x) \wedge i+j > \length(x) \wedge i' = \length(x)-i$\;
			 $\mathit{funApps} \leftarrow \mathit{funApps}[y:=\substring(x, i, i')/y:=\substring(x, i, j)]$\;
		}
		\For{each $y:=\substring(x, i, j) \in \mathcal{J}_3$}
		{
			 $arith \leftarrow arith \wedge i < 0$\;
			 $active \leftarrow active \cup \{y \in \NFA_\varepsilon\}$\;
			 $\mathit{funApps} \leftarrow \mathit{funApps} \setminus \{y:=\substring(x, i, j)\}$\;		 
		}
\caption{$\mathit{substringCaseSplit}$  for case splits in the semantics of $\substring$}\label{alg:substring}
\end{algorithm}





%%%%%% The proof of Theorem~\ref{thm-la-sat-cefa} is removed%%%%%%%%%%
%%%%%% The proof of Theorem~\ref{thm-la-sat-cefa} is removed%%%%%%%%%%
%%%%%% The proof of Theorem~\ref{thm-la-sat-cefa} is removed%%%%%%%%%%
%%%%%% The proof of Theorem~\ref{thm-la-sat-cefa} is removed%%%%%%%%%%
\hide{
\section{Proof of Theorem~\ref{thm-la-sat-cefa}} \label{appendix:thm-la-sat-cefa}

For a $k$-cost-enriched language $L$, we define 
\[
\prjnum(L) = \left\{(n_1, \cdots, n_k) \in \Int^k \mid \mbox{ there exist } w \in \Sigma^*.\ (w,(n_1,\cdots,n_k)) \in L \right\}.
\]

\begin{lemma}\label{lem-cefa-la}
	Let $\CEFA=(Q, \Sigma, R, \delta, I, F)$ be a CEFA with $R= (r_1, \cdots,  r_k)$. Then an existential LIA formula $\phi_\CEFA(r_1, \cdots, r_k)$ such that $\cM(\phi_\CEFA)= \prjnum(\Lang(\CEFA))$ can be computed in linear time from $\CEFA$.
\end{lemma}

\begin{proof}
	Suppose $\delta = \{\tau_1, \cdots, \tau_l\}$ such that $\tau_j = (q_j, a_j, q'_j, \eta_j)$ and $\eta_j(r_i) =  c_{j,i}$ for every $j \in [l]$ and $i \in [k]$.
	
	From the results on NFAs (Theorem~1 in \cite{SSMH04}), we know that for each pair of states $(q, q') \in I \times F$,  an existential LIA formula $\phi_{q,q'}(m_1, \cdots, m_l)$ can be computed in linear time such that $\cM(\phi_{q,q'})$ is the set of Parikh images of the runs of $\NFA$ starting from $q$ and ending at $q'$, where the variables $m_1, \cdots, m_l$ represent the numbers of occurrences of $\tau_1,\cdots, \tau_l$ respectively in the run. 
	
	Then the desired existential LIA formula $\phi_\NFA$ is constructed as follows,
	\[\phi_\NFA(r_1, \cdots, r_k) ::= \bigvee \limits_{(q,q') \in I \times F} \exists m_1 \cdots \exists m_l.\ \left(\varphi_{q,q'}(m_1, \cdots, m_l) \wedge \bigwedge \limits_{i \in [k]} r_i = \sum \limits_{j \in [l]} c_{j,i} m_j \right).\]
\end{proof}

We are ready to prove Theorem~\ref{thm-la-sat-cefa}.
\begin{proof}[Proof of Theorem~\ref{thm-la-sat-cefa}]
	The NP lower bound follows from the fact that the satisfiability problem of existential LIA formulas is NP-complete \cite{BT76,GS78} (see also \cite{Haase18}).
	
	For the upper bound, suppose that $\phi$ is a quantifier-free LIA formula and $\CEFA_1,\cdots,\CEFA_m$ are CEFAs such that 
	\begin{itemize}
		\item	$\CEFA_i=(Q_i, \Sigma, R_i, \delta_i, I_i, F_i)$  with $R_i = (r_{i, 1}, \cdots, r_{i, k_i})$, for every $i\in [m]$,
		\item $R_i \cap R_j = \emptyset$ for every $1 \le i < j \le m$, and
		\item the free variables of $\phi$ are from $\bigcup_{i\in [m]} R_i$.
	\end{itemize}
	From Lemma~\ref{lem-cefa-la}, for every $i \in [m]$, an existential LIA formula $\phi_{\CEFA_i}(r_{i,1}, \cdots, r_{i, k_i})$ such that $\cM(\phi_{\CEFA_i})= \prjnum(\Lang(\CEFA_i))$ can be computed in linear time from $\CEFA_i$.
	
	Then the satisfiability of $\phi$ w.r.t. $\CEFA_1,\cdots, \CEFA_m$ is reduced to the satisfiability problem of the  following existential LIA formula
	\[
	\phi' \equiv \phi \wedge \bigwedge \limits_{i \in [m]} \phi_{\CEFA_i}(r_{i,1}, \cdots, r_{i, k_i}).
	\]
	Since the size of $\phi'$ is linear in the size of $\phi$ and those of $\CEFA_1,\cdots,\CEFA_m$, and the satisfiability problem of existential LIA formulas is NP-complete, we conclude that the satisfiability of $\phi$ w.r.t.  $\CEFA_1,\cdots,\CEFA_m$ can be decided in nondeterministic polynomial time.
	
	It remains to prove the correctness of the reduction, namely, $\phi$ is satisfiable w.r.t. $\CEFA_1,\cdots, \CEFA_m$ iff $\phi'$ is satisfiable.
	
	\smallskip
	
	\noindent \emph{``Only if'' direction}. Suppose $\phi$ is satisfiable w.r.t. $\CEFA_1,\cdots, \CEFA_m$. Then there are an assignment function $\theta: \bigcup \limits_{i \in [m]} R_i \rightarrow \Int$ and strings $w_1, \cdots, w_m$  
	such that  $\phi[(\theta(R_i)/R_i)_{i \in [m]}]$ is evaluated to $true$ and $(w_i, \theta(R_i)) \in \Lang(\NFA_i)$ for every $i \in [m]$. For every $i \in [m]$, from $\cM(\phi_{\CEFA_i})=\prjnum(\Lang(\CEFA_i))$, we know that $\theta(R_i)$ satisfies $\phi_{\CEFA_i}$, namely, $\phi_{\CEFA_i}[\theta(R_i)/R_i]$ is evaluated to $true$. Therefore, the assignment $\theta$ makes $\phi'$ satisfied.
	
	\smallskip 
	
	\noindent \emph{``If'' direction}. Suppose $\phi'$ is satisfiable. Then there is an assignment $\theta: \bigcup \limits_{i \in [m]} R_i \rightarrow \Int$ such that $\phi[(\theta(R_i)/R_i)_{i \in [m]}]$, $\phi_{\CEFA_1}[\theta(R_1)/R_1]$, $\cdots$, and $\phi_{\CEFA_m}[\theta(R_m)/R_m]$ are all evaluated to $true$. For every $i \in [m]$, from $\cM(\phi_{\CEFA_i})=\prjnum(\Lang(\CEFA_i))$,  we know that there is a string $w_i$ such that $(w_i, \theta(R_i)) \in \Lang(\CEFA_i)$. From Definition~\ref{def-la-sat-cefa}, we conclude that $\phi$ is satisfiable w.r.t. $\CEFA_1,\cdots, \CEFA_m$.
\end{proof}
}
%%%%%% The proof of Theorem~\ref{thm-la-sat-cefa} is removed%%%%%%%%%%
%%%%%% The proof of Theorem~\ref{thm-la-sat-cefa} is removed%%%%%%%%%%


\end{appendix}

%\fi

\end{document}
