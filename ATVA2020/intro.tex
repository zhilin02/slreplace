%!TEX root = main.tex


%This paper concerns strings and integers, two fundamental data-types in virtually all programming languages.
String-manipulating programs are notoriously subtle, and their potential bugs %Many subtle programming bugs are caused by , some with
may bring severe security consequences. A typical example is cross-site scripting
(XSS), which is among the OWASP Top 10 Application Security Risks~\cite{owasp17}.
Integer data type occurs naturally and extensively in string-manipulating programs. %For instance, access to lengths or positions of strings is frequently needed. 
%
An effective and increasingly popular method for identifying bugs, including XSS, is symbolic execution~\cite{symbex-survey}.
% and combinations with dynamic analysis
%called \emph{dynamic symbolic execution} \cite{jalangi,DART,EXE,CUTE,KLEE}.
%(See~\cite{symbex-survey} for an excellent survey.)
%Unlike purely random testing,
%which runs only \emph{concrete} program executions on different
%inputs, the techniques of symbolic execution 
In a nutshell, this technique analyses static paths
 through the program being considered.
Each of these paths can be viewed as a constraint~$\varphi$ over
appropriate data domains, and symbolic execution tools
demand fast constraint
solvers to check the satisfiability of $\varphi$. Such constraint
solvers need to support all %native 
data-type operations occurring in
a program.
% (i.e., to check
%the \emph{feasibility} of the static path), which can be used for, e.g.,  generating
%inputs that lead to certain parts of the program or an erroneous behavior.


Typically, mainstream programming languages provide standard string functions such as concatenation, $\replace$, and $\replaceall$. Moreover, Web programming languages usually provide complex string operations (e.g. htmlEscape and trim), which are conveniently modelled as finite transducers, to sanitise malicious user inputs \cite{BEK}. 
%These sanitization operations can be conveniently modeled by finite transducers. 
Nevertheless, apart from these operations involving only the string data type, functions such as $\length$, $\indexof$, and $\substring$, which can convert strings to integers and vice versa, are also heavily used in practice; for instance, it was reported~\cite{Berkeley-JavaScript} that $\length$, $\indexof$, $\substring$, and variants thereof, comprise over 80\% of string function occurrences in 18 popular JavaScript applications, notably outnumbering concatenation. The introduction of integers exacerbates the intricacy of string-manipulating programs, and poses new theoretical and practical challenges in solver development. 

%Typically, mainstream programming languages provide, apart from standard string functions (e.g., concatenation, $\replace$ and $\replaceall$), %string functions involving the integer data type, e.g. 
%functions such as $\length$, $\indexof$, and $\substring$, which can convert strings to integers and vice versa. %, are also widely used in string-manipulating programs. 
%These functions are indeed heavily used in practice; for instance, it was reported~\cite{Berkeley-JavaScript} that $\length$, $\indexof$, $\substring$, and variants thereof, comprise over 80\% of string function occurrences in 18 popular JavaScript applications, notably outnumbering concatenation. The introduction of integers exacerbates the intricacy of string-manipulating programs, and poses new theoretical and practical challenges in solver development. 


%in programs it , typically in the form of lengths or positions of strings. Besides the classical string functions e.g. concatenation $\concat$ and $\replaceall$, string functions involving the integer data type, e.g. $\length$, $\indexof$, and $\substring$, are also widely used in string-manipulating programs. For instance, it was reported in \cite{Berkeley-JavaScript} that $\length$, $\indexof$, $\substring$, and their variants, occupy more than 80 percent of string function occurrences in their subject Javascript applications (even much more frequently used than concatenation). This motivates \emph{string constraint solvers that reason about strings and integers simultaneously}. 



%$\length: \Sigma^* \rightarrow \Int$ (where $\Sigma$ is the alphabet and $\Sigma^*$ is the set of strings over $\Sigma$), $\substring: \Sigma^* \times \Int \times \Int \rightarrow \Sigma^*$, and $\indexof: \Sigma^* \times \Int \rightarrow \Int$.

%Reasoning about strings and integers is important and challenging, but is not well-studied. 
%It is easy to end up with undecidability: even the slightest extension of this theory (e.g. the theory with concatenation and letter counting)  would render the theory undecidable \cite{buchi,GB16}. 

When combining strings and integers, decidability can easily be lost; 
 for instance, the string theory with concatenation and letter counting
functions is undecidable~\cite{buchi,Manea-RP}.
\OMIT{
Although a great deal of research has shown that it is viable to reason about rather complex string operations without breaking decidability by %imposing proper 
restricting the form of constraints (e.g.,~\cite{CCH+18,CHL+19}; cf.\ related work for a brief survey), adding length constraints would immediately lead to undecidability~\cite{CCH+18}. 
}
Remarkably, it is still a major open problem whether the string theory with concatenation (arguably the simplest string operation) and length function 
(arguably the most common string-number function) is 
decidable~\cite{Vijay-length}. 
One promising approach to retain decidability is to enforce a syntactic
restriction to the constraints. In the  literature, these restriction  include solved forms
\cite{Vijay-length}, acyclicity \cite{BFL13,Abdulla14,AbdullaA+19}, and 
straight-line fragment \cite{LB16,CCH+18,CHL+19,HJLRV18}. On the one hand,
such a restriction has led to decidability of string constraint solving with 
complex string
operations (not only concatenation, but also finite transducers) and integer
operations (letter-counting, $\length$, $\indexof$, etc.); see, e.g., 
\cite{LB16}. On the other hand, there is a lot of evidence (e.g. from 
benchmark) that %sufficiently 
many practical string constraints do satisfy 
such syntactic restrictions.
%
%At the same time, the classes of constraints that are required in practice sometimes demand solving techniques which are not necessarily complete. 
%
\OMIT{
Although there have been some recent results on decision procedures for string constraints involving the integer data-type, for instance~\cite{Vijay-length,LeH18,LinM18,LB16}, the theories considered in those approaches are usually quite restricted and do not reflect the constraints that occur in applications well. Overall, much more research is needed to
better understand the decidability of theories combining strings and integers.
}

Approaches to building practical string solvers could essentially be classified
into two categories. Firstly, one could support as many constraints as possible, but 
primarily resort to heuristics, 
offering no completeness/termination guarantee. This is a realistic approach 
since, as
mentioned above, the problem involving both string and integer data types is in 
general undecidable. Many solvers
belong to this category, e.g., 
{\cvc}~\cite{cvc4}, Z3 \cite{BTV09,Z3}, {\zthree}~\cite{Z3-str3}, S3(P) \cite{S3,TCJ16},
Trau~\cite{Abdulla17} (or its variants {\trauplus}~\cite{AbdullaA+19} and {\zthreetrau}~\cite{Z3-trau}), ABC \cite{ABC}, and \slent~\cite{WC+18}.
%\anthony{I think we should mention more here ...}
%
%On a more practical level, 
%
%this leads to a situation where the vast majority of  state-of-the-art string constraint solvers, e.g., 
%resort to heuristics without completeness guarantees to support
%reasoning about the combination of strings and integers. 
\OMIT{
Such solvers
often show excellent performance in applications and on existing
benchmark suites, but 
}
Completeness guarantees are, however, valuable since %it is well-known that 
the 
performance of heuristics can be difficult to predict.
The second approach is to develop solvers for decidable fragments
supporting both strings and integers (e.g. %the fragments of
\cite{Vijay-length,BFL13,Abdulla14,AbdullaA+19,LB16,CCH+18,CHL+19,HJLRV18}). 
Solvers in this category include Norn \cite{Abdulla14}, SLOTH
\cite{HJLRV18}, and OSTRICH \cite{CHL+19}. 
%Unfortunately
%Even though the fragment
%theoretical decidability result of \cite{LB16}  does not
%easily lead to an implementation. 
%(as far as we know, there has not been any implementation of the procedure therein yet).
The fragment \emph{without} complex string operations (e.g. $\replaceall$ and
finite transducers, but $\length$) can be handled quite well by Norn. The fragment \emph{without} length
constraints (but $\replaceall$ and finite transducers)
%constraints but not complex string operations like replaceAll and transducers,
can be handled effectively by OSTRICH and SLOTH. 
Moreover, most existing solvers that belong to the first category do not support
complex string operations like $\replaceall$ and finite transducers as well.
This motivates the following problem: 
\emph{provide a decision procedure that supports both string and integer data type, with completeness guarantee and meanwhile admitting efficient implementation}.

%\anthony{need to say why extending these with length is non-trivial}
We argue that this problem is highly challenging.
A deeper examination into the algorithms used by OSTRICH and SLOTH reveals that,
unlike the case for Norn, it would \emph{not} be straightforward to extend OSTRICH and 
SLOTH with integer constraints. First and foremost, the complexity of the
fragment used by Norn (i.e. without transducers and $\replaceall$) is solvable
in exponential time, even in the presence of integer constraints. This is not
the case for the straight-line fragments with transducers/$\replaceall$, which 
require at least double exponential time (regardless of the integer constraints).
This unfortunately manifests itself in the size of symbolic representations of 
the solutions. SLOTH \cite{HJLRV18} computes 
%a ``global'' 
a representation of all
solutions ``eagerly'' as (alternating) finite transducers. Dealing with integer data type requires one
to compute the Parikh images of these transducers \cite{LB16}, which would 
result in a
quantifier-free linear integer arithmetic formula (LIA for short) of double exponential size, thus giving us a triply-exponential-time algorithm, since LIA formulas are solved in exponential time  (see e.g. \cite{VSS05}).
Lin and Barcelo \cite{LB16} provided a double-exponential upper bound on the 
length of the strings in the solution, and show that the double
exponential-time theoretical complexity could be retained. This, however, does
not result in a practical algorithm since it requires all strings of double
exponential size to be enumerated. OSTRICH \cite{CHL+19} adopted a ``lazy'' approach and computed the pre-images of regular languages step by step,
%step by step, 
which is more scalable than
the ``eager'' approach adopted by SLOTH and results in a highly competitive solver.
It uses \emph{recognizable relations} (a finite union of products of regular languages)
as symbolic representations. Nevertheless, extending this approach to integer
constraints is not obvious since integer constraints break the independence
between the different string variables in the recognizable relations.
%\emph{provide an
%efficient decision procedure that supports string and integer data type, as well
%as complex string operations, with a completeness guarantee}.


%At the moment, 
%The best way to summarize the current
%situation is 
%There are competitive solvers
%belonging to either category, but one
%; for instance, sometimes innocent syntactic modifications of
%constraints can cause solvers to diverge and not to terminate. 
\OMIT{
There is still a gap between the string operations supported by
solvers, and the operations occurring in applications; for instance,
operations like \textsf{replaceAll} and escape/unescape-transformations
occur extremely frequently in programs, but are currently not handled well
by many solvers. 
The \emph{grand challenge} in string constraint solving is to
identify fragments and develop algorithms, that cover the constraints
occurring in \emph{applications}, while preserving \emph{decidability} and
admitting \emph{efficient implementations.}
}


%Nevertheless, solving string constraints involving the integer data type is far from trivial and actually undecidable in general \cite{buchi,CCH+18}. Therefore, all the aforementioned string constraint solvers
%They however inevitably  resort to heuristics without completeness guarantees. 

%Nevertheless, as already argued in~\cite{CHL+19}, despite the excellent performance of some of these solvers on some existing benchmark suites, decision procedures for string constraints with stronger theoretical guarantees, e.g., in the form of decidability (perhaps accompanied by a complexity analysis), are still highly desirable. 

%In a nutshell, there are at least two reasons: (1) they are \emph{theoretically intriguing} as this is essentially to delimit the (un)decidability boundary for the first-order theory of strings not least; and   
%(2) they are \emph{practically appealing} since they could potentially strike a good balance between precision and scalability, if they could be made an efficient implementation. 

%are amenable to implementation. 
%they can provide a kind of robustness guarantees upon which a string constraint solver could further improve and optimise.if they are amenable to implementation, 

%\zhilin{this paragraph is copied from the popl paper} Despite the excellent performance of some of these solvers on several existing benchmarks, there are good reasons for designing decision procedures with stronger theoretical guarantees, e.g., in the form of decidability (perhaps accompanied by a complexity analysis). One such reason is that string constraint solving is a research area in its infancy with an insufficient range of benchmarking examples to convince us that if a string solver works well on existing benchmarks, it will also work well on future benchmarks. A theoretical result provides a kind of robustness guarantee upon which a practical solver could further improve and optimise.

%(concatenation, length and regular constraints) 
%We tackle this issue 

%In contrast, in this paper we aim to develop decision procedures for string constraints involving the integer data type, which %, on the one hand, 
%\emph{provide completeness guarantees} and %on the other hand, 
%\emph{admit efficient implementation}.

\noindent\emph{Contribution.} The main contribution of this paper is a decision procedure for an expressive class of string constraints involving the integer data type, which includes not only concatenation, $\replace$/$\replaceall$, $\reverse$, finite transducers, and regular constraints, but also $\length$, $\indexof$ and $\substring$. The decision procedure utilizes a variant of cost-register automata introduced by Alur et al. \cite{RLJ+13}, which are called \emph{cost-enriched finite automata} (CEFA) for convenience. \zhilin{add several sentences here, describing the idea of CEFAs} Intuitively, each CEFA records the connections between a string variable and the integer variables attached to it, and symbolically represents the set of solutions of these variables. With CEFAs, the concept of recognizable relations is then naturally extended. The integer constraints, on the other hand, are \emph{put alongside CEFAs, instead of as parts of CEFAs}, consequently preserving the independence of string variables in recognizable relations. 
The crux of the decision procedure is to compute the backward images of CEFAs under string functions, in the same flavor as OSTRICH for string constraints \emph{without} the integer data type \cite{CHL+19}. 
%, thus preserving the virtues of the ``lazy'' approach of OSTRICH.
Such an approach %inherits its elegance by 
is able to treat %the aforementioned seemly 
a wide range of string functions in a generic, and yet simple, way. 

To the best of our knowledge, the class of string constraints considered in this paper is currently one of the most expressive string theories involving the integer data type known to enjoy a decision procedure. In particular, it is strictly more expressive than that in \cite{LB16}, where $\reverse$ and $\substring$ are missing. 
%\anthony{add something here perhaps}
%
\OMIT{
Moreover, 
our decision procedure takes a ``local'' approach for step-by-step backward 
computation of pre-images of CEFAs, which extends the framework of {\ostrich}
solver \cite{CHL+19}.

compared to the ``global'' approach in~\cite{LB16} which is based on enumerating
all strings up to a double exponential length (which does not lead to an
efficient implementation), 
}


%\emph{the first decision procedure for string constraints that supports the string functions} $\substring$, $\length$, and $\indexof$, as well as concatenation, $\replaceall$, and finite transducers. \zhilin{improve latter}

%Perhaps more importantly, 
%
We implement the decision procedure based on %an extension of 
the recent {\ostrich} solver \cite{CHL+19},  resulting in {\ostrich}+.  We perform experiments on a wide range of benchmark suites, including those where both $\replace$/$\replaceall$/finite transducers and $\length$/$\indexof$/$\substring$ occur, as well as the well-known benchmarks {\kaluzabench} and {\pyexbench}.
%, to evaluate the performance of {\ostrich}+. 
The results show that  %{\ostrich}+ achieves a nice balance between expressiveness and efficiency in the sense that 
1) {\ostrich}+ so far is the only string constraint solver capable of dealing with finite transducers and integer constraints, and 2) its overall performance is comparable with the best state-of-the-art string constraint solvers (e.g. {\cvc} and {\zthreetrau}) which are short of completeness guarantees. 

\hide{
\smallskip
\section{Related Work}
\label{sec-related}

In this section, we will discuss related results. In particular, we will discuss
(1) results on modelling and reasoning about
\regexp{} constraints, and (2) results on string constraint solving.


\subsection{Modelling and Reasoning about \regexp{}}
%This paper is concerned with string constraint solving in general, but the focus is on the interplay of regular expressions in modern programming language and solving constraints involving complex string functions. Both of them are monumental research fields for which we will only discuss the work which are most pertinent to ours. 

Variants and extensions of regular expressions to capture their usage in programming languages have received attention %been investigated 
in both theory and practice. In formal language theory, regular expressions with capturing groups and backreferences were considered in \cite{CSY03,CN09} and also more recently in \cite{Freydenberger13,Schmid16,BM17b,FS19}, where the expressibility issues and decision problems were investigated. Nevertheless, some basic features of these regular expression, namely, the non-commutative union and the greedy/lazy semantics of Kleene star/plus, were not addressed therein. In the software engineering community, % have also received attention in the software engineering community. 
some empirical studies were recently reported for these regular expressions, including portability across different programing languages \cite{DMC+19} and DDos attacks \cite{SP18}, as well as how programmers write them in practice \cite{MDD+19}.


Prioritized finite-state automata and %prioritized finite-state 
transducers were proposed in \cite{BM17}. Prioritized finite-state transducers add indexed brackets to the input string in order to identify the matches of capturing groups. It is hard---if not impossible---to use prioritized finite-state transducers to model replace(all) function in general, e.g., swapping the first and last name as in Example~\ref{exmp-name-swap}. In contrast, {\PSST}s store the matches of capturing groups in string variables, which can then be referred to, thus allowing us to conveniently model the match and replace(all) function. 
%
Streaming string transducers were also used in \cite{ZAM19} to solve the straight-line string constraints with concatenation, finite-state transducers, and regular constraints.

\subsection{String Constraint Solving}
As we discussed Section \ref{sec-intro}, there has been a wealth of research
activities focussing on the development of string constraint solving algorithms, especially
in the past ten years or so. Solvers are typically using a combination
of different techniques to check the satisfiability of string constraints,
including word-based methods, automata-based methods, and unfolding-based methods
like the translation to bit-vector constraints.
We mention among others the following string solvers:
Z3 \cite{Z3}, CVC4 \cite{cvc4}, Z3-str/2/3/4 \cite{Z3-str,Z3-str2,Z3-str3,BerzishMurphy2021},
 ABC \cite{ABC}, Norn
\cite{Abdulla14}, Trau \cite{Z3-trau,AbdullaACDHRR18-trau,Abdulla17}, OSTRICH
\cite{CHL+19}, S2S \cite{DBLP:conf/aplas/LeH18}, Qzy \cite{cox2017model}, Stranger \cite{Stranger}, Sloth
\cite{HJLRV18,AbdullaA+19},
Slog \cite{fang-yu-circuits}, Slent \cite{WC+18}, Gecode+S \cite{DBLP:conf/cpaior/ScottFPS17}, G-Strings \cite{DBLP:conf/cp/AmadiniGST17}, HAMPI
\cite{HAMPI}, and S3 \cite{S3}. 
Most modern string solvers provide support of concatenation and regular 
constraints. The push (e.g. see
\cite{GB16,Vijay-length,HAMPI,Berkeley-JavaScript,LB16,S3})
towards incorporating other functions---e.g. length, 
string-number conversions, replace, replaceAll---in a string theory has been an
important theme in the area, owing to the desire to be able to reason 
about increasingly complex real-world string-manipulating programs.
These functions, among others, are now part of the SMT-LIB Unicode Strings
standard.\footnote{See
\url{http://smtlib.cs.uiowa.edu/theories-UnicodeStrings.shtml}}

To the best of our knowledge, at the moment there is no solver that
supports \regexp\ features like greedy/lazing matching or capturing
groups (apart from our own solver \ostrich). This was also remarked in
\cite{LMK19}, where the authors try to amend the situation by developing 
\expose{} --- a dynamic symbolic execution engine --- that maps path 
constraints in JavaScript to Z3. The strength of \expose{} is in a thorough
modelling of \regexp{} features, some of which (including backreferences) we do 
not cover in our string constraint language and string solver \ostrich{}. Note, however,
that the features that we do not cover are also rare in practice, according to
\cite{LMK19} --- in fact, around 75\% of all the \regexp{} expressions found in
their benchmarks across 415,487 NPM packages can be covered in our fragment.
The strength of \ostrich{} against \expose{} is in a substantial improvement in
performance (by 30--50 fold) and precision. \expose{} does not terminate 
even for simple examples (e.g. for Example \ref{exmp-name-swap} and Example 
\ref{ex:normalize}), which can be solved by our solver within a few seconds.
%we do cover a huge portion of \regexp{} that arise in practice.

%For a recent comparison of the solvers, we refer the
%reader to the survey \cite{Ama20}.

For string constraint solving in general, we refer the readers to the recent survey \cite{Ama20}. In this work, we consider a string constraint language which is undecidable in general, and propose a propagation-based calculus to solve the constraints. However, we also identified a straight-line fragment including concatenation, extract, replace(All) which turns to be decidable. The decision procedure we use extends the backward-reasoning approach in \cite{CHL+19}, where only standard one-way and two-way finite-state transducers were considered. 

}

%%%%%%%%%%%%%%%%%%%%%%% The running example %%%%%%%%%%%%%%%%%%%%%%%
%%%%%%%%%%%%%%%%%%%%%%% The running example %%%%%%%%%%%%%%%%%%%%%%%
%%%%%%%%%%%%%%%%%%%%%%% The running example %%%%%%%%%%%%%%%%%%%%%%%
%%!TEX root = main.tex

%\begin{example}\label{exmp:running}
\section{Running example}
We use the following JavaScript program as the running example of this paper,
%which demonstrates the use of string functions coupled with integer data type as well as path feasibility of string-manipulating programs.
which defines a function {\urlxsssanitise} (the input parameter {\sf url} specifying a URL). A typical URL consists of a hierarchical sequence of components commonly referred to as protocol, host, path, query, and fragment. For instance, in ``\url{http://www.example.com/some/abc.html?name=john#print}'', the protocol is ``{\tt http}'', the host is ``{\tt www.example.com}'', the path is ``{\tt /some/abc.html}'', the query is ``{\tt name=john}'' (preceded by $?$), and the fragment is ``{\tt print}'' (preceded by $\#$). (Note that both the query and the fragment could be empty in a URL.) The aim of {\urlxsssanitise} is to mitigate \emph{URL reflection attacks} \cite{url-reflect}, by filtering out the dangerous substring ``\url{script}'' from the query and fragment components of  the input URL. URL reflection attacks are a type of cross-site-scripting (XSS) attacks that do not rely on saving malicious code in database, but rather on hiding it in the query or fragment component of URLs, e.g., ``\url{http://www.example.com/some/abc.html?name=<script>alert('xss!');</script>}''.
%\begin{example}
{\small
\begin{minted}[linenos]{javascript}
function urlXssSanitise(url)
{
    var prothostpath='', querfrag = '';
    url = url.trim();
    var qmarkpos = url.indexof('?'), sharppos = url.indexof('#');
    if(qmarkpos >= 0) 
    {   prothostpath = url.substr(0, qmarkpos);
        querfrag = url.substr(qmarkpos); }
    else if(sharppos >= 0)
    {   prothostpath = url.substr(0, sharppos);
        querfrag = url.substr(sharppos); }
    else prothostpath = url;
    querfrag = querfrag.replace(/script/g, '');
    url = prothostpath.concat(querfrag);
    return url;
}
\end{minted}
}
%\end{example}
%
\noindent Note that {\urlxsssanitise} uses the JavaScript sanitisation operation $\sf trim$ that removes whitespace from both ends of a string, which can be conveniently modeled by finite-state transducers. Two string functions involving the integer data type, namely, $\sf indexof$ and $\sf substr$ ($\indexof$ and $\substring$ in this paper), as well as concatenation and $\sf replace$ (with the `g'---global---flag, called $\replaceall$ in this paper), are present. 

%$\footnote{$\sf replace$ with the g flag in Javascript corresponds to the $\replaceall$ function in this paper.}

% \in [\backslash w | \backslash x2E]^*$
%We expect that the returned value of the ``host'' variable contains only the alphanumeric symbols as well as the dot symbol, but actually this is not the case. This question can be reduced to solving the path feasibility problem of the following program of the SSA (single static assignment) form.

The program analysis is to ascertain whether {\urlxsssanitise} indeed works, namely, after applying {\urlxsssanitise} to the input URL, ``\url{script}'' does not appear.  This problem can be reduced to checking whether there is an execution path of {\urlxsssanitise} that produces an output such that its query or fragment component contains occurrences of ``\url{script}''. For instance, assuming  the ``if'' branch is executed, %first execution path of {\urlxsssanitise} is chosen, then the problem is reduced to 
we will need to solve the path feasibility of the following JavaScript program in single static assignment (SSA) form,
%
{\small
\begin{minted}{javascript}
    prothostpath =''; querfrag = '';
    url1 = url.trim(); qmarkpos = url1.indexof('?');
    sharppos = url1.indexof('#'); assert(qmarkpos >= 0); 
    prothostpath1 = url1.substr(0, qmarkpos);
    querfrag1 = url1.substr(qmarkpos);
    querfrag2 = querfrag1.replace(/script/g, '');
    url2 = prothostpath1.concat(querfrag2);
    assert(/script/.test(querfrag2))
\end{minted}
}
\noindent where the $\ASSERT{cond}$ statement checks that the condition $cond$ is satisfied. As one will see later, this can be directly encoded in our constraint language (cf.\ Section~\ref{sec:logic}) and handled by the decision procedure (cf. Section~\ref{sec:dec}). 
%Back to Example~\ref{exmp:running}, 
Our solver  {\ostrich}+ can solve the path feasibility of the aforementioned 
SSA program in several seconds. This is far from trivial since %the program contains not only 
the solver needs to tackle complex string functions such as {\tt trim()} (modeled as finite transducers) and {\tt replaceAll}, %but also those involving the integer data type, namely 
as well as %$\length$, 
$\indexof$  and $\substring$, which is beyond the scope of other existing solvers.  
%\qed
%\end{example}

%%%%%%%%%%%%%%%%%%%%%%% The running example %%%%%%%%%%%%%%%%%%%%%%%
%%%%%%%%%%%%%%%%%%%%%%% The running example %%%%%%%%%%%%%%%%%%%%%%%
%%%%%%%%%%%%%%%%%%%%%%% The running example %%%%%%%%%%%%%%%%%%%%%%%
