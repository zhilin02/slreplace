%!TEX root = main.tex

\section{Prioritized Streaming String Transducers}  \label{sect:psst}
%\zhilin{Move from preliminary to here}

In this section, we introduce prioritized streaming string transducers (PSST), which extend prioritized finite-state automata (PFA,  \cite{BM17}). We shall utilize PSST to model  the non-greedy and greedy semantics of $\regexp$ as well as the behavior of the $\extract$ and $\replaceall$ functions.
%based on which we model the semantics of $\regexp$ defined in Section~\ref{sec:prel} and design the decision procedure in Section~\ref{sec:decision}.

%\paragraph{Prioritized Finite-state automata.}
%
For a finite set $Q$, let $\overline{Q} = \bigcup_{n\in \Nat}\{ (q_1, \ldots, q_n) \mid \forall i \in [n], q_i \in Q \wedge \forall i,j \in[n], i \neq j \rightarrow q_i \neq q_j \}$. Intuitively, $\overline{Q}$ is the set of sequences of non-repetitive elements from $Q$. In particular, the empty sequence $\varepsilon \in \overline{Q}$. Note that the length of each sequence from $\overline{Q}$ is bounded by  $| Q |$. For a sequence $P = (q_1, \ldots, q_n) \in \overline{Q}$ and  $q \in Q$, we write $q \in P$ if  $q = q_i$ for some $i \in [n]$. Moreover, for $P_1 = (q_1, \ldots, q_m) \in \overline{Q}$ and $P_2 = (q'_1, \ldots, q'_n) \in \overline{Q}$, we say $P_1 \cap P_2 = \emptyset$ if $\{q_1, \ldots, q_m\} \cap \{q'_1, \ldots, q'_n\} = \emptyset$.


\begin{definition}[Prioritized Finite-state Automata]\label{def-pfa}
  A \emph{prioritized finite-state automaton} (PFA) over a finite alphabet $\Sigma$ is a tuple $\pnfa=(Q, \Sigma, \delta, \tau, q_0, F)$ where $\delta \in Q
  \times \Sigma \rightarrow \overline{Q}$ and $\tau \in Q \rightarrow \overline{Q} \times \overline{Q}$ such that for every $q \in Q$, if $\tau(q) = (P_1; P_2)$, then $P_1 \cap P_2 = \emptyset$. 
  The definition of $Q$, $q_0$ and $F$ is the same as ordinary FA.
\end{definition}
For $\tau(q) = (P_1; P_2)$, we will use $\pi_1(\tau(q))$ and $\pi_2(\tau(q))$ to denote $P_1$ and $P_2$ respectively.  With slight abuse of notation, we write $q\in (P_1; P_2)$ for $q\in P_1\cup P_2$. Intuitively, $\tau(q)=(P_1; P_2)$ specifies the $\varepsilon$-transitions at $q$, with the intuition that the $\varepsilon$-transitions to the states in $P_1$ resp. $P_2$ have higher resp. lower priorities than the non-$\varepsilon$-transitions out of $q$.
  
A  run of $\pnfa$ on a string $w$ is a sequence $q_0 \sigma'_1 q_1 \ldots \sigma'_m q_m$ such that 
\begin{itemize}
%\item $q_m \in F$,
\item for any $i \in [m]$, either $\sigma'_i \in \Sigma$ and $q_i \in \delta (q_{i - 1}, \sigma'_i)$, or $\sigma'_i = \varepsilon$ and $q_i \in \tau(q_{i-1})$ %\pi_1(\tau(q_{i-1}))\cup \pi_2(\tau(q_{i-1}))$,
\item $w = \sigma'_1 \cdots \sigma'_m$,
%
\item for every subsequence $q_i \sigma'_{i+1} q_{i+1} \ldots \sigma'_{j} q_j$ such that  $i < j$ and $\sigma'_{i+1} = \cdots = \sigma'_j = \varepsilon$, it holds that for every $k, l: i \le k < l < j$, $(q_k, q_{k+1}) \neq (q_l, q_{l+1})$.
%each state $q \in Q$ occurs \emph{at most twice} in the subsequence. 
(Intuitively, each transition occurs at most once in a sequence of $\varepsilon$-transitions.) 
\end{itemize}
Note that it is possible that $\delta(q, \sigma) = ()$, namely, there is no $\sigma$-transition out of $q$. 
It is easy to observe that given a string $w$, the length of a run of $\pnfa$ on $w$ is $O(|w||Q| |\Sigma| |Q|)$ (Note that $\pnfa$ contains at most $|Q| (|\Sigma|+1) |Q|$ transitions). 
For any two runs $p = q_0 \sigma_1 q_1 \ldots \sigma_m q_m$ and $p' =  q_0 \sigma'_1 q_1' \ldots \sigma'_n q'_n$ such that $\sigma_1 \ldots \sigma_m = \sigma'_1 \ldots \sigma'_n$, we say that $p$ is of a higher priority over $p'$ if 
\begin{itemize}
\item either $p'$ is a prefix of $p$ (in this case, the transitions of $p$ after $p'$ are all $\varepsilon$-transitions), 
%
\item or there is an index $j$ satisfying one of the following constraints:
\begin{itemize}
\item $q_0 \sigma_1 q_1 \ldots q_{j-1} \sigma_j = q_0 \sigma'_1 q'_1 \ldots q'_{j-1} \sigma'_j$, $q_j \neq q'_j$, $\sigma_j \in \Sigma$, and $\delta (q_{j - 1}, \sigma_j) =(\ldots, q_j, \ldots, q_j', \ldots)$,
%
\item $q_0 \sigma_1 q_1 \ldots q_{j-1} \sigma_j = q_0 \sigma'_1 q'_1 \ldots q'_{j-1} \sigma'_j$, $q_j \neq q'_j$, $\sigma_j  = \varepsilon$,  and one of the following conditions holds: (i) $\pi_1(\tau(q_{j - 1})) = (\ldots, q_j, \ldots, q_j', \ldots)$, (ii) $\pi_2(\tau(q_{j - 1})) = (\ldots, q_j, \ldots, q_j', \ldots)$, (iii) and $q_j \in \pi_1(\tau(q_{j - 1}))$ and $q'_j \in \pi_2(\tau(q_{j-1}))$, 
%
\item $q_0 \sigma_1 q_1 \ldots q_{j-1}  = q_0 \sigma'_1 q'_1 \ldots q'_{j-1} $, $\sigma_j  = \varepsilon$, $\sigma'_j  \in \Sigma$, $q_j \in \pi_1(\tau(q_{j - 1}))$, and $q'_j \in \delta(q_{j-1}, \sigma'_j)$, 
%
\item $q_0 \sigma_1 q_1 \ldots q_{j-1}  = q_0 \sigma'_1 q'_1 \ldots q'_{j-1} $, $\sigma_j  \in \Sigma$, $\sigma'_j  = \varepsilon$, $q_j \in \delta(q_{j - 1}, \sigma_j)$, and $q'_j \in \pi_2(\tau(q_{j-1}))$.
\end{itemize}
\end{itemize}
From the definition of ``higher priorities" above, we observe that if there is a  run of $\pnfa$ on a string $w$, then there is a unique run of $\pnfa$ on $w$ with the highest priority. 
An \emph{accepting} run of $\pnfa$ on $w$ is a run $q_0 \sigma_1 q_1 \ldots \sigma_m q_m$ of $\pnfa$ on $w$ with the \emph{highest} priority such that $q_m \in F$. (Note that a run $q_0 \sigma_1 q_1 \ldots \sigma_m q_m$ of $\pnfa$ on $w$ with the highest priority may not be accepting, i.e. satisfy $q_m \in F$.) The language of $\pnfa$, denoted as $\Lang(\pnfa)$, is the set of strings on which $\pnfa$ has an accepting run.


Note that PFAs differ from FAs only in the way that a string is accepted; they both define regular languages. 

%\begin{figure}[ht]
%\centering
%\rule{\linewidth}{0cm}
%\includegraphics[scale=0.8]{pfa-epsilon-star.pdf}
%\caption{The PFA for $\varepsilon^\ast$}
%\label{fig-pfa-epsilon-star}
%\end{figure}

\begin{example}\label{exmp-pfa}
The PFAs corresponding to $a^\ast$ and $a^{\ast?}$ respectively are illustrated in Figure~\ref{fig-pfa}(i) and (ii), where the dashed line represents $\pi_2(\tau(q_0))$ (of lower priority than the $a$-transition), the thicker solid line represents $\pi_1(\tau(q_0))$ (of higher priority than the $a$-transition), and the doubly circled state $q_1$ is a final state.

\begin{figure}[ht]
\centering
%\rule{\linewidth}{0cm}
\includegraphics[width=0.6\textwidth]{pfa.pdf}
\caption{The PFAs for $a^\ast$ and $a^{\ast?}$}
\label{fig-pfa}
\end{figure}
\end{example}

%\begin{remark}
%Remark that PFAs in Definition~\ref{def-pfa} are different from pNFAs in \cite{BM17} in the sense that the state set in a pNFA is partitioned into two disjoint subsets and the non-$\varepsilon$-transitions are deterministic, while this is not the case in PFAs. Therefore, PFAs are slightly more flexible than pNFAs in \cite{BM17}. We choose this definition of PFAs as a more natural extension of FAs. 
%\end{remark}

The priorities of PFAs are used to model the greedy and non-greedy semantics of $\regexp$. %, as we shall see in Section~\ref{construction:pnfa}.
%
%\paragraph{Prioritized streaming string transducers.}
%
We then introduce prioritized streaming string transducers, a new class of transducers that combine prioritized transducers \cite{BM17} %which combines the expressive power of 
and streaming string transducers \cite{AC10,AD11}.
  
\begin{definition}[Prioritized Streaming String Transducers]
A \emph{prioritized streaming string transducer} (PSST) is a tuple $\psst = (Q, \Sigma, X, \delta, \tau, E, q_0, F)$, where $Q$ is a
finite set of states, $\Sigma$ is the input and output alphabet, $X$ is a finite set of variables, $\delta \in Q \times \Sigma \rightarrow \overline{Q}$, $\tau \in Q \rightarrow \overline{Q} \times \overline{Q}$, $E$ is a partial function from $Q \times \Sigma^\varepsilon \times
  Q$ to $X \rightarrow \{\nullchar\} \cup (X \cup \Sigma)^{\ast}$, i.e. the set of assignments,
   $q_0 \in Q$ is the initial state, and $F$ is a partial function
  from $Q$ to $(X \cup \Sigma)^{\ast}$.
\end{definition}

A run of $\psst$ on a string $w$ is a sequence $q_0 \sigma_1 s_1 q_1 \ldots \sigma_m s_m q_m$ such that
\begin{itemize}
%\item $q_m \in F$,
%
\item for each $i \in [m]$, 
\begin{itemize}
\item either $\sigma_i \in \Sigma$, $q_i \in \delta (q_{i-1}, \sigma_i)$, and $s_i = E (q_{i - 1}, \sigma_i, q_i)$, 
\item or $\sigma_i = \varepsilon$, $q_i \in \tau(q_{i-1})$ and $s_i = E (q_{i - 1}, \varepsilon, q_i)$,
\end{itemize}

%\item for every subsequence $q_i \sigma_{i+1} s_{i+1} q_{i+1} \ldots \sigma_{j} s_j q_j$ such that  $i < j$ and $\sigma_{i+1} = \cdots = \sigma_j = \varepsilon$, it holds that $q_i, \ldots, q_j$ are mutually distinct. (Intuitively, loops of $\varepsilon$-transitions are forbidden.) 
\item for every subsequence $q_i \sigma_{i+1} s_{i+1} q_{i+1} \ldots \sigma_{j} s_j q_j$ such that  $i < j$ and $\sigma_{i+1} = \cdots = \sigma_j = \varepsilon$,  it holds that for every $k, l: i \le k < l < j$, $(q_k, q_{k+1}) \neq (q_l, q_{l+1})$.
\end{itemize}

%A run of $\psst$ is the sequence $q_0 \sigma_1 s_1 q_1 \ldots \sigma_m s_m q_m$, where $F (q_m)$ is defined and for each $i \in [m], q_i \in \delta (q_{i-1}, \sigma_i)$ and $s_i = E (q_{i - 1}, \sigma_i, q_i)$. 
For any pair of runs $p = q_0 \sigma_1 s_1 \ldots \sigma_m s_m q_m$ and $p' = q_0 \sigma'_1
  s_1' \ldots \sigma'_n s_n' q_n'$ such that $\sigma_1 \ldots \sigma_m = \sigma'_1 \ldots \sigma'_n$, the definition that $p$ is of a higher priority over
  $p'$ is similar to PFAs.
  % $p \neq p'$ and, for the smallest index $j$ with $q_j \neq q_j'$,
 % $\delta (q_{j - 1}, \sigma_j) = \ldots q_j \ldots q_j' \ldots$
  
An accepting run of $\psst$ on an input $w$ is a run of $\psst$ on $w$ with the highest priority, say $q_0 \sigma_1 s_1 \ldots \sigma_m s_m q_m$, such that $F(q_m)$ is defined. The output of $\psst$ on $w$, denoted by $\psst(w)$, is defined as $\eta_m(F(q_m))$, where $\eta_0(x) = \varepsilon$ for each $x \in X$, and $\eta_{i}(x) = \eta_{i-1}(s_{i}(x))$ for every $1 \le i \le m$ and $x \in X$. Note that here we abuse the notation $\eta_m(F(q_m))$ and $\eta_{i-1}(s_{i}(x))$ by taking a function $\eta$ from $X$ to $\{\nullchar\} \cup \Sigma^*$ as a function from $(X \cup \{\nullchar\} \cup \Sigma)^*$ to $\{\nullchar\} \cup \Sigma^*$, which maps each $\nullchar$ to $\nullchar$, $\sigma \in \Sigma$ to $\sigma$, and each $x \in X$ to $\eta(x)$. If there is no accepting run of $\psst$ on $w$, then $\psst(w) = \bot$, namely, the output of $\psst$ on $w$ is undefined. The string relation defined by $\psst$, denoted by $\cR_\psst$,  is $\{(w, \psst(w)) \mid w \in \Sigma^\ast, \psst(w) \neq \bot\}$.

\begin{example}
The PSST $\cT_{\tt match_{decimalReg,1}}=(Q, \Sigma, X, \delta, \tau, E,  q_{0}, F)$ mentioned in Section~\ref{sec:mot} is illustrated in Figure~\ref{fig-psst-exmp}, where $Q = \{q_0, \dots, q_{6}\}$, $\Sigma = \{0,\cdots,9, .\}$, $X= \{x_1\}$ with $x_1$ recording the matches of the 1st capturing group, $F(q_{6}) = x_1$ denotes the final output, and $\delta, \tau, E$ are illustrated by the edges, where the dashed/ticker solid edges denote the $\varepsilon$-transitions of lower/higher priorities than the non-$\varepsilon$-transitions and the symbol $\ell$ is used to denote the currently scanned input letter. For instance, $\delta(q_1, \ell) = (q_1)$ for every $\ell \in \{0, \dots, 9\}$, $\delta(q_1, .) = ()$, $\tau(q_1) = ((); (q_2))$, and $E(q_1, \ell, q_1)(x_1) = x_1 \ell$. Since the $\varepsilon$-transition has lower priority than the $\ell$-transition at the state $q_1$, whenever the currently scanned letter is $\ell \in \{0,\cdots,9\}$ at $q_1$,  $\cT_{\tt match_{decimalReg,1}}$ will choose to go to $q_1$ greedily, until there is no more $\ell  \in \{0,\cdots,9\}$. (In this case, it has to choose the $\epsilon$-transition and goes to $q_2$.) Note that the identity assignments, e.g. $E(q_3, ., q_4)(x_1) = x_1$, are omitted in Figure~\ref{fig-psst-exmp}, for readability. 
%From $\delta(q_4, \backslash s) = q_5q_{6}$, we know that $q_5$ is prior to $q_6$. 
%Therefore, whenever $\cT_{\sf nameReg}$ reads $\backslash$s at the state $q_3$,  it will choose to go the state $q_5$ greedily, unless this choice would lead to the nonacceptance (in this case, $q_6$ will be chosen). 
\begin{figure*}[ht]
\centering
%\rule{\linewidth}{0cm}
\includegraphics[width=0.95\textwidth]{psst-epsilon-exmp-new.pdf}
\caption{The PSST $\cT_{\tt match_{decimalReg,1}}$}
\label{fig-psst-exmp}
\end{figure*}
\end{example}

  
%  $\tmop{Out} (r) =
%  s_{\varepsilon} \circ s_1 \circ s_2 \ldots s_n \circ F (q_n)$ where
%  $s_{\varepsilon}$ is the empty substitution which maps all variables to
%  $\varepsilon$.
  

% Note that in the definition of \NSST, there is no \emph{copyless} restriction.



