%!TEX root = main.tex

\section{Introduction}


% general intro on string constraint solving

%
%Strings are among the most important data types. 
Modern high-level programming languages like JavaScript, Python, Java,
and PHP natively support a variety of string operations, and use
strings to store and process virtually any kind of data or code.
Applied string operations range from concatenation, splitting, and
replacement, to complex functions like regular expression matching and
encoding/decoding.  As a result, string-manipulating programs are
notoriously subtle, error-prone, and their potential bugs may bring
severe security consequences. A typical example is cross-site
scripting (XSS), which is among the OWASP Top 10 Application Security
Risks.
%Regular expressions are widely used in string-manipulating programs. 
An effective and increasingly popular method for identifying such bugs
in the program is symbolic execution, possibly in combination with dynamic
analysis. In a nutshell, this technique analyses a static path in a
a program under consideration, by viewing it as a constraint $\phi$, whose 
feasibility can be checked by constraint solvers.

Regular expression matching is one of the most important string operations
in programming languages \cite{Berkeley-JavaScript,BM17,LMK19,HAMPI}.
Most state-of-the-art string constraint solvers (e.g.
Z3, CVC4, Z3-str/2/3/4, ABC, Norn, Trau, OSTRICH, S2S, Qzy, Stranger, Slog, Slent, Gecode+S, G-Strings, ... \anthony{make sure we add all}) therefore support
the \emph{regular expression constraints}, e.g., matching a string with a 
regular expression, as we know it from formal language theory. Unfortunately, 
\emph{Real-world Regular Expressions} (RWRE) in programming languages are dramatically different from 
\emph{classical Regular Expressions} (RE) in formal language theory. 
%In the sequel, we call the former as \emph{real-word} regular expressions and the latter as \emph{classical} regular expressions. 
Classical regular expressions are built from letters by the operators of concatenation, union, and Kleene star, and have nice compositional semantics. RWREs differ from classical ones mainly in the following two aspects: 1) non-standard semantics of 
operators, e.g., the non-commutative union, the greedy/lazy Kleene star, and 2) new 
features, e.g., capturing groups and back-references.

%The semantics of RWREs are tricky and can be different in different programming languages. 
%Real-world regular expressions are challenging for string constraint solvers. The state-of-the-art string constraint solvers e.g. CVC4 and Z3-str only support classical regular expressions. 
RWREs are in general more expressive than classical REs, e.g., it is known that
with back references one can easily generate languages that are not even 
context-free (e.g. see \cite{FS19}). %It is an open question whether 
Since the state-of-the-art string solvers support only classical REs instead of
RWREs, % are not primitively supported by state-of-the-art string solvers
%(in fact, they are in general ,
existing symbolic execution approaches that handle
string-manipulating programs with RWREs apply a workaround.
We mention here Aratha \cite{aratha} and Expose \cite{LMK19}, both of which are
symbolic execution engines for JavaScript programs.
%symbolic executors of string manipulating programs, e.g. 
Aratha performs a very rough approximation to the 
non-standard semantics of regular expressions, e.g., a back reference
is replaced by the regular expression that accepts every word.
%referred to by the backreference 
%operator. 
On the other hand, Expose attempts to exploit the power of string 
equations and classical REs (as implemented in Z3 \cite{Z3}) supported by string
solvers to capture the 
semantics of RWREs. Unfortunately, the semantics of RWREs cannot 
in general be fully captured by string constraints with REs. 
%This is caused by
%the aforementioned features of RWREs: greedy semantics
%, especially in the
%presence of the greedy semantics of backreferences. 
%It is even an open
%question if even the greedy semantics of RWREs have to be 
For this reason, 
Expose opts to approximate the semantics of RWREs in the style of 
CEGAR (counter-example guided abstraction and refinement). This, however,
results in a rather severe price in both precision and performance: The refinement process may not terminate and the symbolic execution of a simple program with RWREs may need to be refined for tens of times. 
%For one, satisfiability of string equations with regular constraints is
%well-known to be PSPACE-complete \cite{J16,Kozen77,P04}. For another, to the 
%best of our knowledge, no existing string solver is complete for string 
%equations with regular constraints.

%Typical string operations involving RWREs in programming languages include match, exec, test, search/find, and replace.

RWREs are commonly used in combination with
other string operations, e.g. match and replace(all) functions \cite{LMK19}.  
On a given string $s$ and a RWRE $e$, the match function allows one to extract the match of a capturing group from the first match of $e$ in $s$. For the replace function, on a given string $s$, a matching pattern RWRE $e$, and a replacement string $t$, it replaces the first match (or all 
matches, if the global flag is enabled) of $e$ in $s$ by $t$. Here $t$
could contain references to the matches of various capturing groups
in $e$. 
The semantics of RWREs drastically affect the behaviors of these functions. In particular, one must take a special care of the
greedy/lazy semantics of Kleene star, which cannot 
be captured in a complete way as constraints over word equations and classical 
REs. 
\anthony{More to come}


%In particular, 
 %to approach the genuine semantics of real-world regular expressions. 
%Although the CEGAR approach of Expose made a first step towards tackling the semantics of real-world regular expressions in the analysis and verification of string-manipulating programs, it is still unsatisfactory in both the precision and performance: 1) although CEGAR approximates the semantics of real-world regular expressions to a greater precision, it is still imprecise, 2) tens of refinement steps or even more are needed for simple string-manipulating programs containing regular expressions. Direct support of real-world regular expressions in string constraint solvers would facilitate the improvement of both the precision and scalability of symbolic executions of string manipulating programs.

\paragraph*{Contributions}
This paper proposes a novel approach to support real-world regular expressions in string constraint solving. The main idea of our approach is to propose a new automata model, called prioritized streaming string transducers (PSST). The model of PSST extends and combines prioritized finite-state automata \cite{BM17} and streaming string transducers \cite{AC10,AD11}. With PSST, we encode the non-standard semantics of regular expression operators by priorities and deal with capturing groups and back references by string variables. 
The widely used string functions involving regular expressions, e.g. match and replace(all), can be easily transformed into PSSTs. 

We then design a decision procedure for a class of string constraints with real-world regular expressions. The decision procedure extends the backward reasoning approach proposed in \cite{CHL+19} to PSSTs. Specifically, we show that the pre-images of regular languages under PSSTs are regular and can be computed effectively. 

We implement the decision procedure in our new solver \ostrich,
on top of the existing open-source solver~OSTRICH~\cite{CHL+19},
 and carry out extensive experiments to evaluate the performance. For the benchmarks, we generate \zhilin{xxx} Javascript programs from a library of real-world regular expressions \cite{DMC+19}, by using some simple Javascript program template containing match and replace functions.  Then we generate all the path constraints for each Javascript program and put them into one SMT file. We run {\ostrich} on these SMT files. The average running time on each file is \zhilin{xxx} seconds. For comparison, we also run Expose on the Javascript programs. The average running time for each program is \zhilin{xxx} seconds. The huge difference of the running time shows that our approach can reason about RWREs in a much more precise and faster way than the CEGAR-based approach.


\paragraph*{Related work.}

RWREs have been investigated in formal language theory. Regular expressions with capturing groups and back references have been considered in \cite{CSY03,CN09,Freydenberger13,Schmid16,FS19}, where the expressibility issues and decision problems were investigated. Nevertheless, some basic features of RWREs, namely, the non-commutative union and the greedy/lazy semantics of Kleene star, were not addressed therein.

Prioritized finite-state automata and prioritized finite-state transducers were proposed in \cite{BM17}. Prioritized finite-state transducers add indexed brackets to the input string in order to identify the matches of capturing groups. In contrast, PSSTs store the matches of capturing groups into string variables, which can then be referred to later on, thus allowing to conveniently model the match and replace(all) function. 
%
Streaming string transducers were used in \cite{ZAM19} to solve the straight-line string constraints with concatenation, finite-state transducers, and regular constraints.

RWREs have also received the attentions in software engineering community. Some empirical studies were reported for RWREs recently, including portability across different programing languages \cite{DMC+19} and DDos attacks \cite{SP18}, as well as how programmers write RWREs in practice \cite{MDD+19}.

%%%%%%%%%%%%%%%%%%%%%%%%%%%%%%%%%%%%%%%%%%%%%%%%%%
%%%%%%%%%%%%%%%%%%%%%%%%%%%%%%%%%%%%%%%%%%%%%%%%%%
\hide{
Strings are a fundamental data type in virtually all programming languages.
%Their generic nature can, however, lead to many subtle programming
%bugs, some with security consequences, e.g., cross-site scripting
%(XSS), which is among the OWASP Top 10 Application Security Risks
%\cite{owasp17}. 

One effective automatic testing method for identifying subtle programming errors  is based on \emph{symbolic execution}
\cite{king76} and combinations with dynamic analysis
called \emph{dynamic symbolic execution} \cite{jalangi,DART,EXE,CUTE,KLEE}.
See \cite{symbex-survey} for an excellent survey. 

Unlike purely random testing,
which runs only \emph{concrete} program executions on different
inputs, the techniques of symbolic execution analyse \emph{static} paths
(also called symbolic executions) through the software system under test.
Such a path can be viewed as a constraint $\varphi$ (over
appropriate data domains) and the hope is that a fast
solver is available for checking the satisfiability of $\varphi$ (i.e. to check
the \emph{feasibility} of the static path), which can be used for generating
inputs that lead to certain parts of the program or an erroneous behaviour.
%a undesirable program behaviour.
%or an exploration of a new part of the
%system.


%
In this paper, we focus on two string operations with emphasis on practical usage of  regular expressions. Rather than textbook style regular expressions, regular expressions used in programming languages are considerably more involved. On particular feature we consider is the capturing group. This is particularly useful for string pattern matching 
%Many regular expression matching libraries perform matching as a form of parsing by using capturing groups,and thus 
where it can be returned what subexpression matched which substring. 

%This form of regular expression matching requires theoretical un-derpinnings different from classical regular expressions as defined in formal language theory. 


%which effective serves as a register when matching the regular expression to a string. Accompanying to the capturing groups 


%Many regular expression matching libraries perform matching as a form of parsing by using capturing groups,and thus output what subexpression matched which substring[9]. This form of regular expression matching requires theoretical un-derpinnings different from classical regular expressions as defined in formal language theory. 
%
%A popular implementation strategy used for performing regular expression matching (or parsing) with capturing groups, used for example in Java, .NET and the PCRE library[14], is a worst-case exponential time depth-first search strategy. A formal approach to matching with capturing groups can be obtained by using finite state transducers that output annotations on the input string to signify what subexpression matched which substring[16]. 
%
%A complicating factor in this approach is introduced by the fact that the matching semantics dictates a single output string for each input string, obtained by using rules to determine a “highest priority” match among the potentially exponentially many possible ones (in contrast,[6]discusses non-deterministic capturing groups).

The \emph{string-replace function}, 
which may be used to replace all occurrences of a string matching a pattern by 
another string. 

The replace function (especially 
the replace-all functionality) is omnipresent in HTML5 applications
\cite{LB16,TCJ16,YABI14}. 
%\mat{What does it mean for the replace function to be convincingly argued?}

A regular expression (shortened as regex) is a sequence of characters that define a search pattern. Usually such patterns are used by string-searching algorithms for "find" or "find and replace" operations on strings, or for input validation.  

The semantics of regular expressions with capturing groups and back references is rather involved. One of the reasons is that different languages may choose different semantics for a regex to match the string when the regex is served as a pattern. 

To capture the semantics, priority is introduced, giving rise to an extension of the standard finite-state automata. However, this is not sufficient for capturing string operations. For that purpose, we introduce  a new transducer model, prioritized streaming string transducer (PSST) which is a combination of priority which is essential for modelling capturing groups and streaming transducers which are a highly expressive formalism for modelling string operations. 
}
%%%%%%%%%%%%%%%%%%%%%%%%%%%%%%%%%%%%%%%%%%%%%%%%%%
%%%%%%%%%%%%%%%%%%%%%%%%%%%%%%%%%%%%%%%%%%%%%%%%%%
