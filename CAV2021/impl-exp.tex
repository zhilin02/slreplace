%!TEX root = main.tex

\section{Implementation and Experiments}
\label{sect:impl}

We have implemented our decision procedure for $\strline_{\sf reg}$ in the SMT
solver \ostrich.\footnote{Name anonymized for doubly-blind review,
and will be provided in the final version.}
%\cite{CHL+19}, %\url{https://github.com/uuverifiers/ostrich}}
%which provides a modular and easy-to-use framework for extending all
%sorts of string operations. 
As shown in Section \ref{sec:decision},
PSSTs satisfy the conditions required by the backward reasoning
approach of \ostrich, which enables us to integrate our logic with
standard string theory. The resulting extended theory of strings is a
conservative extension of the SMT-LIB theory of Unicode
strings.\footnote{\url{http://smtlib.cs.uiowa.edu/theories-UnicodeStrings.shtml}}

\subsection{Input format}

Our implementation extends classical regular expression in SMT-LIB
with indexed {\sf re.capture} and {\sf re.reference} operators, which
denote capturing groups and references to them. We also add {\sf re.*?}
and {\sf re.+?} for lazy Kleene stars.

The three string operators that use these extended real world regular
expressions are {\sf str.replace\_cg}, {\sf str.replace\_cg\_all}, and
{\sf str.extract}. Operators {\sf str.replace\_cg} and {\sf
  str.replace\_cg\_all} are counterparts of the standard {\sf
  str.replace\_re} and {\sf replace\_re\_all} operators, and allow
capturing groups in the match pattern and references in the
replacement pattern. As an example, the following constraint swaps
lower-case characters~$x$ with a sub-sequent upper-case character~$Y$:
%
\begin{minted}{lisp}
  (= w (str.replace_cg_all v
                   (re.++ ((_ re.capture 1) (re.range "a" "z"))
                          ((_ re.capture 2) (re.range "A" "Z")))
                   (re.++ (_ re.reference 2) (_ re.reference 1))))
\end{minted}
%
The replacement string is written as a regular expression only
containing the operators {\sf re.++}, {\sf str.to\_re}, and {\sf re.reference}. In contrast to the standard operators, it is not allowed to use string variables in the 
replacement parameter.

The indexed operator {\sf str.extract} implements $\extract_{i, e}$ in
$\strline$. For instance,
%
\begin{minted}{lisp}
  ((_ str.extract 1)
        (re.++ (re.*? re.allchar)
               ((_ re.capture 1) (re.+ (re.range "a" "z"))
               re.all)) x)
\end{minted}
%
extracts the left-most, longest sub-string consisting of only lower-case
characters from a string~$x$.

Our implementation is able to handle \textit{anchors} as well. Due to space restrictions we did not introduce them as part of our formalism. Basically, anchors are special symbols that match certain positions of a string without consuming any characters. In most practical programming languages, it is quite common to use \verb!^! and \verb!$! in regular expressions to signify the start and end of a string, respectively. We add \textsf{re.begin-anchor} and \textsf{re.end-anchor} for them.

\subsection{Implementation}

Our implementation revolves around PSSTs. Any of the three string operators mentioned above will be converted into an equivalent PSST (See Appendix~\ref{appendix:sec-extract-replace-to-psst}). {\ostrich} then iterates the dependency graph and repeatedly eliminates the them. More specifically, we first use Lemma \ref{lem:psst_preimage} to back-propagate regular constraints on string variables and check satisfiability, then use forward concrete evaluation to generate a model. 

\subsection{Experimental evaluation}

\begin{table}[tb]
	\begin{center}
	\begin{tabular}{|l@{\quad}c@{\quad}|*{2}{c}|@{\quad}*{2}{c}|}
	\hline
	   & &
	  \multicolumn{2}{c|@{\quad}}{\textbf{ExpoSE+Z3}} &\multicolumn{2}{c@{\quad}|}{\textbf{Aratha+\ostrich}}
	  \\
	   & \#JS-Progs. & ~ \# Executed~  & ~ \#Timeout~ &  ~\#Executed ~ & ~\#Timeout~
	  \\\hline
	  \textbf{ExpoSE} & 94 & 93 & 1 & \textbf{94} & 0 
	  \\
	  && \multicolumn{2}{c|@{\quad}}{Average time: 17.65s} & \multicolumn{2}{c@{\quad}|}{Average time: \textbf{5.53}s}
	  \\\hline
	  \textbf{Match} & 38 &  30&   8& \textbf{38} & 0
	  \\
	  && \multicolumn{2}{c|@{\quad}}{Average time: 13.56s} & \multicolumn{2}{c@{\quad}|}{Average time: \textbf{2.78}s}
	  \\\hline
	  \textbf{Replace} & 40 & 16 & 24 & \textbf{40} & 0
	  \\
	  && \multicolumn{2}{c|@{\quad}}{Average time: 17.65s} & \multicolumn{2}{c@{\quad}|}{Average time: \textbf{2.69}s}
	  \\\hline
	\end{tabular}
	\end{center}
	\caption{Results of Expose+Z3 and Aratha+{\ostrich} on Javascript programs. Average time is wall-clock time. All experiments were done on an Intel-Xeon-E5-2690-@2.90GHz machine, running 64-bit Linux and Java 1.8. Runtime was limited to 1min wall-clock time. }
	\label{tab:exp-r2}

  \begin{center}
  \begin{tabular}{l@{\quad}c@{\quad}|*{6}{c}|@{\quad}c}
     & &
    \multicolumn{6}{c|@{\quad}}{\textbf{\ostrich}} & \textbf{ExpoSE}
    \\
      && \multicolumn{6}{c|@{\quad}}{\# solved queries}
    \\
     & \#Benchm.& ~~0~~ & ~~1~~ & ~~2~~ & ~~3~~ & ~~4~~ & ~~\#Err~~
    \\\hline
    \textbf{Match} & 98117 & 75 & 88 & 176 & 79 & 96565 & 1132
    \\
    && \multicolumn{6}{c|@{\quad}}{Average time: 0.63s}
    \\
    && \multicolumn{6}{c|@{\quad}}{Total \#sat: 108132, \#unsat: 279135}
    \\\hline
    \textbf{Replace} & 98117 & 445 & 229 & 576 & 95735 & --- & 1132
    \\
    && \multicolumn{6}{c|@{\quad}}{Average time: 1.02s}
    \\
    && \multicolumn{6}{c|@{\quad}}{Total \#sat: 273927, \#unsat: 14659}
  \end{tabular}
  \end{center}
  \caption{Results of running \ostrich\ and ExpoSE on regular
    expressions.  Average time is wall-clock time. All experiments
    were done on an AMD Opteron 2220 SE machine, running 64-bit Linux
    and Java~1.8.  Runtime was limited to 1min wall clock time, and
    memory to 2GB.}
  \label{tab:exp-r1}
\end{table}

Our experiments have the purpose of answering the following main questions:\\
\textbf{R1:} Are the semantics of RWREs defined in this paper
suitable to encode regular expressions in programming languages,
for instance ECMAScript expressions~\cite{ECMAScript10}?
\\
\textbf{R2:} How does \ostrich\ compare to other solvers that can
handle real-world regular expressions, e.g., greedy/non-greedy
quantifiers and capturing groups?
\\
\textbf{R3:} How does it perform in the context of symbolic execution,
the prime application of string solving?

For \textbf{R1}, we apply an additional post-processing step to
Aratha+{\ostrich} in the experiments for \textbf{R3}, that is, when
Aratha reports that some path is feasible and {\ostrich} returns a
model, the Javascript program is executed concretely, with the model
as the input, in order to confirm that the concrete execution indeed
follows the targeted path, thus showing that the semantics in this
paper are consistent with that of Javascript.

Concerning \textbf{R2}, we are not aware of other SMT solvers with
support for capturing groups; the same holds for CP solvers (e.g.,
\cite{DBLP:conf/cpaior/ScottFPS17,DBLP:conf/cp/AmadiniGS20}). The
closest related work is the algorithm implemented in the symbolic
execution tool ExpoSE~\cite{DBLP:conf/spin/LoringMK17,LMK19}, which
applies Z3-seq~\cite{Z3} for solving string constraints, but augments
it with a refinement loop to approximate the semantics of
replace(All).\footnote{We also considered to replace Z3 with \ostrich\
  in ExpoSE for the experiments. However, ExpoSE integrates Z3 using
  its C API, and changing to a solver like \ostrich, with native
  support for capture groups, would have required the rewrite of
  substantial parts of ExpoSE.}
%
For \textbf{R2}, we therefore compared \ostrich\ with ExpoSE on
regular expressions taken from the ??? library.

For \textbf{R3}, we plug {\ostrich} into Aratha and compare it with ExpoSE, on the collection of Javascript programs tested by ExpoSE, as well as some other Javascript programs containing match or replace functions extracted from Github. The experimental results are summarized in Table~\ref{tab:exp-r2}. From the results, we can see that on average, Aratha+{\ostrich} is faster than ExpoSE+Z3 by one order of magnitude. Therefore, in the context of symbolic execution, {\ostrich} is much more efficient than the CEGAR-augmented symbolic execution, for dealing with RWREs.
