%!TEX root = main.tex

\section{Implementation and Experiments}
\label{sect:impl}

We have implemented our decision procedure for $\strline$ in the SMT solver OSTRICH \cite{CHL+19}, which provides a modular and easy-to-use framework for extending all sorts of string operations. As shown in Section \ref{sec:decision}, \PSST s satisfy the conditions required by the backwards reasoning approach of OSTRICH, which enables us to integrate our logic with standard string theory.

Our implementation extends classical regular expression in SMTLIB with {\sf re.capture} and {\sf re.reference} operators, which signify capturing groups and reference to them. We also add {\sf re.*?} and {\sf re.+?} for nongreedy Klenne stars.

Three novel string operators that use these extended real world regular expressions are also added:

\begin{enumerate}
  \item {\sf str.replace\_cg} and {\sf str.replace\_cg\_all} are
  counterparts of standard {\sf str.replace\_re} and
  {\sf replace\_re\_all} operators, which allows using
  capturing groups in the match pattern and references
  in the replacement pattern. However, it's not allowed to use symbolic
  variable as replacement pattern, like the standard operators did.
  
  \item {\sf str.extract} behaves like {\sf match} function in
  javascript. It performs a match of a real world regular expression against
  the input string, and extracts the content of a capturing group.
\end{enumerate}

Our implementation is able to handle \textit{anchors} as well. Due to space restrictions we did not introduce them as part of our formalism. Basically, anchors are special symbols that match certain positions of a string without consuming any characters. In most practical programming languages, it's quite common to use \^{} and $\$$ in regular expressions to signify the start and end of a string, respectively. We add \textsf{re.begin-anchor} and \textsf{re.end-anchor} for them.

We evaluated our implementation in XXX ...

\zhilei{stopped here}
