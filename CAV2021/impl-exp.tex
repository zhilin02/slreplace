%!TEX root = main.tex

\section{Implementation and Experiments}
\label{sect:impl}

We have implemented our decision procedure for $\strline_{\sf reg}$ in the SMT
solver \ostrich,\footnote{Name anonymized for doubly-blind review,
and will be provided in the final version.} extending the
open-source solver OSTRICH~\cite{CHL+19}.
%\cite{CHL+19}, %\url{https://github.com/uuverifiers/ostrich}}
%which provides a modular and easy-to-use framework for extending all
%sorts of string operations. 
As shown in Section \ref{sec:decision},
PSSTs satisfy the conditions required by the backward reasoning
approach of OSTRICH, which enables us to integrate our logic with
standard string theory. The resulting extended theory of strings is a
conservative extension of the SMT-LIB theory of Unicode
strings.\footnote{\url{http://smtlib.cs.uiowa.edu/theories-UnicodeStrings.shtml}}

\subsection{Implementation}

Our implementation extends classical regular expressions in SMT-LIB
with indexed {\sf re.capture} and {\sf re.reference} operators, which
denote capturing groups and references to them. We also add {\sf re.*?}
and {\sf re.+?} for lazy Kleene stars.

The three string operators that use these extended real world regular
expressions are {\sf str.replace\_cg}, {\sf str.replace\_cg\_all}, and
{\sf str.extract}. Operators {\sf str.replace\_cg} and {\sf
  str.replace\_cg\_all} are counterparts of the standard {\sf
  str.replace\_re} and {\sf replace\_re\_all} operators, and allow
capturing groups in the match pattern and references in the
replacement pattern. As an example, the following constraint swaps the first name and the last name (see Example~\ref{exmp-name-swap}),
%lower-case characters~$x$ with a sub-sequent upper-case character~$Y$:
%
{\small
\begin{minted}{lisp}
(= w (str.replace_cg_all v (re.++  
 ((_ re.capture 1)(re.+ (re.union (re.range "A" "Z")(re.range "a" "z"))))
 (str.to.re " ") 
 ((_ re.capture 2)(re.+ (re.union (re.range "A" "Z")(re.range "a" "z")))))
 (re.++ (_ re.reference 2) (_ re.reference 1))))
\end{minted}
}
%
The replacement string is written as a regular expression only
containing the operators {\sf re.++}, {\sf str.to\_re}, and {\sf re.reference}. In contrast to the standard operators, using string variables in the 
replacement parameter is not allowed.

The indexed operator {\sf str.extract} implements $\extract_{i, e}$ in
$\strline_{\sf reg}$. For instance,
%
{\small
\begin{minted}{lisp}
  ((_ str.extract 1)
        (re.++ (re.*? re.allchar)
               ((_ re.capture 1) (re.+ (re.range "a" "z")) re.all)) x)
\end{minted}
}
extracts the left-most, longest sub-string consisting of only lower-case
characters from a string~$x$.

Our implementation is able to handle \textit{anchors} as well. Due to space restrictions, we did not introduce them as part of our formalism. Anchors are special symbols that match certain positions of a string without consuming any characters. In most practical programming languages, it is quite common to use \verb!^! and \verb!$! in regular expressions to signify the start and end of a string, respectively. We add \textsf{re.begin-anchor} and \textsf{re.end-anchor} for them.

The implementation revolves around PSSTs. Any of the three string operators mentioned above will be converted into an equivalent PSST (See Appendix~\ref{appendix:sec-extract-replace-to-psst}). {\ostrich} then iterates the dependency graph and repeatedly eliminates them. More specifically, we first use Lemma \ref{lem:psst_preimage} to back-propagate regular constraints on string variables and check satisfiability, and then use forward concrete evaluation to generate a model. We omit the details here due to space restrictions, and refer the readers to the baseline \cite{CHL+19}.

\subsection{Experimental evaluation}

\begin{figure}[tb]
  \scriptsize

  \begin{minipage}{0.49\linewidth}
\begin{minted}{lisp}
(declare-fun x () String)
(define-fun  y () String
   (str.replace_cg_all x <re1> <repl>))
(push 1)
(assert (str.in.re x 
          (re.++ re.all <re1> re.all)))
(assert (str.in.re y 
          (re.++ re.all <re2> re.all)))
(check-sat) (get-model)
(pop 1) (push 1)
(assert (str.in.re x 
          (re.++ re.all <re1> re.all)))
(assert (not (str.in.re y 
          (re.++ re.all <re2> re.all))))
(check-sat) (get-model)
(pop 1) (push 1)
(assert (not (str.in.re x 
          (re.++ re.all <re1> re.all))))
(check-sat) (get-model)
(pop 1)
\end{minted}
  \end{minipage}\hfill
  \raisebox{-25ex}{\rule{0.4pt}{52ex}}\hfill
  \begin{minipage}{0.49\linewidth}
\begin{minted}{js}
function fun(x) {
   if(/<re1>/.test(x)) {
      var y = x.replace(/<re1>/g, <repl>);
      if(/<re2>/.test(y))
        console.log("1");
      else
        console.log("2");
   }
   else
       console.log("3");
}

var S$ = require("S$");
var x = S$.symbol("x", "");
fun(x);
\end{minted}
  \end{minipage}
  
  \caption{Harnesses with replace-all: SMT-LIB for \ostrich,
    and JavaScript for \expose{}.}
  \label{fig:harness}
  \vspace{-2mm}
\end{figure}

Our experiments have the purpose of answering the following main questions:

\medskip
\noindent
\textbf{R1:} Are the  RWREs defined in this paper
suitable to encode regular expressions in programming languages,
for instance ECMAScript regular expressions~\cite{ECMAScript11}?
\\
\textbf{R2:} How does \ostrich\ compare to other solvers that can
handle real-world regular expressions, e.g., greedy/lazy
quantifiers and capturing groups?
\\
\textbf{R3:} How does \ostrich\ perform in the context of symbolic execution,
the primary application of string constraint solving?

%\vspace{-2mm}

\paragraph{For \textbf{R1}:} We implemented a translator from ECMAScript 11th Edition (ES11 for short) regular
expressions to RWREs, and integrated \ostrich\ into the symbolic
execution tool Aratha~\cite{aratha}. We then ran Aratha+\ostrich\ on
the regression test suite of \expose{}~\cite{DBLP:conf/spin/LoringMK17},
as well as some other JavaScript programs containing match or replace
functions extracted from Github. To verify the soundness of
Aratha+\ostrich, we compared the results with those produced by
\expose{}; we also checked the correctness of models computed by
\ostrich\ by concretely executing the JavaScript program under test on
the generated inputs, to confirm that the concrete execution indeed
follows the targeted path. The results are summarized in Table~\ref{tab:exp-r1};
no inconsistencies were observed in the experiments, showing that the
semantics in this paper are indeed suitable for capturing ES11
semantics.

\begin{table}[htb]
\vspace{-2mm}
	\begin{center}
	\begin{tabular}{|l@{\quad}|*{6}{c}|*{6}{c}|}
	\hline
	   & 
	  \multicolumn{6}{c|}{\textbf{Aratha+\ostrich}} &\multicolumn{6}{c|}{\textbf{\expose{}+Z3}}
	  \\
    & 
	  \multicolumn{6}{c|}{~\# paths covered within 60s~~} &\multicolumn{6}{c@{\quad}|}{~\# paths covered within 60s}
	  \\
	   & ~~0~  & ~1~ &  ~2~ & ~3~ & ~$\geq$4~ & ~\#T.O.~ &
	    ~~0~  & ~1~ &  ~2~ & ~3~ & ~$\geq$4~ & ~\#T.O.~
	  \\\hline
	  \textbf{\expose{}} &  14 & 9 & 9 & 2 & 15 & 0  & 15 & 8 & 9 & 2 & 15 & 1  
	  \\
	  \emph{(49 programs)} & \multicolumn{6}{c|}{Average time: \textbf{6.49}s} & \multicolumn{6}{c@{\quad}|}{Average time: 13.46s}\\
	  & \multicolumn{6}{c|}{Total \#paths covered:124}  & \multicolumn{6}{c@{\quad}|}{Total \#paths covered:120}	  
	  \\\hline
	  \textbf{Match} & 3 & 7 & 12 & 6 & 0 & 0  & 3 & 8 & 12 & 5 & 0 & 6
	  \\
	  \emph{(28 programs)} & \multicolumn{6}{c|}{Average time: \textbf{4.30}s} & \multicolumn{6}{c@{\quad}|}{Average time: 23.66s}\\
	  & \multicolumn{6}{c|}{Total \#paths covered: 49}  & \multicolumn{6}{c@{\quad}|}{Total \#paths covered: 47}	  
	  \\\hline
	  \textbf{Replace} & 12 & 20 & 6 & 0 & 0 & 0  & 12 & 23 & 3 & 0 & 0 & 22
	  \\
	  \emph{(38 programs)} & \multicolumn{6}{c|}{Average time: \textbf{2.71}s} & \multicolumn{6}{c@{\quad}|}{Average time: 41.73s}\\
	  & \multicolumn{6}{c|}{Total \#paths covered: 32}  & \multicolumn{6}{c@{\quad}|}{Total \#paths covered: 29}	  
	  \\\hline
	\end{tabular}
	\end{center}
	\caption{Results of Expose+Z3 and Aratha+{\ostrich} on Javascript programs for \textbf{R1} and \textbf{R3}. Experiments were done on an Intel-Xeon-E5-2690-@2.90GHz machine, running 64-bit Linux and Java 1.8. Runtime was limited to 60s wall-clock time. Average time is
    wall-clock time needed per benchmark, and counts timeouts as 60s. \#T.O. is the number of timeouts. Note that some paths may have already been covered before T.O. }
	\label{tab:exp-r1}
\vspace{-1.4cm}	
\end{table}


\paragraph{For \textbf{R2}:} There are no standard string benchmarks
involving RWREs, and we are not aware of other constraint solvers
supporting capturing groups, neither among the SMT nor the CP
solvers. %(e.g.,
%\cite{DBLP:conf/cpaior/ScottFPS17,DBLP:conf/cp/AmadiniGS20}).
The closest related work is the algorithm implemented in \expose{}, which
applies Z3~\cite{Z3} for solving string constraints, but augments
it with a refinement loop to approximate the RWRE
semantics.\footnote{We also considered to replace Z3 with \ostrich\ in
  \expose{} for the experiments. However, \expose{} integrates Z3 using its
  C API, and changing to a solver like \ostrich, with native support
  for capture groups, would have required the rewrite of substantial
  parts of \expose{}.}
%
For \textbf{R2}, we compared \ostrich\ with \expose{} on 98,117
RWREs taken from \cite{DMC+19}.

For each of the regular expressions, we created four harnesses: two in
SMT-LIB, as inputs for \ostrich, and two in JavaScript, as inputs for
\expose{}. The two harnesses shown in Fig.~\ref{fig:harness} use one of the
regular expressions from \cite{DMC+19} (\verb!<re1>!) in combination with
the replace-all function to simulate typical string processing;
\verb!<re2>! is the fixed pattern \verb![a-z]+!, and \verb!<repl>! the
replacement string \verb!"$1"!. The three paths of the JavaScript
function~\verb!fun! correspond to the three queries in the SMT-LIB
script, so that a direct comparison can be made between the results of
the SMT-LIB queries and the set of paths covered by \expose{}. The other
two harnesses are similar to the ones in Fig.~\ref{fig:harness}, but
use the match function instead of replace-all, and contain four
queries and four paths, respectively.

\begin{table}[htb]
\vspace{-3mm}
  \begin{center}
  \begin{tabular}{|l@{~~}|*{6}{c}|*{5}{c}@{~~}|}
    \hline
     & 
    \multicolumn{6}{c|}{\textbf{\ostrich}} &
    \multicolumn{5}{c|}{\textbf{\expose{}+Z3}}
    \\
      & \multicolumn{6}{c|}{\# queries solved within 60s}
      & \multicolumn{5}{c|}{\# paths covered within 60s}
    \\
     & ~~0~~ & ~~1~~ & ~~2~~ & ~~3~~ & ~~4~~ & ~~\#Err~~
     & ~~0~~ & ~~1~~ & ~~2~~ & ~~3~~ & ~~4~~
    \\\hline
    \textbf{Match}  & 172 & 92 & 61 & 534 & 96,126 & 1132
    & ~3,333 & 9,274 & 36,916 & 48,594 &
    \\
     \emph{(98,117} & \multicolumn{6}{c|}{Average time: 1.19s}
    &\multicolumn{5}{c|}{Average time: 28.0s}
    \\
    \emph{~~benchm.)} & \multicolumn{6}{c|}{Total \#sat: 249,975, \#unsat: 136,345}
    & \multicolumn{5}{c|}{Total \#paths covered: 228,888}
    \\\hline
    \textbf{Replace} & 445 & 229 & 576 & 95,735 & --- & 1,132
    & ~5,281 & 18,221 & 69,059 & 5,556 & ---
    \\
    \emph{(98,117} & \multicolumn{6}{c|}{Average time: 1.48s}
    & \multicolumn{5}{c|}{Average time: 55.0s}
    \\
    \emph{~~bench.)} & \multicolumn{6}{c|}{Total \#sat: 273,927, \#unsat: 14,659}
    & \multicolumn{5}{c|}{Total \#paths covered: 173,007}
      \\\hline
  \end{tabular}
  \end{center}
  \caption{The number of queries answered by \ostrich, and number of
    paths covered by \expose{}, in the \textbf{R2} experiments. 
    Experiments were done on an AMD Opteron 2220 SE machine, running
    64-bit Linux and Java~1.8.  Runtime per benchmark was limited to
    60s wall clock time, memory to 2GB, and the number of tests
    executed concurrently by \expose\ to 1.  Average time is
    wall-clock time needed per benchmark, and counts timeouts as 60s.}
  \label{tab:exp-r2}
\vspace{-8mm}
\end{table}



The results of this experiment are shown in
Table~\ref{tab:exp-r2}. \ostrich\ is able to answer all four queries
in 96,126 of the match benchmarks (97.9\%), and all three queries in
95,735 of the replace-all benchmarks (97.6\%). The errors in 1,132 cases
are due to back-references in \verb!<re1>!, which are not handled by
\ostrich. \expose{} can on the match problems cover 228,888 paths in
total (91.5\% of the number of sat results of \ostrich), although the
runtime of \expose{} is on average 23x higher than that of
\ostrich. For replace, \expose{} can cover 173,007 paths (63.2\%),
showing that this class of constraints is harder; the runtime of
\expose{} is on average 37x higher than that of \ostrich.  Overall,
even taking into account that \expose{} has to analyze JavaScript
code, as opposed to the SMT-LIB given to \ostrich, the experiments
show that \ostrich\ is a highly competitive solver for RWREs.


\vspace{-2mm}

\paragraph{For \textbf{R3}:} We compare Aratha+{\ostrich} with
\expose{}+Z3 on the benchmarks from \textbf{R1}. In
Table~\ref{tab:exp-r1}, we can see that Aratha+{\ostrich} can within
60s cover slightly more paths than \expose{}+Z3. Aratha+{\ostrich} can
discover feasible paths more quickly than \expose{}+Z3, however: on
all three families of benchmarks, Aratha+{\ostrich} terminates on
average in less than 10s, and it discovers all paths 
within 35s. \expose{}+Z3 needs full 60s for
29 of the programs (``T.O.'' in the table),
and it finds new paths until the end of the 60s. Since \expose{}+Z3
handles the replace-all operation by unrolling, it is not able to
prove infeasibility of paths involving such operations, and will
therefore not terminate on some programs.
%is faster than \expose{}+Z3 by a factor between 1.7 and 15, while being
%able to cover more paths. The smaller speedup compared to the results
%for \textbf{R2} can be explained by the fact that Aratha should spend time in processing Javascript code, besides using {\ostrich} to solve path constraints.
%SMT-LIB queries
%produced by Aratha are not pure string constraints, but also include
%ADTs, arrays, and bit-vectors. 
Overall, the experiments indicate that {\ostrich} is more
efficient than the CEGAR-augmented symbolic execution for dealing
with RWREs.
