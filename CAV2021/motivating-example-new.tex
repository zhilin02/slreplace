%!TEX root = main.tex

\section{Motivating Example}\label{sec:mot}

\begin{figure}[htbp]
\begin{center}
\begin{minted}[linenos]{javascript}
function normalize(decimal)
{
  const decimalReg = /^(\d+)\.?(\d*)$/;
  var format = decimal.match(decimalReg);
  var result = "";
  if(format)
  {
   var integer = format[1].replace(/^0+/, "");
   var fractional  = format[2].replace(/0+$/, "");
   if(integer !== "") result = integer;
   else result = "0"; 
   if(fractional !== "") result = result + "." + fractional;
  }
  return result;
}
\end{minted}
\end{center}
\caption{Normalize a decimal by removing the leading and trailing zeros}
\label{fig-run-exmp}
\end{figure}

We use the JavaScript program in Figure~\ref{fig-run-exmp} as a motivating example to illustrate the main approach of this paper. 
The function {\tt normalize}  in Figure~\ref{fig-run-exmp} normalize a decimal string by removing the leading and trailing zeros. The input of {\tt normalize} is a string variable {\tt decimal}. For instance,  if $\tt{decimal} =$``02.50'', then $\tt{normalize(decimal)}$ returns ``2.5'', on the other hand, if $\tt{decimal} =$``0250'', then $\tt{normalize(decimal)}$ returns ``250'', moreover, if $\tt{decimal} =$``0.250'', then $\tt{normalize(decimal)}$ returns ``0.25''. 

%\tl{should the "and" in dblp be removed? Alice Brown, John Smith}\zhilin{I am referring to the bibtex style. Both ACM and DBLP bibtex style contain ``and''}

The input {\tt decimal} of the function {\tt normalize} is matched to a regular expression {\tt decimalReg}={\tt /\^{}({\footnotesize\textbackslash}d+){\footnotesize\textbackslash}.?({\footnotesize\textbackslash}d*)\$/}, which requires that  the input comprises a digit sequence representing the integer part of an input, possibly followed by a dot symbol as well as another digit sequence representing the fractional part (where the anchor symbols \^{} and $\$$ denote the beginning and end of an input). Note that  {\tt decimalReg} utilizes two capturing groups, namely, {\tt ({\footnotesize\textbackslash}d+)} and {\tt ({\footnotesize\textbackslash}d*)}, to record the integer and fractional part of the decimal. The expression {\tt {\footnotesize\textbackslash}.?} specifies that the dot symbol is optional, namely, it may not occur in the input. Moreover, it utilizes the \emph{greedy} semantics of the quantifier {\tt +} to enforce that {\tt {\footnotesize\textbackslash}d+} is matched to the whole string if the input decimal does not contain any occurrence of the dot symbol. For instance, if {\tt decimal} = ``0250'', then {\tt ({\footnotesize\textbackslash}d+)} is matched to ``0250'' and  {\tt ({\footnotesize\textbackslash}d*)} is matched to the empty string. Nevertheless, if the standard \emph{nondeterministic} semantics of {\tt +} is used, then the following matching is also valid: {\tt ({\footnotesize\textbackslash}d+)} is matched to ``02'' and {\tt ({\footnotesize\textbackslash}d*)} is matched to ``50''. The result of the matching, which is an array of strings, is stored into the variable {\tt format}. Since there are two capturing groups in {\tt decimalReg}, the array {\tt format} is of length 3.

Then the leading zeros are trimmed by applying {\tt replace(/\^{}0+/, "")} to {\tt format[1]} and the result is stored in the variable {\tt integer}. Dually, the trailing zeros are trimmed by {\tt replace(/{}0+\$/, "")} to {\tt format[2]} and the result is stored in the variable {\tt fractional}. Again, the greedy semantics of {\tt 0+} is used to trim \emph{all} the leading/trailing zeros.

The return value is obtained as follows: If {\tt integer} is empty, then it gets a default value ``0''. If {\tt fractional} is empty, then the return value is {\tt integer}. Othewise, the return value is the join of {\tt integer} and {\tt fractional} with the dot symbol. 

A natural invariant property of {\tt normalize} is that its output contains neither leading nor trailing zeros. The invariant property holds iff for along \emph{every} execution path of the program, the return value of {\tt normalize} satisfies the property. Therefore, the invariant property does not hold if along \emph{some} execution path of the program, the return value of {\tt normalize} does not satisfy the property. Take one typical execution path as an example, for the validity of the invariant property, it is necessary that the following program in single static assignment form is infeasible (namely, there does not exist a value of {\tt decimal} so that the program can execute to the end), 
\begin{figure}[htbp]
\begin{center}
\begin{minted}[linenos]{javascript}
  const decimalReg = /^(\d+)\.?(\d*)$/;
  var format = decimal.match(decimalReg);
  var result = "";
  assert(format!==null);
  var integer = format[1].replace(/^0+/, "");
  var fractional  = format[2].replace(/0+$/, "");
  assert(integer !== ""); 
  result1 = integer;
  assert(fractional !== ""); 
  result2 = result1 + "." + fractional;
  assert(/^0[1-9].*|.*\.\d*0$/.test(result2));
\end{minted}
\end{center}
\label{fig-run-exmp-path}
\end{figure}
where the regular expression {\tt /\^{}0[1-9].*|.*{\footnotesize\textbackslash}.d*0\$/} specifies the existence of leading or trailing zeros.

For a symbolic execution of the JavaScript program in Figure~\ref{fig-run-exmp-path}, one needs to model the greedy semantics of {\tt +} and the matchings of capturing groups. For this purpose, we introduce prioritized streaming string transducers (PSST, see Section~\ref{sect:psst}). Then {\tt format[1]}, namely {\tt decimal.match(decimalReg)[1]}, can be modelled by a PSST $\cT_{\tt match_{decimalReg,1}}$, where the priorities are used to model the greedy semantics of $+$ (see Definition~\ref{def-regex-semantics} in Section~\ref{sec:prel}) and the string variables are used to record the matchings of capturing groups. Similarly for  {\tt format[2]}. Moreover, the function {\tt replace(/\^{}0+/, "")} can also be modeled by a PSST $\cT_{\tt replace(\mbox{\tt /\^{}0+/, ""})}$. Then the program in Figure~\ref{fig-run-exmp-path} is transformed and simplified into the following program
\begin{eqnarray}\label{eqn:exmp}
& & \ASSERT{\tt decimal \in \Aut_{decimalReg}};\nonumber \\
& & \sf integer  := \tt  \cT_{\tt replace(\mbox{\scriptsize \tt /\^{}0+/, ""})}(\cT_{\tt match_{decimalReg,1}}(decimal));\nonumber \\
& & \sf fractional  := \tt  \cT_{\scriptsize\tt replace(\mbox{\tt /\^{}0+/, ""})}(\cT_{\tt match_{decimalReg,2}}(decimal));\nonumber \\
&&  \ASSERT{\tt integer \in \Aut_{\scriptsize\mbox{\tt.+}}}; 
%&&  \tt result1 := integer;\nonumber\\
\ASSERT{\tt fractional \in \Aut_{\scriptsize\mbox{\tt.+}}}; \nonumber\\
 && \tt result2 := integer \concat ``." \concat fractional; \nonumber\\
 && \ASSERT{\tt result2 \in \Aut_{\scriptsize\mbox{\tt /\^{}0[1-9].*|.*{\scriptsize\textbackslash}.d*0\$/}}}; 
\end{eqnarray}
where $\Aut_{\scriptsize\mbox{\tt.+}}$, $\Aut_{\tt decimalReg}$ and $\Aut_{\scriptsize\mbox{\tt /\^{}0[1-9].*|.*{\scriptsize\textbackslash}.d*0\$/}}$ are the finite automata corresponding to the three regular expressions.

The path feasibility problem of the program in Equation~(\ref{eqn:exmp}) is solved by a ``backward'' reasoning as follows (see Section~\ref{sec:decision} for the details):  
\begin{itemize}
\item At first, we compute the pre-image of $(\Aut_{\scriptsize\mbox{\tt /\^{}0[1-9].*|.*{\scriptsize\textbackslash}.d*0\$/}})$ under the concatenation $\concat$, which is a finite union of products of regular languages, remove $\tt result2 := integer \concat ``." \concat fractional$, select one disjunct of the union, say $(\Aut'_1, \Aut'_2)$, add the assertion $\ASSERT{\tt integer \in \Aut'_1};\ASSERT{\tt fractional \in \Aut'_2}$, resulting into the following program,
\begin{eqnarray}\label{eqn:exmp-2}
& & \ASSERT{\tt decimal \in \Aut_{decimalReg}};\nonumber \\
& & \tt integer  := \tt  \cT_{\tt replace(\mbox{\scriptsize \tt /\^{}0+/, ""})}(\cT_{\tt match_{decimalReg,1}}(decimal));\nonumber \\
& & \tt fractional  := \tt  \cT_{\scriptsize\tt replace(\mbox{\tt /\^{}0+/, ""})}(\cT_{\tt match_{decimalReg,2}}(decimal));\nonumber \\
&&  \ASSERT{\tt integer \in \Aut_{\scriptsize\mbox{\tt.+}}}; 
%\nonumber\\
%&&  \tt result1 := integer;\nonumber\\
  \ASSERT{\tt fractional \in \Aut_{\scriptsize\mbox{\tt.+}}}; \nonumber\\
% && \tt result2 := integer \concat ``." \concat fractional; \nonumber\\
 && \ASSERT{\tt result2 \in \Aut_{\scriptsize\mbox{\tt /\^{}0[1-9].*|.*{\scriptsize\textbackslash}.d*0\$/}}}; \nonumber\\
  && \ASSERT{\tt integer \in \Aut'_1};\ASSERT{\tt fractional \in \Aut'_2}; 
\end{eqnarray}
%
\item Next, we compute the pre-image of $\Aut'_2$ under $\cT_{\tt replace(\mbox{\scriptsize \tt /\^{}0+/, ""})} \circ \cT_{\tt match_{decimalReg,2}}$ (see Theorem~\ref{theorem:psst_preimage}), denoted by $\cB_1$, similarly we compute the pre-image of $\Aut_{\scriptsize\mbox{\tt.+}}$ under $\cT_{\tt replace(\mbox{\scriptsize \tt /\^{}0+/, ""})} \circ \cT_{\tt match_{decimalReg,2}}$,  denoted by $\cB_2$, then remove the assignment for  $\tt fractional$, and add $\ASSERT{\tt decimal \in \cB_1};\ASSERT{\tt decimal \in \cB_2}$. Similarly, we compute the pre-images, remove the assignment for {\tt integer}, and add the assertions $\ASSERT{\tt decimal \in \cC_1};\ASSERT{\tt decimal \in \cC_2}$. Then we obtain the following program containing no assignment statements, 
\begin{eqnarray}\label{eqn:exmp-2}
& & \ASSERT{\tt decimal \in \Aut_{decimalReg}};\nonumber \\
%& & \tt integer  := \tt  \cT_{\tt replace(\mbox{\scriptsize \tt /\^{}0+/, ""})}(\cT_{\tt match_{decimalReg,1}}(decimal));\nonumber \\
%& & \tt fractional  := \tt  \cT_{\scriptsize\tt replace(\mbox{\tt /\^{}0+/, ""})}(\cT_{\tt match_{decimalReg,2}}(decimal));\nonumber \\
&&  \ASSERT{\tt integer \in \Aut_{\scriptsize\mbox{\tt.+}}}; 
%\nonumber\\
%&&  \tt result1 := integer;\nonumber\\
  \ASSERT{\tt fractional \in \Aut_{\scriptsize\mbox{\tt.+}}}; \nonumber\\
% && \tt result2 := integer \concat ``." \concat fractional; \nonumber\\
 && \ASSERT{\tt result2 \in \Aut_{\scriptsize\mbox{\tt /\^{}0[1-9].*|.*{\scriptsize\textbackslash}.d*0\$/}}}; \nonumber\\
  && \ASSERT{\tt integer \in \Aut'_1};\ASSERT{\tt fractional \in \Aut'_2}; \nonumber\\
    && \ASSERT{\tt decimal \in \cB_1};\ASSERT{\tt decimal \in \cB_2}; \nonumber\\
    && \ASSERT{\tt decimal \in \cC_1};\ASSERT{\tt decimal \in \cC_2}; 
\end{eqnarray}
%
\item Finally, we check the nonemptiness of the intersection of the regular languages for the input variable $\tt decimal$, namely, $\Lang(\Aut_{\tt decimalReg})$, $\Lang(\cB_1)$, $\Lang(\cB_2)$, $\Lang(\cC_1)$, and $\Lang(\cC_2)$. If the intersection is nonempty, then the invariant property does \emph{not} hold.
\end{itemize}
