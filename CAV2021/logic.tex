%!TEX root = main.tex

\section{The string logic}\label{sec:logic}



%%%%%%%%%%%%%%%%%%%%%%%%%%%%%%%%%%%%%%%%


%\subsection{Prioritized Streaming String Transducer}
%Below, we introduce  a new class of prioritized transducer \cite{BM17} which combines the expressive power of streaming string transducer \cite{AC10,AD11}.
%
%\begin{definition}[Prioritized Streaming String Transducer]
%	A \emph{prioritized streaming string transducer} (PSST) is a tuple $\psst = (Q, \Sigma, X, E, \delta, q_0, F)$, where $Q$ a
%	finite set of states, $\Sigma$ is the input and output alphabet, and $X$ a finite set of variables. $E$ is a partial function from $Q \times \Sigma \times
%	Q$ to $X \rightarrow (X \cup \Sigma)^{\ast}$, i.e. the set of assignment,
%	$\delta \in Q \times \Sigma \rightarrow \overline{Q}$ and $F$ is a partial function
%	from $Q$ to $(X \cup \Sigma)^{\ast}$.
%\end{definition}
%
%A run of $\psst$ is the sequence $q_0 \sigma_1 s_1 q_1 \ldots \sigma_m s_m q_m$, where $F (q_m)$ is defined and for each $i \in [m], q_i \in \delta (q_{i-1}, \sigma_i)$ and $s_i = E (q_{i - 1}, \sigma_i, q_i)$. For any two runs on $w = \sigma_1 \ldots \sigma_m$, denoted by $p = q_0 \sigma_1 s_1 \ldots \sigma_m s_m q_m$ and $p' = q_0 \sigma_1
%s_1' \ldots \sigma_m s_m' q_m'$, we say that $p$ is of a higher priority over
%$p'$ if $p \neq p'$ and, for the smallest index $j$ with $q_j \neq q_j'$,
%$\delta (q_{j - 1}, \sigma_j) = \ldots q_j \ldots q_j' \ldots$
%
%The accepting run of $\psst$ on input $w$ is the run of the highest priority. The output of $T$ on w, denoted by $T(w)$, is defined as $\pi_m(F(q_m))$, where $\pi_0(x) = \varepsilon$ for each $x \in X$, and $\pi_{i}(x) = \pi_{i-1}(s_{i}(x))$ for $1 \le i \le m$ and $x \in X$. Note that here we abuse the notation  $\pi_m(F(q_m))$ and $\pi_{i-1}(s_{i}(x))$ by taking a function $\pi$ from $X$ to $\Sigma^*$ as a function from $(X \cup \Sigma)^*$ to $\Sigma^*$, which maps each $\sigma \in \Sigma$ to $\sigma$ and each $x \in X$ to $\pi(x)$.  
%
%%  $\tmop{Out} (r) =
%%  s_{\varepsilon} \circ s_1 \circ s_2 \ldots s_n \circ F (q_n)$ where
%%  $s_{\varepsilon}$ is the empty substitution which maps all variables to
%%  $\varepsilon$.
%
%\begin{definition}[pre-image]
%	For a string relation $R \subseteq \Sigma^* \times \Sigma^*$ and $L \subseteq \Sigma^*$, we define the \emph{pre-image} of $L$ under $R$ as $R^{-1}(L):=\{w \in \Sigma^* \mid \exists w'.\ w' \in L \mbox{ and } (w, w') \in R\}$. 
%\end{definition}
%
%\begin{theorem}[pre-image of \PSST{}]
%	\label{theorem:psst_preimage}
%	Given a \PSST{} $\psst = (Q_T, \Sigma$, $X, E, \delta_T, q_{0, T}, F_T)$ and \FA{} $A
%	= (Q_A, \Sigma, \delta_A, q_{0, A}, F_A)$, we can compute an \FA{} $B = (Q_B,
%	\Sigma, \delta_B, q_{0, B}, F_B)$ in exponential time  such that $\Lang(B) = \cR^{-1}_T(\Lang(\Aut))$.
%\end{theorem}
%
%\begin{proof}
%	Intuitively, $B$ simulates the run of $\psst$ on $w$, and, for each $x \in X$, records the set of state pairs $(p, q) \in Q_A \times Q_A$ such that starting from $p$, $A$ can reach $q$ after reading the string stored in $x$. Moreover, $B$ also records all the states accessible from a run with higher priority to ensure the current run is the accepting one of $\psst$.
%	
%	Formally, $Q_B = Q_T \times (\cP(Q_A \times Q_A ))^{X} \times \cP(Q_T)  $, $q_{0, B} = (q_{0, T}, \rho_{\varepsilon}, \emptyset)$ where $\rho_{\varepsilon} (x) = \{(q, q) \mid q \in Q\}$ for each $x \in X$, and $\delta_{B}$ comprises the tuples $((q, \rho, S), a, (q_i, \rho', S'))$ such that there exists $s \in \left((X \cup \Sigma\right)^*)^X$ satisfying
%	\begin{itemize}
%		\item $\delta_T (q, a) = (q_1 \ldots q_i \ldots q_m)$, 
%		\item $s = E(q,a,q_i)$.
%		\item $S' = \delta_T^{\ast} (S, a) \cup \{ q_1, \ldots, q_{i - 1} \}$, where $\delta_T^{\ast}(S,a) = \{q' \mid \exists q \in S, q' \in \delta_T(q,a)\}$.
%		\item and $\rho'$ is obtained from $\rho$ and $s$ as follows: for each $x \in X$, if $s(x) = \varepsilon$, then $\rho'(x) = \{(p, p) \mid p \in Q_A\}$, otherwise, let $s(x) = b_1 \cdots b_\ell$ with $b_i \in \Sigma \cup X$ for each $i \in [\ell]$, then $\rho'(x) = \theta_1 \circ \cdots \circ \theta_\ell$, where $\theta_i = \delta^{(b_i)}_A$ if $b_i \in \Sigma$, and $\theta_i = \rho(b_i)$ otherwise.
%		%
%		%$\rho'(x) = \theta_\ell$ such that $\theta_0 = \{(p,p) \mid p \in Q_A\}$, and for each $i \in [\ell]$, if $b_i \in \Sigma$, then $\theta_i = \{(p, p') \mid (p, p'') \in \theta_{i-1}, (p'', b_i, p') \in \delta_A \mbox{ for some } p''\}$, otherwise, $\theta_i = \theta_{i-1} \cdot \rho(x)$. 
%	\end{itemize}
%	
%	Moreover, $F_B$ is the set of states $(q, \rho, S) \in Q_B$ such that
%	\begin{enumerate}
%		\item $F_T (q)$ is defined,
%		\item For any $q' \in S$, $F_T (q')$ is not defined
%		
%		\item if $F_T(q) = \varepsilon$, then $q_{0, A}  \in F_A$, otherwise, 
%		let $F_T(q) = b_1 \cdots b_\ell$ with $b_i \in \Sigma \cup X$ for each $i \in [\ell]$, then $(\theta_1 \circ \cdots \circ \theta_\ell) \cap (\{q_{0,A}\} \times F_A) \neq \emptyset$, where for each $i \in [\ell]$, if $b_i \in \Sigma$, then $\theta_i = \delta^{(b_i)}_A$, otherwise, $\theta_i = \rho(b_i)$.
%	\end{enumerate}
%\end{proof}
%
%Note that the above construction  does not utilize the so-called \tmtextit{copyless} property \cite{AC10,AD11},
%thus it works for general, or \tmtextit{copyful} \PSST{} \cite{FR17}.

% Note that in the definition of \NSST, there is no \emph{copyless} restriction.



%%%%%%%%%%%%%%%%%%%%%%%%%%%%%%%%%%%%%%%%%%%%%%%%%%%%%


%We will use $\$ 1, \$2, \cdots$ to denote the references to capturing groups in regular expressions.

%We define the set of reference expressions as follows: 

%We consider the symbolic execution of the string-manipulating programs.

%\begin{definition}[The constraint language $\strline$] 
We define the string-manipulating language $\strline$ considered in this paper as follows. 
%The $\strline$ language is defined by
\[
\begin{array}{l}
S \eqdef  z:= x \concat y \mid y := \extract_{i, e}(x) \mid  
%& &  
%y := \reverse(x) 
y := \replaceall_{\pat, \rep}(x)   \mid 
%y := \Transducer(x)\  \mid\\
 \ASSERT{x \in e} \mid S; S\
\label{eq:SL}
%a ::= f(x_1,\ldots,x_n), \qquad b ::= g(x_1,\ldots,x_n)
\end{array}
\]
%\tl{to avoid confusion, write  $\ASSERT{x \in A}$?} 
where 
\begin{itemize}
	\item $\concat$ is the string concatenation operation which concatenates two strings,
%
\item for the $\extract$ function, $i \in \Nat$, $e \in \cgexp$,
%
	\item  for the $\replaceall$ operation, $\pat\in \cgexp$, $\rep \in \refexp$, %$\replaceall$ is the replace-all function to be defined shortly,
%	\item $\reverse$ is the string function which reverses a string; 
%	\item $\Transducer$ is a \PSST,
%
	\item for assertions, $e \in \regexp$.
\end{itemize} 
%and $R$ is a recognisable relation represented by a collection of tuples of \FA{}s.
%\end{definition}
%
%\zhilei{Should we add NSST constraints? Basically NSST can express more than PSST.}
%\tl{maybe just use NSST to replace PSST?}
%
%\zhilei{PSST is needed for decision procedure. NSST can be decided too, but the algorithm is very similar, so maybe too tedious to add both }

%It is evident that the $\reverse$ function is subsumed by \PSST{}s.

%
%\begin{remark}
%	Zhilin mentioned that we might introduce a function which takes a string and a pattern with capturing groups, and does sort of pattern matching to extra substrings. This function can be captured by the transducer $T$. We will formalise this later.
%\end{remark}

The $\extract$ function is used to model the regular-expression match function in programming languages.
%, e.g. $\sf str.match(regexp)$ function in Javascript. 
Specifically, the $\extract_{i, e}(x)$ function extracts the match of the $i$-th capturing group in the accepting match of $e$ to $x$ for $x \in \Lang(e)$ (otherwise, the return value of the function is undefined). Note that $\extract_{i, e}(x)$ returns $x$ if $i=0$. For instance, assuming $e = [[([\Gamma^+])\concat .?] \concat ([\Gamma^*])]$,   $\extract_{1, e}(0250)=0250$ and $\extract_{2, e}(0250)=\varepsilon$, as shown in Example~\ref{exmp-regex-semantics}. 

\begin{remark}
The match function in programming languages, e.g. $\sf str.match(reg)$ function in JavaScript, finds the first match of $\sf reg$ in $\sf str$. We can use $\extract$ to express the first match of $\sf reg$ in $\sf str$ by adding $[\Sigma^{*?}]$ and $[\Sigma^*]$ before and after $\sf reg$ respectively. More generally, the value of the $i$-th capturing group in the first match of a $\regexp$ $\sf reg$ in $\sf str$ can be specified as $\extract_{i+1, {\sf reg'}}({\sf str})$, where ${\sf reg'} = [[[\Sigma^{*?}] \concat ({\sf reg})] \concat [\Sigma^*]]$.
\end{remark}

The $\replaceall_{\pat, \rep}(x)$ function is parameterized by the  %\emph{subject} string, the second parameter is a 
\emph{pattern} $\pat \in \cgexp$ and the \emph{replacement} string $\rep \in \refexp$. For a given input string $x$, the function identifies all the %the first, second, $\dots$, 
matches of $\pat$ in $x$ and replace them with the corresponding strings specified by the replacement string. (In the replacement string,  references may be used which refer to %are replaced by 
the corresponding matches of the capturing groups.)  For instance, let $\pat = [[([\Gamma^+])\concat .?] \concat ([\Gamma^*])]$ and $\rep = \$1$. We have $\replaceall_{\pat, \rep}(2.5,3.4) = 2,3$. 

Without loss of generality, we assume that all the $\strline$ programs are in single static assignment (SSA) form, that is, each variable $x$ is assigned at most once. Moreover, if it is assigned, all its occurrences on the right hand sides of the assignment statements or in assertions are after the assignment statement of $x$.
%
For an $\strline$ program $S$, a variable $x$ occurring in $S$ is said to be an \emph{input} variable if $x$ does not occur on the left hand sides of assignment statements. The \emph{path feasibility} problem of an $\strline$ program is to decide whether there are valuations of the input variables such that the program can execute to the end.


%
%For the semantics of $\replaceall$ function, in particular when the pattern is a regular expression, we adopt the \emph{leftmost and longest} matching. 




%For instance, $\replaceall(aababaab, (ab)^+, c) =ac\cdot \replaceall(aab, (ab)^+, c)= acac$, since the leftmost and longest matching of $(ab)^+$ in $aababaab$ is $abab$. Here we require that the language defined by the pattern parameter does \emph{not} contain the empty string, in order to avoid the troublesome definition of the semantics of the matching of the empty string. We refer the reader to \cite{CCHLW18} for the formal semantics of the $\replaceall$ function. To be consistent with the notation in this paper, for each regular expression $e$, we define
%the string function $\replaceall_e: \ialphabet^* \times \ialphabet^* \rightarrow \ialphabet^*$ such that for $u, v \in \ialphabet^*$, $\replaceall_e(u, v) = \replaceall(u, e, v)$, and we write $\replaceall(x, e, y)$ as $\replaceall_e(x,y)$.

It turns out that the path feasibility problem is undecidable, attributed to the the back references in assertion statements. 

\begin{proposition}\label{prop-und}
The path feasibility problem of $\strline$ is undecidable.
\end{proposition}

We shall show that the path feasibility problem is decidable, if the uses of back references in assertion statements are forbidden, which turns out to be the situation in practice.\footnote{A partial evidence is that the occurrences of regular expressions with back references occupy only less than $1\%$ in the NPM package, according to the statistics collected in \cite{LMK19}.} In the sequel, we will use $\strline_{\sf reg}$ to denote the collection of $\strline$ programs which are free of %where no 
back references in assertion statements. %We state 
The main result of this paper is as follows.

\begin{theorem}\label{thm-main}
The path feasibility of $\strline_{\sf reg}$ is decidable in XXX. \zhilin{complexity should be added}
\end{theorem}
The decision procedure for $\strline_{\sf reg}$ utilizes a new model called prioritized streaming string transducers, which will be defined in the next section.
